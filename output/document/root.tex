\documentclass[11pt,a4paper]{article}
\usepackage[T1]{fontenc}
\usepackage{isabelle,isabellesym}

% further packages required for unusual symbols (see also
% isabellesym.sty), use only when needed

%\usepackage{amssymb}
  %for \<leadsto>, \<box>, \<diamond>, \<sqsupset>, \<mho>, \<Join>,
  %\<lhd>, \<lesssim>, \<greatersim>, \<lessapprox>, \<greaterapprox>,
  %\<triangleq>, \<yen>, \<lozenge>

%\usepackage{eurosym}
  %for \<euro>

%\usepackage[only,bigsqcap,bigparallel,fatsemi,interleave,sslash]{stmaryrd}
  %for \<Sqinter>, \<Parallel>, \<Zsemi>, \<Parallel>, \<sslash>

%\usepackage{eufrak}
  %for \<AA> ... \<ZZ>, \<aa> ... \<zz> (also included in amssymb)

%\usepackage{textcomp}
  %for \<onequarter>, \<onehalf>, \<threequarters>, \<degree>, \<cent>,
  %\<currency>

% this should be the last package used
\usepackage{pdfsetup}

% urls in roman style, theory text in math-similar italics
\urlstyle{rm}
\isabellestyle{it}

% for uniform font size
%\renewcommand{\isastyle}{\isastyleminor}


\begin{document}

\title{Verified Enumeration of Trees}
\author{Nils Cremer}
\maketitle

\begin{abstract}
This thesis presents the verification of enumeration algorithms for trees.
The first algorithm is based on the well known Prüfer-correspondence and allows the enumeration of all possible labeled trees over a fixed finite set of vertices.
The second algorithm enumerates rooted, unlabeled trees of a specified size up to graph isomorphisms.
It allows for the efficient enumeration without the use of an intermediate encoding of the trees with level sequences, unlike the algorithm by Beyer and Hedetniemi~\cite{beyer} it is based on.
Both algorithms are formalized and verified in Isabelle/HOL.
The formalization of trees and other graph theoretic results is also presented.
\end{abstract}

\tableofcontents

% sane default for proof documents
\parindent 0pt\parskip 0.5ex

% generated text of all theories
%
\begin{isabellebody}%
\setisabellecontext{Tree{\isacharunderscore}{\kern0pt}Graph}%
%
\isadelimdocument
%
\endisadelimdocument
%
\isatagdocument
%
\isamarkupsection{Trees%
}
\isamarkuptrue%
%
\endisatagdocument
{\isafolddocument}%
%
\isadelimdocument
%
\endisadelimdocument
%
\isadelimtheory
%
\endisadelimtheory
%
\isatagtheory
\isacommand{theory}\isamarkupfalse%
\ Tree{\isacharunderscore}{\kern0pt}Graph\isanewline
\ \ \isakeyword{imports}\ Undirected{\isacharunderscore}{\kern0pt}Graph{\isacharunderscore}{\kern0pt}Theory{\isachardot}{\kern0pt}Undirected{\isacharunderscore}{\kern0pt}Graphs{\isacharunderscore}{\kern0pt}Root\isanewline
\isakeyword{begin}%
\endisatagtheory
{\isafoldtheory}%
%
\isadelimtheory
%
\endisadelimtheory
%
\isadelimdocument
%
\endisadelimdocument
%
\isatagdocument
%
\isamarkupsubsection{Misc%
}
\isamarkuptrue%
%
\endisatagdocument
{\isafolddocument}%
%
\isadelimdocument
%
\endisadelimdocument
\isacommand{definition}\isamarkupfalse%
\ {\isacharparenleft}{\kern0pt}\isakeyword{in}\ ulgraph{\isacharparenright}{\kern0pt}\ loops\ {\isacharcolon}{\kern0pt}{\isacharcolon}{\kern0pt}\ {\isachardoublequoteopen}{\isacharprime}{\kern0pt}a\ edge\ set{\isachardoublequoteclose}\ \isakeyword{where}\isanewline
\ \ {\isachardoublequoteopen}loops\ {\isacharequal}{\kern0pt}\ {\isacharbraceleft}{\kern0pt}e{\isasymin}E{\isachardot}{\kern0pt}\ is{\isacharunderscore}{\kern0pt}loop\ e{\isacharbraceright}{\kern0pt}{\isachardoublequoteclose}\isanewline
\isanewline
\isacommand{definition}\isamarkupfalse%
\ {\isacharparenleft}{\kern0pt}\isakeyword{in}\ ulgraph{\isacharparenright}{\kern0pt}\ sedges\ {\isacharcolon}{\kern0pt}{\isacharcolon}{\kern0pt}\ {\isachardoublequoteopen}{\isacharprime}{\kern0pt}a\ edge\ set{\isachardoublequoteclose}\ \isakeyword{where}\isanewline
\ \ {\isachardoublequoteopen}sedges\ {\isacharequal}{\kern0pt}\ {\isacharbraceleft}{\kern0pt}e{\isasymin}E{\isachardot}{\kern0pt}\ is{\isacharunderscore}{\kern0pt}sedge\ e{\isacharbraceright}{\kern0pt}{\isachardoublequoteclose}\isanewline
\isanewline
\isacommand{lemma}\isamarkupfalse%
\ {\isacharparenleft}{\kern0pt}\isakeyword{in}\ ulgraph{\isacharparenright}{\kern0pt}\ union{\isacharunderscore}{\kern0pt}loops{\isacharunderscore}{\kern0pt}sedges{\isacharcolon}{\kern0pt}\ {\isachardoublequoteopen}loops\ {\isasymunion}\ sedges\ {\isacharequal}{\kern0pt}\ E{\isachardoublequoteclose}\isanewline
%
\isadelimproof
\ \ %
\endisadelimproof
%
\isatagproof
\isacommand{unfolding}\isamarkupfalse%
\ loops{\isacharunderscore}{\kern0pt}def\ sedges{\isacharunderscore}{\kern0pt}def\ is{\isacharunderscore}{\kern0pt}loop{\isacharunderscore}{\kern0pt}def\ is{\isacharunderscore}{\kern0pt}sedge{\isacharunderscore}{\kern0pt}def\ \isacommand{using}\isamarkupfalse%
\ alt{\isacharunderscore}{\kern0pt}edge{\isacharunderscore}{\kern0pt}size\ \isacommand{by}\isamarkupfalse%
\ blast%
\endisatagproof
{\isafoldproof}%
%
\isadelimproof
\isanewline
%
\endisadelimproof
\isanewline
\isacommand{lemma}\isamarkupfalse%
\ {\isacharparenleft}{\kern0pt}\isakeyword{in}\ ulgraph{\isacharparenright}{\kern0pt}\ disjnt{\isacharunderscore}{\kern0pt}loops{\isacharunderscore}{\kern0pt}sedges{\isacharcolon}{\kern0pt}\ {\isachardoublequoteopen}disjnt\ loops\ sedges{\isachardoublequoteclose}\isanewline
%
\isadelimproof
\ \ %
\endisadelimproof
%
\isatagproof
\isacommand{unfolding}\isamarkupfalse%
\ disjnt{\isacharunderscore}{\kern0pt}def\ loops{\isacharunderscore}{\kern0pt}def\ sedges{\isacharunderscore}{\kern0pt}def\ is{\isacharunderscore}{\kern0pt}loop{\isacharunderscore}{\kern0pt}def\ is{\isacharunderscore}{\kern0pt}sedge{\isacharunderscore}{\kern0pt}def\ \isacommand{by}\isamarkupfalse%
\ auto%
\endisatagproof
{\isafoldproof}%
%
\isadelimproof
\isanewline
%
\endisadelimproof
\isanewline
\isacommand{lemma}\isamarkupfalse%
\ {\isacharparenleft}{\kern0pt}\isakeyword{in}\ fin{\isacharunderscore}{\kern0pt}ulgraph{\isacharparenright}{\kern0pt}\ finite{\isacharunderscore}{\kern0pt}loops{\isacharcolon}{\kern0pt}\ {\isachardoublequoteopen}finite\ loops{\isachardoublequoteclose}\isanewline
%
\isadelimproof
\ \ %
\endisadelimproof
%
\isatagproof
\isacommand{unfolding}\isamarkupfalse%
\ loops{\isacharunderscore}{\kern0pt}def\ \isacommand{using}\isamarkupfalse%
\ fin{\isacharunderscore}{\kern0pt}edges\ \isacommand{by}\isamarkupfalse%
\ auto%
\endisatagproof
{\isafoldproof}%
%
\isadelimproof
\isanewline
%
\endisadelimproof
\isanewline
\isacommand{lemma}\isamarkupfalse%
\ {\isacharparenleft}{\kern0pt}\isakeyword{in}\ fin{\isacharunderscore}{\kern0pt}ulgraph{\isacharparenright}{\kern0pt}\ finite{\isacharunderscore}{\kern0pt}sedges{\isacharcolon}{\kern0pt}\ {\isachardoublequoteopen}finite\ sedges{\isachardoublequoteclose}\isanewline
%
\isadelimproof
\ \ %
\endisadelimproof
%
\isatagproof
\isacommand{unfolding}\isamarkupfalse%
\ sedges{\isacharunderscore}{\kern0pt}def\ \isacommand{using}\isamarkupfalse%
\ fin{\isacharunderscore}{\kern0pt}edges\ \isacommand{by}\isamarkupfalse%
\ auto%
\endisatagproof
{\isafoldproof}%
%
\isadelimproof
\isanewline
%
\endisadelimproof
\isanewline
\isacommand{lemma}\isamarkupfalse%
\ {\isacharparenleft}{\kern0pt}\isakeyword{in}\ ulgraph{\isacharparenright}{\kern0pt}\ edge{\isacharunderscore}{\kern0pt}incident{\isacharunderscore}{\kern0pt}vert{\isacharcolon}{\kern0pt}\ {\isachardoublequoteopen}e\ {\isasymin}\ E\ {\isasymLongrightarrow}\ {\isasymexists}v{\isasymin}V{\isachardot}{\kern0pt}\ incident\ v\ e{\isachardoublequoteclose}\isanewline
%
\isadelimproof
\ \ %
\endisadelimproof
%
\isatagproof
\isacommand{using}\isamarkupfalse%
\ edge{\isacharunderscore}{\kern0pt}size\ wellformed\ \isacommand{by}\isamarkupfalse%
\ {\isacharparenleft}{\kern0pt}metis\ empty{\isacharunderscore}{\kern0pt}not{\isacharunderscore}{\kern0pt}edge\ equals{\isadigit{0}}I\ incident{\isacharunderscore}{\kern0pt}def\ incident{\isacharunderscore}{\kern0pt}edge{\isacharunderscore}{\kern0pt}in{\isacharunderscore}{\kern0pt}wf{\isacharparenright}{\kern0pt}%
\endisatagproof
{\isafoldproof}%
%
\isadelimproof
\isanewline
%
\endisadelimproof
\isanewline
\isacommand{lemma}\isamarkupfalse%
\ {\isacharparenleft}{\kern0pt}\isakeyword{in}\ ulgraph{\isacharparenright}{\kern0pt}\ Union{\isacharunderscore}{\kern0pt}incident{\isacharunderscore}{\kern0pt}edges{\isacharcolon}{\kern0pt}\ {\isachardoublequoteopen}{\isacharparenleft}{\kern0pt}{\isasymUnion}v{\isasymin}V{\isachardot}{\kern0pt}\ incident{\isacharunderscore}{\kern0pt}edges\ v{\isacharparenright}{\kern0pt}\ {\isacharequal}{\kern0pt}\ E{\isachardoublequoteclose}\isanewline
%
\isadelimproof
\ \ %
\endisadelimproof
%
\isatagproof
\isacommand{unfolding}\isamarkupfalse%
\ incident{\isacharunderscore}{\kern0pt}edges{\isacharunderscore}{\kern0pt}def\ \isacommand{using}\isamarkupfalse%
\ edge{\isacharunderscore}{\kern0pt}incident{\isacharunderscore}{\kern0pt}vert\ \isacommand{by}\isamarkupfalse%
\ auto%
\endisatagproof
{\isafoldproof}%
%
\isadelimproof
\isanewline
%
\endisadelimproof
\isanewline
\isanewline
\isacommand{lemma}\isamarkupfalse%
\ {\isacharparenleft}{\kern0pt}\isakeyword{in}\ ulgraph{\isacharparenright}{\kern0pt}\ induced{\isacharunderscore}{\kern0pt}edges{\isacharunderscore}{\kern0pt}mono{\isacharcolon}{\kern0pt}\ {\isachardoublequoteopen}V\isactrlsub {\isadigit{1}}\ {\isasymsubseteq}\ V\isactrlsub {\isadigit{2}}\ {\isasymLongrightarrow}\ induced{\isacharunderscore}{\kern0pt}edges\ V\isactrlsub {\isadigit{1}}\ {\isasymsubseteq}\ induced{\isacharunderscore}{\kern0pt}edges\ V\isactrlsub {\isadigit{2}}{\isachardoublequoteclose}\isanewline
%
\isadelimproof
\ \ %
\endisadelimproof
%
\isatagproof
\isacommand{using}\isamarkupfalse%
\ induced{\isacharunderscore}{\kern0pt}edges{\isacharunderscore}{\kern0pt}def\ \isacommand{by}\isamarkupfalse%
\ auto%
\endisatagproof
{\isafoldproof}%
%
\isadelimproof
\isanewline
%
\endisadelimproof
\isanewline
\isacommand{definition}\isamarkupfalse%
\ {\isacharparenleft}{\kern0pt}\isakeyword{in}\ graph{\isacharunderscore}{\kern0pt}system{\isacharparenright}{\kern0pt}\ remove{\isacharunderscore}{\kern0pt}vertex\ {\isacharcolon}{\kern0pt}{\isacharcolon}{\kern0pt}\ {\isachardoublequoteopen}{\isacharprime}{\kern0pt}a\ {\isasymRightarrow}\ {\isacharprime}{\kern0pt}a\ pregraph{\isachardoublequoteclose}\ \isakeyword{where}\isanewline
\ \ {\isachardoublequoteopen}remove{\isacharunderscore}{\kern0pt}vertex\ v\ {\isacharequal}{\kern0pt}\ {\isacharparenleft}{\kern0pt}V\ {\isacharminus}{\kern0pt}\ {\isacharbraceleft}{\kern0pt}v{\isacharbraceright}{\kern0pt}{\isacharcomma}{\kern0pt}\ {\isacharbraceleft}{\kern0pt}e{\isasymin}E{\isachardot}{\kern0pt}\ {\isasymnot}\ incident\ v\ e{\isacharbraceright}{\kern0pt}{\isacharparenright}{\kern0pt}{\isachardoublequoteclose}%
\isadelimdocument
%
\endisadelimdocument
%
\isatagdocument
%
\isamarkupsubsection{Degree%
}
\isamarkuptrue%
%
\endisatagdocument
{\isafolddocument}%
%
\isadelimdocument
%
\endisadelimdocument
\isacommand{lemma}\isamarkupfalse%
\ {\isacharparenleft}{\kern0pt}\isakeyword{in}\ ulgraph{\isacharparenright}{\kern0pt}\ empty{\isacharunderscore}{\kern0pt}E{\isacharunderscore}{\kern0pt}degree{\isacharunderscore}{\kern0pt}{\isadigit{0}}{\isacharcolon}{\kern0pt}\ {\isachardoublequoteopen}E\ {\isacharequal}{\kern0pt}\ {\isacharbraceleft}{\kern0pt}{\isacharbraceright}{\kern0pt}\ {\isasymLongrightarrow}\ degree\ v\ {\isacharequal}{\kern0pt}\ {\isadigit{0}}{\isachardoublequoteclose}\isanewline
%
\isadelimproof
\ \ %
\endisadelimproof
%
\isatagproof
\isacommand{using}\isamarkupfalse%
\ incident{\isacharunderscore}{\kern0pt}edges{\isacharunderscore}{\kern0pt}empty\ degree{\isadigit{0}}{\isacharunderscore}{\kern0pt}inc{\isacharunderscore}{\kern0pt}edges{\isacharunderscore}{\kern0pt}empt{\isacharunderscore}{\kern0pt}iff\ \isacommand{unfolding}\isamarkupfalse%
\ incident{\isacharunderscore}{\kern0pt}edges{\isacharunderscore}{\kern0pt}def\ \isacommand{by}\isamarkupfalse%
\ simp%
\endisatagproof
{\isafoldproof}%
%
\isadelimproof
\isanewline
%
\endisadelimproof
\isanewline
\isacommand{lemma}\isamarkupfalse%
\ {\isacharparenleft}{\kern0pt}\isakeyword{in}\ fin{\isacharunderscore}{\kern0pt}ulgraph{\isacharparenright}{\kern0pt}\ handshaking{\isacharcolon}{\kern0pt}\ {\isachardoublequoteopen}{\isacharparenleft}{\kern0pt}{\isasymSum}v{\isasymin}V{\isachardot}{\kern0pt}\ degree\ v{\isacharparenright}{\kern0pt}\ {\isacharequal}{\kern0pt}\ {\isadigit{2}}\ {\isacharasterisk}{\kern0pt}\ card\ E{\isachardoublequoteclose}\isanewline
%
\isadelimproof
\ \ %
\endisadelimproof
%
\isatagproof
\isacommand{using}\isamarkupfalse%
\ fin{\isacharunderscore}{\kern0pt}edges\ fin{\isacharunderscore}{\kern0pt}ulgraph{\isacharunderscore}{\kern0pt}axioms\isanewline
\isacommand{proof}\isamarkupfalse%
\ {\isacharparenleft}{\kern0pt}induction\ E{\isacharparenright}{\kern0pt}\isanewline
\ \ \isacommand{case}\isamarkupfalse%
\ empty\isanewline
\ \ \isacommand{then}\isamarkupfalse%
\ \isacommand{interpret}\isamarkupfalse%
\ g{\isacharcolon}{\kern0pt}\ fin{\isacharunderscore}{\kern0pt}ulgraph\ V\ {\isachardoublequoteopen}{\isacharbraceleft}{\kern0pt}{\isacharbraceright}{\kern0pt}{\isachardoublequoteclose}\ \isacommand{{\isachardot}{\kern0pt}}\isamarkupfalse%
\isanewline
\ \ \isacommand{show}\isamarkupfalse%
\ {\isacharquery}{\kern0pt}case\ \isacommand{using}\isamarkupfalse%
\ g{\isachardot}{\kern0pt}empty{\isacharunderscore}{\kern0pt}E{\isacharunderscore}{\kern0pt}degree{\isacharunderscore}{\kern0pt}{\isadigit{0}}\ \isacommand{by}\isamarkupfalse%
\ simp\isanewline
\isacommand{next}\isamarkupfalse%
\isanewline
\ \ \isacommand{case}\isamarkupfalse%
\ {\isacharparenleft}{\kern0pt}insert\ e\ E{\isacharprime}{\kern0pt}{\isacharparenright}{\kern0pt}\isanewline
\ \ \isacommand{then}\isamarkupfalse%
\ \isacommand{interpret}\isamarkupfalse%
\ g{\isacharprime}{\kern0pt}{\isacharcolon}{\kern0pt}\ fin{\isacharunderscore}{\kern0pt}ulgraph\ V\ {\isachardoublequoteopen}insert\ e\ E{\isacharprime}{\kern0pt}{\isachardoublequoteclose}\ \isacommand{by}\isamarkupfalse%
\ blast\isanewline
\ \ \isacommand{interpret}\isamarkupfalse%
\ g{\isacharcolon}{\kern0pt}\ fin{\isacharunderscore}{\kern0pt}ulgraph\ V\ E{\isacharprime}{\kern0pt}\ \isacommand{using}\isamarkupfalse%
\ g{\isacharprime}{\kern0pt}{\isachardot}{\kern0pt}wellformed\ g{\isacharprime}{\kern0pt}{\isachardot}{\kern0pt}edge{\isacharunderscore}{\kern0pt}size\ finV\ \isacommand{by}\isamarkupfalse%
\ {\isacharparenleft}{\kern0pt}unfold{\isacharunderscore}{\kern0pt}locales{\isacharcomma}{\kern0pt}\ auto{\isacharparenright}{\kern0pt}\isanewline
\ \ \isacommand{show}\isamarkupfalse%
\ {\isacharquery}{\kern0pt}case\isanewline
\ \ \isacommand{proof}\isamarkupfalse%
\ {\isacharparenleft}{\kern0pt}cases\ {\isachardoublequoteopen}is{\isacharunderscore}{\kern0pt}loop\ e{\isachardoublequoteclose}{\isacharparenright}{\kern0pt}\isanewline
\ \ \ \ \isacommand{case}\isamarkupfalse%
\ True\isanewline
\ \ \ \ \isacommand{then}\isamarkupfalse%
\ \isacommand{obtain}\isamarkupfalse%
\ u\ \isakeyword{where}\ e{\isacharcolon}{\kern0pt}\ {\isachardoublequoteopen}e\ {\isacharequal}{\kern0pt}\ {\isacharbraceleft}{\kern0pt}u{\isacharbraceright}{\kern0pt}{\isachardoublequoteclose}\ \isacommand{using}\isamarkupfalse%
\ card{\isacharunderscore}{\kern0pt}{\isadigit{1}}{\isacharunderscore}{\kern0pt}singletonE\ is{\isacharunderscore}{\kern0pt}loop{\isacharunderscore}{\kern0pt}def\ \isacommand{by}\isamarkupfalse%
\ blast\isanewline
\ \ \ \ \isacommand{then}\isamarkupfalse%
\ \isacommand{have}\isamarkupfalse%
\ inc{\isacharunderscore}{\kern0pt}sedges{\isacharcolon}{\kern0pt}\ {\isachardoublequoteopen}{\isasymAnd}v{\isachardot}{\kern0pt}\ g{\isacharprime}{\kern0pt}{\isachardot}{\kern0pt}incident{\isacharunderscore}{\kern0pt}sedges\ v\ {\isacharequal}{\kern0pt}\ g{\isachardot}{\kern0pt}incident{\isacharunderscore}{\kern0pt}sedges\ v{\isachardoublequoteclose}\ \isacommand{unfolding}\isamarkupfalse%
\ g{\isacharprime}{\kern0pt}{\isachardot}{\kern0pt}incident{\isacharunderscore}{\kern0pt}sedges{\isacharunderscore}{\kern0pt}def\ g{\isachardot}{\kern0pt}incident{\isacharunderscore}{\kern0pt}sedges{\isacharunderscore}{\kern0pt}def\ \isacommand{by}\isamarkupfalse%
\ auto\isanewline
\ \ \ \ \isacommand{have}\isamarkupfalse%
\ {\isachardoublequoteopen}{\isasymAnd}v{\isachardot}{\kern0pt}\ v\ {\isasymnoteq}\ u\ {\isasymLongrightarrow}\ g{\isacharprime}{\kern0pt}{\isachardot}{\kern0pt}incident{\isacharunderscore}{\kern0pt}loops\ v\ {\isacharequal}{\kern0pt}\ g{\isachardot}{\kern0pt}incident{\isacharunderscore}{\kern0pt}loops\ v{\isachardoublequoteclose}\ \isacommand{unfolding}\isamarkupfalse%
\ g{\isacharprime}{\kern0pt}{\isachardot}{\kern0pt}incident{\isacharunderscore}{\kern0pt}loops{\isacharunderscore}{\kern0pt}def\ g{\isachardot}{\kern0pt}incident{\isacharunderscore}{\kern0pt}loops{\isacharunderscore}{\kern0pt}def\ \isacommand{using}\isamarkupfalse%
\ e\ \isacommand{by}\isamarkupfalse%
\ auto\isanewline
\ \ \ \ \isacommand{then}\isamarkupfalse%
\ \isacommand{have}\isamarkupfalse%
\ degree{\isacharunderscore}{\kern0pt}not{\isacharunderscore}{\kern0pt}u{\isacharcolon}{\kern0pt}\ {\isachardoublequoteopen}{\isasymAnd}v{\isachardot}{\kern0pt}\ v\ {\isasymnoteq}\ u\ {\isasymLongrightarrow}\ g{\isacharprime}{\kern0pt}{\isachardot}{\kern0pt}degree\ v\ {\isacharequal}{\kern0pt}\ g{\isachardot}{\kern0pt}degree\ v{\isachardoublequoteclose}\ \isacommand{using}\isamarkupfalse%
\ inc{\isacharunderscore}{\kern0pt}sedges\ \isacommand{unfolding}\isamarkupfalse%
\ g{\isacharprime}{\kern0pt}{\isachardot}{\kern0pt}degree{\isacharunderscore}{\kern0pt}def\ g{\isachardot}{\kern0pt}degree{\isacharunderscore}{\kern0pt}def\ \isacommand{by}\isamarkupfalse%
\ auto\isanewline
\ \ \ \ \isacommand{have}\isamarkupfalse%
\ {\isachardoublequoteopen}g{\isacharprime}{\kern0pt}{\isachardot}{\kern0pt}incident{\isacharunderscore}{\kern0pt}loops\ u\ {\isacharequal}{\kern0pt}\ g{\isachardot}{\kern0pt}incident{\isacharunderscore}{\kern0pt}loops\ u\ {\isasymunion}\ {\isacharbraceleft}{\kern0pt}e{\isacharbraceright}{\kern0pt}{\isachardoublequoteclose}\ \isacommand{unfolding}\isamarkupfalse%
\ g{\isacharprime}{\kern0pt}{\isachardot}{\kern0pt}incident{\isacharunderscore}{\kern0pt}loops{\isacharunderscore}{\kern0pt}def\ g{\isachardot}{\kern0pt}incident{\isacharunderscore}{\kern0pt}loops{\isacharunderscore}{\kern0pt}def\ \isacommand{using}\isamarkupfalse%
\ e\ \isacommand{by}\isamarkupfalse%
\ auto\isanewline
\ \ \ \ \isacommand{then}\isamarkupfalse%
\ \isacommand{have}\isamarkupfalse%
\ degree{\isacharunderscore}{\kern0pt}u{\isacharcolon}{\kern0pt}\ {\isachardoublequoteopen}g{\isacharprime}{\kern0pt}{\isachardot}{\kern0pt}degree\ u\ {\isacharequal}{\kern0pt}\ g{\isachardot}{\kern0pt}degree\ u\ {\isacharplus}{\kern0pt}\ {\isadigit{2}}{\isachardoublequoteclose}\ \isacommand{using}\isamarkupfalse%
\ inc{\isacharunderscore}{\kern0pt}sedges\ insert{\isacharparenleft}{\kern0pt}{\isadigit{2}}{\isacharparenright}{\kern0pt}\ g{\isachardot}{\kern0pt}finite{\isacharunderscore}{\kern0pt}incident{\isacharunderscore}{\kern0pt}loops\ g{\isachardot}{\kern0pt}incident{\isacharunderscore}{\kern0pt}loops{\isacharunderscore}{\kern0pt}def\ \isacommand{unfolding}\isamarkupfalse%
\ g{\isacharprime}{\kern0pt}{\isachardot}{\kern0pt}degree{\isacharunderscore}{\kern0pt}def\ g{\isachardot}{\kern0pt}degree{\isacharunderscore}{\kern0pt}def\ \isacommand{by}\isamarkupfalse%
\ auto\isanewline
\ \ \ \ \isacommand{have}\isamarkupfalse%
\ {\isachardoublequoteopen}u\ {\isasymin}\ V{\isachardoublequoteclose}\ \isacommand{using}\isamarkupfalse%
\ e\ g{\isacharprime}{\kern0pt}{\isachardot}{\kern0pt}wellformed\ \isacommand{by}\isamarkupfalse%
\ blast\isanewline
\ \ \ \ \isacommand{then}\isamarkupfalse%
\ \isacommand{have}\isamarkupfalse%
\ {\isachardoublequoteopen}{\isacharparenleft}{\kern0pt}{\isasymSum}v{\isasymin}V{\isachardot}{\kern0pt}\ g{\isacharprime}{\kern0pt}{\isachardot}{\kern0pt}degree\ v{\isacharparenright}{\kern0pt}\ {\isacharequal}{\kern0pt}\ g{\isacharprime}{\kern0pt}{\isachardot}{\kern0pt}degree\ u\ {\isacharplus}{\kern0pt}\ {\isacharparenleft}{\kern0pt}{\isasymSum}v{\isasymin}V{\isacharminus}{\kern0pt}{\isacharbraceleft}{\kern0pt}u{\isacharbraceright}{\kern0pt}{\isachardot}{\kern0pt}\ g{\isacharprime}{\kern0pt}{\isachardot}{\kern0pt}degree\ v{\isacharparenright}{\kern0pt}{\isachardoublequoteclose}\isanewline
\ \ \ \ \ \ \isacommand{by}\isamarkupfalse%
\ {\isacharparenleft}{\kern0pt}simp\ add{\isacharcolon}{\kern0pt}\ finV\ sum{\isachardot}{\kern0pt}remove{\isacharparenright}{\kern0pt}\isanewline
\ \ \ \ \isacommand{also}\isamarkupfalse%
\ \isacommand{have}\isamarkupfalse%
\ {\isachardoublequoteopen}{\isasymdots}\ {\isacharequal}{\kern0pt}\ {\isacharparenleft}{\kern0pt}{\isasymSum}v{\isasymin}V{\isachardot}{\kern0pt}\ g{\isachardot}{\kern0pt}degree\ v{\isacharparenright}{\kern0pt}\ {\isacharplus}{\kern0pt}\ {\isadigit{2}}{\isachardoublequoteclose}\ \isacommand{using}\isamarkupfalse%
\ degree{\isacharunderscore}{\kern0pt}not{\isacharunderscore}{\kern0pt}u\ degree{\isacharunderscore}{\kern0pt}u\ sum{\isachardot}{\kern0pt}remove{\isacharbrackleft}{\kern0pt}OF\ finV\ {\isacartoucheopen}u{\isasymin}V{\isacartoucheclose}{\isacharcomma}{\kern0pt}\ of\ g{\isachardot}{\kern0pt}degree{\isacharbrackright}{\kern0pt}\ \isacommand{by}\isamarkupfalse%
\ auto\isanewline
\ \ \ \ \isacommand{also}\isamarkupfalse%
\ \isacommand{have}\isamarkupfalse%
\ {\isachardoublequoteopen}{\isasymdots}\ {\isacharequal}{\kern0pt}\ {\isadigit{2}}\ {\isacharasterisk}{\kern0pt}\ card\ {\isacharparenleft}{\kern0pt}insert\ e\ E{\isacharprime}{\kern0pt}{\isacharparenright}{\kern0pt}{\isachardoublequoteclose}\ \isacommand{using}\isamarkupfalse%
\ insert\ g{\isachardot}{\kern0pt}fin{\isacharunderscore}{\kern0pt}ulgraph{\isacharunderscore}{\kern0pt}axioms\ \isacommand{by}\isamarkupfalse%
\ auto\isanewline
\ \ \ \ \isacommand{finally}\isamarkupfalse%
\ \isacommand{show}\isamarkupfalse%
\ {\isacharquery}{\kern0pt}thesis\ \isacommand{{\isachardot}{\kern0pt}}\isamarkupfalse%
\isanewline
\ \ \isacommand{next}\isamarkupfalse%
\isanewline
\ \ \ \ \isacommand{case}\isamarkupfalse%
\ False\isanewline
\ \ \ \ \isacommand{obtain}\isamarkupfalse%
\ u\ w\ \isakeyword{where}\ e{\isacharcolon}{\kern0pt}\ {\isachardoublequoteopen}e\ {\isacharequal}{\kern0pt}\ {\isacharbraceleft}{\kern0pt}u{\isacharcomma}{\kern0pt}w{\isacharbraceright}{\kern0pt}{\isachardoublequoteclose}\ \isacommand{using}\isamarkupfalse%
\ g{\isacharprime}{\kern0pt}{\isachardot}{\kern0pt}obtain{\isacharunderscore}{\kern0pt}edge{\isacharunderscore}{\kern0pt}pair{\isacharunderscore}{\kern0pt}adj\ \isacommand{by}\isamarkupfalse%
\ fastforce\isanewline
\ \ \ \ \isacommand{then}\isamarkupfalse%
\ \isacommand{have}\isamarkupfalse%
\ card{\isacharunderscore}{\kern0pt}e{\isacharcolon}{\kern0pt}\ {\isachardoublequoteopen}card\ e\ {\isacharequal}{\kern0pt}\ {\isadigit{2}}{\isachardoublequoteclose}\ \isacommand{using}\isamarkupfalse%
\ False\ g{\isacharprime}{\kern0pt}{\isachardot}{\kern0pt}alt{\isacharunderscore}{\kern0pt}edge{\isacharunderscore}{\kern0pt}size\ is{\isacharunderscore}{\kern0pt}loop{\isacharunderscore}{\kern0pt}def\ \isacommand{by}\isamarkupfalse%
\ auto\isanewline
\ \ \ \ \isacommand{then}\isamarkupfalse%
\ \isacommand{have}\isamarkupfalse%
\ {\isachardoublequoteopen}u\ {\isasymnoteq}\ w{\isachardoublequoteclose}\ \isacommand{using}\isamarkupfalse%
\ card{\isacharunderscore}{\kern0pt}{\isadigit{2}}{\isacharunderscore}{\kern0pt}iff\ \isacommand{using}\isamarkupfalse%
\ e\ \isacommand{by}\isamarkupfalse%
\ fastforce\isanewline
\ \ \ \ \isacommand{have}\isamarkupfalse%
\ inc{\isacharunderscore}{\kern0pt}loops{\isacharcolon}{\kern0pt}\ {\isachardoublequoteopen}{\isasymAnd}v{\isachardot}{\kern0pt}\ g{\isacharprime}{\kern0pt}{\isachardot}{\kern0pt}incident{\isacharunderscore}{\kern0pt}loops\ v\ {\isacharequal}{\kern0pt}\ g{\isachardot}{\kern0pt}incident{\isacharunderscore}{\kern0pt}loops\ v{\isachardoublequoteclose}\isanewline
\ \ \ \ \ \ \isacommand{unfolding}\isamarkupfalse%
\ g{\isacharprime}{\kern0pt}{\isachardot}{\kern0pt}incident{\isacharunderscore}{\kern0pt}loops{\isacharunderscore}{\kern0pt}alt\ g{\isachardot}{\kern0pt}incident{\isacharunderscore}{\kern0pt}loops{\isacharunderscore}{\kern0pt}alt\ \isacommand{using}\isamarkupfalse%
\ False\ is{\isacharunderscore}{\kern0pt}loop{\isacharunderscore}{\kern0pt}def\ \isacommand{by}\isamarkupfalse%
\ auto\isanewline
\ \ \ \ \isacommand{have}\isamarkupfalse%
\ {\isachardoublequoteopen}{\isasymAnd}v{\isachardot}{\kern0pt}\ v\ {\isasymnoteq}\ u\ {\isasymLongrightarrow}\ v\ {\isasymnoteq}\ w\ {\isasymLongrightarrow}\ g{\isacharprime}{\kern0pt}{\isachardot}{\kern0pt}incident{\isacharunderscore}{\kern0pt}sedges\ v\ {\isacharequal}{\kern0pt}\ g{\isachardot}{\kern0pt}incident{\isacharunderscore}{\kern0pt}sedges\ v{\isachardoublequoteclose}\isanewline
\ \ \ \ \ \ \isacommand{unfolding}\isamarkupfalse%
\ g{\isacharprime}{\kern0pt}{\isachardot}{\kern0pt}incident{\isacharunderscore}{\kern0pt}sedges{\isacharunderscore}{\kern0pt}def\ g{\isachardot}{\kern0pt}incident{\isacharunderscore}{\kern0pt}sedges{\isacharunderscore}{\kern0pt}def\ g{\isachardot}{\kern0pt}incident{\isacharunderscore}{\kern0pt}def\ \isacommand{using}\isamarkupfalse%
\ e\ \isacommand{by}\isamarkupfalse%
\ auto\isanewline
\ \ \ \ \isacommand{then}\isamarkupfalse%
\ \isacommand{have}\isamarkupfalse%
\ degree{\isacharunderscore}{\kern0pt}not{\isacharunderscore}{\kern0pt}u{\isacharunderscore}{\kern0pt}w{\isacharcolon}{\kern0pt}\ {\isachardoublequoteopen}{\isasymAnd}v{\isachardot}{\kern0pt}\ v\ {\isasymnoteq}\ u\ {\isasymLongrightarrow}\ v\ {\isasymnoteq}\ w\ {\isasymLongrightarrow}\ g{\isacharprime}{\kern0pt}{\isachardot}{\kern0pt}degree\ v\ {\isacharequal}{\kern0pt}\ g{\isachardot}{\kern0pt}degree\ v{\isachardoublequoteclose}\isanewline
\ \ \ \ \ \ \isacommand{unfolding}\isamarkupfalse%
\ g{\isacharprime}{\kern0pt}{\isachardot}{\kern0pt}degree{\isacharunderscore}{\kern0pt}def\ g{\isachardot}{\kern0pt}degree{\isacharunderscore}{\kern0pt}def\ \isacommand{using}\isamarkupfalse%
\ inc{\isacharunderscore}{\kern0pt}loops\ \isacommand{by}\isamarkupfalse%
\ auto\isanewline
\ \ \ \ \isacommand{have}\isamarkupfalse%
\ {\isachardoublequoteopen}g{\isacharprime}{\kern0pt}{\isachardot}{\kern0pt}incident{\isacharunderscore}{\kern0pt}sedges\ u\ {\isacharequal}{\kern0pt}\ g{\isachardot}{\kern0pt}incident{\isacharunderscore}{\kern0pt}sedges\ u\ {\isasymunion}\ {\isacharbraceleft}{\kern0pt}e{\isacharbraceright}{\kern0pt}{\isachardoublequoteclose}\isanewline
\ \ \ \ \ \ \isacommand{unfolding}\isamarkupfalse%
\ g{\isacharprime}{\kern0pt}{\isachardot}{\kern0pt}incident{\isacharunderscore}{\kern0pt}sedges{\isacharunderscore}{\kern0pt}def\ g{\isachardot}{\kern0pt}incident{\isacharunderscore}{\kern0pt}sedges{\isacharunderscore}{\kern0pt}def\ g{\isachardot}{\kern0pt}incident{\isacharunderscore}{\kern0pt}def\ \isacommand{using}\isamarkupfalse%
\ e\ card{\isacharunderscore}{\kern0pt}e\ \isacommand{by}\isamarkupfalse%
\ auto\isanewline
\ \ \ \ \isacommand{then}\isamarkupfalse%
\ \isacommand{have}\isamarkupfalse%
\ degree{\isacharunderscore}{\kern0pt}u{\isacharcolon}{\kern0pt}\ {\isachardoublequoteopen}g{\isacharprime}{\kern0pt}{\isachardot}{\kern0pt}degree\ u\ {\isacharequal}{\kern0pt}\ g{\isachardot}{\kern0pt}degree\ u\ {\isacharplus}{\kern0pt}\ {\isadigit{1}}{\isachardoublequoteclose}\isanewline
\ \ \ \ \ \ \isacommand{using}\isamarkupfalse%
\ inc{\isacharunderscore}{\kern0pt}loops\ insert{\isacharparenleft}{\kern0pt}{\isadigit{2}}{\isacharparenright}{\kern0pt}\ g{\isachardot}{\kern0pt}fin{\isacharunderscore}{\kern0pt}edges\ g{\isachardot}{\kern0pt}finite{\isacharunderscore}{\kern0pt}inc{\isacharunderscore}{\kern0pt}sedges\ g{\isachardot}{\kern0pt}incident{\isacharunderscore}{\kern0pt}sedges{\isacharunderscore}{\kern0pt}def\isanewline
\ \ \ \ \ \ \isacommand{unfolding}\isamarkupfalse%
\ g{\isacharprime}{\kern0pt}{\isachardot}{\kern0pt}degree{\isacharunderscore}{\kern0pt}def\ g{\isachardot}{\kern0pt}degree{\isacharunderscore}{\kern0pt}def\ \isacommand{by}\isamarkupfalse%
\ auto\isanewline
\ \ \ \ \isacommand{have}\isamarkupfalse%
\ {\isachardoublequoteopen}g{\isacharprime}{\kern0pt}{\isachardot}{\kern0pt}incident{\isacharunderscore}{\kern0pt}sedges\ w\ {\isacharequal}{\kern0pt}\ g{\isachardot}{\kern0pt}incident{\isacharunderscore}{\kern0pt}sedges\ w\ {\isasymunion}\ {\isacharbraceleft}{\kern0pt}e{\isacharbraceright}{\kern0pt}{\isachardoublequoteclose}\isanewline
\ \ \ \ \ \ \isacommand{unfolding}\isamarkupfalse%
\ g{\isacharprime}{\kern0pt}{\isachardot}{\kern0pt}incident{\isacharunderscore}{\kern0pt}sedges{\isacharunderscore}{\kern0pt}def\ g{\isachardot}{\kern0pt}incident{\isacharunderscore}{\kern0pt}sedges{\isacharunderscore}{\kern0pt}def\ g{\isachardot}{\kern0pt}incident{\isacharunderscore}{\kern0pt}def\ \isacommand{using}\isamarkupfalse%
\ e\ card{\isacharunderscore}{\kern0pt}e\ \isacommand{by}\isamarkupfalse%
\ auto\isanewline
\ \ \ \ \isacommand{then}\isamarkupfalse%
\ \isacommand{have}\isamarkupfalse%
\ degree{\isacharunderscore}{\kern0pt}w{\isacharcolon}{\kern0pt}\ {\isachardoublequoteopen}g{\isacharprime}{\kern0pt}{\isachardot}{\kern0pt}degree\ w\ {\isacharequal}{\kern0pt}\ g{\isachardot}{\kern0pt}degree\ w\ {\isacharplus}{\kern0pt}\ {\isadigit{1}}{\isachardoublequoteclose}\isanewline
\ \ \ \ \ \ \isacommand{using}\isamarkupfalse%
\ inc{\isacharunderscore}{\kern0pt}loops\ insert{\isacharparenleft}{\kern0pt}{\isadigit{2}}{\isacharparenright}{\kern0pt}\ g{\isachardot}{\kern0pt}fin{\isacharunderscore}{\kern0pt}edges\ g{\isachardot}{\kern0pt}finite{\isacharunderscore}{\kern0pt}inc{\isacharunderscore}{\kern0pt}sedges\ g{\isachardot}{\kern0pt}incident{\isacharunderscore}{\kern0pt}sedges{\isacharunderscore}{\kern0pt}def\isanewline
\ \ \ \ \ \ \isacommand{unfolding}\isamarkupfalse%
\ g{\isacharprime}{\kern0pt}{\isachardot}{\kern0pt}degree{\isacharunderscore}{\kern0pt}def\ g{\isachardot}{\kern0pt}degree{\isacharunderscore}{\kern0pt}def\ \isacommand{by}\isamarkupfalse%
\ auto\isanewline
\ \ \ \ \isacommand{have}\isamarkupfalse%
\ inV{\isacharcolon}{\kern0pt}\ {\isachardoublequoteopen}u\ {\isasymin}\ V{\isachardoublequoteclose}\ {\isachardoublequoteopen}w\ {\isasymin}\ V{\isacharminus}{\kern0pt}{\isacharbraceleft}{\kern0pt}u{\isacharbraceright}{\kern0pt}{\isachardoublequoteclose}\ \isacommand{using}\isamarkupfalse%
\ e\ g{\isacharprime}{\kern0pt}{\isachardot}{\kern0pt}wellformed\ {\isacartoucheopen}u{\isasymnoteq}w{\isacartoucheclose}\ \isacommand{by}\isamarkupfalse%
\ auto\isanewline
\ \ \ \ \isacommand{then}\isamarkupfalse%
\ \isacommand{have}\isamarkupfalse%
\ {\isachardoublequoteopen}{\isacharparenleft}{\kern0pt}{\isasymSum}v{\isasymin}V{\isachardot}{\kern0pt}\ g{\isacharprime}{\kern0pt}{\isachardot}{\kern0pt}degree\ v{\isacharparenright}{\kern0pt}\ {\isacharequal}{\kern0pt}\ g{\isacharprime}{\kern0pt}{\isachardot}{\kern0pt}degree\ u\ {\isacharplus}{\kern0pt}\ g{\isacharprime}{\kern0pt}{\isachardot}{\kern0pt}degree\ w\ {\isacharplus}{\kern0pt}\ {\isacharparenleft}{\kern0pt}{\isasymSum}v{\isasymin}V{\isacharminus}{\kern0pt}{\isacharbraceleft}{\kern0pt}u{\isacharbraceright}{\kern0pt}{\isacharminus}{\kern0pt}{\isacharbraceleft}{\kern0pt}w{\isacharbraceright}{\kern0pt}{\isachardot}{\kern0pt}\ g{\isacharprime}{\kern0pt}{\isachardot}{\kern0pt}degree\ v{\isacharparenright}{\kern0pt}{\isachardoublequoteclose}\isanewline
\ \ \ \ \ \ \isacommand{using}\isamarkupfalse%
\ sum{\isachardot}{\kern0pt}remove\ finV\ \isacommand{by}\isamarkupfalse%
\ {\isacharparenleft}{\kern0pt}metis\ add{\isachardot}{\kern0pt}assoc\ finite{\isacharunderscore}{\kern0pt}Diff{\isacharparenright}{\kern0pt}\isanewline
\ \ \ \ \isacommand{also}\isamarkupfalse%
\ \isacommand{have}\isamarkupfalse%
\ {\isachardoublequoteopen}{\isasymdots}\ {\isacharequal}{\kern0pt}\ g{\isachardot}{\kern0pt}degree\ u\ {\isacharplus}{\kern0pt}\ g{\isachardot}{\kern0pt}degree\ w\ {\isacharplus}{\kern0pt}\ {\isacharparenleft}{\kern0pt}{\isasymSum}v{\isasymin}V{\isacharminus}{\kern0pt}{\isacharbraceleft}{\kern0pt}u{\isacharbraceright}{\kern0pt}{\isacharminus}{\kern0pt}{\isacharbraceleft}{\kern0pt}w{\isacharbraceright}{\kern0pt}{\isachardot}{\kern0pt}\ g{\isachardot}{\kern0pt}degree\ v{\isacharparenright}{\kern0pt}\ {\isacharplus}{\kern0pt}\ {\isadigit{2}}{\isachardoublequoteclose}\isanewline
\ \ \ \ \ \ \isacommand{using}\isamarkupfalse%
\ degree{\isacharunderscore}{\kern0pt}not{\isacharunderscore}{\kern0pt}u{\isacharunderscore}{\kern0pt}w\ degree{\isacharunderscore}{\kern0pt}u\ degree{\isacharunderscore}{\kern0pt}w\ \isacommand{by}\isamarkupfalse%
\ simp\isanewline
\ \ \ \ \isacommand{also}\isamarkupfalse%
\ \isacommand{have}\isamarkupfalse%
\ {\isachardoublequoteopen}{\isasymdots}\ {\isacharequal}{\kern0pt}\ {\isacharparenleft}{\kern0pt}{\isasymSum}v{\isasymin}V{\isachardot}{\kern0pt}\ g{\isachardot}{\kern0pt}degree\ v{\isacharparenright}{\kern0pt}\ {\isacharplus}{\kern0pt}\ {\isadigit{2}}{\isachardoublequoteclose}\ \isacommand{using}\isamarkupfalse%
\ sum{\isachardot}{\kern0pt}remove\ finV\ inV\ \isacommand{by}\isamarkupfalse%
\ {\isacharparenleft}{\kern0pt}metis\ add{\isachardot}{\kern0pt}assoc\ finite{\isacharunderscore}{\kern0pt}Diff{\isacharparenright}{\kern0pt}\isanewline
\ \ \ \ \isacommand{also}\isamarkupfalse%
\ \isacommand{have}\isamarkupfalse%
\ {\isachardoublequoteopen}{\isasymdots}\ {\isacharequal}{\kern0pt}\ {\isadigit{2}}\ {\isacharasterisk}{\kern0pt}\ card\ {\isacharparenleft}{\kern0pt}insert\ e\ E{\isacharprime}{\kern0pt}{\isacharparenright}{\kern0pt}{\isachardoublequoteclose}\ \isacommand{using}\isamarkupfalse%
\ insert\ g{\isachardot}{\kern0pt}fin{\isacharunderscore}{\kern0pt}ulgraph{\isacharunderscore}{\kern0pt}axioms\ \isacommand{by}\isamarkupfalse%
\ auto\isanewline
\ \ \ \ \isacommand{finally}\isamarkupfalse%
\ \isacommand{show}\isamarkupfalse%
\ {\isacharquery}{\kern0pt}thesis\ \isacommand{{\isachardot}{\kern0pt}}\isamarkupfalse%
\isanewline
\ \ \isacommand{qed}\isamarkupfalse%
\isanewline
\isacommand{qed}\isamarkupfalse%
%
\endisatagproof
{\isafoldproof}%
%
\isadelimproof
%
\endisadelimproof
%
\isadelimdocument
%
\endisadelimdocument
%
\isatagdocument
%
\isamarkupsubsection{Walks%
}
\isamarkuptrue%
%
\endisatagdocument
{\isafolddocument}%
%
\isadelimdocument
%
\endisadelimdocument
\isacommand{lemma}\isamarkupfalse%
\ {\isacharparenleft}{\kern0pt}\isakeyword{in}\ ulgraph{\isacharparenright}{\kern0pt}\ walk{\isacharunderscore}{\kern0pt}edges{\isacharunderscore}{\kern0pt}induced{\isacharunderscore}{\kern0pt}edges{\isacharcolon}{\kern0pt}\ {\isachardoublequoteopen}is{\isacharunderscore}{\kern0pt}walk\ p\ {\isasymLongrightarrow}\ set\ {\isacharparenleft}{\kern0pt}walk{\isacharunderscore}{\kern0pt}edges\ p{\isacharparenright}{\kern0pt}\ {\isasymsubseteq}\ induced{\isacharunderscore}{\kern0pt}edges\ {\isacharparenleft}{\kern0pt}set\ p{\isacharparenright}{\kern0pt}{\isachardoublequoteclose}\isanewline
%
\isadelimproof
\ \ %
\endisadelimproof
%
\isatagproof
\isacommand{unfolding}\isamarkupfalse%
\ induced{\isacharunderscore}{\kern0pt}edges{\isacharunderscore}{\kern0pt}def\ is{\isacharunderscore}{\kern0pt}walk{\isacharunderscore}{\kern0pt}def\ \isacommand{by}\isamarkupfalse%
\ {\isacharparenleft}{\kern0pt}induction\ p\ rule{\isacharcolon}{\kern0pt}\ walk{\isacharunderscore}{\kern0pt}edges{\isachardot}{\kern0pt}induct{\isacharparenright}{\kern0pt}\ auto%
\endisatagproof
{\isafoldproof}%
%
\isadelimproof
\isanewline
%
\endisadelimproof
\isanewline
\isacommand{lemma}\isamarkupfalse%
\ {\isacharparenleft}{\kern0pt}\isakeyword{in}\ ulgraph{\isacharparenright}{\kern0pt}\ walk{\isacharunderscore}{\kern0pt}edges{\isacharunderscore}{\kern0pt}in{\isacharunderscore}{\kern0pt}verts{\isacharcolon}{\kern0pt}\ {\isachardoublequoteopen}e\ {\isasymin}\ set\ {\isacharparenleft}{\kern0pt}walk{\isacharunderscore}{\kern0pt}edges\ xs{\isacharparenright}{\kern0pt}\ {\isasymLongrightarrow}\ e\ {\isasymsubseteq}\ set\ xs{\isachardoublequoteclose}\isanewline
%
\isadelimproof
\ \ %
\endisadelimproof
%
\isatagproof
\isacommand{by}\isamarkupfalse%
\ {\isacharparenleft}{\kern0pt}induction\ xs\ rule{\isacharcolon}{\kern0pt}\ walk{\isacharunderscore}{\kern0pt}edges{\isachardot}{\kern0pt}induct{\isacharparenright}{\kern0pt}\ auto%
\endisatagproof
{\isafoldproof}%
%
\isadelimproof
\isanewline
%
\endisadelimproof
\isanewline
\isacommand{lemma}\isamarkupfalse%
\ {\isacharparenleft}{\kern0pt}\isakeyword{in}\ ulgraph{\isacharparenright}{\kern0pt}\ is{\isacharunderscore}{\kern0pt}walk{\isacharunderscore}{\kern0pt}prefix{\isacharcolon}{\kern0pt}\ {\isachardoublequoteopen}is{\isacharunderscore}{\kern0pt}walk\ {\isacharparenleft}{\kern0pt}xs{\isacharat}{\kern0pt}ys{\isacharparenright}{\kern0pt}\ {\isasymLongrightarrow}\ xs\ {\isasymnoteq}\ {\isacharbrackleft}{\kern0pt}{\isacharbrackright}{\kern0pt}\ {\isasymLongrightarrow}\ is{\isacharunderscore}{\kern0pt}walk\ xs{\isachardoublequoteclose}\isanewline
%
\isadelimproof
\ \ %
\endisadelimproof
%
\isatagproof
\isacommand{unfolding}\isamarkupfalse%
\ is{\isacharunderscore}{\kern0pt}walk{\isacharunderscore}{\kern0pt}def\ \isacommand{using}\isamarkupfalse%
\ walk{\isacharunderscore}{\kern0pt}edges{\isacharunderscore}{\kern0pt}append{\isacharunderscore}{\kern0pt}ss{\isadigit{2}}\ \isacommand{by}\isamarkupfalse%
\ fastforce%
\endisatagproof
{\isafoldproof}%
%
\isadelimproof
\isanewline
%
\endisadelimproof
\isanewline
\isacommand{lemma}\isamarkupfalse%
\ {\isacharparenleft}{\kern0pt}\isakeyword{in}\ ulgraph{\isacharparenright}{\kern0pt}\ split{\isacharunderscore}{\kern0pt}walk{\isacharunderscore}{\kern0pt}edge{\isacharcolon}{\kern0pt}\ {\isachardoublequoteopen}{\isacharbraceleft}{\kern0pt}x{\isacharcomma}{\kern0pt}y{\isacharbraceright}{\kern0pt}\ {\isasymin}\ set\ {\isacharparenleft}{\kern0pt}walk{\isacharunderscore}{\kern0pt}edges\ p{\isacharparenright}{\kern0pt}\ {\isasymLongrightarrow}\isanewline
\ \ {\isasymexists}xs\ ys{\isachardot}{\kern0pt}\ p\ {\isacharequal}{\kern0pt}\ xs\ {\isacharat}{\kern0pt}\ x\ {\isacharhash}{\kern0pt}\ y\ {\isacharhash}{\kern0pt}\ ys\ {\isasymor}\ p\ {\isacharequal}{\kern0pt}\ xs\ {\isacharat}{\kern0pt}\ y\ {\isacharhash}{\kern0pt}\ x\ {\isacharhash}{\kern0pt}\ ys{\isachardoublequoteclose}\isanewline
%
\isadelimproof
\ \ %
\endisadelimproof
%
\isatagproof
\isacommand{by}\isamarkupfalse%
\ {\isacharparenleft}{\kern0pt}induction\ p\ rule{\isacharcolon}{\kern0pt}\ walk{\isacharunderscore}{\kern0pt}edges{\isachardot}{\kern0pt}induct{\isacharparenright}{\kern0pt}\ {\isacharparenleft}{\kern0pt}auto{\isacharcomma}{\kern0pt}\ metis\ append{\isacharunderscore}{\kern0pt}Nil\ doubleton{\isacharunderscore}{\kern0pt}eq{\isacharunderscore}{\kern0pt}iff{\isacharcomma}{\kern0pt}\ {\isacharparenleft}{\kern0pt}metis\ append{\isacharunderscore}{\kern0pt}Cons{\isacharparenright}{\kern0pt}{\isacharplus}{\kern0pt}{\isacharparenright}{\kern0pt}%
\endisatagproof
{\isafoldproof}%
%
\isadelimproof
%
\endisadelimproof
%
\isadelimdocument
%
\endisadelimdocument
%
\isatagdocument
%
\isamarkupsubsection{Paths%
}
\isamarkuptrue%
%
\endisatagdocument
{\isafolddocument}%
%
\isadelimdocument
%
\endisadelimdocument
\isacommand{lemma}\isamarkupfalse%
\ {\isacharparenleft}{\kern0pt}\isakeyword{in}\ ulgraph{\isacharparenright}{\kern0pt}\ is{\isacharunderscore}{\kern0pt}gen{\isacharunderscore}{\kern0pt}path{\isacharunderscore}{\kern0pt}wf{\isacharcolon}{\kern0pt}\ {\isachardoublequoteopen}is{\isacharunderscore}{\kern0pt}gen{\isacharunderscore}{\kern0pt}path\ p\ {\isasymLongrightarrow}\ set\ p\ {\isasymsubseteq}\ V{\isachardoublequoteclose}\isanewline
%
\isadelimproof
\ \ %
\endisadelimproof
%
\isatagproof
\isacommand{unfolding}\isamarkupfalse%
\ is{\isacharunderscore}{\kern0pt}gen{\isacharunderscore}{\kern0pt}path{\isacharunderscore}{\kern0pt}def\ \isacommand{using}\isamarkupfalse%
\ is{\isacharunderscore}{\kern0pt}walk{\isacharunderscore}{\kern0pt}wf\ \isacommand{by}\isamarkupfalse%
\ auto%
\endisatagproof
{\isafoldproof}%
%
\isadelimproof
\isanewline
%
\endisadelimproof
\isanewline
\isacommand{lemma}\isamarkupfalse%
\ {\isacharparenleft}{\kern0pt}\isakeyword{in}\ ulgraph{\isacharparenright}{\kern0pt}\ path{\isacharunderscore}{\kern0pt}wf{\isacharcolon}{\kern0pt}\ {\isachardoublequoteopen}is{\isacharunderscore}{\kern0pt}path\ p\ {\isasymLongrightarrow}\ set\ p\ {\isasymsubseteq}\ V{\isachardoublequoteclose}\isanewline
%
\isadelimproof
\ \ %
\endisadelimproof
%
\isatagproof
\isacommand{by}\isamarkupfalse%
\ {\isacharparenleft}{\kern0pt}simp\ add{\isacharcolon}{\kern0pt}\ is{\isacharunderscore}{\kern0pt}path{\isacharunderscore}{\kern0pt}walk\ is{\isacharunderscore}{\kern0pt}walk{\isacharunderscore}{\kern0pt}wf{\isacharparenright}{\kern0pt}%
\endisatagproof
{\isafoldproof}%
%
\isadelimproof
\isanewline
%
\endisadelimproof
\isanewline
\isacommand{lemma}\isamarkupfalse%
\ {\isacharparenleft}{\kern0pt}\isakeyword{in}\ fin{\isacharunderscore}{\kern0pt}ulgraph{\isacharparenright}{\kern0pt}\ length{\isacharunderscore}{\kern0pt}gen{\isacharunderscore}{\kern0pt}path{\isacharunderscore}{\kern0pt}card{\isacharunderscore}{\kern0pt}V{\isacharcolon}{\kern0pt}\ {\isachardoublequoteopen}is{\isacharunderscore}{\kern0pt}gen{\isacharunderscore}{\kern0pt}path\ p\ {\isasymLongrightarrow}\ walk{\isacharunderscore}{\kern0pt}length\ p\ {\isasymle}\ card\ V{\isachardoublequoteclose}\isanewline
%
\isadelimproof
\ \ %
\endisadelimproof
%
\isatagproof
\isacommand{by}\isamarkupfalse%
\ {\isacharparenleft}{\kern0pt}metis\ card{\isacharunderscore}{\kern0pt}mono\ distinct{\isacharunderscore}{\kern0pt}card\ distinct{\isacharunderscore}{\kern0pt}tl\ finV\ is{\isacharunderscore}{\kern0pt}gen{\isacharunderscore}{\kern0pt}path{\isacharunderscore}{\kern0pt}def\ is{\isacharunderscore}{\kern0pt}walk{\isacharunderscore}{\kern0pt}def\ length{\isacharunderscore}{\kern0pt}tl\isanewline
\ \ \ \ \ \ list{\isachardot}{\kern0pt}exhaust{\isacharunderscore}{\kern0pt}sel\ order{\isacharunderscore}{\kern0pt}trans\ set{\isacharunderscore}{\kern0pt}subset{\isacharunderscore}{\kern0pt}Cons\ walk{\isacharunderscore}{\kern0pt}length{\isacharunderscore}{\kern0pt}conv{\isacharparenright}{\kern0pt}%
\endisatagproof
{\isafoldproof}%
%
\isadelimproof
\isanewline
%
\endisadelimproof
\isanewline
\isacommand{lemma}\isamarkupfalse%
\ {\isacharparenleft}{\kern0pt}\isakeyword{in}\ fin{\isacharunderscore}{\kern0pt}ulgraph{\isacharparenright}{\kern0pt}\ length{\isacharunderscore}{\kern0pt}path{\isacharunderscore}{\kern0pt}card{\isacharunderscore}{\kern0pt}V{\isacharcolon}{\kern0pt}\ {\isachardoublequoteopen}is{\isacharunderscore}{\kern0pt}path\ p\ {\isasymLongrightarrow}\ length\ p\ {\isasymle}\ card\ V{\isachardoublequoteclose}\isanewline
%
\isadelimproof
\ \ %
\endisadelimproof
%
\isatagproof
\isacommand{by}\isamarkupfalse%
\ {\isacharparenleft}{\kern0pt}metis\ path{\isacharunderscore}{\kern0pt}wf\ card{\isacharunderscore}{\kern0pt}mono\ distinct{\isacharunderscore}{\kern0pt}card\ finV\ is{\isacharunderscore}{\kern0pt}path{\isacharunderscore}{\kern0pt}def{\isacharparenright}{\kern0pt}%
\endisatagproof
{\isafoldproof}%
%
\isadelimproof
\isanewline
%
\endisadelimproof
\isanewline
\isacommand{lemma}\isamarkupfalse%
\ {\isacharparenleft}{\kern0pt}\isakeyword{in}\ ulgraph{\isacharparenright}{\kern0pt}\ is{\isacharunderscore}{\kern0pt}gen{\isacharunderscore}{\kern0pt}path{\isacharunderscore}{\kern0pt}prefix{\isacharcolon}{\kern0pt}\ {\isachardoublequoteopen}is{\isacharunderscore}{\kern0pt}gen{\isacharunderscore}{\kern0pt}path\ {\isacharparenleft}{\kern0pt}xs{\isacharat}{\kern0pt}ys{\isacharparenright}{\kern0pt}\ {\isasymLongrightarrow}\ xs\ {\isasymnoteq}\ {\isacharbrackleft}{\kern0pt}{\isacharbrackright}{\kern0pt}\ {\isasymLongrightarrow}\ is{\isacharunderscore}{\kern0pt}gen{\isacharunderscore}{\kern0pt}path\ {\isacharparenleft}{\kern0pt}xs{\isacharparenright}{\kern0pt}{\isachardoublequoteclose}\isanewline
%
\isadelimproof
\ \ %
\endisadelimproof
%
\isatagproof
\isacommand{unfolding}\isamarkupfalse%
\ is{\isacharunderscore}{\kern0pt}gen{\isacharunderscore}{\kern0pt}path{\isacharunderscore}{\kern0pt}def\ \isacommand{using}\isamarkupfalse%
\ is{\isacharunderscore}{\kern0pt}walk{\isacharunderscore}{\kern0pt}prefix\ \isacommand{apply}\isamarkupfalse%
\ auto\isanewline
\ \ \isacommand{by}\isamarkupfalse%
\ {\isacharparenleft}{\kern0pt}metis\ Int{\isacharunderscore}{\kern0pt}iff\ distinct{\isachardot}{\kern0pt}simps{\isacharparenleft}{\kern0pt}{\isadigit{2}}{\isacharparenright}{\kern0pt}\ emptyE\ last{\isacharunderscore}{\kern0pt}appendL\ last{\isacharunderscore}{\kern0pt}appendR\ last{\isacharunderscore}{\kern0pt}in{\isacharunderscore}{\kern0pt}set\ list{\isachardot}{\kern0pt}collapse{\isacharparenright}{\kern0pt}%
\endisatagproof
{\isafoldproof}%
%
\isadelimproof
\isanewline
%
\endisadelimproof
\isanewline
\isacommand{lemma}\isamarkupfalse%
\ {\isacharparenleft}{\kern0pt}\isakeyword{in}\ ulgraph{\isacharparenright}{\kern0pt}\ connecting{\isacharunderscore}{\kern0pt}path{\isacharunderscore}{\kern0pt}append{\isacharcolon}{\kern0pt}\ {\isachardoublequoteopen}connecting{\isacharunderscore}{\kern0pt}path\ u\ w\ {\isacharparenleft}{\kern0pt}xs{\isacharat}{\kern0pt}ys{\isacharparenright}{\kern0pt}\ {\isasymLongrightarrow}\ xs\ {\isasymnoteq}\ {\isacharbrackleft}{\kern0pt}{\isacharbrackright}{\kern0pt}\ {\isasymLongrightarrow}\ connecting{\isacharunderscore}{\kern0pt}path\ u\ {\isacharparenleft}{\kern0pt}last\ xs{\isacharparenright}{\kern0pt}\ xs{\isachardoublequoteclose}\isanewline
%
\isadelimproof
\ \ %
\endisadelimproof
%
\isatagproof
\isacommand{unfolding}\isamarkupfalse%
\ connecting{\isacharunderscore}{\kern0pt}path{\isacharunderscore}{\kern0pt}def\ \isacommand{using}\isamarkupfalse%
\ is{\isacharunderscore}{\kern0pt}gen{\isacharunderscore}{\kern0pt}path{\isacharunderscore}{\kern0pt}prefix\ \isacommand{by}\isamarkupfalse%
\ auto%
\endisatagproof
{\isafoldproof}%
%
\isadelimproof
\isanewline
%
\endisadelimproof
\isanewline
\isacommand{lemma}\isamarkupfalse%
\ {\isacharparenleft}{\kern0pt}\isakeyword{in}\ ulgraph{\isacharparenright}{\kern0pt}\ connecting{\isacharunderscore}{\kern0pt}path{\isacharunderscore}{\kern0pt}tl{\isacharcolon}{\kern0pt}\ {\isachardoublequoteopen}connecting{\isacharunderscore}{\kern0pt}path\ u\ v\ {\isacharparenleft}{\kern0pt}u\ {\isacharhash}{\kern0pt}\ w\ {\isacharhash}{\kern0pt}\ xs{\isacharparenright}{\kern0pt}\ {\isasymLongrightarrow}\ connecting{\isacharunderscore}{\kern0pt}path\ w\ v\ {\isacharparenleft}{\kern0pt}w\ {\isacharhash}{\kern0pt}\ xs{\isacharparenright}{\kern0pt}{\isachardoublequoteclose}\isanewline
%
\isadelimproof
\ \ %
\endisadelimproof
%
\isatagproof
\isacommand{unfolding}\isamarkupfalse%
\ connecting{\isacharunderscore}{\kern0pt}path{\isacharunderscore}{\kern0pt}def\ is{\isacharunderscore}{\kern0pt}gen{\isacharunderscore}{\kern0pt}path{\isacharunderscore}{\kern0pt}def\ \isacommand{using}\isamarkupfalse%
\ is{\isacharunderscore}{\kern0pt}walk{\isacharunderscore}{\kern0pt}drop{\isacharunderscore}{\kern0pt}hd\ distinct{\isacharunderscore}{\kern0pt}tl\ \isacommand{by}\isamarkupfalse%
\ auto%
\endisatagproof
{\isafoldproof}%
%
\isadelimproof
\isanewline
%
\endisadelimproof
\isanewline
\isacommand{lemma}\isamarkupfalse%
\ {\isacharparenleft}{\kern0pt}\isakeyword{in}\ fin{\isacharunderscore}{\kern0pt}ulgraph{\isacharparenright}{\kern0pt}\ obtain{\isacharunderscore}{\kern0pt}longest{\isacharunderscore}{\kern0pt}path{\isacharcolon}{\kern0pt}\isanewline
\ \ \isakeyword{assumes}\ {\isachardoublequoteopen}e\ {\isasymin}\ E{\isachardoublequoteclose}\isanewline
\ \ \ \ \isakeyword{and}\ sedge{\isacharcolon}{\kern0pt}\ {\isachardoublequoteopen}is{\isacharunderscore}{\kern0pt}sedge\ e{\isachardoublequoteclose}\isanewline
\ \ \isakeyword{obtains}\ p\ \isakeyword{where}\ {\isachardoublequoteopen}is{\isacharunderscore}{\kern0pt}path\ p{\isachardoublequoteclose}\ {\isachardoublequoteopen}{\isasymforall}s{\isachardot}{\kern0pt}\ is{\isacharunderscore}{\kern0pt}path\ s\ {\isasymlongrightarrow}\ length\ s\ {\isasymle}\ length\ p{\isachardoublequoteclose}\isanewline
%
\isadelimproof
%
\endisadelimproof
%
\isatagproof
\isacommand{proof}\isamarkupfalse%
{\isacharminus}{\kern0pt}\isanewline
\ \ \isacommand{let}\isamarkupfalse%
\ {\isacharquery}{\kern0pt}longest{\isacharunderscore}{\kern0pt}path\ {\isacharequal}{\kern0pt}\ {\isachardoublequoteopen}ARG{\isacharunderscore}{\kern0pt}MAX\ length\ p{\isachardot}{\kern0pt}\ is{\isacharunderscore}{\kern0pt}path\ p{\isachardoublequoteclose}\isanewline
\ \ \isacommand{obtain}\isamarkupfalse%
\ u\ v\ \isakeyword{where}\ e{\isacharcolon}{\kern0pt}\ {\isachardoublequoteopen}u\ {\isasymnoteq}\ v{\isachardoublequoteclose}\ {\isachardoublequoteopen}e\ {\isacharequal}{\kern0pt}\ {\isacharbraceleft}{\kern0pt}u{\isacharcomma}{\kern0pt}v{\isacharbraceright}{\kern0pt}{\isachardoublequoteclose}\ \isacommand{using}\isamarkupfalse%
\ sedge\ card{\isacharunderscore}{\kern0pt}{\isadigit{2}}{\isacharunderscore}{\kern0pt}iff\ \isacommand{unfolding}\isamarkupfalse%
\ is{\isacharunderscore}{\kern0pt}sedge{\isacharunderscore}{\kern0pt}def\ \isacommand{by}\isamarkupfalse%
\ metis\isanewline
\ \ \isacommand{then}\isamarkupfalse%
\ \isacommand{have}\isamarkupfalse%
\ inV{\isacharcolon}{\kern0pt}\ {\isachardoublequoteopen}u\ {\isasymin}\ V{\isachardoublequoteclose}\ {\isachardoublequoteopen}v\ {\isasymin}\ V{\isachardoublequoteclose}\ \isacommand{using}\isamarkupfalse%
\ {\isacartoucheopen}e{\isasymin}E{\isacartoucheclose}\ wellformed\ \isacommand{by}\isamarkupfalse%
\ auto\isanewline
\ \ \isacommand{then}\isamarkupfalse%
\ \isacommand{have}\isamarkupfalse%
\ path{\isacharunderscore}{\kern0pt}ex{\isacharcolon}{\kern0pt}\ {\isachardoublequoteopen}is{\isacharunderscore}{\kern0pt}path\ {\isacharbrackleft}{\kern0pt}u{\isacharcomma}{\kern0pt}v{\isacharbrackright}{\kern0pt}{\isachardoublequoteclose}\ \isacommand{using}\isamarkupfalse%
\ e\ {\isacartoucheopen}e{\isasymin}E{\isacartoucheclose}\ \isacommand{unfolding}\isamarkupfalse%
\ is{\isacharunderscore}{\kern0pt}path{\isacharunderscore}{\kern0pt}def\ is{\isacharunderscore}{\kern0pt}open{\isacharunderscore}{\kern0pt}walk{\isacharunderscore}{\kern0pt}def\ is{\isacharunderscore}{\kern0pt}walk{\isacharunderscore}{\kern0pt}def\ \isacommand{by}\isamarkupfalse%
\ simp\isanewline
\ \ \isacommand{obtain}\isamarkupfalse%
\ p\ \isakeyword{where}\ p{\isacharunderscore}{\kern0pt}is{\isacharunderscore}{\kern0pt}path{\isacharcolon}{\kern0pt}\ {\isachardoublequoteopen}is{\isacharunderscore}{\kern0pt}path\ p{\isachardoublequoteclose}\ \isakeyword{and}\ p{\isacharunderscore}{\kern0pt}longest{\isacharunderscore}{\kern0pt}path{\isacharcolon}{\kern0pt}\ {\isachardoublequoteopen}{\isasymforall}s{\isachardot}{\kern0pt}\ is{\isacharunderscore}{\kern0pt}path\ s\ {\isasymlongrightarrow}\ length\ s\ {\isasymle}\ length\ p{\isachardoublequoteclose}\isanewline
\ \ \ \ \isacommand{using}\isamarkupfalse%
\ path{\isacharunderscore}{\kern0pt}ex\ length{\isacharunderscore}{\kern0pt}path{\isacharunderscore}{\kern0pt}card{\isacharunderscore}{\kern0pt}V\ ex{\isacharunderscore}{\kern0pt}has{\isacharunderscore}{\kern0pt}greatest{\isacharunderscore}{\kern0pt}nat{\isacharbrackleft}{\kern0pt}of\ is{\isacharunderscore}{\kern0pt}path\ {\isachardoublequoteopen}{\isacharbrackleft}{\kern0pt}u{\isacharcomma}{\kern0pt}v{\isacharbrackright}{\kern0pt}{\isachardoublequoteclose}\ length\ order{\isacharbrackright}{\kern0pt}\ \isacommand{by}\isamarkupfalse%
\ force\isanewline
\ \ \isacommand{then}\isamarkupfalse%
\ \isacommand{show}\isamarkupfalse%
\ {\isacharquery}{\kern0pt}thesis\ \isacommand{{\isachardot}{\kern0pt}{\isachardot}{\kern0pt}}\isamarkupfalse%
\isanewline
\isacommand{qed}\isamarkupfalse%
%
\endisatagproof
{\isafoldproof}%
%
\isadelimproof
%
\endisadelimproof
%
\isadelimdocument
%
\endisadelimdocument
%
\isatagdocument
%
\isamarkupsubsection{Cycles%
}
\isamarkuptrue%
%
\endisatagdocument
{\isafolddocument}%
%
\isadelimdocument
%
\endisadelimdocument
\isacommand{context}\isamarkupfalse%
\ ulgraph\isanewline
\isakeyword{begin}\isanewline
\isanewline
\isacommand{definition}\isamarkupfalse%
\ is{\isacharunderscore}{\kern0pt}cycle{\isadigit{2}}\ {\isacharcolon}{\kern0pt}{\isacharcolon}{\kern0pt}\ {\isachardoublequoteopen}{\isacharprime}{\kern0pt}a\ list\ {\isasymRightarrow}\ bool{\isachardoublequoteclose}\ \isakeyword{where}\isanewline
\ \ {\isachardoublequoteopen}is{\isacharunderscore}{\kern0pt}cycle{\isadigit{2}}\ xs\ {\isasymlongleftrightarrow}\ is{\isacharunderscore}{\kern0pt}cycle\ xs\ {\isasymand}\ distinct\ {\isacharparenleft}{\kern0pt}walk{\isacharunderscore}{\kern0pt}edges\ xs{\isacharparenright}{\kern0pt}{\isachardoublequoteclose}\isanewline
\isanewline
\isacommand{lemma}\isamarkupfalse%
\ loop{\isacharunderscore}{\kern0pt}is{\isacharunderscore}{\kern0pt}cycle{\isadigit{2}}{\isacharcolon}{\kern0pt}\ {\isachardoublequoteopen}{\isacharbraceleft}{\kern0pt}v{\isacharbraceright}{\kern0pt}\ {\isasymin}\ E\ {\isasymLongrightarrow}\ is{\isacharunderscore}{\kern0pt}cycle{\isadigit{2}}\ {\isacharbrackleft}{\kern0pt}v{\isacharcomma}{\kern0pt}\ v{\isacharbrackright}{\kern0pt}{\isachardoublequoteclose}\isanewline
%
\isadelimproof
\ \ %
\endisadelimproof
%
\isatagproof
\isacommand{unfolding}\isamarkupfalse%
\ is{\isacharunderscore}{\kern0pt}cycle{\isadigit{2}}{\isacharunderscore}{\kern0pt}def\ is{\isacharunderscore}{\kern0pt}cycle{\isacharunderscore}{\kern0pt}alt\ is{\isacharunderscore}{\kern0pt}walk{\isacharunderscore}{\kern0pt}def\ \isacommand{using}\isamarkupfalse%
\ wellformed\ walk{\isacharunderscore}{\kern0pt}length{\isacharunderscore}{\kern0pt}conv\ \isacommand{by}\isamarkupfalse%
\ auto%
\endisatagproof
{\isafoldproof}%
%
\isadelimproof
\isanewline
%
\endisadelimproof
\isanewline
\isacommand{end}\isamarkupfalse%
\isanewline
\isanewline
\isacommand{lemma}\isamarkupfalse%
\ {\isacharparenleft}{\kern0pt}\isakeyword{in}\ sgraph{\isacharparenright}{\kern0pt}\ cycle{\isadigit{2}}{\isacharunderscore}{\kern0pt}min{\isacharunderscore}{\kern0pt}length{\isacharcolon}{\kern0pt}\isanewline
\ \ \isakeyword{assumes}\ cycle{\isacharcolon}{\kern0pt}\ {\isachardoublequoteopen}is{\isacharunderscore}{\kern0pt}cycle{\isadigit{2}}\ c{\isachardoublequoteclose}\isanewline
\ \ \isakeyword{shows}\ {\isachardoublequoteopen}walk{\isacharunderscore}{\kern0pt}length\ c\ {\isasymge}\ {\isadigit{3}}{\isachardoublequoteclose}\isanewline
%
\isadelimproof
%
\endisadelimproof
%
\isatagproof
\isacommand{proof}\isamarkupfalse%
{\isacharminus}{\kern0pt}\isanewline
\ \ \isacommand{consider}\isamarkupfalse%
\ {\isachardoublequoteopen}c\ {\isacharequal}{\kern0pt}\ {\isacharbrackleft}{\kern0pt}{\isacharbrackright}{\kern0pt}{\isachardoublequoteclose}\ {\isacharbar}{\kern0pt}\ {\isachardoublequoteopen}{\isasymexists}v{\isadigit{1}}{\isachardot}{\kern0pt}\ c\ {\isacharequal}{\kern0pt}\ {\isacharbrackleft}{\kern0pt}v{\isadigit{1}}{\isacharbrackright}{\kern0pt}{\isachardoublequoteclose}\ {\isacharbar}{\kern0pt}\ {\isachardoublequoteopen}{\isasymexists}v{\isadigit{1}}\ v{\isadigit{2}}{\isachardot}{\kern0pt}\ c\ {\isacharequal}{\kern0pt}\ {\isacharbrackleft}{\kern0pt}v{\isadigit{1}}{\isacharcomma}{\kern0pt}\ v{\isadigit{2}}{\isacharbrackright}{\kern0pt}{\isachardoublequoteclose}\ {\isacharbar}{\kern0pt}\ {\isachardoublequoteopen}{\isasymexists}v{\isadigit{1}}\ v{\isadigit{2}}\ v{\isadigit{3}}{\isachardot}{\kern0pt}\ c\ {\isacharequal}{\kern0pt}\ {\isacharbrackleft}{\kern0pt}v{\isadigit{1}}{\isacharcomma}{\kern0pt}\ v{\isadigit{2}}{\isacharcomma}{\kern0pt}\ v{\isadigit{3}}{\isacharbrackright}{\kern0pt}{\isachardoublequoteclose}\ {\isacharbar}{\kern0pt}\ {\isachardoublequoteopen}{\isasymexists}v{\isadigit{1}}\ v{\isadigit{2}}\ v{\isadigit{3}}\ v{\isadigit{4}}\ vs{\isachardot}{\kern0pt}\ c\ {\isacharequal}{\kern0pt}\ v{\isadigit{1}}{\isacharhash}{\kern0pt}v{\isadigit{2}}{\isacharhash}{\kern0pt}v{\isadigit{3}}{\isacharhash}{\kern0pt}v{\isadigit{4}}{\isacharhash}{\kern0pt}vs{\isachardoublequoteclose}\isanewline
\ \ \ \ \isacommand{by}\isamarkupfalse%
\ {\isacharparenleft}{\kern0pt}metis\ list{\isachardot}{\kern0pt}exhaust{\isacharunderscore}{\kern0pt}sel{\isacharparenright}{\kern0pt}\isanewline
\ \ \isacommand{then}\isamarkupfalse%
\ \isacommand{show}\isamarkupfalse%
\ {\isacharquery}{\kern0pt}thesis\ \isacommand{using}\isamarkupfalse%
\ cycle\ walk{\isacharunderscore}{\kern0pt}length{\isacharunderscore}{\kern0pt}conv\ singleton{\isacharunderscore}{\kern0pt}not{\isacharunderscore}{\kern0pt}edge\ \isacommand{unfolding}\isamarkupfalse%
\ is{\isacharunderscore}{\kern0pt}cycle{\isadigit{2}}{\isacharunderscore}{\kern0pt}def\ is{\isacharunderscore}{\kern0pt}cycle{\isacharunderscore}{\kern0pt}alt\ is{\isacharunderscore}{\kern0pt}walk{\isacharunderscore}{\kern0pt}def\ \isacommand{by}\isamarkupfalse%
\ {\isacharparenleft}{\kern0pt}cases{\isacharcomma}{\kern0pt}\ auto{\isacharparenright}{\kern0pt}\isanewline
\isacommand{qed}\isamarkupfalse%
%
\endisatagproof
{\isafoldproof}%
%
\isadelimproof
\isanewline
%
\endisadelimproof
\isanewline
\isacommand{lemma}\isamarkupfalse%
\ {\isacharparenleft}{\kern0pt}\isakeyword{in}\ fin{\isacharunderscore}{\kern0pt}ulgraph{\isacharparenright}{\kern0pt}\ length{\isacharunderscore}{\kern0pt}cycle{\isacharunderscore}{\kern0pt}card{\isacharunderscore}{\kern0pt}V{\isacharcolon}{\kern0pt}\ {\isachardoublequoteopen}is{\isacharunderscore}{\kern0pt}cycle\ c\ {\isasymLongrightarrow}\ walk{\isacharunderscore}{\kern0pt}length\ c\ {\isasymle}\ Suc\ {\isacharparenleft}{\kern0pt}card\ V{\isacharparenright}{\kern0pt}{\isachardoublequoteclose}\isanewline
%
\isadelimproof
\ \ %
\endisadelimproof
%
\isatagproof
\isacommand{using}\isamarkupfalse%
\ length{\isacharunderscore}{\kern0pt}gen{\isacharunderscore}{\kern0pt}path{\isacharunderscore}{\kern0pt}card{\isacharunderscore}{\kern0pt}V\ \isacommand{unfolding}\isamarkupfalse%
\ is{\isacharunderscore}{\kern0pt}gen{\isacharunderscore}{\kern0pt}path{\isacharunderscore}{\kern0pt}def\ is{\isacharunderscore}{\kern0pt}cycle{\isacharunderscore}{\kern0pt}alt\ \isacommand{by}\isamarkupfalse%
\ fastforce%
\endisatagproof
{\isafoldproof}%
%
\isadelimproof
\isanewline
%
\endisadelimproof
\isanewline
\isacommand{lemma}\isamarkupfalse%
\ {\isacharparenleft}{\kern0pt}\isakeyword{in}\ ulgraph{\isacharparenright}{\kern0pt}\ is{\isacharunderscore}{\kern0pt}cycle{\isacharunderscore}{\kern0pt}connecting{\isacharunderscore}{\kern0pt}path{\isacharcolon}{\kern0pt}\ {\isachardoublequoteopen}is{\isacharunderscore}{\kern0pt}cycle\ {\isacharparenleft}{\kern0pt}u{\isacharhash}{\kern0pt}v{\isacharhash}{\kern0pt}xs{\isacharparenright}{\kern0pt}\ {\isasymLongrightarrow}\ connecting{\isacharunderscore}{\kern0pt}path\ v\ u\ {\isacharparenleft}{\kern0pt}v{\isacharhash}{\kern0pt}xs{\isacharparenright}{\kern0pt}{\isachardoublequoteclose}\isanewline
%
\isadelimproof
\ \ %
\endisadelimproof
%
\isatagproof
\isacommand{unfolding}\isamarkupfalse%
\ is{\isacharunderscore}{\kern0pt}cycle{\isacharunderscore}{\kern0pt}def\ connecting{\isacharunderscore}{\kern0pt}path{\isacharunderscore}{\kern0pt}def\ is{\isacharunderscore}{\kern0pt}closed{\isacharunderscore}{\kern0pt}walk{\isacharunderscore}{\kern0pt}def\ is{\isacharunderscore}{\kern0pt}gen{\isacharunderscore}{\kern0pt}path{\isacharunderscore}{\kern0pt}def\ \isacommand{using}\isamarkupfalse%
\ is{\isacharunderscore}{\kern0pt}walk{\isacharunderscore}{\kern0pt}drop{\isacharunderscore}{\kern0pt}hd\ \isacommand{by}\isamarkupfalse%
\ auto%
\endisatagproof
{\isafoldproof}%
%
\isadelimproof
\isanewline
%
\endisadelimproof
\isanewline
\isacommand{lemma}\isamarkupfalse%
\ {\isacharparenleft}{\kern0pt}\isakeyword{in}\ ulgraph{\isacharparenright}{\kern0pt}\ cycle{\isacharunderscore}{\kern0pt}edges{\isacharunderscore}{\kern0pt}notin{\isacharunderscore}{\kern0pt}tl{\isacharcolon}{\kern0pt}\ {\isachardoublequoteopen}is{\isacharunderscore}{\kern0pt}cycle{\isadigit{2}}\ {\isacharparenleft}{\kern0pt}u{\isacharhash}{\kern0pt}v{\isacharhash}{\kern0pt}xs{\isacharparenright}{\kern0pt}\ {\isasymLongrightarrow}\ {\isacharbraceleft}{\kern0pt}u{\isacharcomma}{\kern0pt}v{\isacharbraceright}{\kern0pt}\ {\isasymnotin}\ set\ {\isacharparenleft}{\kern0pt}walk{\isacharunderscore}{\kern0pt}edges\ {\isacharparenleft}{\kern0pt}v{\isacharhash}{\kern0pt}xs{\isacharparenright}{\kern0pt}{\isacharparenright}{\kern0pt}{\isachardoublequoteclose}\isanewline
%
\isadelimproof
\ \ %
\endisadelimproof
%
\isatagproof
\isacommand{unfolding}\isamarkupfalse%
\ is{\isacharunderscore}{\kern0pt}cycle{\isadigit{2}}{\isacharunderscore}{\kern0pt}def\ \isacommand{by}\isamarkupfalse%
\ simp%
\endisatagproof
{\isafoldproof}%
%
\isadelimproof
%
\endisadelimproof
%
\isadelimdocument
%
\endisadelimdocument
%
\isatagdocument
%
\isamarkupsubsection{Subgraphs%
}
\isamarkuptrue%
%
\endisatagdocument
{\isafolddocument}%
%
\isadelimdocument
%
\endisadelimdocument
\isacommand{locale}\isamarkupfalse%
\ ulsubgraph\ {\isacharequal}{\kern0pt}\ subgraph\ V\isactrlsub H\ E\isactrlsub H\ V\isactrlsub G\ E\isactrlsub G\ {\isacharplus}{\kern0pt}\isanewline
\ \ G{\isacharcolon}{\kern0pt}\ ulgraph\ V\isactrlsub G\ E\isactrlsub G\ \isakeyword{for}\ V\isactrlsub H\ E\isactrlsub H\ V\isactrlsub G\ E\isactrlsub G\isanewline
\isakeyword{begin}\isanewline
\isanewline
\isacommand{interpretation}\isamarkupfalse%
\ H{\isacharcolon}{\kern0pt}\ ulgraph\ V\isactrlsub H\ E\isactrlsub H\isanewline
%
\isadelimproof
\ \ %
\endisadelimproof
%
\isatagproof
\isacommand{using}\isamarkupfalse%
\ is{\isacharunderscore}{\kern0pt}subgraph{\isacharunderscore}{\kern0pt}ulgraph\ G{\isachardot}{\kern0pt}ulgraph{\isacharunderscore}{\kern0pt}axioms\ \isacommand{by}\isamarkupfalse%
\ auto%
\endisatagproof
{\isafoldproof}%
%
\isadelimproof
\isanewline
%
\endisadelimproof
\isanewline
\isacommand{lemma}\isamarkupfalse%
\ is{\isacharunderscore}{\kern0pt}walk{\isacharcolon}{\kern0pt}\ {\isachardoublequoteopen}H{\isachardot}{\kern0pt}is{\isacharunderscore}{\kern0pt}walk\ xs\ {\isasymLongrightarrow}\ G{\isachardot}{\kern0pt}is{\isacharunderscore}{\kern0pt}walk\ xs{\isachardoublequoteclose}\isanewline
%
\isadelimproof
\ \ %
\endisadelimproof
%
\isatagproof
\isacommand{unfolding}\isamarkupfalse%
\ H{\isachardot}{\kern0pt}is{\isacharunderscore}{\kern0pt}walk{\isacharunderscore}{\kern0pt}def\ G{\isachardot}{\kern0pt}is{\isacharunderscore}{\kern0pt}walk{\isacharunderscore}{\kern0pt}def\ \isacommand{using}\isamarkupfalse%
\ verts{\isacharunderscore}{\kern0pt}ss\ edges{\isacharunderscore}{\kern0pt}ss\ \isacommand{by}\isamarkupfalse%
\ blast%
\endisatagproof
{\isafoldproof}%
%
\isadelimproof
\isanewline
%
\endisadelimproof
\isanewline
\isacommand{lemma}\isamarkupfalse%
\ is{\isacharunderscore}{\kern0pt}closed{\isacharunderscore}{\kern0pt}walk{\isacharcolon}{\kern0pt}\ {\isachardoublequoteopen}H{\isachardot}{\kern0pt}is{\isacharunderscore}{\kern0pt}closed{\isacharunderscore}{\kern0pt}walk\ xs\ {\isasymLongrightarrow}\ G{\isachardot}{\kern0pt}is{\isacharunderscore}{\kern0pt}closed{\isacharunderscore}{\kern0pt}walk\ xs{\isachardoublequoteclose}\isanewline
%
\isadelimproof
\ \ %
\endisadelimproof
%
\isatagproof
\isacommand{unfolding}\isamarkupfalse%
\ H{\isachardot}{\kern0pt}is{\isacharunderscore}{\kern0pt}closed{\isacharunderscore}{\kern0pt}walk{\isacharunderscore}{\kern0pt}def\ G{\isachardot}{\kern0pt}is{\isacharunderscore}{\kern0pt}closed{\isacharunderscore}{\kern0pt}walk{\isacharunderscore}{\kern0pt}def\ \isacommand{using}\isamarkupfalse%
\ is{\isacharunderscore}{\kern0pt}walk\ \isacommand{by}\isamarkupfalse%
\ blast%
\endisatagproof
{\isafoldproof}%
%
\isadelimproof
\isanewline
%
\endisadelimproof
\isanewline
\isacommand{lemma}\isamarkupfalse%
\ is{\isacharunderscore}{\kern0pt}gen{\isacharunderscore}{\kern0pt}path{\isacharcolon}{\kern0pt}\ {\isachardoublequoteopen}H{\isachardot}{\kern0pt}is{\isacharunderscore}{\kern0pt}gen{\isacharunderscore}{\kern0pt}path\ p\ {\isasymLongrightarrow}\ G{\isachardot}{\kern0pt}is{\isacharunderscore}{\kern0pt}gen{\isacharunderscore}{\kern0pt}path\ p{\isachardoublequoteclose}\isanewline
%
\isadelimproof
\ \ %
\endisadelimproof
%
\isatagproof
\isacommand{unfolding}\isamarkupfalse%
\ H{\isachardot}{\kern0pt}is{\isacharunderscore}{\kern0pt}gen{\isacharunderscore}{\kern0pt}path{\isacharunderscore}{\kern0pt}def\ G{\isachardot}{\kern0pt}is{\isacharunderscore}{\kern0pt}gen{\isacharunderscore}{\kern0pt}path{\isacharunderscore}{\kern0pt}def\ \isacommand{using}\isamarkupfalse%
\ is{\isacharunderscore}{\kern0pt}walk\ \isacommand{by}\isamarkupfalse%
\ blast%
\endisatagproof
{\isafoldproof}%
%
\isadelimproof
\isanewline
%
\endisadelimproof
\isanewline
\isacommand{lemma}\isamarkupfalse%
\ connecting{\isacharunderscore}{\kern0pt}path{\isacharcolon}{\kern0pt}\ {\isachardoublequoteopen}H{\isachardot}{\kern0pt}connecting{\isacharunderscore}{\kern0pt}path\ u\ v\ p\ {\isasymLongrightarrow}\ G{\isachardot}{\kern0pt}connecting{\isacharunderscore}{\kern0pt}path\ u\ v\ p{\isachardoublequoteclose}\isanewline
%
\isadelimproof
\ \ %
\endisadelimproof
%
\isatagproof
\isacommand{unfolding}\isamarkupfalse%
\ H{\isachardot}{\kern0pt}connecting{\isacharunderscore}{\kern0pt}path{\isacharunderscore}{\kern0pt}def\ G{\isachardot}{\kern0pt}connecting{\isacharunderscore}{\kern0pt}path{\isacharunderscore}{\kern0pt}def\ \isacommand{using}\isamarkupfalse%
\ is{\isacharunderscore}{\kern0pt}gen{\isacharunderscore}{\kern0pt}path\ \isacommand{by}\isamarkupfalse%
\ blast%
\endisatagproof
{\isafoldproof}%
%
\isadelimproof
\isanewline
%
\endisadelimproof
\isanewline
\isacommand{lemma}\isamarkupfalse%
\ is{\isacharunderscore}{\kern0pt}cycle{\isacharcolon}{\kern0pt}\ {\isachardoublequoteopen}H{\isachardot}{\kern0pt}is{\isacharunderscore}{\kern0pt}cycle\ c\ {\isasymLongrightarrow}\ G{\isachardot}{\kern0pt}is{\isacharunderscore}{\kern0pt}cycle\ c{\isachardoublequoteclose}\isanewline
%
\isadelimproof
\ \ %
\endisadelimproof
%
\isatagproof
\isacommand{unfolding}\isamarkupfalse%
\ H{\isachardot}{\kern0pt}is{\isacharunderscore}{\kern0pt}cycle{\isacharunderscore}{\kern0pt}def\ G{\isachardot}{\kern0pt}is{\isacharunderscore}{\kern0pt}cycle{\isacharunderscore}{\kern0pt}def\ \isacommand{using}\isamarkupfalse%
\ is{\isacharunderscore}{\kern0pt}closed{\isacharunderscore}{\kern0pt}walk\ \isacommand{by}\isamarkupfalse%
\ blast%
\endisatagproof
{\isafoldproof}%
%
\isadelimproof
\isanewline
%
\endisadelimproof
\isanewline
\isacommand{lemma}\isamarkupfalse%
\ is{\isacharunderscore}{\kern0pt}cycle{\isadigit{2}}{\isacharcolon}{\kern0pt}\ {\isachardoublequoteopen}H{\isachardot}{\kern0pt}is{\isacharunderscore}{\kern0pt}cycle{\isadigit{2}}\ c\ {\isasymLongrightarrow}\ G{\isachardot}{\kern0pt}is{\isacharunderscore}{\kern0pt}cycle{\isadigit{2}}\ c{\isachardoublequoteclose}\isanewline
%
\isadelimproof
\ \ %
\endisadelimproof
%
\isatagproof
\isacommand{unfolding}\isamarkupfalse%
\ H{\isachardot}{\kern0pt}is{\isacharunderscore}{\kern0pt}cycle{\isadigit{2}}{\isacharunderscore}{\kern0pt}def\ G{\isachardot}{\kern0pt}is{\isacharunderscore}{\kern0pt}cycle{\isadigit{2}}{\isacharunderscore}{\kern0pt}def\ \isacommand{using}\isamarkupfalse%
\ is{\isacharunderscore}{\kern0pt}cycle\ \isacommand{by}\isamarkupfalse%
\ blast%
\endisatagproof
{\isafoldproof}%
%
\isadelimproof
\isanewline
%
\endisadelimproof
\isanewline
\isacommand{lemma}\isamarkupfalse%
\ vert{\isacharunderscore}{\kern0pt}connected{\isacharcolon}{\kern0pt}\ {\isachardoublequoteopen}H{\isachardot}{\kern0pt}vert{\isacharunderscore}{\kern0pt}connected\ u\ v\ {\isasymLongrightarrow}\ G{\isachardot}{\kern0pt}vert{\isacharunderscore}{\kern0pt}connected\ u\ v{\isachardoublequoteclose}\isanewline
%
\isadelimproof
\ \ %
\endisadelimproof
%
\isatagproof
\isacommand{unfolding}\isamarkupfalse%
\ H{\isachardot}{\kern0pt}vert{\isacharunderscore}{\kern0pt}connected{\isacharunderscore}{\kern0pt}def\ G{\isachardot}{\kern0pt}vert{\isacharunderscore}{\kern0pt}connected{\isacharunderscore}{\kern0pt}def\ \isacommand{using}\isamarkupfalse%
\ connecting{\isacharunderscore}{\kern0pt}path\ \isacommand{by}\isamarkupfalse%
\ blast%
\endisatagproof
{\isafoldproof}%
%
\isadelimproof
\isanewline
%
\endisadelimproof
\isanewline
\isacommand{lemma}\isamarkupfalse%
\ is{\isacharunderscore}{\kern0pt}connected{\isacharunderscore}{\kern0pt}set{\isacharcolon}{\kern0pt}\ {\isachardoublequoteopen}H{\isachardot}{\kern0pt}is{\isacharunderscore}{\kern0pt}connected{\isacharunderscore}{\kern0pt}set\ V{\isacharprime}{\kern0pt}\ {\isasymLongrightarrow}\ G{\isachardot}{\kern0pt}is{\isacharunderscore}{\kern0pt}connected{\isacharunderscore}{\kern0pt}set\ V{\isacharprime}{\kern0pt}{\isachardoublequoteclose}\isanewline
%
\isadelimproof
\ \ %
\endisadelimproof
%
\isatagproof
\isacommand{unfolding}\isamarkupfalse%
\ H{\isachardot}{\kern0pt}is{\isacharunderscore}{\kern0pt}connected{\isacharunderscore}{\kern0pt}set{\isacharunderscore}{\kern0pt}def\ G{\isachardot}{\kern0pt}is{\isacharunderscore}{\kern0pt}connected{\isacharunderscore}{\kern0pt}set{\isacharunderscore}{\kern0pt}def\ \isacommand{using}\isamarkupfalse%
\ vert{\isacharunderscore}{\kern0pt}connected\ \isacommand{by}\isamarkupfalse%
\ blast%
\endisatagproof
{\isafoldproof}%
%
\isadelimproof
\isanewline
%
\endisadelimproof
\isanewline
\isacommand{end}\isamarkupfalse%
\isanewline
\isanewline
\isacommand{lemma}\isamarkupfalse%
\ {\isacharparenleft}{\kern0pt}\isakeyword{in}\ graph{\isacharunderscore}{\kern0pt}system{\isacharparenright}{\kern0pt}\ subgraph{\isacharunderscore}{\kern0pt}remove{\isacharunderscore}{\kern0pt}vertex{\isacharcolon}{\kern0pt}\ {\isachardoublequoteopen}remove{\isacharunderscore}{\kern0pt}vertex\ v\ {\isacharequal}{\kern0pt}\ {\isacharparenleft}{\kern0pt}V{\isacharprime}{\kern0pt}{\isacharcomma}{\kern0pt}\ E{\isacharprime}{\kern0pt}{\isacharparenright}{\kern0pt}\ {\isasymLongrightarrow}\ subgraph\ V{\isacharprime}{\kern0pt}\ E{\isacharprime}{\kern0pt}\ V\ E{\isachardoublequoteclose}\isanewline
%
\isadelimproof
\ \ %
\endisadelimproof
%
\isatagproof
\isacommand{using}\isamarkupfalse%
\ wellformed\ \isacommand{unfolding}\isamarkupfalse%
\ remove{\isacharunderscore}{\kern0pt}vertex{\isacharunderscore}{\kern0pt}def\ incident{\isacharunderscore}{\kern0pt}def\ \isacommand{by}\isamarkupfalse%
\ {\isacharparenleft}{\kern0pt}unfold{\isacharunderscore}{\kern0pt}locales{\isacharcomma}{\kern0pt}\ auto{\isacharparenright}{\kern0pt}%
\endisatagproof
{\isafoldproof}%
%
\isadelimproof
%
\endisadelimproof
%
\isadelimdocument
%
\endisadelimdocument
%
\isatagdocument
%
\isamarkupsubsection{Connectivity%
}
\isamarkuptrue%
%
\endisatagdocument
{\isafolddocument}%
%
\isadelimdocument
%
\endisadelimdocument
\isacommand{lemma}\isamarkupfalse%
\ {\isacharparenleft}{\kern0pt}\isakeyword{in}\ ulgraph{\isacharparenright}{\kern0pt}\ connecting{\isacharunderscore}{\kern0pt}path{\isacharunderscore}{\kern0pt}connected{\isacharunderscore}{\kern0pt}set{\isacharcolon}{\kern0pt}\isanewline
\ \ \isakeyword{assumes}\ conn{\isacharunderscore}{\kern0pt}path{\isacharcolon}{\kern0pt}\ {\isachardoublequoteopen}connecting{\isacharunderscore}{\kern0pt}path\ u\ v\ p{\isachardoublequoteclose}\isanewline
\ \ \isakeyword{shows}\ {\isachardoublequoteopen}is{\isacharunderscore}{\kern0pt}connected{\isacharunderscore}{\kern0pt}set\ {\isacharparenleft}{\kern0pt}set\ p{\isacharparenright}{\kern0pt}{\isachardoublequoteclose}\isanewline
%
\isadelimproof
%
\endisadelimproof
%
\isatagproof
\isacommand{proof}\isamarkupfalse%
{\isacharminus}{\kern0pt}\isanewline
\ \ \isacommand{have}\isamarkupfalse%
\ {\isachardoublequoteopen}{\isasymforall}w{\isasymin}set\ p{\isachardot}{\kern0pt}\ vert{\isacharunderscore}{\kern0pt}connected\ u\ w{\isachardoublequoteclose}\isanewline
\ \ \isacommand{proof}\isamarkupfalse%
\isanewline
\ \ \ \ \isacommand{fix}\isamarkupfalse%
\ w\ \isacommand{assume}\isamarkupfalse%
\ {\isachardoublequoteopen}w\ {\isasymin}\ set\ p{\isachardoublequoteclose}\isanewline
\ \ \ \ \isacommand{then}\isamarkupfalse%
\ \isacommand{obtain}\isamarkupfalse%
\ xs\ ys\ \isakeyword{where}\ {\isachardoublequoteopen}p\ {\isacharequal}{\kern0pt}\ xs{\isacharat}{\kern0pt}{\isacharbrackleft}{\kern0pt}w{\isacharbrackright}{\kern0pt}{\isacharat}{\kern0pt}ys{\isachardoublequoteclose}\ \isacommand{using}\isamarkupfalse%
\ split{\isacharunderscore}{\kern0pt}list\ \isacommand{by}\isamarkupfalse%
\ fastforce\isanewline
\ \ \ \ \isacommand{then}\isamarkupfalse%
\ \isacommand{have}\isamarkupfalse%
\ {\isachardoublequoteopen}connecting{\isacharunderscore}{\kern0pt}path\ u\ w\ {\isacharparenleft}{\kern0pt}xs{\isacharat}{\kern0pt}{\isacharbrackleft}{\kern0pt}w{\isacharbrackright}{\kern0pt}{\isacharparenright}{\kern0pt}{\isachardoublequoteclose}\ \isacommand{using}\isamarkupfalse%
\ conn{\isacharunderscore}{\kern0pt}path\ \isacommand{unfolding}\isamarkupfalse%
\ connecting{\isacharunderscore}{\kern0pt}path{\isacharunderscore}{\kern0pt}def\ \isacommand{using}\isamarkupfalse%
\ is{\isacharunderscore}{\kern0pt}gen{\isacharunderscore}{\kern0pt}path{\isacharunderscore}{\kern0pt}prefix\ \isacommand{by}\isamarkupfalse%
\ {\isacharparenleft}{\kern0pt}auto\ simp{\isacharcolon}{\kern0pt}\ hd{\isacharunderscore}{\kern0pt}append{\isacharparenright}{\kern0pt}\isanewline
\ \ \ \ \isacommand{then}\isamarkupfalse%
\ \isacommand{show}\isamarkupfalse%
\ {\isachardoublequoteopen}vert{\isacharunderscore}{\kern0pt}connected\ u\ w{\isachardoublequoteclose}\ \isacommand{unfolding}\isamarkupfalse%
\ vert{\isacharunderscore}{\kern0pt}connected{\isacharunderscore}{\kern0pt}def\ \isacommand{by}\isamarkupfalse%
\ blast\isanewline
\ \ \isacommand{qed}\isamarkupfalse%
\isanewline
\ \ \isacommand{then}\isamarkupfalse%
\ \isacommand{show}\isamarkupfalse%
\ {\isacharquery}{\kern0pt}thesis\ \isacommand{using}\isamarkupfalse%
\ vert{\isacharunderscore}{\kern0pt}connected{\isacharunderscore}{\kern0pt}rev\ vert{\isacharunderscore}{\kern0pt}connected{\isacharunderscore}{\kern0pt}trans\ \isacommand{unfolding}\isamarkupfalse%
\ is{\isacharunderscore}{\kern0pt}connected{\isacharunderscore}{\kern0pt}set{\isacharunderscore}{\kern0pt}def\ \isacommand{by}\isamarkupfalse%
\ blast\isanewline
\isacommand{qed}\isamarkupfalse%
%
\endisatagproof
{\isafoldproof}%
%
\isadelimproof
\isanewline
%
\endisadelimproof
\isanewline
\isacommand{lemma}\isamarkupfalse%
\ {\isacharparenleft}{\kern0pt}\isakeyword{in}\ ulgraph{\isacharparenright}{\kern0pt}\ vert{\isacharunderscore}{\kern0pt}connected{\isacharunderscore}{\kern0pt}neighbors{\isacharcolon}{\kern0pt}\isanewline
\ \ \isakeyword{assumes}\ {\isachardoublequoteopen}{\isacharbraceleft}{\kern0pt}v{\isacharcomma}{\kern0pt}u{\isacharbraceright}{\kern0pt}\ {\isasymin}\ E{\isachardoublequoteclose}\isanewline
\ \ \isakeyword{shows}\ {\isachardoublequoteopen}vert{\isacharunderscore}{\kern0pt}connected\ v\ u{\isachardoublequoteclose}\isanewline
%
\isadelimproof
%
\endisadelimproof
%
\isatagproof
\isacommand{proof}\isamarkupfalse%
{\isacharminus}{\kern0pt}\isanewline
\ \ \isacommand{have}\isamarkupfalse%
\ {\isachardoublequoteopen}connecting{\isacharunderscore}{\kern0pt}path\ v\ u\ {\isacharbrackleft}{\kern0pt}v{\isacharcomma}{\kern0pt}u{\isacharbrackright}{\kern0pt}{\isachardoublequoteclose}\ \isacommand{unfolding}\isamarkupfalse%
\ connecting{\isacharunderscore}{\kern0pt}path{\isacharunderscore}{\kern0pt}def\ is{\isacharunderscore}{\kern0pt}gen{\isacharunderscore}{\kern0pt}path{\isacharunderscore}{\kern0pt}def\ is{\isacharunderscore}{\kern0pt}walk{\isacharunderscore}{\kern0pt}def\ \isacommand{using}\isamarkupfalse%
\ assms\ wellformed\ \isacommand{by}\isamarkupfalse%
\ auto\isanewline
\ \ \isacommand{then}\isamarkupfalse%
\ \isacommand{show}\isamarkupfalse%
\ {\isacharquery}{\kern0pt}thesis\ \isacommand{unfolding}\isamarkupfalse%
\ vert{\isacharunderscore}{\kern0pt}connected{\isacharunderscore}{\kern0pt}def\ \isacommand{by}\isamarkupfalse%
\ auto\isanewline
\isacommand{qed}\isamarkupfalse%
%
\endisatagproof
{\isafoldproof}%
%
\isadelimproof
\isanewline
%
\endisadelimproof
\isanewline
\isacommand{lemma}\isamarkupfalse%
\ {\isacharparenleft}{\kern0pt}\isakeyword{in}\ ulgraph{\isacharparenright}{\kern0pt}\ connected{\isacharunderscore}{\kern0pt}empty{\isacharunderscore}{\kern0pt}E{\isacharcolon}{\kern0pt}\isanewline
\ \ \isakeyword{assumes}\ empty{\isacharcolon}{\kern0pt}\ {\isachardoublequoteopen}E\ {\isacharequal}{\kern0pt}\ {\isacharbraceleft}{\kern0pt}{\isacharbraceright}{\kern0pt}{\isachardoublequoteclose}\isanewline
\ \ \ \ \isakeyword{and}\ connected{\isacharcolon}{\kern0pt}\ {\isachardoublequoteopen}vert{\isacharunderscore}{\kern0pt}connected\ u\ v{\isachardoublequoteclose}\isanewline
\ \ \isakeyword{shows}\ {\isachardoublequoteopen}u\ {\isacharequal}{\kern0pt}\ v{\isachardoublequoteclose}\isanewline
%
\isadelimproof
%
\endisadelimproof
%
\isatagproof
\isacommand{proof}\isamarkupfalse%
\ {\isacharparenleft}{\kern0pt}rule\ ccontr{\isacharparenright}{\kern0pt}\isanewline
\ \ \isacommand{assume}\isamarkupfalse%
\ {\isachardoublequoteopen}u\ {\isasymnoteq}\ v{\isachardoublequoteclose}\isanewline
\ \ \isacommand{then}\isamarkupfalse%
\ \isacommand{obtain}\isamarkupfalse%
\ p\ \isakeyword{where}\ conn{\isacharunderscore}{\kern0pt}path{\isacharcolon}{\kern0pt}\ {\isachardoublequoteopen}connecting{\isacharunderscore}{\kern0pt}path\ u\ v\ p{\isachardoublequoteclose}\ \isacommand{using}\isamarkupfalse%
\ connected\ \isacommand{unfolding}\isamarkupfalse%
\ vert{\isacharunderscore}{\kern0pt}connected{\isacharunderscore}{\kern0pt}def\ \isacommand{by}\isamarkupfalse%
\ blast\isanewline
\ \ \isacommand{then}\isamarkupfalse%
\ \isacommand{obtain}\isamarkupfalse%
\ e\ \isakeyword{where}\ {\isachardoublequoteopen}e\ {\isasymin}\ set\ {\isacharparenleft}{\kern0pt}walk{\isacharunderscore}{\kern0pt}edges\ p{\isacharparenright}{\kern0pt}{\isachardoublequoteclose}\ \isacommand{using}\isamarkupfalse%
\ {\isacartoucheopen}u{\isasymnoteq}v{\isacartoucheclose}\ connecting{\isacharunderscore}{\kern0pt}path{\isacharunderscore}{\kern0pt}length{\isacharunderscore}{\kern0pt}bound\ \isacommand{unfolding}\isamarkupfalse%
\ walk{\isacharunderscore}{\kern0pt}length{\isacharunderscore}{\kern0pt}def\ \isacommand{by}\isamarkupfalse%
\ fastforce\isanewline
\ \ \isacommand{then}\isamarkupfalse%
\ \isacommand{have}\isamarkupfalse%
\ {\isachardoublequoteopen}e\ {\isasymin}\ E{\isachardoublequoteclose}\ \isacommand{using}\isamarkupfalse%
\ conn{\isacharunderscore}{\kern0pt}path\ \isacommand{unfolding}\isamarkupfalse%
\ connecting{\isacharunderscore}{\kern0pt}path{\isacharunderscore}{\kern0pt}def\ is{\isacharunderscore}{\kern0pt}gen{\isacharunderscore}{\kern0pt}path{\isacharunderscore}{\kern0pt}def\ is{\isacharunderscore}{\kern0pt}walk{\isacharunderscore}{\kern0pt}def\ \isacommand{by}\isamarkupfalse%
\ blast\isanewline
\ \ \isacommand{then}\isamarkupfalse%
\ \isacommand{show}\isamarkupfalse%
\ False\ \isacommand{using}\isamarkupfalse%
\ empty\ \isacommand{by}\isamarkupfalse%
\ blast\isanewline
\isacommand{qed}\isamarkupfalse%
%
\endisatagproof
{\isafoldproof}%
%
\isadelimproof
\isanewline
%
\endisadelimproof
\isanewline
\isacommand{lemma}\isamarkupfalse%
\ {\isacharparenleft}{\kern0pt}\isakeyword{in}\ fin{\isacharunderscore}{\kern0pt}ulgraph{\isacharparenright}{\kern0pt}\ degree{\isacharunderscore}{\kern0pt}{\isadigit{0}}{\isacharunderscore}{\kern0pt}not{\isacharunderscore}{\kern0pt}connected{\isacharcolon}{\kern0pt}\isanewline
\ \ \isakeyword{assumes}\ degree{\isacharunderscore}{\kern0pt}{\isadigit{0}}{\isacharcolon}{\kern0pt}\ {\isachardoublequoteopen}degree\ v\ {\isacharequal}{\kern0pt}\ {\isadigit{0}}{\isachardoublequoteclose}\isanewline
\ \ \ \ \isakeyword{and}\ {\isachardoublequoteopen}u\ {\isasymnoteq}\ v{\isachardoublequoteclose}\isanewline
\ \ \isakeyword{shows}\ {\isachardoublequoteopen}{\isasymnot}\ vert{\isacharunderscore}{\kern0pt}connected\ v\ u{\isachardoublequoteclose}\isanewline
%
\isadelimproof
%
\endisadelimproof
%
\isatagproof
\isacommand{proof}\isamarkupfalse%
\isanewline
\ \ \isacommand{assume}\isamarkupfalse%
\ connected{\isacharcolon}{\kern0pt}\ {\isachardoublequoteopen}vert{\isacharunderscore}{\kern0pt}connected\ v\ u{\isachardoublequoteclose}\isanewline
\ \ \isacommand{then}\isamarkupfalse%
\ \isacommand{obtain}\isamarkupfalse%
\ p\ \isakeyword{where}\ conn{\isacharunderscore}{\kern0pt}path{\isacharcolon}{\kern0pt}\ {\isachardoublequoteopen}connecting{\isacharunderscore}{\kern0pt}path\ v\ u\ p{\isachardoublequoteclose}\ \isacommand{unfolding}\isamarkupfalse%
\ vert{\isacharunderscore}{\kern0pt}connected{\isacharunderscore}{\kern0pt}def\ \isacommand{by}\isamarkupfalse%
\ blast\isanewline
\ \ \isacommand{then}\isamarkupfalse%
\ \isacommand{have}\isamarkupfalse%
\ {\isachardoublequoteopen}walk{\isacharunderscore}{\kern0pt}length\ p\ {\isasymge}\ {\isadigit{1}}{\isachardoublequoteclose}\ \isacommand{using}\isamarkupfalse%
\ {\isacartoucheopen}u{\isasymnoteq}v{\isacartoucheclose}\ connecting{\isacharunderscore}{\kern0pt}path{\isacharunderscore}{\kern0pt}length{\isacharunderscore}{\kern0pt}bound\ \isacommand{by}\isamarkupfalse%
\ metis\isanewline
\ \ \isacommand{then}\isamarkupfalse%
\ \isacommand{have}\isamarkupfalse%
\ {\isachardoublequoteopen}length\ p\ {\isasymge}\ {\isadigit{2}}{\isachardoublequoteclose}\ \isacommand{using}\isamarkupfalse%
\ walk{\isacharunderscore}{\kern0pt}length{\isacharunderscore}{\kern0pt}conv\ \isacommand{by}\isamarkupfalse%
\ simp\isanewline
\ \ \isacommand{then}\isamarkupfalse%
\ \isacommand{obtain}\isamarkupfalse%
\ w\ p{\isacharprime}{\kern0pt}\ \isakeyword{where}\ {\isachardoublequoteopen}p\ {\isacharequal}{\kern0pt}\ v{\isacharhash}{\kern0pt}w{\isacharhash}{\kern0pt}p{\isacharprime}{\kern0pt}{\isachardoublequoteclose}\ \isacommand{using}\isamarkupfalse%
\ walk{\isacharunderscore}{\kern0pt}length{\isacharunderscore}{\kern0pt}conv\ conn{\isacharunderscore}{\kern0pt}path\ \isacommand{unfolding}\isamarkupfalse%
\ connecting{\isacharunderscore}{\kern0pt}path{\isacharunderscore}{\kern0pt}def\isanewline
\ \ \ \ \isacommand{by}\isamarkupfalse%
\ {\isacharparenleft}{\kern0pt}metis\ assms{\isacharparenleft}{\kern0pt}{\isadigit{2}}{\isacharparenright}{\kern0pt}\ is{\isacharunderscore}{\kern0pt}gen{\isacharunderscore}{\kern0pt}path{\isacharunderscore}{\kern0pt}def\ is{\isacharunderscore}{\kern0pt}walk{\isacharunderscore}{\kern0pt}not{\isacharunderscore}{\kern0pt}empty{\isadigit{2}}\ last{\isacharunderscore}{\kern0pt}ConsL\ list{\isachardot}{\kern0pt}collapse{\isacharparenright}{\kern0pt}\isanewline
\ \ \isacommand{then}\isamarkupfalse%
\ \isacommand{have}\isamarkupfalse%
\ inE{\isacharcolon}{\kern0pt}\ {\isachardoublequoteopen}{\isacharbraceleft}{\kern0pt}v{\isacharcomma}{\kern0pt}w{\isacharbraceright}{\kern0pt}\ {\isasymin}\ E{\isachardoublequoteclose}\ \isacommand{using}\isamarkupfalse%
\ conn{\isacharunderscore}{\kern0pt}path\ \isacommand{unfolding}\isamarkupfalse%
\ connecting{\isacharunderscore}{\kern0pt}path{\isacharunderscore}{\kern0pt}def\ is{\isacharunderscore}{\kern0pt}gen{\isacharunderscore}{\kern0pt}path{\isacharunderscore}{\kern0pt}def\ is{\isacharunderscore}{\kern0pt}walk{\isacharunderscore}{\kern0pt}def\ \isacommand{by}\isamarkupfalse%
\ simp\isanewline
\ \ \isacommand{then}\isamarkupfalse%
\ \isacommand{have}\isamarkupfalse%
\ {\isachardoublequoteopen}{\isacharbraceleft}{\kern0pt}v{\isacharcomma}{\kern0pt}w{\isacharbraceright}{\kern0pt}\ {\isasymin}\ incident{\isacharunderscore}{\kern0pt}edges\ v{\isachardoublequoteclose}\ \isacommand{unfolding}\isamarkupfalse%
\ incident{\isacharunderscore}{\kern0pt}edges{\isacharunderscore}{\kern0pt}def\ incident{\isacharunderscore}{\kern0pt}def\ \isacommand{by}\isamarkupfalse%
\ simp\isanewline
\ \ \isacommand{then}\isamarkupfalse%
\ \isacommand{show}\isamarkupfalse%
\ False\ \isacommand{using}\isamarkupfalse%
\ degree{\isadigit{0}}{\isacharunderscore}{\kern0pt}inc{\isacharunderscore}{\kern0pt}edges{\isacharunderscore}{\kern0pt}empt{\isacharunderscore}{\kern0pt}iff\ fin{\isacharunderscore}{\kern0pt}edges\ degree{\isacharunderscore}{\kern0pt}{\isadigit{0}}\ \isacommand{by}\isamarkupfalse%
\ blast\isanewline
\isacommand{qed}\isamarkupfalse%
%
\endisatagproof
{\isafoldproof}%
%
\isadelimproof
\isanewline
%
\endisadelimproof
\isanewline
\isacommand{lemma}\isamarkupfalse%
\ {\isacharparenleft}{\kern0pt}\isakeyword{in}\ fin{\isacharunderscore}{\kern0pt}connected{\isacharunderscore}{\kern0pt}ulgraph{\isacharparenright}{\kern0pt}\ degree{\isacharunderscore}{\kern0pt}not{\isacharunderscore}{\kern0pt}{\isadigit{0}}{\isacharcolon}{\kern0pt}\isanewline
\ \ \isakeyword{assumes}\ {\isachardoublequoteopen}card\ V\ {\isasymge}\ {\isadigit{2}}{\isachardoublequoteclose}\isanewline
\ \ \ \ \isakeyword{and}\ inV{\isacharcolon}{\kern0pt}\ {\isachardoublequoteopen}v\ {\isasymin}\ V{\isachardoublequoteclose}\isanewline
\ \ \isakeyword{shows}\ {\isachardoublequoteopen}degree\ v\ {\isasymnoteq}\ {\isadigit{0}}{\isachardoublequoteclose}\isanewline
%
\isadelimproof
%
\endisadelimproof
%
\isatagproof
\isacommand{proof}\isamarkupfalse%
{\isacharminus}{\kern0pt}\isanewline
\ \ \isacommand{obtain}\isamarkupfalse%
\ u\ \isakeyword{where}\ {\isachardoublequoteopen}u\ {\isasymin}\ V{\isachardoublequoteclose}\ \isakeyword{and}\ {\isachardoublequoteopen}u\ {\isasymnoteq}\ v{\isachardoublequoteclose}\ \isacommand{using}\isamarkupfalse%
\ assms\isanewline
\ \ \ \ \isacommand{by}\isamarkupfalse%
\ {\isacharparenleft}{\kern0pt}metis\ card{\isacharunderscore}{\kern0pt}eq{\isacharunderscore}{\kern0pt}{\isadigit{0}}{\isacharunderscore}{\kern0pt}iff\ card{\isacharunderscore}{\kern0pt}le{\isacharunderscore}{\kern0pt}Suc{\isadigit{0}}{\isacharunderscore}{\kern0pt}iff{\isacharunderscore}{\kern0pt}eq\ less{\isacharunderscore}{\kern0pt}eq{\isacharunderscore}{\kern0pt}Suc{\isacharunderscore}{\kern0pt}le\ nat{\isacharunderscore}{\kern0pt}less{\isacharunderscore}{\kern0pt}le\ not{\isacharunderscore}{\kern0pt}less{\isacharunderscore}{\kern0pt}eq{\isacharunderscore}{\kern0pt}eq\ numeral{\isacharunderscore}{\kern0pt}{\isadigit{2}}{\isacharunderscore}{\kern0pt}eq{\isacharunderscore}{\kern0pt}{\isadigit{2}}{\isacharparenright}{\kern0pt}\isanewline
\ \ \isacommand{then}\isamarkupfalse%
\ \isacommand{show}\isamarkupfalse%
\ {\isacharquery}{\kern0pt}thesis\ \isacommand{using}\isamarkupfalse%
\ degree{\isacharunderscore}{\kern0pt}{\isadigit{0}}{\isacharunderscore}{\kern0pt}not{\isacharunderscore}{\kern0pt}connected\ inV\ vertices{\isacharunderscore}{\kern0pt}connected\ \isacommand{by}\isamarkupfalse%
\ blast\isanewline
\isacommand{qed}\isamarkupfalse%
%
\endisatagproof
{\isafoldproof}%
%
\isadelimproof
\isanewline
%
\endisadelimproof
\isanewline
\isacommand{lemma}\isamarkupfalse%
\ {\isacharparenleft}{\kern0pt}\isakeyword{in}\ connected{\isacharunderscore}{\kern0pt}ulgraph{\isacharparenright}{\kern0pt}\ V{\isacharunderscore}{\kern0pt}E{\isacharunderscore}{\kern0pt}empty{\isacharcolon}{\kern0pt}\ {\isachardoublequoteopen}E\ {\isacharequal}{\kern0pt}\ {\isacharbraceleft}{\kern0pt}{\isacharbraceright}{\kern0pt}\ {\isasymLongrightarrow}\ {\isasymexists}v{\isachardot}{\kern0pt}\ V\ {\isacharequal}{\kern0pt}\ {\isacharbraceleft}{\kern0pt}v{\isacharbraceright}{\kern0pt}{\isachardoublequoteclose}\isanewline
%
\isadelimproof
\ \ %
\endisadelimproof
%
\isatagproof
\isacommand{using}\isamarkupfalse%
\ connected{\isacharunderscore}{\kern0pt}empty{\isacharunderscore}{\kern0pt}E\ connected\ not{\isacharunderscore}{\kern0pt}empty\ \isacommand{unfolding}\isamarkupfalse%
\ is{\isacharunderscore}{\kern0pt}connected{\isacharunderscore}{\kern0pt}set{\isacharunderscore}{\kern0pt}def\isanewline
\ \ \isacommand{by}\isamarkupfalse%
\ {\isacharparenleft}{\kern0pt}metis\ ex{\isacharunderscore}{\kern0pt}in{\isacharunderscore}{\kern0pt}conv\ insert{\isacharunderscore}{\kern0pt}iff\ mk{\isacharunderscore}{\kern0pt}disjoint{\isacharunderscore}{\kern0pt}insert{\isacharparenright}{\kern0pt}%
\endisatagproof
{\isafoldproof}%
%
\isadelimproof
\isanewline
%
\endisadelimproof
\isanewline
\isacommand{lemma}\isamarkupfalse%
\ {\isacharparenleft}{\kern0pt}\isakeyword{in}\ connected{\isacharunderscore}{\kern0pt}ulgraph{\isacharparenright}{\kern0pt}\ vert{\isacharunderscore}{\kern0pt}connected{\isacharunderscore}{\kern0pt}remove{\isacharunderscore}{\kern0pt}edge{\isacharcolon}{\kern0pt}\isanewline
\ \ \isakeyword{assumes}\ e{\isacharcolon}{\kern0pt}\ {\isachardoublequoteopen}{\isacharbraceleft}{\kern0pt}u{\isacharcomma}{\kern0pt}v{\isacharbraceright}{\kern0pt}\ {\isasymin}\ E{\isachardoublequoteclose}\isanewline
\ \ \isakeyword{shows}\ {\isachardoublequoteopen}{\isasymforall}w{\isasymin}V{\isachardot}{\kern0pt}\ ulgraph{\isachardot}{\kern0pt}vert{\isacharunderscore}{\kern0pt}connected\ V\ {\isacharparenleft}{\kern0pt}E\ {\isacharminus}{\kern0pt}\ {\isacharbraceleft}{\kern0pt}{\isacharbraceleft}{\kern0pt}u{\isacharcomma}{\kern0pt}v{\isacharbraceright}{\kern0pt}{\isacharbraceright}{\kern0pt}{\isacharparenright}{\kern0pt}\ w\ u\ {\isasymor}\ ulgraph{\isachardot}{\kern0pt}vert{\isacharunderscore}{\kern0pt}connected\ V\ {\isacharparenleft}{\kern0pt}E\ {\isacharminus}{\kern0pt}\ {\isacharbraceleft}{\kern0pt}{\isacharbraceleft}{\kern0pt}u{\isacharcomma}{\kern0pt}v{\isacharbraceright}{\kern0pt}{\isacharbraceright}{\kern0pt}{\isacharparenright}{\kern0pt}\ w\ v{\isachardoublequoteclose}\isanewline
%
\isadelimproof
%
\endisadelimproof
%
\isatagproof
\isacommand{proof}\isamarkupfalse%
\isanewline
\ \ \isacommand{fix}\isamarkupfalse%
\ w\ \isacommand{assume}\isamarkupfalse%
\ {\isachardoublequoteopen}w{\isasymin}V{\isachardoublequoteclose}\isanewline
\ \ \isacommand{interpret}\isamarkupfalse%
\ g{\isacharprime}{\kern0pt}{\isacharcolon}{\kern0pt}\ ulgraph\ V\ {\isachardoublequoteopen}E\ {\isacharminus}{\kern0pt}\ {\isacharbraceleft}{\kern0pt}{\isacharbraceleft}{\kern0pt}u{\isacharcomma}{\kern0pt}v{\isacharbraceright}{\kern0pt}{\isacharbraceright}{\kern0pt}{\isachardoublequoteclose}\ \isacommand{using}\isamarkupfalse%
\ wellformed\ edge{\isacharunderscore}{\kern0pt}size\ \isacommand{by}\isamarkupfalse%
\ {\isacharparenleft}{\kern0pt}unfold{\isacharunderscore}{\kern0pt}locales{\isacharcomma}{\kern0pt}\ auto{\isacharparenright}{\kern0pt}\isanewline
\ \ \isacommand{have}\isamarkupfalse%
\ inV{\isacharcolon}{\kern0pt}\ {\isachardoublequoteopen}u\ {\isasymin}\ V{\isachardoublequoteclose}\ {\isachardoublequoteopen}v\ {\isasymin}\ V{\isachardoublequoteclose}\ \isacommand{using}\isamarkupfalse%
\ e\ wellformed\ \isacommand{by}\isamarkupfalse%
\ auto\isanewline
\ \ \isacommand{obtain}\isamarkupfalse%
\ p\ \isakeyword{where}\ conn{\isacharunderscore}{\kern0pt}path{\isacharcolon}{\kern0pt}\ {\isachardoublequoteopen}connecting{\isacharunderscore}{\kern0pt}path\ w\ v\ p{\isachardoublequoteclose}\ \isacommand{using}\isamarkupfalse%
\ connected\ inV\ {\isacartoucheopen}w{\isasymin}V{\isacartoucheclose}\ \isacommand{unfolding}\isamarkupfalse%
\ is{\isacharunderscore}{\kern0pt}connected{\isacharunderscore}{\kern0pt}set{\isacharunderscore}{\kern0pt}def\ vert{\isacharunderscore}{\kern0pt}connected{\isacharunderscore}{\kern0pt}def\ \isacommand{by}\isamarkupfalse%
\ blast\isanewline
\ \ \isacommand{then}\isamarkupfalse%
\ \isacommand{show}\isamarkupfalse%
\ {\isachardoublequoteopen}g{\isacharprime}{\kern0pt}{\isachardot}{\kern0pt}vert{\isacharunderscore}{\kern0pt}connected\ w\ u\ {\isasymor}\ g{\isacharprime}{\kern0pt}{\isachardot}{\kern0pt}vert{\isacharunderscore}{\kern0pt}connected\ w\ v{\isachardoublequoteclose}\isanewline
\ \ \isacommand{proof}\isamarkupfalse%
\ {\isacharparenleft}{\kern0pt}cases\ {\isachardoublequoteopen}{\isacharbraceleft}{\kern0pt}u{\isacharcomma}{\kern0pt}v{\isacharbraceright}{\kern0pt}\ {\isasymin}\ set\ {\isacharparenleft}{\kern0pt}walk{\isacharunderscore}{\kern0pt}edges\ p{\isacharparenright}{\kern0pt}{\isachardoublequoteclose}{\isacharparenright}{\kern0pt}\isanewline
\ \ \ \ \isacommand{case}\isamarkupfalse%
\ True\isanewline
\ \ \ \ \isacommand{assume}\isamarkupfalse%
\ walk{\isacharunderscore}{\kern0pt}edge{\isacharcolon}{\kern0pt}\ {\isachardoublequoteopen}{\isacharbraceleft}{\kern0pt}u{\isacharcomma}{\kern0pt}v{\isacharbraceright}{\kern0pt}\ {\isasymin}\ set\ {\isacharparenleft}{\kern0pt}walk{\isacharunderscore}{\kern0pt}edges\ p{\isacharparenright}{\kern0pt}{\isachardoublequoteclose}\isanewline
\ \ \ \ \isacommand{then}\isamarkupfalse%
\ \isacommand{show}\isamarkupfalse%
\ {\isacharquery}{\kern0pt}thesis\isanewline
\ \ \ \ \isacommand{proof}\isamarkupfalse%
\ {\isacharparenleft}{\kern0pt}cases\ {\isachardoublequoteopen}w\ {\isacharequal}{\kern0pt}\ v{\isachardoublequoteclose}{\isacharparenright}{\kern0pt}\isanewline
\ \ \ \ \ \ \isacommand{case}\isamarkupfalse%
\ True\isanewline
\ \ \ \ \ \ \isacommand{then}\isamarkupfalse%
\ \isacommand{show}\isamarkupfalse%
\ {\isacharquery}{\kern0pt}thesis\ \isacommand{using}\isamarkupfalse%
\ inV\ g{\isacharprime}{\kern0pt}{\isachardot}{\kern0pt}vert{\isacharunderscore}{\kern0pt}connected{\isacharunderscore}{\kern0pt}id\ \isacommand{by}\isamarkupfalse%
\ blast\isanewline
\ \ \ \ \isacommand{next}\isamarkupfalse%
\isanewline
\ \ \ \ \ \ \isacommand{case}\isamarkupfalse%
\ False\isanewline
\ \ \ \ \ \ \isacommand{then}\isamarkupfalse%
\ \isacommand{have}\isamarkupfalse%
\ distinct{\isacharcolon}{\kern0pt}\ {\isachardoublequoteopen}distinct\ p{\isachardoublequoteclose}\ \isacommand{using}\isamarkupfalse%
\ conn{\isacharunderscore}{\kern0pt}path\ \isacommand{by}\isamarkupfalse%
\ {\isacharparenleft}{\kern0pt}simp\ add{\isacharcolon}{\kern0pt}\ connecting{\isacharunderscore}{\kern0pt}path{\isacharunderscore}{\kern0pt}def\ is{\isacharunderscore}{\kern0pt}gen{\isacharunderscore}{\kern0pt}path{\isacharunderscore}{\kern0pt}distinct{\isacharparenright}{\kern0pt}\isanewline
\ \ \ \ \ \ \isacommand{have}\isamarkupfalse%
\ {\isachardoublequoteopen}u\ {\isasymin}\ set\ p{\isachardoublequoteclose}\ \isacommand{using}\isamarkupfalse%
\ walk{\isacharunderscore}{\kern0pt}edge\ walk{\isacharunderscore}{\kern0pt}edges{\isacharunderscore}{\kern0pt}in{\isacharunderscore}{\kern0pt}verts\ \isacommand{by}\isamarkupfalse%
\ blast\isanewline
\ \ \ \ \ \ \isacommand{obtain}\isamarkupfalse%
\ xs\ ys\ \isakeyword{where}\ p{\isacharunderscore}{\kern0pt}split{\isacharcolon}{\kern0pt}\ {\isachardoublequoteopen}p\ {\isacharequal}{\kern0pt}\ xs\ {\isacharat}{\kern0pt}\ u\ {\isacharhash}{\kern0pt}\ v\ {\isacharhash}{\kern0pt}\ ys\ {\isasymor}\ p\ {\isacharequal}{\kern0pt}\ xs\ {\isacharat}{\kern0pt}\ v\ {\isacharhash}{\kern0pt}\ u\ {\isacharhash}{\kern0pt}\ ys{\isachardoublequoteclose}\ \isacommand{using}\isamarkupfalse%
\ split{\isacharunderscore}{\kern0pt}walk{\isacharunderscore}{\kern0pt}edge{\isacharbrackleft}{\kern0pt}OF\ walk{\isacharunderscore}{\kern0pt}edge{\isacharbrackright}{\kern0pt}\ \isacommand{by}\isamarkupfalse%
\ blast\isanewline
\ \ \ \ \ \ \isacommand{have}\isamarkupfalse%
\ v{\isacharunderscore}{\kern0pt}notin{\isacharunderscore}{\kern0pt}ys{\isacharcolon}{\kern0pt}\ {\isachardoublequoteopen}v\ {\isasymnotin}\ set\ ys{\isachardoublequoteclose}\ \isacommand{using}\isamarkupfalse%
\ distinct\ p{\isacharunderscore}{\kern0pt}split\ \isacommand{by}\isamarkupfalse%
\ auto\isanewline
\ \ \ \ \ \ \isacommand{have}\isamarkupfalse%
\ {\isachardoublequoteopen}last\ p\ {\isacharequal}{\kern0pt}\ v{\isachardoublequoteclose}\ \isacommand{using}\isamarkupfalse%
\ conn{\isacharunderscore}{\kern0pt}path\ \isacommand{unfolding}\isamarkupfalse%
\ connecting{\isacharunderscore}{\kern0pt}path{\isacharunderscore}{\kern0pt}def\ \isacommand{by}\isamarkupfalse%
\ simp\isanewline
\ \ \ \ \ \ \isacommand{then}\isamarkupfalse%
\ \isacommand{have}\isamarkupfalse%
\ p{\isacharcolon}{\kern0pt}\ {\isachardoublequoteopen}p\ {\isacharequal}{\kern0pt}\ {\isacharparenleft}{\kern0pt}xs{\isacharat}{\kern0pt}{\isacharbrackleft}{\kern0pt}u{\isacharbrackright}{\kern0pt}{\isacharparenright}{\kern0pt}\ {\isacharat}{\kern0pt}\ {\isacharbrackleft}{\kern0pt}v{\isacharbrackright}{\kern0pt}{\isachardoublequoteclose}\ \isacommand{using}\isamarkupfalse%
\ v{\isacharunderscore}{\kern0pt}notin{\isacharunderscore}{\kern0pt}ys\ p{\isacharunderscore}{\kern0pt}split\ last{\isacharunderscore}{\kern0pt}in{\isacharunderscore}{\kern0pt}set\ last{\isacharunderscore}{\kern0pt}appendR\isanewline
\ \ \ \ \ \ \ \ \isacommand{by}\isamarkupfalse%
\ {\isacharparenleft}{\kern0pt}metis\ append{\isachardot}{\kern0pt}assoc\ append{\isacharunderscore}{\kern0pt}Cons\ last{\isachardot}{\kern0pt}simps\ list{\isachardot}{\kern0pt}discI\ self{\isacharunderscore}{\kern0pt}append{\isacharunderscore}{\kern0pt}conv{\isadigit{2}}{\isacharparenright}{\kern0pt}\isanewline
\ \ \ \ \ \ \isacommand{then}\isamarkupfalse%
\ \isacommand{have}\isamarkupfalse%
\ conn{\isacharunderscore}{\kern0pt}path{\isacharunderscore}{\kern0pt}u{\isacharcolon}{\kern0pt}\ {\isachardoublequoteopen}connecting{\isacharunderscore}{\kern0pt}path\ w\ u\ {\isacharparenleft}{\kern0pt}xs{\isacharat}{\kern0pt}{\isacharbrackleft}{\kern0pt}u{\isacharbrackright}{\kern0pt}{\isacharparenright}{\kern0pt}{\isachardoublequoteclose}\ \isacommand{using}\isamarkupfalse%
\ connecting{\isacharunderscore}{\kern0pt}path{\isacharunderscore}{\kern0pt}append\ conn{\isacharunderscore}{\kern0pt}path\ \isacommand{by}\isamarkupfalse%
\ fastforce\isanewline
\ \ \ \ \ \ \isacommand{have}\isamarkupfalse%
\ {\isachardoublequoteopen}v\ {\isasymnotin}\ set\ {\isacharparenleft}{\kern0pt}xs{\isacharat}{\kern0pt}{\isacharbrackleft}{\kern0pt}u{\isacharbrackright}{\kern0pt}{\isacharparenright}{\kern0pt}{\isachardoublequoteclose}\ \isacommand{using}\isamarkupfalse%
\ p\ distinct\ \isacommand{by}\isamarkupfalse%
\ auto\isanewline
\ \ \ \ \ \ \isacommand{then}\isamarkupfalse%
\ \isacommand{have}\isamarkupfalse%
\ {\isachardoublequoteopen}{\isacharbraceleft}{\kern0pt}u{\isacharcomma}{\kern0pt}v{\isacharbraceright}{\kern0pt}\ {\isasymnotin}\ set\ {\isacharparenleft}{\kern0pt}walk{\isacharunderscore}{\kern0pt}edges\ {\isacharparenleft}{\kern0pt}xs{\isacharat}{\kern0pt}{\isacharbrackleft}{\kern0pt}u{\isacharbrackright}{\kern0pt}{\isacharparenright}{\kern0pt}{\isacharparenright}{\kern0pt}{\isachardoublequoteclose}\ \isacommand{using}\isamarkupfalse%
\ walk{\isacharunderscore}{\kern0pt}edges{\isacharunderscore}{\kern0pt}in{\isacharunderscore}{\kern0pt}verts\ \isacommand{by}\isamarkupfalse%
\ blast\isanewline
\ \ \ \ \ \ \isacommand{then}\isamarkupfalse%
\ \isacommand{have}\isamarkupfalse%
\ {\isachardoublequoteopen}g{\isacharprime}{\kern0pt}{\isachardot}{\kern0pt}connecting{\isacharunderscore}{\kern0pt}path\ w\ u\ {\isacharparenleft}{\kern0pt}xs{\isacharat}{\kern0pt}{\isacharbrackleft}{\kern0pt}u{\isacharbrackright}{\kern0pt}{\isacharparenright}{\kern0pt}{\isachardoublequoteclose}\ \isacommand{using}\isamarkupfalse%
\ conn{\isacharunderscore}{\kern0pt}path{\isacharunderscore}{\kern0pt}u\isanewline
\ \ \ \ \ \ \ \ \isacommand{unfolding}\isamarkupfalse%
\ g{\isacharprime}{\kern0pt}{\isachardot}{\kern0pt}connecting{\isacharunderscore}{\kern0pt}path{\isacharunderscore}{\kern0pt}def\ connecting{\isacharunderscore}{\kern0pt}path{\isacharunderscore}{\kern0pt}def\ g{\isacharprime}{\kern0pt}{\isachardot}{\kern0pt}is{\isacharunderscore}{\kern0pt}gen{\isacharunderscore}{\kern0pt}path{\isacharunderscore}{\kern0pt}def\ is{\isacharunderscore}{\kern0pt}gen{\isacharunderscore}{\kern0pt}path{\isacharunderscore}{\kern0pt}def\ g{\isacharprime}{\kern0pt}{\isachardot}{\kern0pt}is{\isacharunderscore}{\kern0pt}walk{\isacharunderscore}{\kern0pt}def\ is{\isacharunderscore}{\kern0pt}walk{\isacharunderscore}{\kern0pt}def\ \isacommand{by}\isamarkupfalse%
\ blast\isanewline
\ \ \ \ \ \ \isacommand{then}\isamarkupfalse%
\ \isacommand{show}\isamarkupfalse%
\ {\isacharquery}{\kern0pt}thesis\ \isacommand{unfolding}\isamarkupfalse%
\ g{\isacharprime}{\kern0pt}{\isachardot}{\kern0pt}vert{\isacharunderscore}{\kern0pt}connected{\isacharunderscore}{\kern0pt}def\ \isacommand{by}\isamarkupfalse%
\ blast\isanewline
\ \ \ \ \isacommand{qed}\isamarkupfalse%
\isanewline
\ \ \isacommand{next}\isamarkupfalse%
\isanewline
\ \ \ \ \isacommand{case}\isamarkupfalse%
\ False\isanewline
\ \ \ \ \isacommand{then}\isamarkupfalse%
\ \isacommand{have}\isamarkupfalse%
\ {\isachardoublequoteopen}g{\isacharprime}{\kern0pt}{\isachardot}{\kern0pt}connecting{\isacharunderscore}{\kern0pt}path\ w\ v\ p{\isachardoublequoteclose}\ \isacommand{using}\isamarkupfalse%
\ conn{\isacharunderscore}{\kern0pt}path\isanewline
\ \ \ \ \ \ \isacommand{unfolding}\isamarkupfalse%
\ g{\isacharprime}{\kern0pt}{\isachardot}{\kern0pt}connecting{\isacharunderscore}{\kern0pt}path{\isacharunderscore}{\kern0pt}def\ connecting{\isacharunderscore}{\kern0pt}path{\isacharunderscore}{\kern0pt}def\ g{\isacharprime}{\kern0pt}{\isachardot}{\kern0pt}is{\isacharunderscore}{\kern0pt}gen{\isacharunderscore}{\kern0pt}path{\isacharunderscore}{\kern0pt}def\ is{\isacharunderscore}{\kern0pt}gen{\isacharunderscore}{\kern0pt}path{\isacharunderscore}{\kern0pt}def\ g{\isacharprime}{\kern0pt}{\isachardot}{\kern0pt}is{\isacharunderscore}{\kern0pt}walk{\isacharunderscore}{\kern0pt}def\ is{\isacharunderscore}{\kern0pt}walk{\isacharunderscore}{\kern0pt}def\ \isacommand{by}\isamarkupfalse%
\ blast\isanewline
\ \ \ \ \isacommand{then}\isamarkupfalse%
\ \isacommand{show}\isamarkupfalse%
\ {\isacharquery}{\kern0pt}thesis\ \isacommand{unfolding}\isamarkupfalse%
\ g{\isacharprime}{\kern0pt}{\isachardot}{\kern0pt}vert{\isacharunderscore}{\kern0pt}connected{\isacharunderscore}{\kern0pt}def\ \isacommand{by}\isamarkupfalse%
\ blast\isanewline
\ \ \isacommand{qed}\isamarkupfalse%
\isanewline
\isacommand{qed}\isamarkupfalse%
%
\endisatagproof
{\isafoldproof}%
%
\isadelimproof
\isanewline
%
\endisadelimproof
\isanewline
\isacommand{lemma}\isamarkupfalse%
\ {\isacharparenleft}{\kern0pt}\isakeyword{in}\ ulgraph{\isacharparenright}{\kern0pt}\ vert{\isacharunderscore}{\kern0pt}connected{\isacharunderscore}{\kern0pt}remove{\isacharunderscore}{\kern0pt}cycle{\isacharunderscore}{\kern0pt}edge{\isacharcolon}{\kern0pt}\isanewline
\ \ \isakeyword{assumes}\ cycle{\isacharcolon}{\kern0pt}\ {\isachardoublequoteopen}is{\isacharunderscore}{\kern0pt}cycle{\isadigit{2}}\ {\isacharparenleft}{\kern0pt}u{\isacharhash}{\kern0pt}v{\isacharhash}{\kern0pt}xs{\isacharparenright}{\kern0pt}{\isachardoublequoteclose}\isanewline
\ \ \ \ \isakeyword{shows}\ {\isachardoublequoteopen}ulgraph{\isachardot}{\kern0pt}vert{\isacharunderscore}{\kern0pt}connected\ V\ {\isacharparenleft}{\kern0pt}E\ {\isacharminus}{\kern0pt}\ {\isacharbraceleft}{\kern0pt}{\isacharbraceleft}{\kern0pt}u{\isacharcomma}{\kern0pt}v{\isacharbraceright}{\kern0pt}{\isacharbraceright}{\kern0pt}{\isacharparenright}{\kern0pt}\ u\ v{\isachardoublequoteclose}\isanewline
%
\isadelimproof
%
\endisadelimproof
%
\isatagproof
\isacommand{proof}\isamarkupfalse%
{\isacharminus}{\kern0pt}\isanewline
\ \ \isacommand{interpret}\isamarkupfalse%
\ g{\isacharprime}{\kern0pt}{\isacharcolon}{\kern0pt}\ ulgraph\ V\ {\isachardoublequoteopen}E\ {\isacharminus}{\kern0pt}\ {\isacharbraceleft}{\kern0pt}{\isacharbraceleft}{\kern0pt}u{\isacharcomma}{\kern0pt}v{\isacharbraceright}{\kern0pt}{\isacharbraceright}{\kern0pt}{\isachardoublequoteclose}\ \isacommand{using}\isamarkupfalse%
\ wellformed\ edge{\isacharunderscore}{\kern0pt}size\ \isacommand{by}\isamarkupfalse%
\ {\isacharparenleft}{\kern0pt}unfold{\isacharunderscore}{\kern0pt}locales{\isacharcomma}{\kern0pt}\ auto{\isacharparenright}{\kern0pt}\isanewline
\ \ \isacommand{have}\isamarkupfalse%
\ conn{\isacharunderscore}{\kern0pt}path{\isacharcolon}{\kern0pt}\ {\isachardoublequoteopen}connecting{\isacharunderscore}{\kern0pt}path\ v\ u\ {\isacharparenleft}{\kern0pt}v{\isacharhash}{\kern0pt}xs{\isacharparenright}{\kern0pt}{\isachardoublequoteclose}\ \isacommand{using}\isamarkupfalse%
\ cycle\ is{\isacharunderscore}{\kern0pt}cycle{\isacharunderscore}{\kern0pt}connecting{\isacharunderscore}{\kern0pt}path\ \isacommand{unfolding}\isamarkupfalse%
\ is{\isacharunderscore}{\kern0pt}cycle{\isadigit{2}}{\isacharunderscore}{\kern0pt}def\ \isacommand{by}\isamarkupfalse%
\ blast\isanewline
\ \ \isacommand{have}\isamarkupfalse%
\ {\isachardoublequoteopen}{\isacharbraceleft}{\kern0pt}u{\isacharcomma}{\kern0pt}v{\isacharbraceright}{\kern0pt}\ {\isasymnotin}\ set\ {\isacharparenleft}{\kern0pt}walk{\isacharunderscore}{\kern0pt}edges\ {\isacharparenleft}{\kern0pt}v{\isacharhash}{\kern0pt}xs{\isacharparenright}{\kern0pt}{\isacharparenright}{\kern0pt}{\isachardoublequoteclose}\ \isacommand{using}\isamarkupfalse%
\ cycle\ \isacommand{unfolding}\isamarkupfalse%
\ is{\isacharunderscore}{\kern0pt}cycle{\isadigit{2}}{\isacharunderscore}{\kern0pt}def\ \isacommand{by}\isamarkupfalse%
\ simp\isanewline
\ \ \isacommand{then}\isamarkupfalse%
\ \isacommand{have}\isamarkupfalse%
\ {\isachardoublequoteopen}g{\isacharprime}{\kern0pt}{\isachardot}{\kern0pt}connecting{\isacharunderscore}{\kern0pt}path\ v\ u\ {\isacharparenleft}{\kern0pt}v{\isacharhash}{\kern0pt}xs{\isacharparenright}{\kern0pt}{\isachardoublequoteclose}\ \isacommand{using}\isamarkupfalse%
\ conn{\isacharunderscore}{\kern0pt}path\isanewline
\ \ \ \ \isacommand{unfolding}\isamarkupfalse%
\ g{\isacharprime}{\kern0pt}{\isachardot}{\kern0pt}connecting{\isacharunderscore}{\kern0pt}path{\isacharunderscore}{\kern0pt}def\ connecting{\isacharunderscore}{\kern0pt}path{\isacharunderscore}{\kern0pt}def\ g{\isacharprime}{\kern0pt}{\isachardot}{\kern0pt}is{\isacharunderscore}{\kern0pt}gen{\isacharunderscore}{\kern0pt}path{\isacharunderscore}{\kern0pt}def\ is{\isacharunderscore}{\kern0pt}gen{\isacharunderscore}{\kern0pt}path{\isacharunderscore}{\kern0pt}def\ g{\isacharprime}{\kern0pt}{\isachardot}{\kern0pt}is{\isacharunderscore}{\kern0pt}walk{\isacharunderscore}{\kern0pt}def\ is{\isacharunderscore}{\kern0pt}walk{\isacharunderscore}{\kern0pt}def\ \isacommand{by}\isamarkupfalse%
\ blast\isanewline
\ \ \isacommand{then}\isamarkupfalse%
\ \isacommand{show}\isamarkupfalse%
\ {\isacharquery}{\kern0pt}thesis\ \isacommand{using}\isamarkupfalse%
\ g{\isacharprime}{\kern0pt}{\isachardot}{\kern0pt}vert{\isacharunderscore}{\kern0pt}connected{\isacharunderscore}{\kern0pt}rev\ \isacommand{unfolding}\isamarkupfalse%
\ g{\isacharprime}{\kern0pt}{\isachardot}{\kern0pt}vert{\isacharunderscore}{\kern0pt}connected{\isacharunderscore}{\kern0pt}def\ \isacommand{by}\isamarkupfalse%
\ blast\isanewline
\isacommand{qed}\isamarkupfalse%
%
\endisatagproof
{\isafoldproof}%
%
\isadelimproof
\isanewline
%
\endisadelimproof
\isanewline
\isacommand{lemma}\isamarkupfalse%
\ {\isacharparenleft}{\kern0pt}\isakeyword{in}\ connected{\isacharunderscore}{\kern0pt}ulgraph{\isacharparenright}{\kern0pt}\ connected{\isacharunderscore}{\kern0pt}remove{\isacharunderscore}{\kern0pt}cycle{\isacharunderscore}{\kern0pt}edges{\isacharcolon}{\kern0pt}\isanewline
\ \ \isakeyword{assumes}\ cycle{\isacharcolon}{\kern0pt}\ {\isachardoublequoteopen}is{\isacharunderscore}{\kern0pt}cycle{\isadigit{2}}\ {\isacharparenleft}{\kern0pt}u{\isacharhash}{\kern0pt}v{\isacharhash}{\kern0pt}xs{\isacharparenright}{\kern0pt}{\isachardoublequoteclose}\isanewline
\ \ \isakeyword{shows}\ {\isachardoublequoteopen}connected{\isacharunderscore}{\kern0pt}ulgraph\ V\ {\isacharparenleft}{\kern0pt}E\ {\isacharminus}{\kern0pt}\ {\isacharbraceleft}{\kern0pt}{\isacharbraceleft}{\kern0pt}u{\isacharcomma}{\kern0pt}v{\isacharbraceright}{\kern0pt}{\isacharbraceright}{\kern0pt}{\isacharparenright}{\kern0pt}{\isachardoublequoteclose}\isanewline
%
\isadelimproof
%
\endisadelimproof
%
\isatagproof
\isacommand{proof}\isamarkupfalse%
{\isacharminus}{\kern0pt}\isanewline
\ \ \isacommand{interpret}\isamarkupfalse%
\ g{\isacharprime}{\kern0pt}{\isacharcolon}{\kern0pt}\ ulgraph\ V\ {\isachardoublequoteopen}E\ {\isacharminus}{\kern0pt}\ {\isacharbraceleft}{\kern0pt}{\isacharbraceleft}{\kern0pt}u{\isacharcomma}{\kern0pt}v{\isacharbraceright}{\kern0pt}{\isacharbraceright}{\kern0pt}{\isachardoublequoteclose}\ \isacommand{using}\isamarkupfalse%
\ wellformed\ edge{\isacharunderscore}{\kern0pt}size\ \isacommand{by}\isamarkupfalse%
\ {\isacharparenleft}{\kern0pt}unfold{\isacharunderscore}{\kern0pt}locales{\isacharcomma}{\kern0pt}\ auto{\isacharparenright}{\kern0pt}\isanewline
\ \ \isacommand{have}\isamarkupfalse%
\ {\isachardoublequoteopen}g{\isacharprime}{\kern0pt}{\isachardot}{\kern0pt}vert{\isacharunderscore}{\kern0pt}connected\ x\ y{\isachardoublequoteclose}\ \isakeyword{if}\ inV{\isacharcolon}{\kern0pt}\ {\isachardoublequoteopen}x\ {\isasymin}\ V{\isachardoublequoteclose}\ {\isachardoublequoteopen}y\ {\isasymin}\ V{\isachardoublequoteclose}\ \isakeyword{for}\ x\ y\isanewline
\ \ \isacommand{proof}\isamarkupfalse%
{\isacharminus}{\kern0pt}\isanewline
\ \ \ \ \isacommand{have}\isamarkupfalse%
\ e{\isacharcolon}{\kern0pt}\ {\isachardoublequoteopen}{\isacharbraceleft}{\kern0pt}u{\isacharcomma}{\kern0pt}v{\isacharbraceright}{\kern0pt}\ {\isasymin}\ E{\isachardoublequoteclose}\ \isacommand{using}\isamarkupfalse%
\ cycle\ \isacommand{unfolding}\isamarkupfalse%
\ is{\isacharunderscore}{\kern0pt}cycle{\isadigit{2}}{\isacharunderscore}{\kern0pt}def\ is{\isacharunderscore}{\kern0pt}cycle{\isacharunderscore}{\kern0pt}alt\ is{\isacharunderscore}{\kern0pt}walk{\isacharunderscore}{\kern0pt}def\ \isacommand{by}\isamarkupfalse%
\ auto\isanewline
\ \ \ \ \isacommand{show}\isamarkupfalse%
\ {\isacharquery}{\kern0pt}thesis\ \isacommand{using}\isamarkupfalse%
\ vert{\isacharunderscore}{\kern0pt}connected{\isacharunderscore}{\kern0pt}remove{\isacharunderscore}{\kern0pt}cycle{\isacharunderscore}{\kern0pt}edge{\isacharbrackleft}{\kern0pt}OF\ cycle{\isacharbrackright}{\kern0pt}\ vert{\isacharunderscore}{\kern0pt}connected{\isacharunderscore}{\kern0pt}remove{\isacharunderscore}{\kern0pt}edge{\isacharbrackleft}{\kern0pt}OF\ e{\isacharbrackright}{\kern0pt}\ g{\isacharprime}{\kern0pt}{\isachardot}{\kern0pt}vert{\isacharunderscore}{\kern0pt}connected{\isacharunderscore}{\kern0pt}trans\ g{\isacharprime}{\kern0pt}{\isachardot}{\kern0pt}vert{\isacharunderscore}{\kern0pt}connected{\isacharunderscore}{\kern0pt}rev\ inV\ \isacommand{by}\isamarkupfalse%
\ metis\isanewline
\ \ \isacommand{qed}\isamarkupfalse%
\isanewline
\ \ \isacommand{then}\isamarkupfalse%
\ \isacommand{show}\isamarkupfalse%
\ {\isacharquery}{\kern0pt}thesis\ \isacommand{using}\isamarkupfalse%
\ not{\isacharunderscore}{\kern0pt}empty\ \isacommand{by}\isamarkupfalse%
\ {\isacharparenleft}{\kern0pt}unfold{\isacharunderscore}{\kern0pt}locales{\isacharcomma}{\kern0pt}\ auto\ simp{\isacharcolon}{\kern0pt}\ g{\isacharprime}{\kern0pt}{\isachardot}{\kern0pt}is{\isacharunderscore}{\kern0pt}connected{\isacharunderscore}{\kern0pt}set{\isacharunderscore}{\kern0pt}def{\isacharparenright}{\kern0pt}\isanewline
\isacommand{qed}\isamarkupfalse%
%
\endisatagproof
{\isafoldproof}%
%
\isadelimproof
\isanewline
%
\endisadelimproof
\isanewline
\isacommand{lemma}\isamarkupfalse%
\ {\isacharparenleft}{\kern0pt}\isakeyword{in}\ connected{\isacharunderscore}{\kern0pt}ulgraph{\isacharparenright}{\kern0pt}\ connected{\isacharunderscore}{\kern0pt}remove{\isacharunderscore}{\kern0pt}leaf{\isacharcolon}{\kern0pt}\isanewline
\ \ \isakeyword{assumes}\ degree{\isacharcolon}{\kern0pt}\ {\isachardoublequoteopen}degree\ l\ {\isacharequal}{\kern0pt}\ {\isadigit{1}}{\isachardoublequoteclose}\isanewline
\ \ \ \ \isakeyword{and}\ remove{\isacharunderscore}{\kern0pt}vertex{\isacharcolon}{\kern0pt}\ {\isachardoublequoteopen}remove{\isacharunderscore}{\kern0pt}vertex\ l\ {\isacharequal}{\kern0pt}\ {\isacharparenleft}{\kern0pt}V{\isacharprime}{\kern0pt}{\isacharcomma}{\kern0pt}\ E{\isacharprime}{\kern0pt}{\isacharparenright}{\kern0pt}{\isachardoublequoteclose}\isanewline
\ \ \isakeyword{shows}\ {\isachardoublequoteopen}ulgraph{\isachardot}{\kern0pt}is{\isacharunderscore}{\kern0pt}connected{\isacharunderscore}{\kern0pt}set\ V{\isacharprime}{\kern0pt}\ E{\isacharprime}{\kern0pt}\ V{\isacharprime}{\kern0pt}{\isachardoublequoteclose}\isanewline
%
\isadelimproof
%
\endisadelimproof
%
\isatagproof
\isacommand{proof}\isamarkupfalse%
{\isacharminus}{\kern0pt}\isanewline
\ \ \isacommand{interpret}\isamarkupfalse%
\ g{\isacharprime}{\kern0pt}{\isacharcolon}{\kern0pt}\ ulgraph\ V{\isacharprime}{\kern0pt}\ E{\isacharprime}{\kern0pt}\ \isacommand{using}\isamarkupfalse%
\ remove{\isacharunderscore}{\kern0pt}vertex\ wellformed\ edge{\isacharunderscore}{\kern0pt}size\isanewline
\ \ \ \ \isacommand{unfolding}\isamarkupfalse%
\ remove{\isacharunderscore}{\kern0pt}vertex{\isacharunderscore}{\kern0pt}def\ incident{\isacharunderscore}{\kern0pt}def\ \isacommand{by}\isamarkupfalse%
\ {\isacharparenleft}{\kern0pt}unfold{\isacharunderscore}{\kern0pt}locales{\isacharcomma}{\kern0pt}\ auto{\isacharparenright}{\kern0pt}\isanewline
\ \ \isacommand{have}\isamarkupfalse%
\ V{\isacharprime}{\kern0pt}{\isacharcolon}{\kern0pt}\ {\isachardoublequoteopen}V{\isacharprime}{\kern0pt}\ {\isacharequal}{\kern0pt}\ V\ {\isacharminus}{\kern0pt}\ {\isacharbraceleft}{\kern0pt}l{\isacharbraceright}{\kern0pt}{\isachardoublequoteclose}\ \isacommand{using}\isamarkupfalse%
\ remove{\isacharunderscore}{\kern0pt}vertex\ \isacommand{unfolding}\isamarkupfalse%
\ remove{\isacharunderscore}{\kern0pt}vertex{\isacharunderscore}{\kern0pt}def\ \isacommand{by}\isamarkupfalse%
\ simp\isanewline
\ \ \isacommand{have}\isamarkupfalse%
\ E{\isacharprime}{\kern0pt}{\isacharcolon}{\kern0pt}\ {\isachardoublequoteopen}E{\isacharprime}{\kern0pt}\ {\isacharequal}{\kern0pt}\ {\isacharbraceleft}{\kern0pt}e{\isasymin}E{\isachardot}{\kern0pt}\ l\ {\isasymnotin}\ e{\isacharbraceright}{\kern0pt}{\isachardoublequoteclose}\ \isacommand{using}\isamarkupfalse%
\ remove{\isacharunderscore}{\kern0pt}vertex\ \isacommand{unfolding}\isamarkupfalse%
\ remove{\isacharunderscore}{\kern0pt}vertex{\isacharunderscore}{\kern0pt}def\ incident{\isacharunderscore}{\kern0pt}def\ \isacommand{by}\isamarkupfalse%
\ simp\isanewline
\ \ \isacommand{have}\isamarkupfalse%
\ {\isachardoublequoteopen}u\ {\isasymin}\ V{\isacharprime}{\kern0pt}\ {\isasymLongrightarrow}\ v\ {\isasymin}\ V{\isacharprime}{\kern0pt}\ {\isasymLongrightarrow}\ g{\isacharprime}{\kern0pt}{\isachardot}{\kern0pt}vert{\isacharunderscore}{\kern0pt}connected\ u\ v{\isachardoublequoteclose}\ \isakeyword{for}\ u\ v\isanewline
\ \ \isacommand{proof}\isamarkupfalse%
{\isacharminus}{\kern0pt}\isanewline
\ \ \ \ \isacommand{assume}\isamarkupfalse%
\ inV{\isacharprime}{\kern0pt}{\isacharcolon}{\kern0pt}\ {\isachardoublequoteopen}u\ {\isasymin}\ V{\isacharprime}{\kern0pt}{\isachardoublequoteclose}\ {\isachardoublequoteopen}v\ {\isasymin}\ V{\isacharprime}{\kern0pt}{\isachardoublequoteclose}\isanewline
\ \ \ \ \isacommand{then}\isamarkupfalse%
\ \isacommand{have}\isamarkupfalse%
\ inV{\isacharcolon}{\kern0pt}\ {\isachardoublequoteopen}u\ {\isasymin}\ V{\isachardoublequoteclose}\ {\isachardoublequoteopen}v\ {\isasymin}\ V{\isachardoublequoteclose}\ \isacommand{using}\isamarkupfalse%
\ remove{\isacharunderscore}{\kern0pt}vertex\ \isacommand{unfolding}\isamarkupfalse%
\ remove{\isacharunderscore}{\kern0pt}vertex{\isacharunderscore}{\kern0pt}def\ \isacommand{by}\isamarkupfalse%
\ auto\isanewline
\ \ \ \ \isacommand{then}\isamarkupfalse%
\ \isacommand{obtain}\isamarkupfalse%
\ p\ \isakeyword{where}\ conn{\isacharunderscore}{\kern0pt}path{\isacharcolon}{\kern0pt}\ {\isachardoublequoteopen}connecting{\isacharunderscore}{\kern0pt}path\ u\ v\ p{\isachardoublequoteclose}\ \isacommand{using}\isamarkupfalse%
\ vertices{\isacharunderscore}{\kern0pt}connected{\isacharunderscore}{\kern0pt}path\ \isacommand{by}\isamarkupfalse%
\ blast\isanewline
\ \ \ \ \isacommand{show}\isamarkupfalse%
\ {\isacharquery}{\kern0pt}thesis\isanewline
\ \ \ \ \isacommand{proof}\isamarkupfalse%
\ {\isacharparenleft}{\kern0pt}cases\ {\isachardoublequoteopen}u\ {\isacharequal}{\kern0pt}\ v{\isachardoublequoteclose}{\isacharparenright}{\kern0pt}\isanewline
\ \ \ \ \ \ \isacommand{case}\isamarkupfalse%
\ True\isanewline
\ \ \ \ \ \ \isacommand{then}\isamarkupfalse%
\ \isacommand{show}\isamarkupfalse%
\ {\isacharquery}{\kern0pt}thesis\ \isacommand{using}\isamarkupfalse%
\ g{\isacharprime}{\kern0pt}{\isachardot}{\kern0pt}vert{\isacharunderscore}{\kern0pt}connected{\isacharunderscore}{\kern0pt}id\ inV{\isacharprime}{\kern0pt}\ \isacommand{by}\isamarkupfalse%
\ simp\isanewline
\ \ \ \ \isacommand{next}\isamarkupfalse%
\isanewline
\ \ \ \ \ \ \isacommand{case}\isamarkupfalse%
\ False\isanewline
\ \ \ \ \ \ \isacommand{then}\isamarkupfalse%
\ \isacommand{have}\isamarkupfalse%
\ distinct{\isacharcolon}{\kern0pt}\ {\isachardoublequoteopen}distinct\ p{\isachardoublequoteclose}\ \isacommand{using}\isamarkupfalse%
\ conn{\isacharunderscore}{\kern0pt}path\ \isacommand{unfolding}\isamarkupfalse%
\ connecting{\isacharunderscore}{\kern0pt}path{\isacharunderscore}{\kern0pt}def\ is{\isacharunderscore}{\kern0pt}gen{\isacharunderscore}{\kern0pt}path{\isacharunderscore}{\kern0pt}def\ \isacommand{by}\isamarkupfalse%
\ blast\isanewline
\ \ \ \ \ \ \isacommand{have}\isamarkupfalse%
\ l{\isacharunderscore}{\kern0pt}notin{\isacharunderscore}{\kern0pt}p{\isacharcolon}{\kern0pt}\ {\isachardoublequoteopen}l\ {\isasymnotin}\ set\ p{\isachardoublequoteclose}\isanewline
\ \ \ \ \ \ \isacommand{proof}\isamarkupfalse%
\isanewline
\ \ \ \ \ \ \ \ \isacommand{assume}\isamarkupfalse%
\ l{\isacharunderscore}{\kern0pt}in{\isacharunderscore}{\kern0pt}p{\isacharcolon}{\kern0pt}\ {\isachardoublequoteopen}l\ {\isasymin}\ set\ p{\isachardoublequoteclose}\isanewline
\ \ \ \ \ \ \ \ \isacommand{then}\isamarkupfalse%
\ \isacommand{obtain}\isamarkupfalse%
\ xs\ ys\ \isakeyword{where}\ p{\isacharcolon}{\kern0pt}\ {\isachardoublequoteopen}p\ {\isacharequal}{\kern0pt}\ xs\ {\isacharat}{\kern0pt}\ l\ {\isacharhash}{\kern0pt}\ ys{\isachardoublequoteclose}\ \isacommand{by}\isamarkupfalse%
\ {\isacharparenleft}{\kern0pt}meson\ split{\isacharunderscore}{\kern0pt}list{\isacharparenright}{\kern0pt}\isanewline
\ \ \ \ \ \ \ \ \isacommand{have}\isamarkupfalse%
\ {\isachardoublequoteopen}l\ {\isasymnoteq}\ u{\isachardoublequoteclose}\ {\isachardoublequoteopen}l\ {\isasymnoteq}\ v{\isachardoublequoteclose}\ \isacommand{using}\isamarkupfalse%
\ inV{\isacharprime}{\kern0pt}\ remove{\isacharunderscore}{\kern0pt}vertex\ \isacommand{unfolding}\isamarkupfalse%
\ remove{\isacharunderscore}{\kern0pt}vertex{\isacharunderscore}{\kern0pt}def\ \isacommand{by}\isamarkupfalse%
\ auto\isanewline
\ \ \ \ \ \ \ \ \isacommand{then}\isamarkupfalse%
\ \isacommand{have}\isamarkupfalse%
\ {\isachardoublequoteopen}xs\ {\isasymnoteq}\ {\isacharbrackleft}{\kern0pt}{\isacharbrackright}{\kern0pt}{\isachardoublequoteclose}\ \isacommand{using}\isamarkupfalse%
\ p\ conn{\isacharunderscore}{\kern0pt}path\ \isacommand{unfolding}\isamarkupfalse%
\ connecting{\isacharunderscore}{\kern0pt}path{\isacharunderscore}{\kern0pt}def\ \isacommand{by}\isamarkupfalse%
\ fastforce\isanewline
\ \ \ \ \ \ \ \ \isacommand{then}\isamarkupfalse%
\ \isacommand{obtain}\isamarkupfalse%
\ xs{\isacharprime}{\kern0pt}\ x\ \isakeyword{where}\ xs{\isacharcolon}{\kern0pt}\ {\isachardoublequoteopen}xs\ {\isacharequal}{\kern0pt}\ xs{\isacharprime}{\kern0pt}{\isacharat}{\kern0pt}{\isacharbrackleft}{\kern0pt}x{\isacharbrackright}{\kern0pt}{\isachardoublequoteclose}\ \isacommand{by}\isamarkupfalse%
\ {\isacharparenleft}{\kern0pt}meson\ rev{\isacharunderscore}{\kern0pt}exhaust{\isacharparenright}{\kern0pt}\isanewline
\ \ \ \ \ \ \ \ \isacommand{then}\isamarkupfalse%
\ \isacommand{have}\isamarkupfalse%
\ {\isachardoublequoteopen}x\ {\isasymnoteq}\ l{\isachardoublequoteclose}\ \isacommand{using}\isamarkupfalse%
\ distinct\ p\ \isacommand{by}\isamarkupfalse%
\ simp\isanewline
\ \ \ \ \ \ \ \ \isacommand{have}\isamarkupfalse%
\ {\isachardoublequoteopen}{\isacharbraceleft}{\kern0pt}x{\isacharcomma}{\kern0pt}l{\isacharbraceright}{\kern0pt}\ {\isasymin}\ set\ {\isacharparenleft}{\kern0pt}walk{\isacharunderscore}{\kern0pt}edges\ p{\isacharparenright}{\kern0pt}{\isachardoublequoteclose}\ \isacommand{using}\isamarkupfalse%
\ conn{\isacharunderscore}{\kern0pt}path\ walk{\isacharunderscore}{\kern0pt}edges{\isacharunderscore}{\kern0pt}append{\isacharunderscore}{\kern0pt}union\ p\ xs\isanewline
\ \ \ \ \ \ \ \ \ \ \isacommand{by}\isamarkupfalse%
\ {\isacharparenleft}{\kern0pt}smt\ {\isacharparenleft}{\kern0pt}verit{\isacharparenright}{\kern0pt}\ Un{\isacharunderscore}{\kern0pt}insert{\isacharunderscore}{\kern0pt}right\ {\isacartoucheopen}xs\ {\isasymnoteq}\ {\isacharbrackleft}{\kern0pt}{\isacharbrackright}{\kern0pt}{\isacartoucheclose}\ comp{\isacharunderscore}{\kern0pt}sgraph{\isachardot}{\kern0pt}walk{\isacharunderscore}{\kern0pt}edges{\isacharunderscore}{\kern0pt}append{\isacharunderscore}{\kern0pt}union\ insert{\isacharunderscore}{\kern0pt}iff\isanewline
\ \ \ \ \ \ \ \ \ \ \ \ \ \ last{\isacharunderscore}{\kern0pt}snoc\ list{\isachardot}{\kern0pt}discI\ list{\isachardot}{\kern0pt}sel{\isacharparenleft}{\kern0pt}{\isadigit{1}}{\isacharparenright}{\kern0pt}{\isacharparenright}{\kern0pt}\isanewline
\ \ \ \ \ \ \ \ \isacommand{then}\isamarkupfalse%
\ \isacommand{have}\isamarkupfalse%
\ xl{\isacharunderscore}{\kern0pt}incident{\isacharcolon}{\kern0pt}\ {\isachardoublequoteopen}{\isacharbraceleft}{\kern0pt}x{\isacharcomma}{\kern0pt}l{\isacharbraceright}{\kern0pt}\ {\isasymin}\ incident{\isacharunderscore}{\kern0pt}sedges\ l{\isachardoublequoteclose}\ \isacommand{using}\isamarkupfalse%
\ conn{\isacharunderscore}{\kern0pt}path\ {\isacartoucheopen}x{\isasymnoteq}l{\isacartoucheclose}\isanewline
\ \ \ \ \ \ \ \ \ \ \isacommand{unfolding}\isamarkupfalse%
\ connecting{\isacharunderscore}{\kern0pt}path{\isacharunderscore}{\kern0pt}def\ is{\isacharunderscore}{\kern0pt}gen{\isacharunderscore}{\kern0pt}path{\isacharunderscore}{\kern0pt}def\ is{\isacharunderscore}{\kern0pt}walk{\isacharunderscore}{\kern0pt}def\ incident{\isacharunderscore}{\kern0pt}sedges{\isacharunderscore}{\kern0pt}def\ incident{\isacharunderscore}{\kern0pt}def\ \isacommand{by}\isamarkupfalse%
\ auto\isanewline
\isanewline
\ \ \ \ \ \ \ \ \isacommand{have}\isamarkupfalse%
\ {\isachardoublequoteopen}ys\ {\isasymnoteq}\ {\isacharbrackleft}{\kern0pt}{\isacharbrackright}{\kern0pt}{\isachardoublequoteclose}\ \isacommand{using}\isamarkupfalse%
\ {\isacartoucheopen}l{\isasymnoteq}v{\isacartoucheclose}\ p\ conn{\isacharunderscore}{\kern0pt}path\ \isacommand{unfolding}\isamarkupfalse%
\ connecting{\isacharunderscore}{\kern0pt}path{\isacharunderscore}{\kern0pt}def\ \isacommand{by}\isamarkupfalse%
\ fastforce\isanewline
\ \ \ \ \ \ \ \ \isacommand{then}\isamarkupfalse%
\ \isacommand{obtain}\isamarkupfalse%
\ y\ ys{\isacharprime}{\kern0pt}\ \isakeyword{where}\ ys{\isacharcolon}{\kern0pt}\ {\isachardoublequoteopen}ys\ {\isacharequal}{\kern0pt}\ y\ {\isacharhash}{\kern0pt}\ ys{\isacharprime}{\kern0pt}{\isachardoublequoteclose}\ \isacommand{by}\isamarkupfalse%
\ {\isacharparenleft}{\kern0pt}meson\ list{\isachardot}{\kern0pt}exhaust{\isacharparenright}{\kern0pt}\isanewline
\ \ \ \ \ \ \ \ \isacommand{then}\isamarkupfalse%
\ \isacommand{have}\isamarkupfalse%
\ {\isachardoublequoteopen}y\ {\isasymnoteq}\ l{\isachardoublequoteclose}\ \isacommand{using}\isamarkupfalse%
\ distinct\ p\ \isacommand{by}\isamarkupfalse%
\ auto\isanewline
\ \ \ \ \ \ \ \ \isacommand{then}\isamarkupfalse%
\ \isacommand{have}\isamarkupfalse%
\ {\isachardoublequoteopen}{\isacharbraceleft}{\kern0pt}y{\isacharcomma}{\kern0pt}l{\isacharbraceright}{\kern0pt}\ {\isasymin}\ set\ {\isacharparenleft}{\kern0pt}walk{\isacharunderscore}{\kern0pt}edges\ p{\isacharparenright}{\kern0pt}{\isachardoublequoteclose}\ \isacommand{using}\isamarkupfalse%
\ p\ ys\ conn{\isacharunderscore}{\kern0pt}path\ walk{\isacharunderscore}{\kern0pt}edges{\isacharunderscore}{\kern0pt}append{\isacharunderscore}{\kern0pt}ss{\isadigit{1}}\ \isacommand{by}\isamarkupfalse%
\ fastforce\isanewline
\ \ \ \ \ \ \ \ \isacommand{then}\isamarkupfalse%
\ \isacommand{have}\isamarkupfalse%
\ yl{\isacharunderscore}{\kern0pt}incident{\isacharcolon}{\kern0pt}\ {\isachardoublequoteopen}{\isacharbraceleft}{\kern0pt}y{\isacharcomma}{\kern0pt}l{\isacharbraceright}{\kern0pt}\ {\isasymin}\ incident{\isacharunderscore}{\kern0pt}sedges\ l{\isachardoublequoteclose}\ \isacommand{using}\isamarkupfalse%
\ conn{\isacharunderscore}{\kern0pt}path\ {\isacartoucheopen}y{\isasymnoteq}l{\isacartoucheclose}\isanewline
\ \ \ \ \ \ \ \ \ \ \isacommand{unfolding}\isamarkupfalse%
\ connecting{\isacharunderscore}{\kern0pt}path{\isacharunderscore}{\kern0pt}def\ is{\isacharunderscore}{\kern0pt}gen{\isacharunderscore}{\kern0pt}path{\isacharunderscore}{\kern0pt}def\ is{\isacharunderscore}{\kern0pt}walk{\isacharunderscore}{\kern0pt}def\ incident{\isacharunderscore}{\kern0pt}sedges{\isacharunderscore}{\kern0pt}def\ incident{\isacharunderscore}{\kern0pt}def\ \isacommand{by}\isamarkupfalse%
\ auto\isanewline
\isanewline
\ \ \ \ \ \ \ \ \isacommand{have}\isamarkupfalse%
\ card{\isacharunderscore}{\kern0pt}loops{\isacharcolon}{\kern0pt}\ {\isachardoublequoteopen}card\ {\isacharparenleft}{\kern0pt}incident{\isacharunderscore}{\kern0pt}loops\ l{\isacharparenright}{\kern0pt}\ {\isacharequal}{\kern0pt}\ {\isadigit{0}}{\isachardoublequoteclose}\ \isacommand{using}\isamarkupfalse%
\ degree\ \isacommand{unfolding}\isamarkupfalse%
\ degree{\isacharunderscore}{\kern0pt}def\ \isacommand{by}\isamarkupfalse%
\ auto\isanewline
\ \ \ \ \ \ \ \ \isacommand{have}\isamarkupfalse%
\ {\isachardoublequoteopen}x\ {\isasymnoteq}\ y{\isachardoublequoteclose}\ \isacommand{using}\isamarkupfalse%
\ distinct\ \isacommand{unfolding}\isamarkupfalse%
\ p\ xs\ ys\ \isacommand{by}\isamarkupfalse%
\ simp\isanewline
\ \ \ \ \ \ \ \ \isacommand{then}\isamarkupfalse%
\ \isacommand{have}\isamarkupfalse%
\ {\isachardoublequoteopen}{\isacharbraceleft}{\kern0pt}x{\isacharcomma}{\kern0pt}l{\isacharbraceright}{\kern0pt}\ {\isasymnoteq}\ {\isacharbraceleft}{\kern0pt}y{\isacharcomma}{\kern0pt}l{\isacharbraceright}{\kern0pt}{\isachardoublequoteclose}\ \isacommand{by}\isamarkupfalse%
\ {\isacharparenleft}{\kern0pt}metis\ doubleton{\isacharunderscore}{\kern0pt}eq{\isacharunderscore}{\kern0pt}iff{\isacharparenright}{\kern0pt}\isanewline
\ \ \ \ \ \ \ \ \isacommand{then}\isamarkupfalse%
\ \isacommand{have}\isamarkupfalse%
\ {\isachardoublequoteopen}card\ {\isacharparenleft}{\kern0pt}incident{\isacharunderscore}{\kern0pt}sedges\ l{\isacharparenright}{\kern0pt}\ {\isasymnoteq}\ {\isadigit{1}}{\isachardoublequoteclose}\ \isacommand{using}\isamarkupfalse%
\ xl{\isacharunderscore}{\kern0pt}incident\ yl{\isacharunderscore}{\kern0pt}incident\isanewline
\ \ \ \ \ \ \ \ \ \ \isacommand{by}\isamarkupfalse%
\ {\isacharparenleft}{\kern0pt}metis\ card{\isacharunderscore}{\kern0pt}{\isadigit{1}}{\isacharunderscore}{\kern0pt}singletonE\ singletonD{\isacharparenright}{\kern0pt}\isanewline
\ \ \ \ \ \ \ \ \isacommand{then}\isamarkupfalse%
\ \isacommand{have}\isamarkupfalse%
\ {\isachardoublequoteopen}degree\ l\ {\isasymnoteq}\ {\isadigit{1}}{\isachardoublequoteclose}\ \isacommand{using}\isamarkupfalse%
\ card{\isacharunderscore}{\kern0pt}loops\ \isacommand{unfolding}\isamarkupfalse%
\ degree{\isacharunderscore}{\kern0pt}def\ \isacommand{by}\isamarkupfalse%
\ simp\isanewline
\ \ \ \ \ \ \ \ \isacommand{then}\isamarkupfalse%
\ \isacommand{show}\isamarkupfalse%
\ False\ \isacommand{using}\isamarkupfalse%
\ degree\ \isacommand{{\isachardot}{\kern0pt}{\isachardot}{\kern0pt}}\isamarkupfalse%
\isanewline
\ \ \ \ \ \ \isacommand{qed}\isamarkupfalse%
\isanewline
\ \ \ \ \ \ \isacommand{then}\isamarkupfalse%
\ \isacommand{have}\isamarkupfalse%
\ {\isachardoublequoteopen}set\ {\isacharparenleft}{\kern0pt}walk{\isacharunderscore}{\kern0pt}edges\ p{\isacharparenright}{\kern0pt}\ {\isasymsubseteq}\ E{\isacharprime}{\kern0pt}{\isachardoublequoteclose}\ \isacommand{using}\isamarkupfalse%
\ walk{\isacharunderscore}{\kern0pt}edges{\isacharunderscore}{\kern0pt}in{\isacharunderscore}{\kern0pt}verts\ conn{\isacharunderscore}{\kern0pt}path\ E{\isacharprime}{\kern0pt}\ \isacommand{unfolding}\isamarkupfalse%
\ connecting{\isacharunderscore}{\kern0pt}path{\isacharunderscore}{\kern0pt}def\ is{\isacharunderscore}{\kern0pt}gen{\isacharunderscore}{\kern0pt}path{\isacharunderscore}{\kern0pt}def\ is{\isacharunderscore}{\kern0pt}walk{\isacharunderscore}{\kern0pt}def\ \isacommand{by}\isamarkupfalse%
\ blast\isanewline
\ \ \ \ \ \ \isacommand{then}\isamarkupfalse%
\ \isacommand{have}\isamarkupfalse%
\ {\isachardoublequoteopen}g{\isacharprime}{\kern0pt}{\isachardot}{\kern0pt}connecting{\isacharunderscore}{\kern0pt}path\ u\ v\ p{\isachardoublequoteclose}\ \isacommand{using}\isamarkupfalse%
\ conn{\isacharunderscore}{\kern0pt}path\ V{\isacharprime}{\kern0pt}\ l{\isacharunderscore}{\kern0pt}notin{\isacharunderscore}{\kern0pt}p\isanewline
\ \ \ \ \ \ \ \ \isacommand{unfolding}\isamarkupfalse%
\ g{\isacharprime}{\kern0pt}{\isachardot}{\kern0pt}connecting{\isacharunderscore}{\kern0pt}path{\isacharunderscore}{\kern0pt}def\ connecting{\isacharunderscore}{\kern0pt}path{\isacharunderscore}{\kern0pt}def\ g{\isacharprime}{\kern0pt}{\isachardot}{\kern0pt}is{\isacharunderscore}{\kern0pt}gen{\isacharunderscore}{\kern0pt}path{\isacharunderscore}{\kern0pt}def\ is{\isacharunderscore}{\kern0pt}gen{\isacharunderscore}{\kern0pt}path{\isacharunderscore}{\kern0pt}def\ g{\isacharprime}{\kern0pt}{\isachardot}{\kern0pt}is{\isacharunderscore}{\kern0pt}walk{\isacharunderscore}{\kern0pt}def\ is{\isacharunderscore}{\kern0pt}walk{\isacharunderscore}{\kern0pt}def\ \isacommand{by}\isamarkupfalse%
\ blast\isanewline
\ \ \ \ \ \ \isacommand{then}\isamarkupfalse%
\ \isacommand{show}\isamarkupfalse%
\ {\isacharquery}{\kern0pt}thesis\ \isacommand{unfolding}\isamarkupfalse%
\ g{\isacharprime}{\kern0pt}{\isachardot}{\kern0pt}vert{\isacharunderscore}{\kern0pt}connected{\isacharunderscore}{\kern0pt}def\ \isacommand{by}\isamarkupfalse%
\ blast\isanewline
\ \ \ \ \isacommand{qed}\isamarkupfalse%
\isanewline
\ \ \isacommand{qed}\isamarkupfalse%
\isanewline
\ \ \isacommand{then}\isamarkupfalse%
\ \isacommand{show}\isamarkupfalse%
\ {\isacharquery}{\kern0pt}thesis\ \isacommand{unfolding}\isamarkupfalse%
\ g{\isacharprime}{\kern0pt}{\isachardot}{\kern0pt}is{\isacharunderscore}{\kern0pt}connected{\isacharunderscore}{\kern0pt}set{\isacharunderscore}{\kern0pt}def\ \isacommand{by}\isamarkupfalse%
\ blast\isanewline
\isacommand{qed}\isamarkupfalse%
%
\endisatagproof
{\isafoldproof}%
%
\isadelimproof
%
\endisadelimproof
%
\isadelimdocument
%
\endisadelimdocument
%
\isatagdocument
%
\isamarkupsubsection{Connected components%
}
\isamarkuptrue%
%
\endisatagdocument
{\isafolddocument}%
%
\isadelimdocument
%
\endisadelimdocument
\isacommand{context}\isamarkupfalse%
\ ulgraph\isanewline
\isakeyword{begin}\isanewline
\isanewline
\isacommand{abbreviation}\isamarkupfalse%
\ {\isachardoublequoteopen}vert{\isacharunderscore}{\kern0pt}connected{\isacharunderscore}{\kern0pt}rel\ {\isasymequiv}\ {\isacharbraceleft}{\kern0pt}{\isacharparenleft}{\kern0pt}u{\isacharcomma}{\kern0pt}v{\isacharparenright}{\kern0pt}{\isachardot}{\kern0pt}\ vert{\isacharunderscore}{\kern0pt}connected\ u\ v{\isacharbraceright}{\kern0pt}{\isachardoublequoteclose}\isanewline
\isanewline
\isacommand{definition}\isamarkupfalse%
\ connected{\isacharunderscore}{\kern0pt}components\ {\isacharcolon}{\kern0pt}{\isacharcolon}{\kern0pt}\ {\isachardoublequoteopen}{\isacharprime}{\kern0pt}a\ set\ set{\isachardoublequoteclose}\ \isakeyword{where}\isanewline
\ \ {\isachardoublequoteopen}connected{\isacharunderscore}{\kern0pt}components\ {\isacharequal}{\kern0pt}\ V\ {\isacharslash}{\kern0pt}{\isacharslash}{\kern0pt}\ vert{\isacharunderscore}{\kern0pt}connected{\isacharunderscore}{\kern0pt}rel{\isachardoublequoteclose}\isanewline
\isanewline
\isacommand{definition}\isamarkupfalse%
\ connected{\isacharunderscore}{\kern0pt}component{\isacharunderscore}{\kern0pt}of\ {\isacharcolon}{\kern0pt}{\isacharcolon}{\kern0pt}\ {\isachardoublequoteopen}{\isacharprime}{\kern0pt}a\ {\isasymRightarrow}\ {\isacharprime}{\kern0pt}a\ set{\isachardoublequoteclose}\ \isakeyword{where}\isanewline
\ \ {\isachardoublequoteopen}connected{\isacharunderscore}{\kern0pt}component{\isacharunderscore}{\kern0pt}of\ v\ {\isacharequal}{\kern0pt}\ vert{\isacharunderscore}{\kern0pt}connected{\isacharunderscore}{\kern0pt}rel\ {\isacharbackquote}{\kern0pt}{\isacharbackquote}{\kern0pt}\ {\isacharbraceleft}{\kern0pt}v{\isacharbraceright}{\kern0pt}{\isachardoublequoteclose}\isanewline
\isanewline
\isacommand{lemma}\isamarkupfalse%
\ vert{\isacharunderscore}{\kern0pt}connected{\isacharunderscore}{\kern0pt}rel{\isacharunderscore}{\kern0pt}on{\isacharunderscore}{\kern0pt}V{\isacharcolon}{\kern0pt}\ {\isachardoublequoteopen}vert{\isacharunderscore}{\kern0pt}connected{\isacharunderscore}{\kern0pt}rel\ {\isasymsubseteq}\ V\ {\isasymtimes}\ V{\isachardoublequoteclose}\isanewline
%
\isadelimproof
\ \ %
\endisadelimproof
%
\isatagproof
\isacommand{using}\isamarkupfalse%
\ vert{\isacharunderscore}{\kern0pt}connected{\isacharunderscore}{\kern0pt}wf\ \isacommand{by}\isamarkupfalse%
\ auto%
\endisatagproof
{\isafoldproof}%
%
\isadelimproof
\isanewline
%
\endisadelimproof
\isanewline
\isacommand{lemma}\isamarkupfalse%
\ vert{\isacharunderscore}{\kern0pt}connected{\isacharunderscore}{\kern0pt}rel{\isacharunderscore}{\kern0pt}refl{\isacharcolon}{\kern0pt}\ {\isachardoublequoteopen}refl{\isacharunderscore}{\kern0pt}on\ V\ vert{\isacharunderscore}{\kern0pt}connected{\isacharunderscore}{\kern0pt}rel{\isachardoublequoteclose}\isanewline
%
\isadelimproof
\ \ %
\endisadelimproof
%
\isatagproof
\isacommand{unfolding}\isamarkupfalse%
\ refl{\isacharunderscore}{\kern0pt}on{\isacharunderscore}{\kern0pt}def\ \isacommand{using}\isamarkupfalse%
\ vert{\isacharunderscore}{\kern0pt}connected{\isacharunderscore}{\kern0pt}rel{\isacharunderscore}{\kern0pt}on{\isacharunderscore}{\kern0pt}V\ vert{\isacharunderscore}{\kern0pt}connected{\isacharunderscore}{\kern0pt}id\ \isacommand{by}\isamarkupfalse%
\ simp%
\endisatagproof
{\isafoldproof}%
%
\isadelimproof
\isanewline
%
\endisadelimproof
\isanewline
\isacommand{lemma}\isamarkupfalse%
\ vert{\isacharunderscore}{\kern0pt}connected{\isacharunderscore}{\kern0pt}rel{\isacharunderscore}{\kern0pt}sym{\isacharcolon}{\kern0pt}\ {\isachardoublequoteopen}sym\ vert{\isacharunderscore}{\kern0pt}connected{\isacharunderscore}{\kern0pt}rel{\isachardoublequoteclose}\isanewline
%
\isadelimproof
\ \ %
\endisadelimproof
%
\isatagproof
\isacommand{unfolding}\isamarkupfalse%
\ sym{\isacharunderscore}{\kern0pt}def\ \isacommand{using}\isamarkupfalse%
\ vert{\isacharunderscore}{\kern0pt}connected{\isacharunderscore}{\kern0pt}rev\ \isacommand{by}\isamarkupfalse%
\ simp%
\endisatagproof
{\isafoldproof}%
%
\isadelimproof
\isanewline
%
\endisadelimproof
\isanewline
\isacommand{lemma}\isamarkupfalse%
\ vert{\isacharunderscore}{\kern0pt}connected{\isacharunderscore}{\kern0pt}rel{\isacharunderscore}{\kern0pt}trans{\isacharcolon}{\kern0pt}\ {\isachardoublequoteopen}trans\ vert{\isacharunderscore}{\kern0pt}connected{\isacharunderscore}{\kern0pt}rel{\isachardoublequoteclose}\isanewline
%
\isadelimproof
\ \ %
\endisadelimproof
%
\isatagproof
\isacommand{unfolding}\isamarkupfalse%
\ trans{\isacharunderscore}{\kern0pt}def\ \isacommand{using}\isamarkupfalse%
\ vert{\isacharunderscore}{\kern0pt}connected{\isacharunderscore}{\kern0pt}trans\ \isacommand{by}\isamarkupfalse%
\ blast%
\endisatagproof
{\isafoldproof}%
%
\isadelimproof
\isanewline
%
\endisadelimproof
\isanewline
\isacommand{lemma}\isamarkupfalse%
\ equiv{\isacharunderscore}{\kern0pt}vert{\isacharunderscore}{\kern0pt}connected{\isacharcolon}{\kern0pt}\ {\isachardoublequoteopen}equiv\ V\ vert{\isacharunderscore}{\kern0pt}connected{\isacharunderscore}{\kern0pt}rel{\isachardoublequoteclose}\isanewline
%
\isadelimproof
\ \ %
\endisadelimproof
%
\isatagproof
\isacommand{unfolding}\isamarkupfalse%
\ equiv{\isacharunderscore}{\kern0pt}def\ \isacommand{using}\isamarkupfalse%
\ vert{\isacharunderscore}{\kern0pt}connected{\isacharunderscore}{\kern0pt}rel{\isacharunderscore}{\kern0pt}refl\ vert{\isacharunderscore}{\kern0pt}connected{\isacharunderscore}{\kern0pt}rel{\isacharunderscore}{\kern0pt}sym\ vert{\isacharunderscore}{\kern0pt}connected{\isacharunderscore}{\kern0pt}rel{\isacharunderscore}{\kern0pt}trans\ \isacommand{by}\isamarkupfalse%
\ blast%
\endisatagproof
{\isafoldproof}%
%
\isadelimproof
\isanewline
%
\endisadelimproof
\isanewline
\isacommand{lemma}\isamarkupfalse%
\ connected{\isacharunderscore}{\kern0pt}component{\isacharunderscore}{\kern0pt}non{\isacharunderscore}{\kern0pt}empty{\isacharcolon}{\kern0pt}\ {\isachardoublequoteopen}V{\isacharprime}{\kern0pt}\ {\isasymin}\ connected{\isacharunderscore}{\kern0pt}components\ {\isasymLongrightarrow}\ V{\isacharprime}{\kern0pt}\ {\isasymnoteq}\ {\isacharbraceleft}{\kern0pt}{\isacharbraceright}{\kern0pt}{\isachardoublequoteclose}\isanewline
%
\isadelimproof
\ \ %
\endisadelimproof
%
\isatagproof
\isacommand{unfolding}\isamarkupfalse%
\ connected{\isacharunderscore}{\kern0pt}components{\isacharunderscore}{\kern0pt}def\ \isacommand{using}\isamarkupfalse%
\ equiv{\isacharunderscore}{\kern0pt}vert{\isacharunderscore}{\kern0pt}connected\ in{\isacharunderscore}{\kern0pt}quotient{\isacharunderscore}{\kern0pt}imp{\isacharunderscore}{\kern0pt}non{\isacharunderscore}{\kern0pt}empty\ \isacommand{by}\isamarkupfalse%
\ auto%
\endisatagproof
{\isafoldproof}%
%
\isadelimproof
\isanewline
%
\endisadelimproof
\isanewline
\isacommand{lemma}\isamarkupfalse%
\ connected{\isacharunderscore}{\kern0pt}component{\isacharunderscore}{\kern0pt}connected{\isacharcolon}{\kern0pt}\ {\isachardoublequoteopen}V{\isacharprime}{\kern0pt}\ {\isasymin}\ connected{\isacharunderscore}{\kern0pt}components\ {\isasymLongrightarrow}\ is{\isacharunderscore}{\kern0pt}connected{\isacharunderscore}{\kern0pt}set\ V{\isacharprime}{\kern0pt}{\isachardoublequoteclose}\isanewline
%
\isadelimproof
\ \ %
\endisadelimproof
%
\isatagproof
\isacommand{unfolding}\isamarkupfalse%
\ connected{\isacharunderscore}{\kern0pt}components{\isacharunderscore}{\kern0pt}def\ is{\isacharunderscore}{\kern0pt}connected{\isacharunderscore}{\kern0pt}set{\isacharunderscore}{\kern0pt}def\ \isacommand{using}\isamarkupfalse%
\ quotient{\isacharunderscore}{\kern0pt}eq{\isacharunderscore}{\kern0pt}iff{\isacharbrackleft}{\kern0pt}OF\ equiv{\isacharunderscore}{\kern0pt}vert{\isacharunderscore}{\kern0pt}connected{\isacharcomma}{\kern0pt}\ of\ V{\isacharprime}{\kern0pt}\ V{\isacharprime}{\kern0pt}{\isacharbrackright}{\kern0pt}\ \isacommand{by}\isamarkupfalse%
\ simp%
\endisatagproof
{\isafoldproof}%
%
\isadelimproof
\isanewline
%
\endisadelimproof
\isanewline
\isacommand{lemma}\isamarkupfalse%
\ connected{\isacharunderscore}{\kern0pt}component{\isacharunderscore}{\kern0pt}wf{\isacharcolon}{\kern0pt}\ {\isachardoublequoteopen}V{\isacharprime}{\kern0pt}\ {\isasymin}\ connected{\isacharunderscore}{\kern0pt}components\ {\isasymLongrightarrow}\ V{\isacharprime}{\kern0pt}\ {\isasymsubseteq}\ V{\isachardoublequoteclose}\isanewline
%
\isadelimproof
\ \ %
\endisadelimproof
%
\isatagproof
\isacommand{by}\isamarkupfalse%
\ {\isacharparenleft}{\kern0pt}simp\ add{\isacharcolon}{\kern0pt}\ connected{\isacharunderscore}{\kern0pt}component{\isacharunderscore}{\kern0pt}connected\ is{\isacharunderscore}{\kern0pt}connected{\isacharunderscore}{\kern0pt}set{\isacharunderscore}{\kern0pt}wf{\isacharparenright}{\kern0pt}%
\endisatagproof
{\isafoldproof}%
%
\isadelimproof
\isanewline
%
\endisadelimproof
\isanewline
\isacommand{lemma}\isamarkupfalse%
\ connected{\isacharunderscore}{\kern0pt}component{\isacharunderscore}{\kern0pt}of{\isacharunderscore}{\kern0pt}self{\isacharcolon}{\kern0pt}\ {\isachardoublequoteopen}v\ {\isasymin}\ V\ {\isasymLongrightarrow}\ v\ {\isasymin}\ connected{\isacharunderscore}{\kern0pt}component{\isacharunderscore}{\kern0pt}of\ v{\isachardoublequoteclose}\isanewline
%
\isadelimproof
\ \ %
\endisadelimproof
%
\isatagproof
\isacommand{unfolding}\isamarkupfalse%
\ connected{\isacharunderscore}{\kern0pt}component{\isacharunderscore}{\kern0pt}of{\isacharunderscore}{\kern0pt}def\ \isacommand{using}\isamarkupfalse%
\ vert{\isacharunderscore}{\kern0pt}connected{\isacharunderscore}{\kern0pt}id\ \isacommand{by}\isamarkupfalse%
\ blast%
\endisatagproof
{\isafoldproof}%
%
\isadelimproof
\isanewline
%
\endisadelimproof
\isanewline
\isacommand{lemma}\isamarkupfalse%
\ conn{\isacharunderscore}{\kern0pt}comp{\isacharunderscore}{\kern0pt}of{\isacharunderscore}{\kern0pt}conn{\isacharunderscore}{\kern0pt}comps{\isacharcolon}{\kern0pt}\ {\isachardoublequoteopen}v\ {\isasymin}\ V\ {\isasymLongrightarrow}\ connected{\isacharunderscore}{\kern0pt}component{\isacharunderscore}{\kern0pt}of\ v\ {\isasymin}\ connected{\isacharunderscore}{\kern0pt}components{\isachardoublequoteclose}\isanewline
%
\isadelimproof
\ \ %
\endisadelimproof
%
\isatagproof
\isacommand{unfolding}\isamarkupfalse%
\ connected{\isacharunderscore}{\kern0pt}components{\isacharunderscore}{\kern0pt}def\ quotient{\isacharunderscore}{\kern0pt}def\ connected{\isacharunderscore}{\kern0pt}component{\isacharunderscore}{\kern0pt}of{\isacharunderscore}{\kern0pt}def\ \isacommand{by}\isamarkupfalse%
\ blast%
\endisatagproof
{\isafoldproof}%
%
\isadelimproof
\isanewline
%
\endisadelimproof
\isanewline
\isacommand{lemma}\isamarkupfalse%
\ Un{\isacharunderscore}{\kern0pt}connected{\isacharunderscore}{\kern0pt}components{\isacharcolon}{\kern0pt}\ {\isachardoublequoteopen}connected{\isacharunderscore}{\kern0pt}components\ {\isacharequal}{\kern0pt}\ connected{\isacharunderscore}{\kern0pt}component{\isacharunderscore}{\kern0pt}of\ {\isacharbackquote}{\kern0pt}\ V{\isachardoublequoteclose}\isanewline
%
\isadelimproof
\ \ %
\endisadelimproof
%
\isatagproof
\isacommand{unfolding}\isamarkupfalse%
\ connected{\isacharunderscore}{\kern0pt}components{\isacharunderscore}{\kern0pt}def\ connected{\isacharunderscore}{\kern0pt}component{\isacharunderscore}{\kern0pt}of{\isacharunderscore}{\kern0pt}def\ quotient{\isacharunderscore}{\kern0pt}def\ \isacommand{by}\isamarkupfalse%
\ blast%
\endisatagproof
{\isafoldproof}%
%
\isadelimproof
\isanewline
%
\endisadelimproof
\isanewline
\isacommand{lemma}\isamarkupfalse%
\ connected{\isacharunderscore}{\kern0pt}component{\isacharunderscore}{\kern0pt}subgraph{\isacharcolon}{\kern0pt}\ {\isachardoublequoteopen}V{\isacharprime}{\kern0pt}\ {\isasymin}\ connected{\isacharunderscore}{\kern0pt}components\ {\isasymLongrightarrow}\ subgraph\ V{\isacharprime}{\kern0pt}\ {\isacharparenleft}{\kern0pt}induced{\isacharunderscore}{\kern0pt}edges\ V{\isacharprime}{\kern0pt}{\isacharparenright}{\kern0pt}\ V\ E{\isachardoublequoteclose}\isanewline
%
\isadelimproof
\ \ %
\endisadelimproof
%
\isatagproof
\isacommand{using}\isamarkupfalse%
\ induced{\isacharunderscore}{\kern0pt}is{\isacharunderscore}{\kern0pt}subgraph\ connected{\isacharunderscore}{\kern0pt}component{\isacharunderscore}{\kern0pt}wf\ \isacommand{by}\isamarkupfalse%
\ simp%
\endisatagproof
{\isafoldproof}%
%
\isadelimproof
\isanewline
%
\endisadelimproof
\isanewline
\isacommand{lemma}\isamarkupfalse%
\ connected{\isacharunderscore}{\kern0pt}components{\isacharunderscore}{\kern0pt}connected{\isadigit{2}}{\isacharcolon}{\kern0pt}\isanewline
\ \ \isakeyword{assumes}\ conn{\isacharunderscore}{\kern0pt}comp{\isacharcolon}{\kern0pt}\ {\isachardoublequoteopen}V{\isacharprime}{\kern0pt}\ {\isasymin}\ connected{\isacharunderscore}{\kern0pt}components{\isachardoublequoteclose}\isanewline
\ \ \isakeyword{shows}\ {\isachardoublequoteopen}ulgraph{\isachardot}{\kern0pt}is{\isacharunderscore}{\kern0pt}connected{\isacharunderscore}{\kern0pt}set\ V{\isacharprime}{\kern0pt}\ {\isacharparenleft}{\kern0pt}induced{\isacharunderscore}{\kern0pt}edges\ V{\isacharprime}{\kern0pt}{\isacharparenright}{\kern0pt}\ V{\isacharprime}{\kern0pt}{\isachardoublequoteclose}\isanewline
%
\isadelimproof
%
\endisadelimproof
%
\isatagproof
\isacommand{proof}\isamarkupfalse%
{\isacharminus}{\kern0pt}\isanewline
\ \ \isacommand{interpret}\isamarkupfalse%
\ subg{\isacharcolon}{\kern0pt}\ subgraph\ V{\isacharprime}{\kern0pt}\ {\isachardoublequoteopen}induced{\isacharunderscore}{\kern0pt}edges\ V{\isacharprime}{\kern0pt}{\isachardoublequoteclose}\ V\ E\ \isacommand{using}\isamarkupfalse%
\ connected{\isacharunderscore}{\kern0pt}component{\isacharunderscore}{\kern0pt}subgraph\ conn{\isacharunderscore}{\kern0pt}comp\ \isacommand{by}\isamarkupfalse%
\ simp\isanewline
\ \ \isacommand{interpret}\isamarkupfalse%
\ g{\isacharprime}{\kern0pt}{\isacharcolon}{\kern0pt}\ ulgraph\ V{\isacharprime}{\kern0pt}\ {\isachardoublequoteopen}induced{\isacharunderscore}{\kern0pt}edges\ V{\isacharprime}{\kern0pt}{\isachardoublequoteclose}\ \isacommand{using}\isamarkupfalse%
\ subg{\isachardot}{\kern0pt}is{\isacharunderscore}{\kern0pt}subgraph{\isacharunderscore}{\kern0pt}ulgraph\ ulgraph{\isacharunderscore}{\kern0pt}axioms\ \isacommand{by}\isamarkupfalse%
\ simp\isanewline
\ \ \isacommand{have}\isamarkupfalse%
\ {\isachardoublequoteopen}{\isasymAnd}u\ v{\isachardot}{\kern0pt}\ u\ {\isasymin}\ V{\isacharprime}{\kern0pt}\ {\isasymLongrightarrow}\ v\ {\isasymin}\ V{\isacharprime}{\kern0pt}\ {\isasymLongrightarrow}\ g{\isacharprime}{\kern0pt}{\isachardot}{\kern0pt}vert{\isacharunderscore}{\kern0pt}connected\ u\ v{\isachardoublequoteclose}\isanewline
\ \ \isacommand{proof}\isamarkupfalse%
{\isacharminus}{\kern0pt}\isanewline
\ \ \ \ \isacommand{fix}\isamarkupfalse%
\ u\ v\ \isacommand{assume}\isamarkupfalse%
\ {\isachardoublequoteopen}u\ {\isasymin}\ V{\isacharprime}{\kern0pt}{\isachardoublequoteclose}\ {\isachardoublequoteopen}v\ {\isasymin}\ V{\isacharprime}{\kern0pt}{\isachardoublequoteclose}\isanewline
\ \ \ \ \isacommand{then}\isamarkupfalse%
\ \isacommand{obtain}\isamarkupfalse%
\ p\ \isakeyword{where}\ conn{\isacharunderscore}{\kern0pt}path{\isacharcolon}{\kern0pt}\ {\isachardoublequoteopen}connecting{\isacharunderscore}{\kern0pt}path\ u\ v\ p{\isachardoublequoteclose}\ \isacommand{using}\isamarkupfalse%
\ connected{\isacharunderscore}{\kern0pt}component{\isacharunderscore}{\kern0pt}connected\ conn{\isacharunderscore}{\kern0pt}comp\ \isacommand{unfolding}\isamarkupfalse%
\ is{\isacharunderscore}{\kern0pt}connected{\isacharunderscore}{\kern0pt}set{\isacharunderscore}{\kern0pt}def\ vert{\isacharunderscore}{\kern0pt}connected{\isacharunderscore}{\kern0pt}def\ \isacommand{by}\isamarkupfalse%
\ blast\isanewline
\ \ \ \ \isacommand{then}\isamarkupfalse%
\ \isacommand{have}\isamarkupfalse%
\ u{\isacharunderscore}{\kern0pt}in{\isacharunderscore}{\kern0pt}p{\isacharcolon}{\kern0pt}\ {\isachardoublequoteopen}u\ {\isasymin}\ set\ p{\isachardoublequoteclose}\ \isacommand{unfolding}\isamarkupfalse%
\ connecting{\isacharunderscore}{\kern0pt}path{\isacharunderscore}{\kern0pt}def\ is{\isacharunderscore}{\kern0pt}gen{\isacharunderscore}{\kern0pt}path{\isacharunderscore}{\kern0pt}def\ is{\isacharunderscore}{\kern0pt}walk{\isacharunderscore}{\kern0pt}def\ \isacommand{by}\isamarkupfalse%
\ force\isanewline
\ \ \ \ \isacommand{then}\isamarkupfalse%
\ \isacommand{have}\isamarkupfalse%
\ set{\isacharunderscore}{\kern0pt}p{\isacharcolon}{\kern0pt}\ {\isachardoublequoteopen}set\ p\ {\isasymsubseteq}\ V{\isacharprime}{\kern0pt}{\isachardoublequoteclose}\ \isacommand{using}\isamarkupfalse%
\ connecting{\isacharunderscore}{\kern0pt}path{\isacharunderscore}{\kern0pt}connected{\isacharunderscore}{\kern0pt}set{\isacharbrackleft}{\kern0pt}OF\ conn{\isacharunderscore}{\kern0pt}path{\isacharbrackright}{\kern0pt}\isanewline
\ \ \ \ \ \ \ \ in{\isacharunderscore}{\kern0pt}quotient{\isacharunderscore}{\kern0pt}imp{\isacharunderscore}{\kern0pt}closed{\isacharbrackleft}{\kern0pt}OF\ equiv{\isacharunderscore}{\kern0pt}vert{\isacharunderscore}{\kern0pt}connected{\isacharbrackright}{\kern0pt}\ conn{\isacharunderscore}{\kern0pt}comp\ {\isacartoucheopen}u\ {\isasymin}\ V{\isacharprime}{\kern0pt}{\isacartoucheclose}\isanewline
\ \ \ \ \ \ \isacommand{unfolding}\isamarkupfalse%
\ is{\isacharunderscore}{\kern0pt}connected{\isacharunderscore}{\kern0pt}set{\isacharunderscore}{\kern0pt}def\ connected{\isacharunderscore}{\kern0pt}components{\isacharunderscore}{\kern0pt}def\ \isacommand{by}\isamarkupfalse%
\ blast\isanewline
\ \ \ \ \isacommand{then}\isamarkupfalse%
\ \isacommand{have}\isamarkupfalse%
\ {\isachardoublequoteopen}set\ {\isacharparenleft}{\kern0pt}g{\isacharprime}{\kern0pt}{\isachardot}{\kern0pt}walk{\isacharunderscore}{\kern0pt}edges\ p{\isacharparenright}{\kern0pt}\ {\isasymsubseteq}\ induced{\isacharunderscore}{\kern0pt}edges\ V{\isacharprime}{\kern0pt}{\isachardoublequoteclose}\isanewline
\ \ \ \ \ \ \isacommand{using}\isamarkupfalse%
\ walk{\isacharunderscore}{\kern0pt}edges{\isacharunderscore}{\kern0pt}induced{\isacharunderscore}{\kern0pt}edges\ induced{\isacharunderscore}{\kern0pt}edges{\isacharunderscore}{\kern0pt}mono\ conn{\isacharunderscore}{\kern0pt}path\ \isacommand{unfolding}\isamarkupfalse%
\ connecting{\isacharunderscore}{\kern0pt}path{\isacharunderscore}{\kern0pt}def\ is{\isacharunderscore}{\kern0pt}gen{\isacharunderscore}{\kern0pt}path{\isacharunderscore}{\kern0pt}def\ \isacommand{by}\isamarkupfalse%
\ blast\isanewline
\ \ \ \ \isacommand{then}\isamarkupfalse%
\ \isacommand{have}\isamarkupfalse%
\ {\isachardoublequoteopen}g{\isacharprime}{\kern0pt}{\isachardot}{\kern0pt}connecting{\isacharunderscore}{\kern0pt}path\ u\ v\ p{\isachardoublequoteclose}\isanewline
\ \ \ \ \ \ \isacommand{using}\isamarkupfalse%
\ set{\isacharunderscore}{\kern0pt}p\ conn{\isacharunderscore}{\kern0pt}path\isanewline
\ \ \ \ \ \ \isacommand{unfolding}\isamarkupfalse%
\ g{\isacharprime}{\kern0pt}{\isachardot}{\kern0pt}connecting{\isacharunderscore}{\kern0pt}path{\isacharunderscore}{\kern0pt}def\ g{\isacharprime}{\kern0pt}{\isachardot}{\kern0pt}connecting{\isacharunderscore}{\kern0pt}path{\isacharunderscore}{\kern0pt}def\ g{\isacharprime}{\kern0pt}{\isachardot}{\kern0pt}is{\isacharunderscore}{\kern0pt}gen{\isacharunderscore}{\kern0pt}path{\isacharunderscore}{\kern0pt}def\ g{\isacharprime}{\kern0pt}{\isachardot}{\kern0pt}is{\isacharunderscore}{\kern0pt}walk{\isacharunderscore}{\kern0pt}def\isanewline
\ \ \ \ \ \ \isacommand{unfolding}\isamarkupfalse%
\ connecting{\isacharunderscore}{\kern0pt}path{\isacharunderscore}{\kern0pt}def\ connecting{\isacharunderscore}{\kern0pt}path{\isacharunderscore}{\kern0pt}def\ is{\isacharunderscore}{\kern0pt}gen{\isacharunderscore}{\kern0pt}path{\isacharunderscore}{\kern0pt}def\ is{\isacharunderscore}{\kern0pt}walk{\isacharunderscore}{\kern0pt}def\ \isacommand{by}\isamarkupfalse%
\ auto\isanewline
\ \ \ \ \isacommand{then}\isamarkupfalse%
\ \isacommand{show}\isamarkupfalse%
\ {\isachardoublequoteopen}g{\isacharprime}{\kern0pt}{\isachardot}{\kern0pt}vert{\isacharunderscore}{\kern0pt}connected\ u\ v{\isachardoublequoteclose}\ \isacommand{unfolding}\isamarkupfalse%
\ g{\isacharprime}{\kern0pt}{\isachardot}{\kern0pt}vert{\isacharunderscore}{\kern0pt}connected{\isacharunderscore}{\kern0pt}def\ \isacommand{by}\isamarkupfalse%
\ blast\isanewline
\ \ \isacommand{qed}\isamarkupfalse%
\isanewline
\ \ \isacommand{then}\isamarkupfalse%
\ \isacommand{show}\isamarkupfalse%
\ {\isacharquery}{\kern0pt}thesis\ \isacommand{unfolding}\isamarkupfalse%
\ g{\isacharprime}{\kern0pt}{\isachardot}{\kern0pt}is{\isacharunderscore}{\kern0pt}connected{\isacharunderscore}{\kern0pt}set{\isacharunderscore}{\kern0pt}def\ \isacommand{by}\isamarkupfalse%
\ blast\isanewline
\isacommand{qed}\isamarkupfalse%
%
\endisatagproof
{\isafoldproof}%
%
\isadelimproof
\isanewline
%
\endisadelimproof
\isanewline
\isacommand{lemma}\isamarkupfalse%
\ vert{\isacharunderscore}{\kern0pt}connected{\isacharunderscore}{\kern0pt}connected{\isacharunderscore}{\kern0pt}component{\isacharcolon}{\kern0pt}\ {\isachardoublequoteopen}C\ {\isasymin}\ connected{\isacharunderscore}{\kern0pt}components\ {\isasymLongrightarrow}\ u\ {\isasymin}\ C\ {\isasymLongrightarrow}\ vert{\isacharunderscore}{\kern0pt}connected\ u\ v\ {\isasymLongrightarrow}\ v\ {\isasymin}\ C{\isachardoublequoteclose}\isanewline
%
\isadelimproof
\ \ %
\endisadelimproof
%
\isatagproof
\isacommand{unfolding}\isamarkupfalse%
\ connected{\isacharunderscore}{\kern0pt}components{\isacharunderscore}{\kern0pt}def\ \isacommand{using}\isamarkupfalse%
\ equiv{\isacharunderscore}{\kern0pt}vert{\isacharunderscore}{\kern0pt}connected\ in{\isacharunderscore}{\kern0pt}quotient{\isacharunderscore}{\kern0pt}imp{\isacharunderscore}{\kern0pt}closed\ \isacommand{by}\isamarkupfalse%
\ fastforce%
\endisatagproof
{\isafoldproof}%
%
\isadelimproof
\isanewline
%
\endisadelimproof
\isanewline
\isacommand{lemma}\isamarkupfalse%
\ connected{\isacharunderscore}{\kern0pt}components{\isacharunderscore}{\kern0pt}connected{\isacharunderscore}{\kern0pt}ulgraphs{\isacharcolon}{\kern0pt}\isanewline
\ \ \isakeyword{assumes}\ conn{\isacharunderscore}{\kern0pt}comp{\isacharcolon}{\kern0pt}\ {\isachardoublequoteopen}V{\isacharprime}{\kern0pt}\ {\isasymin}\ connected{\isacharunderscore}{\kern0pt}components{\isachardoublequoteclose}\isanewline
\ \ \isakeyword{shows}\ {\isachardoublequoteopen}connected{\isacharunderscore}{\kern0pt}ulgraph\ V{\isacharprime}{\kern0pt}\ {\isacharparenleft}{\kern0pt}induced{\isacharunderscore}{\kern0pt}edges\ V{\isacharprime}{\kern0pt}{\isacharparenright}{\kern0pt}{\isachardoublequoteclose}\isanewline
%
\isadelimproof
%
\endisadelimproof
%
\isatagproof
\isacommand{proof}\isamarkupfalse%
{\isacharminus}{\kern0pt}\isanewline
\ \ \isacommand{interpret}\isamarkupfalse%
\ subg{\isacharcolon}{\kern0pt}\ subgraph\ V{\isacharprime}{\kern0pt}\ {\isachardoublequoteopen}induced{\isacharunderscore}{\kern0pt}edges\ V{\isacharprime}{\kern0pt}{\isachardoublequoteclose}\ V\ E\ \isacommand{using}\isamarkupfalse%
\ connected{\isacharunderscore}{\kern0pt}component{\isacharunderscore}{\kern0pt}subgraph\ conn{\isacharunderscore}{\kern0pt}comp\ \isacommand{by}\isamarkupfalse%
\ simp\isanewline
\ \ \isacommand{interpret}\isamarkupfalse%
\ g{\isacharprime}{\kern0pt}{\isacharcolon}{\kern0pt}\ ulgraph\ V{\isacharprime}{\kern0pt}\ {\isachardoublequoteopen}induced{\isacharunderscore}{\kern0pt}edges\ V{\isacharprime}{\kern0pt}{\isachardoublequoteclose}\ \isacommand{using}\isamarkupfalse%
\ subg{\isachardot}{\kern0pt}is{\isacharunderscore}{\kern0pt}subgraph{\isacharunderscore}{\kern0pt}ulgraph\ ulgraph{\isacharunderscore}{\kern0pt}axioms\ \isacommand{by}\isamarkupfalse%
\ simp\isanewline
\ \ \isacommand{show}\isamarkupfalse%
\ {\isacharquery}{\kern0pt}thesis\ \isacommand{using}\isamarkupfalse%
\ conn{\isacharunderscore}{\kern0pt}comp\ connected{\isacharunderscore}{\kern0pt}component{\isacharunderscore}{\kern0pt}non{\isacharunderscore}{\kern0pt}empty\ connected{\isacharunderscore}{\kern0pt}components{\isacharunderscore}{\kern0pt}connected{\isadigit{2}}\ \isacommand{by}\isamarkupfalse%
\ {\isacharparenleft}{\kern0pt}unfold{\isacharunderscore}{\kern0pt}locales{\isacharcomma}{\kern0pt}\ auto{\isacharparenright}{\kern0pt}\isanewline
\isacommand{qed}\isamarkupfalse%
%
\endisatagproof
{\isafoldproof}%
%
\isadelimproof
\isanewline
%
\endisadelimproof
\isanewline
\isacommand{lemma}\isamarkupfalse%
\ connected{\isacharunderscore}{\kern0pt}components{\isacharunderscore}{\kern0pt}partition{\isacharunderscore}{\kern0pt}on{\isacharunderscore}{\kern0pt}V{\isacharcolon}{\kern0pt}\ {\isachardoublequoteopen}partition{\isacharunderscore}{\kern0pt}on\ V\ connected{\isacharunderscore}{\kern0pt}components{\isachardoublequoteclose}\isanewline
%
\isadelimproof
\ \ %
\endisadelimproof
%
\isatagproof
\isacommand{using}\isamarkupfalse%
\ partition{\isacharunderscore}{\kern0pt}on{\isacharunderscore}{\kern0pt}quotient\ equiv{\isacharunderscore}{\kern0pt}vert{\isacharunderscore}{\kern0pt}connected\ \isacommand{unfolding}\isamarkupfalse%
\ connected{\isacharunderscore}{\kern0pt}components{\isacharunderscore}{\kern0pt}def\ \isacommand{by}\isamarkupfalse%
\ blast%
\endisatagproof
{\isafoldproof}%
%
\isadelimproof
\isanewline
%
\endisadelimproof
\isanewline
\isacommand{lemma}\isamarkupfalse%
\ Union{\isacharunderscore}{\kern0pt}connected{\isacharunderscore}{\kern0pt}components{\isacharcolon}{\kern0pt}\ {\isachardoublequoteopen}{\isasymUnion}connected{\isacharunderscore}{\kern0pt}components\ {\isacharequal}{\kern0pt}\ V{\isachardoublequoteclose}\isanewline
%
\isadelimproof
\ \ %
\endisadelimproof
%
\isatagproof
\isacommand{using}\isamarkupfalse%
\ connected{\isacharunderscore}{\kern0pt}components{\isacharunderscore}{\kern0pt}partition{\isacharunderscore}{\kern0pt}on{\isacharunderscore}{\kern0pt}V\ \isacommand{unfolding}\isamarkupfalse%
\ partition{\isacharunderscore}{\kern0pt}on{\isacharunderscore}{\kern0pt}def\ \isacommand{by}\isamarkupfalse%
\ blast%
\endisatagproof
{\isafoldproof}%
%
\isadelimproof
\isanewline
%
\endisadelimproof
\isanewline
\isacommand{lemma}\isamarkupfalse%
\ disjoint{\isacharunderscore}{\kern0pt}connected{\isacharunderscore}{\kern0pt}components{\isacharcolon}{\kern0pt}\ {\isachardoublequoteopen}disjoint\ connected{\isacharunderscore}{\kern0pt}components{\isachardoublequoteclose}\isanewline
%
\isadelimproof
\ \ %
\endisadelimproof
%
\isatagproof
\isacommand{using}\isamarkupfalse%
\ connected{\isacharunderscore}{\kern0pt}components{\isacharunderscore}{\kern0pt}partition{\isacharunderscore}{\kern0pt}on{\isacharunderscore}{\kern0pt}V\ \isacommand{unfolding}\isamarkupfalse%
\ partition{\isacharunderscore}{\kern0pt}on{\isacharunderscore}{\kern0pt}def\ \isacommand{by}\isamarkupfalse%
\ blast%
\endisatagproof
{\isafoldproof}%
%
\isadelimproof
\isanewline
%
\endisadelimproof
\isanewline
\isacommand{lemma}\isamarkupfalse%
\ Union{\isacharunderscore}{\kern0pt}induced{\isacharunderscore}{\kern0pt}edges{\isacharunderscore}{\kern0pt}connected{\isacharunderscore}{\kern0pt}components{\isacharcolon}{\kern0pt}\ {\isachardoublequoteopen}{\isasymUnion}{\isacharparenleft}{\kern0pt}induced{\isacharunderscore}{\kern0pt}edges\ {\isacharbackquote}{\kern0pt}\ connected{\isacharunderscore}{\kern0pt}components{\isacharparenright}{\kern0pt}\ {\isacharequal}{\kern0pt}\ E{\isachardoublequoteclose}\isanewline
%
\isadelimproof
%
\endisadelimproof
%
\isatagproof
\isacommand{proof}\isamarkupfalse%
{\isacharminus}{\kern0pt}\isanewline
\ \ \isacommand{have}\isamarkupfalse%
\ {\isachardoublequoteopen}{\isasymexists}C{\isasymin}connected{\isacharunderscore}{\kern0pt}components{\isachardot}{\kern0pt}\ e\ {\isasymin}\ induced{\isacharunderscore}{\kern0pt}edges\ C{\isachardoublequoteclose}\ \isakeyword{if}\ {\isachardoublequoteopen}e\ {\isasymin}\ E{\isachardoublequoteclose}\ \isakeyword{for}\ e\isanewline
\ \ \isacommand{proof}\isamarkupfalse%
{\isacharminus}{\kern0pt}\isanewline
\ \ \ \ \isacommand{obtain}\isamarkupfalse%
\ u\ v\ \isakeyword{where}\ e{\isacharcolon}{\kern0pt}\ {\isachardoublequoteopen}e\ {\isacharequal}{\kern0pt}\ {\isacharbraceleft}{\kern0pt}u{\isacharcomma}{\kern0pt}v{\isacharbraceright}{\kern0pt}{\isachardoublequoteclose}\ \isacommand{by}\isamarkupfalse%
\ {\isacharparenleft}{\kern0pt}meson\ {\isacartoucheopen}e\ {\isasymin}\ E{\isacartoucheclose}\ obtain{\isacharunderscore}{\kern0pt}edge{\isacharunderscore}{\kern0pt}pair{\isacharunderscore}{\kern0pt}adj{\isacharparenright}{\kern0pt}\isanewline
\ \ \ \ \isacommand{then}\isamarkupfalse%
\ \isacommand{have}\isamarkupfalse%
\ {\isachardoublequoteopen}vert{\isacharunderscore}{\kern0pt}connected\ u\ v{\isachardoublequoteclose}\ \isacommand{using}\isamarkupfalse%
\ that\ vert{\isacharunderscore}{\kern0pt}connected{\isacharunderscore}{\kern0pt}neighbors\ \isacommand{by}\isamarkupfalse%
\ blast\isanewline
\ \ \ \ \isacommand{then}\isamarkupfalse%
\ \isacommand{have}\isamarkupfalse%
\ {\isachardoublequoteopen}v\ {\isasymin}\ connected{\isacharunderscore}{\kern0pt}component{\isacharunderscore}{\kern0pt}of\ u{\isachardoublequoteclose}\ \isacommand{unfolding}\isamarkupfalse%
\ connected{\isacharunderscore}{\kern0pt}component{\isacharunderscore}{\kern0pt}of{\isacharunderscore}{\kern0pt}def\ \isacommand{by}\isamarkupfalse%
\ simp\isanewline
\ \ \ \ \isacommand{then}\isamarkupfalse%
\ \isacommand{have}\isamarkupfalse%
\ {\isachardoublequoteopen}e\ {\isasymin}\ induced{\isacharunderscore}{\kern0pt}edges\ {\isacharparenleft}{\kern0pt}connected{\isacharunderscore}{\kern0pt}component{\isacharunderscore}{\kern0pt}of\ u{\isacharparenright}{\kern0pt}{\isachardoublequoteclose}\ \isacommand{using}\isamarkupfalse%
\ connected{\isacharunderscore}{\kern0pt}component{\isacharunderscore}{\kern0pt}of{\isacharunderscore}{\kern0pt}self\ wellformed\ {\isacartoucheopen}e{\isasymin}E{\isacartoucheclose}\ \isacommand{unfolding}\isamarkupfalse%
\ e\ induced{\isacharunderscore}{\kern0pt}edges{\isacharunderscore}{\kern0pt}def\ \isacommand{by}\isamarkupfalse%
\ auto\isanewline
\ \ \ \ \isacommand{then}\isamarkupfalse%
\ \isacommand{show}\isamarkupfalse%
\ {\isacharquery}{\kern0pt}thesis\ \isacommand{using}\isamarkupfalse%
\ conn{\isacharunderscore}{\kern0pt}comp{\isacharunderscore}{\kern0pt}of{\isacharunderscore}{\kern0pt}conn{\isacharunderscore}{\kern0pt}comps\ e\ wellformed\ {\isacartoucheopen}e{\isasymin}E{\isacartoucheclose}\ \isacommand{by}\isamarkupfalse%
\ auto\isanewline
\ \ \isacommand{qed}\isamarkupfalse%
\ \isanewline
\ \ \isacommand{then}\isamarkupfalse%
\ \isacommand{show}\isamarkupfalse%
\ {\isacharquery}{\kern0pt}thesis\ \isacommand{using}\isamarkupfalse%
\ connected{\isacharunderscore}{\kern0pt}component{\isacharunderscore}{\kern0pt}wf\ induced{\isacharunderscore}{\kern0pt}edges{\isacharunderscore}{\kern0pt}ss\ \isacommand{by}\isamarkupfalse%
\ blast\isanewline
\isacommand{qed}\isamarkupfalse%
%
\endisatagproof
{\isafoldproof}%
%
\isadelimproof
\isanewline
%
\endisadelimproof
\isanewline
\isacommand{lemma}\isamarkupfalse%
\ connected{\isacharunderscore}{\kern0pt}components{\isacharunderscore}{\kern0pt}empty{\isacharunderscore}{\kern0pt}E{\isacharcolon}{\kern0pt}\isanewline
\ \ \isakeyword{assumes}\ empty{\isacharcolon}{\kern0pt}\ {\isachardoublequoteopen}E\ {\isacharequal}{\kern0pt}\ {\isacharbraceleft}{\kern0pt}{\isacharbraceright}{\kern0pt}{\isachardoublequoteclose}\isanewline
\ \ \isakeyword{shows}\ {\isachardoublequoteopen}connected{\isacharunderscore}{\kern0pt}components\ {\isacharequal}{\kern0pt}\ {\isacharbraceleft}{\kern0pt}{\isacharbraceleft}{\kern0pt}v{\isacharbraceright}{\kern0pt}\ {\isacharbar}{\kern0pt}\ v{\isachardot}{\kern0pt}\ v{\isasymin}V{\isacharbraceright}{\kern0pt}{\isachardoublequoteclose}\isanewline
%
\isadelimproof
%
\endisadelimproof
%
\isatagproof
\isacommand{proof}\isamarkupfalse%
{\isacharminus}{\kern0pt}\isanewline
\ \ \isacommand{have}\isamarkupfalse%
\ {\isachardoublequoteopen}{\isasymforall}v{\isasymin}V{\isachardot}{\kern0pt}\ vert{\isacharunderscore}{\kern0pt}connected{\isacharunderscore}{\kern0pt}rel{\isacharbackquote}{\kern0pt}{\isacharbackquote}{\kern0pt}{\isacharbraceleft}{\kern0pt}v{\isacharbraceright}{\kern0pt}\ {\isacharequal}{\kern0pt}\ {\isacharbraceleft}{\kern0pt}v{\isacharbraceright}{\kern0pt}{\isachardoublequoteclose}\ \isacommand{using}\isamarkupfalse%
\ vert{\isacharunderscore}{\kern0pt}connected{\isacharunderscore}{\kern0pt}id\ connected{\isacharunderscore}{\kern0pt}empty{\isacharunderscore}{\kern0pt}E\ empty\ \isacommand{by}\isamarkupfalse%
\ auto\isanewline
\ \ \isacommand{then}\isamarkupfalse%
\ \isacommand{show}\isamarkupfalse%
\ {\isacharquery}{\kern0pt}thesis\ \isacommand{unfolding}\isamarkupfalse%
\ connected{\isacharunderscore}{\kern0pt}components{\isacharunderscore}{\kern0pt}def\ quotient{\isacharunderscore}{\kern0pt}def\ \isacommand{by}\isamarkupfalse%
\ auto\isanewline
\isacommand{qed}\isamarkupfalse%
%
\endisatagproof
{\isafoldproof}%
%
\isadelimproof
\isanewline
%
\endisadelimproof
\isanewline
\isacommand{lemma}\isamarkupfalse%
\ connected{\isacharunderscore}{\kern0pt}iff{\isacharunderscore}{\kern0pt}connected{\isacharunderscore}{\kern0pt}components{\isacharcolon}{\kern0pt}\isanewline
\ \ \isakeyword{assumes}\ non{\isacharunderscore}{\kern0pt}empty{\isacharcolon}{\kern0pt}\ {\isachardoublequoteopen}V\ {\isasymnoteq}\ {\isacharbraceleft}{\kern0pt}{\isacharbraceright}{\kern0pt}{\isachardoublequoteclose}\isanewline
\ \ \ \ \isakeyword{shows}\ {\isachardoublequoteopen}is{\isacharunderscore}{\kern0pt}connected{\isacharunderscore}{\kern0pt}set\ V\ {\isasymlongleftrightarrow}\ connected{\isacharunderscore}{\kern0pt}components\ {\isacharequal}{\kern0pt}\ {\isacharbraceleft}{\kern0pt}V{\isacharbraceright}{\kern0pt}{\isachardoublequoteclose}\isanewline
%
\isadelimproof
%
\endisadelimproof
%
\isatagproof
\isacommand{proof}\isamarkupfalse%
\isanewline
\ \ \isacommand{assume}\isamarkupfalse%
\ {\isachardoublequoteopen}is{\isacharunderscore}{\kern0pt}connected{\isacharunderscore}{\kern0pt}set\ V{\isachardoublequoteclose}\isanewline
\ \ \isacommand{then}\isamarkupfalse%
\ \isacommand{have}\isamarkupfalse%
\ {\isachardoublequoteopen}{\isasymforall}v{\isasymin}V{\isachardot}{\kern0pt}\ connected{\isacharunderscore}{\kern0pt}component{\isacharunderscore}{\kern0pt}of\ v\ {\isacharequal}{\kern0pt}\ V{\isachardoublequoteclose}\ \isacommand{unfolding}\isamarkupfalse%
\ connected{\isacharunderscore}{\kern0pt}component{\isacharunderscore}{\kern0pt}of{\isacharunderscore}{\kern0pt}def\ is{\isacharunderscore}{\kern0pt}connected{\isacharunderscore}{\kern0pt}set{\isacharunderscore}{\kern0pt}def\ \isacommand{using}\isamarkupfalse%
\ vert{\isacharunderscore}{\kern0pt}connected{\isacharunderscore}{\kern0pt}wf\ \isacommand{by}\isamarkupfalse%
\ blast\isanewline
\ \ \isacommand{then}\isamarkupfalse%
\ \isacommand{show}\isamarkupfalse%
\ {\isachardoublequoteopen}connected{\isacharunderscore}{\kern0pt}components\ {\isacharequal}{\kern0pt}\ {\isacharbraceleft}{\kern0pt}V{\isacharbraceright}{\kern0pt}{\isachardoublequoteclose}\ \isacommand{unfolding}\isamarkupfalse%
\ quotient{\isacharunderscore}{\kern0pt}def\ connected{\isacharunderscore}{\kern0pt}component{\isacharunderscore}{\kern0pt}of{\isacharunderscore}{\kern0pt}def\ connected{\isacharunderscore}{\kern0pt}components{\isacharunderscore}{\kern0pt}def\ \isacommand{using}\isamarkupfalse%
\ non{\isacharunderscore}{\kern0pt}empty\ \isacommand{by}\isamarkupfalse%
\ auto\isanewline
\isacommand{next}\isamarkupfalse%
\isanewline
\ \ \isacommand{show}\isamarkupfalse%
\ {\isachardoublequoteopen}connected{\isacharunderscore}{\kern0pt}components\ {\isacharequal}{\kern0pt}\ {\isacharbraceleft}{\kern0pt}V{\isacharbraceright}{\kern0pt}\ {\isasymLongrightarrow}\ is{\isacharunderscore}{\kern0pt}connected{\isacharunderscore}{\kern0pt}set\ V{\isachardoublequoteclose}\isanewline
\ \ \ \ \isacommand{using}\isamarkupfalse%
\ connected{\isacharunderscore}{\kern0pt}component{\isacharunderscore}{\kern0pt}connected\ \isacommand{unfolding}\isamarkupfalse%
\ connected{\isacharunderscore}{\kern0pt}components{\isacharunderscore}{\kern0pt}def\ is{\isacharunderscore}{\kern0pt}connected{\isacharunderscore}{\kern0pt}set{\isacharunderscore}{\kern0pt}def\ \isacommand{by}\isamarkupfalse%
\ auto\isanewline
\isacommand{qed}\isamarkupfalse%
%
\endisatagproof
{\isafoldproof}%
%
\isadelimproof
\isanewline
%
\endisadelimproof
\isanewline
\isacommand{end}\isamarkupfalse%
\isanewline
\isanewline
\isacommand{lemma}\isamarkupfalse%
\ {\isacharparenleft}{\kern0pt}\isakeyword{in}\ connected{\isacharunderscore}{\kern0pt}ulgraph{\isacharparenright}{\kern0pt}\ connected{\isacharunderscore}{\kern0pt}components{\isacharbrackleft}{\kern0pt}simp{\isacharbrackright}{\kern0pt}{\isacharcolon}{\kern0pt}\ {\isachardoublequoteopen}connected{\isacharunderscore}{\kern0pt}components\ {\isacharequal}{\kern0pt}\ {\isacharbraceleft}{\kern0pt}V{\isacharbraceright}{\kern0pt}{\isachardoublequoteclose}\isanewline
%
\isadelimproof
\ \ %
\endisadelimproof
%
\isatagproof
\isacommand{using}\isamarkupfalse%
\ connected\ connected{\isacharunderscore}{\kern0pt}iff{\isacharunderscore}{\kern0pt}connected{\isacharunderscore}{\kern0pt}components\ not{\isacharunderscore}{\kern0pt}empty\ \isacommand{by}\isamarkupfalse%
\ simp%
\endisatagproof
{\isafoldproof}%
%
\isadelimproof
\isanewline
%
\endisadelimproof
\isanewline
\isacommand{lemma}\isamarkupfalse%
\ {\isacharparenleft}{\kern0pt}\isakeyword{in}\ fin{\isacharunderscore}{\kern0pt}ulgraph{\isacharparenright}{\kern0pt}\ finite{\isacharunderscore}{\kern0pt}connected{\isacharunderscore}{\kern0pt}components{\isacharcolon}{\kern0pt}\ {\isachardoublequoteopen}finite\ connected{\isacharunderscore}{\kern0pt}components{\isachardoublequoteclose}\isanewline
%
\isadelimproof
\ \ %
\endisadelimproof
%
\isatagproof
\isacommand{unfolding}\isamarkupfalse%
\ connected{\isacharunderscore}{\kern0pt}components{\isacharunderscore}{\kern0pt}def\ \isacommand{using}\isamarkupfalse%
\ finV\ vert{\isacharunderscore}{\kern0pt}connected{\isacharunderscore}{\kern0pt}rel{\isacharunderscore}{\kern0pt}on{\isacharunderscore}{\kern0pt}V\ finite{\isacharunderscore}{\kern0pt}quotient\ \isacommand{by}\isamarkupfalse%
\ blast%
\endisatagproof
{\isafoldproof}%
%
\isadelimproof
\isanewline
%
\endisadelimproof
\isanewline
\isacommand{lemma}\isamarkupfalse%
\ {\isacharparenleft}{\kern0pt}\isakeyword{in}\ fin{\isacharunderscore}{\kern0pt}ulgraph{\isacharparenright}{\kern0pt}\ finite{\isacharunderscore}{\kern0pt}connected{\isacharunderscore}{\kern0pt}component{\isacharcolon}{\kern0pt}\ {\isachardoublequoteopen}C\ {\isasymin}\ connected{\isacharunderscore}{\kern0pt}components\ {\isasymLongrightarrow}\ finite\ C{\isachardoublequoteclose}\isanewline
%
\isadelimproof
\ \ %
\endisadelimproof
%
\isatagproof
\isacommand{using}\isamarkupfalse%
\ connected{\isacharunderscore}{\kern0pt}component{\isacharunderscore}{\kern0pt}wf\ finV\ finite{\isacharunderscore}{\kern0pt}subset\ \isacommand{by}\isamarkupfalse%
\ blast%
\endisatagproof
{\isafoldproof}%
%
\isadelimproof
\isanewline
%
\endisadelimproof
\isanewline
\isacommand{lemma}\isamarkupfalse%
\ {\isacharparenleft}{\kern0pt}\isakeyword{in}\ connected{\isacharunderscore}{\kern0pt}ulgraph{\isacharparenright}{\kern0pt}\ connected{\isacharunderscore}{\kern0pt}components{\isacharunderscore}{\kern0pt}remove{\isacharunderscore}{\kern0pt}edges{\isacharcolon}{\kern0pt}\isanewline
\ \ \isakeyword{assumes}\ edge{\isacharcolon}{\kern0pt}\ {\isachardoublequoteopen}{\isacharbraceleft}{\kern0pt}u{\isacharcomma}{\kern0pt}v{\isacharbraceright}{\kern0pt}\ {\isasymin}\ E{\isachardoublequoteclose}\isanewline
\ \ \isakeyword{shows}\ {\isachardoublequoteopen}ulgraph{\isachardot}{\kern0pt}connected{\isacharunderscore}{\kern0pt}components\ V\ {\isacharparenleft}{\kern0pt}E\ {\isacharminus}{\kern0pt}\ {\isacharbraceleft}{\kern0pt}{\isacharbraceleft}{\kern0pt}u{\isacharcomma}{\kern0pt}v{\isacharbraceright}{\kern0pt}{\isacharbraceright}{\kern0pt}{\isacharparenright}{\kern0pt}\ {\isacharequal}{\kern0pt}\isanewline
\ \ \ \ {\isacharbraceleft}{\kern0pt}ulgraph{\isachardot}{\kern0pt}connected{\isacharunderscore}{\kern0pt}component{\isacharunderscore}{\kern0pt}of\ V\ {\isacharparenleft}{\kern0pt}E\ {\isacharminus}{\kern0pt}\ {\isacharbraceleft}{\kern0pt}{\isacharbraceleft}{\kern0pt}u{\isacharcomma}{\kern0pt}v{\isacharbraceright}{\kern0pt}{\isacharbraceright}{\kern0pt}{\isacharparenright}{\kern0pt}\ u{\isacharcomma}{\kern0pt}\ ulgraph{\isachardot}{\kern0pt}connected{\isacharunderscore}{\kern0pt}component{\isacharunderscore}{\kern0pt}of\ V\ {\isacharparenleft}{\kern0pt}E\ {\isacharminus}{\kern0pt}\ {\isacharbraceleft}{\kern0pt}{\isacharbraceleft}{\kern0pt}u{\isacharcomma}{\kern0pt}v{\isacharbraceright}{\kern0pt}{\isacharbraceright}{\kern0pt}{\isacharparenright}{\kern0pt}\ v{\isacharbraceright}{\kern0pt}{\isachardoublequoteclose}\isanewline
%
\isadelimproof
%
\endisadelimproof
%
\isatagproof
\isacommand{proof}\isamarkupfalse%
{\isacharminus}{\kern0pt}\isanewline
\ \ \isacommand{interpret}\isamarkupfalse%
\ g{\isacharprime}{\kern0pt}{\isacharcolon}{\kern0pt}\ ulgraph\ V\ {\isachardoublequoteopen}E\ {\isacharminus}{\kern0pt}\ {\isacharbraceleft}{\kern0pt}{\isacharbraceleft}{\kern0pt}u{\isacharcomma}{\kern0pt}v{\isacharbraceright}{\kern0pt}{\isacharbraceright}{\kern0pt}{\isachardoublequoteclose}\ \isacommand{using}\isamarkupfalse%
\ wellformed\ edge{\isacharunderscore}{\kern0pt}size\ \isacommand{by}\isamarkupfalse%
\ {\isacharparenleft}{\kern0pt}unfold{\isacharunderscore}{\kern0pt}locales{\isacharcomma}{\kern0pt}\ auto{\isacharparenright}{\kern0pt}\isanewline
\ \ \isacommand{have}\isamarkupfalse%
\ inV{\isacharcolon}{\kern0pt}\ {\isachardoublequoteopen}u\ {\isasymin}\ V{\isachardoublequoteclose}\ {\isachardoublequoteopen}v\ {\isasymin}\ V{\isachardoublequoteclose}\ \isacommand{using}\isamarkupfalse%
\ edge\ wellformed\ \isacommand{by}\isamarkupfalse%
\ auto\isanewline
\ \ \isacommand{have}\isamarkupfalse%
\ {\isachardoublequoteopen}{\isasymforall}w{\isasymin}V{\isachardot}{\kern0pt}\ g{\isacharprime}{\kern0pt}{\isachardot}{\kern0pt}connected{\isacharunderscore}{\kern0pt}component{\isacharunderscore}{\kern0pt}of\ w\ {\isacharequal}{\kern0pt}\ g{\isacharprime}{\kern0pt}{\isachardot}{\kern0pt}connected{\isacharunderscore}{\kern0pt}component{\isacharunderscore}{\kern0pt}of\ u\ {\isasymor}\ g{\isacharprime}{\kern0pt}{\isachardot}{\kern0pt}connected{\isacharunderscore}{\kern0pt}component{\isacharunderscore}{\kern0pt}of\ w\ {\isacharequal}{\kern0pt}\ g{\isacharprime}{\kern0pt}{\isachardot}{\kern0pt}connected{\isacharunderscore}{\kern0pt}component{\isacharunderscore}{\kern0pt}of\ v{\isachardoublequoteclose}\isanewline
\ \ \ \ \isacommand{using}\isamarkupfalse%
\ vert{\isacharunderscore}{\kern0pt}connected{\isacharunderscore}{\kern0pt}remove{\isacharunderscore}{\kern0pt}edge{\isacharbrackleft}{\kern0pt}OF\ edge{\isacharbrackright}{\kern0pt}\ g{\isacharprime}{\kern0pt}{\isachardot}{\kern0pt}equiv{\isacharunderscore}{\kern0pt}vert{\isacharunderscore}{\kern0pt}connected\ equiv{\isacharunderscore}{\kern0pt}class{\isacharunderscore}{\kern0pt}eq\ \isacommand{unfolding}\isamarkupfalse%
\ g{\isacharprime}{\kern0pt}{\isachardot}{\kern0pt}connected{\isacharunderscore}{\kern0pt}component{\isacharunderscore}{\kern0pt}of{\isacharunderscore}{\kern0pt}def\ \isacommand{by}\isamarkupfalse%
\ fast\isanewline
\ \ \isacommand{then}\isamarkupfalse%
\ \isacommand{show}\isamarkupfalse%
\ {\isacharquery}{\kern0pt}thesis\ \isacommand{unfolding}\isamarkupfalse%
\ g{\isacharprime}{\kern0pt}{\isachardot}{\kern0pt}connected{\isacharunderscore}{\kern0pt}components{\isacharunderscore}{\kern0pt}def\ quotient{\isacharunderscore}{\kern0pt}def\ g{\isacharprime}{\kern0pt}{\isachardot}{\kern0pt}connected{\isacharunderscore}{\kern0pt}component{\isacharunderscore}{\kern0pt}of{\isacharunderscore}{\kern0pt}def\ \isacommand{using}\isamarkupfalse%
\ inV\ \isacommand{by}\isamarkupfalse%
\ auto\isanewline
\isacommand{qed}\isamarkupfalse%
%
\endisatagproof
{\isafoldproof}%
%
\isadelimproof
\isanewline
%
\endisadelimproof
\isanewline
\isacommand{lemma}\isamarkupfalse%
\ {\isacharparenleft}{\kern0pt}\isakeyword{in}\ ulgraph{\isacharparenright}{\kern0pt}\ connected{\isacharunderscore}{\kern0pt}set{\isacharunderscore}{\kern0pt}connected{\isacharunderscore}{\kern0pt}component{\isacharcolon}{\kern0pt}\isanewline
\ \ \isakeyword{assumes}\ conn{\isacharunderscore}{\kern0pt}set{\isacharcolon}{\kern0pt}\ {\isachardoublequoteopen}is{\isacharunderscore}{\kern0pt}connected{\isacharunderscore}{\kern0pt}set\ C{\isachardoublequoteclose}\isanewline
\ \ \ \ \isakeyword{and}\ non{\isacharunderscore}{\kern0pt}empty{\isacharcolon}{\kern0pt}\ {\isachardoublequoteopen}C\ {\isasymnoteq}\ {\isacharbraceleft}{\kern0pt}{\isacharbraceright}{\kern0pt}{\isachardoublequoteclose}\isanewline
\ \ \ \ \isakeyword{and}\ {\isachardoublequoteopen}{\isasymAnd}u\ v{\isachardot}{\kern0pt}\ {\isacharbraceleft}{\kern0pt}u{\isacharcomma}{\kern0pt}v{\isacharbraceright}{\kern0pt}\ {\isasymin}\ E\ {\isasymLongrightarrow}\ u\ {\isasymin}\ C\ {\isasymLongrightarrow}\ v\ {\isasymin}\ C{\isachardoublequoteclose}\isanewline
\ \ \isakeyword{shows}\ {\isachardoublequoteopen}C\ {\isasymin}\ connected{\isacharunderscore}{\kern0pt}components{\isachardoublequoteclose}\isanewline
%
\isadelimproof
%
\endisadelimproof
%
\isatagproof
\isacommand{proof}\isamarkupfalse%
{\isacharminus}{\kern0pt}\isanewline
\ \ \isacommand{have}\isamarkupfalse%
\ walk{\isacharunderscore}{\kern0pt}subset{\isacharunderscore}{\kern0pt}C{\isacharcolon}{\kern0pt}\ {\isachardoublequoteopen}is{\isacharunderscore}{\kern0pt}walk\ xs\ {\isasymLongrightarrow}\ \ hd\ xs\ {\isasymin}\ C\ {\isasymLongrightarrow}\ set\ xs\ {\isasymsubseteq}\ C{\isachardoublequoteclose}\ \isakeyword{for}\ xs\isanewline
\ \ \isacommand{proof}\isamarkupfalse%
\ {\isacharparenleft}{\kern0pt}induction\ xs\ rule{\isacharcolon}{\kern0pt}\ rev{\isacharunderscore}{\kern0pt}induct{\isacharparenright}{\kern0pt}\isanewline
\ \ \ \ \isacommand{case}\isamarkupfalse%
\ Nil\isanewline
\ \ \ \ \isacommand{then}\isamarkupfalse%
\ \isacommand{show}\isamarkupfalse%
\ {\isacharquery}{\kern0pt}case\ \isacommand{by}\isamarkupfalse%
\ auto\isanewline
\ \ \isacommand{next}\isamarkupfalse%
\isanewline
\ \ \ \ \isacommand{case}\isamarkupfalse%
\ {\isacharparenleft}{\kern0pt}snoc\ x\ xs{\isacharparenright}{\kern0pt}\isanewline
\ \ \ \ \isacommand{then}\isamarkupfalse%
\ \isacommand{show}\isamarkupfalse%
\ {\isacharquery}{\kern0pt}case\isanewline
\ \ \ \ \isacommand{proof}\isamarkupfalse%
\ {\isacharparenleft}{\kern0pt}cases\ xs\ rule{\isacharcolon}{\kern0pt}\ rev{\isacharunderscore}{\kern0pt}exhaust{\isacharparenright}{\kern0pt}\isanewline
\ \ \ \ \ \ \isacommand{case}\isamarkupfalse%
\ Nil\isanewline
\ \ \ \ \ \ \isacommand{then}\isamarkupfalse%
\ \isacommand{show}\isamarkupfalse%
\ {\isacharquery}{\kern0pt}thesis\ \isacommand{using}\isamarkupfalse%
\ snoc\ \isacommand{by}\isamarkupfalse%
\ auto\isanewline
\ \ \ \ \isacommand{next}\isamarkupfalse%
\isanewline
\ \ \ \ \ \ \isacommand{fix}\isamarkupfalse%
\ ys\ y\ \isacommand{assume}\isamarkupfalse%
\ xs{\isacharcolon}{\kern0pt}\ {\isachardoublequoteopen}xs\ {\isacharequal}{\kern0pt}\ ys\ {\isacharat}{\kern0pt}\ {\isacharbrackleft}{\kern0pt}y{\isacharbrackright}{\kern0pt}{\isachardoublequoteclose}\isanewline
\ \ \ \ \ \ \isacommand{then}\isamarkupfalse%
\ \isacommand{have}\isamarkupfalse%
\ {\isachardoublequoteopen}is{\isacharunderscore}{\kern0pt}walk\ xs{\isachardoublequoteclose}\ \isacommand{using}\isamarkupfalse%
\ is{\isacharunderscore}{\kern0pt}walk{\isacharunderscore}{\kern0pt}prefix\ snoc{\isacharparenleft}{\kern0pt}{\isadigit{2}}{\isacharparenright}{\kern0pt}\ \isacommand{by}\isamarkupfalse%
\ blast\isanewline
\ \ \ \ \ \ \isacommand{then}\isamarkupfalse%
\ \isacommand{have}\isamarkupfalse%
\ set{\isacharunderscore}{\kern0pt}xs{\isacharunderscore}{\kern0pt}C{\isacharcolon}{\kern0pt}\ {\isachardoublequoteopen}set\ xs\ {\isasymsubseteq}\ C{\isachardoublequoteclose}\ \isacommand{using}\isamarkupfalse%
\ snoc\ xs\ is{\isacharunderscore}{\kern0pt}walk{\isacharunderscore}{\kern0pt}not{\isacharunderscore}{\kern0pt}empty{\isadigit{2}}\ hd{\isacharunderscore}{\kern0pt}append{\isadigit{2}}\ \isacommand{by}\isamarkupfalse%
\ metis\isanewline
\ \ \ \ \ \ \isacommand{have}\isamarkupfalse%
\ yx{\isacharunderscore}{\kern0pt}E{\isacharcolon}{\kern0pt}\ {\isachardoublequoteopen}{\isacharbraceleft}{\kern0pt}y{\isacharcomma}{\kern0pt}x{\isacharbraceright}{\kern0pt}\ {\isasymin}\ E{\isachardoublequoteclose}\ \isacommand{using}\isamarkupfalse%
\ snoc{\isacharparenleft}{\kern0pt}{\isadigit{2}}{\isacharparenright}{\kern0pt}\ walk{\isacharunderscore}{\kern0pt}edges{\isacharunderscore}{\kern0pt}app\ \isacommand{unfolding}\isamarkupfalse%
\ xs\ is{\isacharunderscore}{\kern0pt}walk{\isacharunderscore}{\kern0pt}def\ \isacommand{by}\isamarkupfalse%
\ simp\isanewline
\ \ \ \ \ \ \isacommand{have}\isamarkupfalse%
\ {\isachardoublequoteopen}x\ {\isasymin}\ C{\isachardoublequoteclose}\ \isacommand{using}\isamarkupfalse%
\ assms{\isacharparenleft}{\kern0pt}{\isadigit{3}}{\isacharparenright}{\kern0pt}{\isacharbrackleft}{\kern0pt}OF\ yx{\isacharunderscore}{\kern0pt}E{\isacharbrackright}{\kern0pt}\ set{\isacharunderscore}{\kern0pt}xs{\isacharunderscore}{\kern0pt}C\ \isacommand{unfolding}\isamarkupfalse%
\ xs\ \isacommand{by}\isamarkupfalse%
\ simp\isanewline
\ \ \ \ \ \ \isacommand{then}\isamarkupfalse%
\ \isacommand{show}\isamarkupfalse%
\ {\isacharquery}{\kern0pt}thesis\ \isacommand{using}\isamarkupfalse%
\ set{\isacharunderscore}{\kern0pt}xs{\isacharunderscore}{\kern0pt}C\ \isacommand{by}\isamarkupfalse%
\ simp\isanewline
\ \ \ \ \isacommand{qed}\isamarkupfalse%
\isanewline
\ \ \isacommand{qed}\isamarkupfalse%
\isanewline
\ \ \isacommand{obtain}\isamarkupfalse%
\ u\ \isakeyword{where}\ {\isachardoublequoteopen}u\ {\isasymin}\ C{\isachardoublequoteclose}\ \isacommand{using}\isamarkupfalse%
\ non{\isacharunderscore}{\kern0pt}empty\ \isacommand{by}\isamarkupfalse%
\ blast\isanewline
\ \ \isacommand{then}\isamarkupfalse%
\ \isacommand{have}\isamarkupfalse%
\ {\isachardoublequoteopen}u\ {\isasymin}\ V{\isachardoublequoteclose}\ \isacommand{using}\isamarkupfalse%
\ conn{\isacharunderscore}{\kern0pt}set\ is{\isacharunderscore}{\kern0pt}connected{\isacharunderscore}{\kern0pt}set{\isacharunderscore}{\kern0pt}wf\ \isacommand{by}\isamarkupfalse%
\ blast\isanewline
\ \ \isacommand{have}\isamarkupfalse%
\ {\isachardoublequoteopen}v\ {\isasymin}\ C{\isachardoublequoteclose}\ \isakeyword{if}\ vert{\isacharunderscore}{\kern0pt}connected{\isacharcolon}{\kern0pt}\ {\isachardoublequoteopen}vert{\isacharunderscore}{\kern0pt}connected\ u\ v{\isachardoublequoteclose}\ \isakeyword{for}\ v\isanewline
\ \ \isacommand{proof}\isamarkupfalse%
{\isacharminus}{\kern0pt}\isanewline
\ \ \ \ \isacommand{obtain}\isamarkupfalse%
\ p\ \isakeyword{where}\ {\isachardoublequoteopen}connecting{\isacharunderscore}{\kern0pt}path\ u\ v\ p{\isachardoublequoteclose}\ \isacommand{using}\isamarkupfalse%
\ vert{\isacharunderscore}{\kern0pt}connected\ \isacommand{unfolding}\isamarkupfalse%
\ vert{\isacharunderscore}{\kern0pt}connected{\isacharunderscore}{\kern0pt}def\ \isacommand{by}\isamarkupfalse%
\ blast\isanewline
\ \ \ \ \isacommand{then}\isamarkupfalse%
\ \isacommand{show}\isamarkupfalse%
\ {\isacharquery}{\kern0pt}thesis\ \isacommand{using}\isamarkupfalse%
\ walk{\isacharunderscore}{\kern0pt}subset{\isacharunderscore}{\kern0pt}C{\isacharbrackleft}{\kern0pt}of\ p{\isacharbrackright}{\kern0pt}\ {\isacartoucheopen}u{\isasymin}C{\isacartoucheclose}\ is{\isacharunderscore}{\kern0pt}walk{\isacharunderscore}{\kern0pt}def\ last{\isacharunderscore}{\kern0pt}in{\isacharunderscore}{\kern0pt}set\ \isacommand{unfolding}\isamarkupfalse%
\ connecting{\isacharunderscore}{\kern0pt}path{\isacharunderscore}{\kern0pt}def\ is{\isacharunderscore}{\kern0pt}gen{\isacharunderscore}{\kern0pt}path{\isacharunderscore}{\kern0pt}def\ \isacommand{by}\isamarkupfalse%
\ auto\isanewline
\ \ \isacommand{qed}\isamarkupfalse%
\isanewline
\ \ \isacommand{then}\isamarkupfalse%
\ \isacommand{have}\isamarkupfalse%
\ {\isachardoublequoteopen}connected{\isacharunderscore}{\kern0pt}component{\isacharunderscore}{\kern0pt}of\ u\ {\isacharequal}{\kern0pt}\ C{\isachardoublequoteclose}\ \isacommand{using}\isamarkupfalse%
\ assms\ {\isacartoucheopen}u{\isasymin}C{\isacartoucheclose}\ \isacommand{unfolding}\isamarkupfalse%
\ connected{\isacharunderscore}{\kern0pt}component{\isacharunderscore}{\kern0pt}of{\isacharunderscore}{\kern0pt}def\ is{\isacharunderscore}{\kern0pt}connected{\isacharunderscore}{\kern0pt}set{\isacharunderscore}{\kern0pt}def\ \isacommand{by}\isamarkupfalse%
\ auto\isanewline
\ \ \isacommand{then}\isamarkupfalse%
\ \isacommand{show}\isamarkupfalse%
\ {\isacharquery}{\kern0pt}thesis\ \isacommand{using}\isamarkupfalse%
\ conn{\isacharunderscore}{\kern0pt}comp{\isacharunderscore}{\kern0pt}of{\isacharunderscore}{\kern0pt}conn{\isacharunderscore}{\kern0pt}comps\ {\isacartoucheopen}u{\isasymin}V{\isacartoucheclose}\ \isacommand{by}\isamarkupfalse%
\ blast\isanewline
\isacommand{qed}\isamarkupfalse%
%
\endisatagproof
{\isafoldproof}%
%
\isadelimproof
\isanewline
%
\endisadelimproof
\isanewline
\isacommand{lemma}\isamarkupfalse%
\ {\isacharparenleft}{\kern0pt}\isakeyword{in}\ ulgraph{\isacharparenright}{\kern0pt}\ subset{\isacharunderscore}{\kern0pt}conn{\isacharunderscore}{\kern0pt}comps{\isacharunderscore}{\kern0pt}if{\isacharunderscore}{\kern0pt}Union{\isacharcolon}{\kern0pt}\isanewline
\ \ \isakeyword{assumes}\ A{\isacharunderscore}{\kern0pt}subset{\isacharunderscore}{\kern0pt}conn{\isacharunderscore}{\kern0pt}comps{\isacharcolon}{\kern0pt}\ {\isachardoublequoteopen}A\ {\isasymsubseteq}\ connected{\isacharunderscore}{\kern0pt}components{\isachardoublequoteclose}\isanewline
\ \ \ \ \isakeyword{and}\ Un{\isacharunderscore}{\kern0pt}A{\isacharcolon}{\kern0pt}\ {\isachardoublequoteopen}{\isasymUnion}A\ {\isacharequal}{\kern0pt}\ V{\isachardoublequoteclose}\isanewline
\ \ \isakeyword{shows}\ {\isachardoublequoteopen}A\ {\isacharequal}{\kern0pt}\ connected{\isacharunderscore}{\kern0pt}components{\isachardoublequoteclose}\isanewline
%
\isadelimproof
%
\endisadelimproof
%
\isatagproof
\isacommand{proof}\isamarkupfalse%
\ {\isacharparenleft}{\kern0pt}rule\ ccontr{\isacharparenright}{\kern0pt}\isanewline
\ \ \isacommand{assume}\isamarkupfalse%
\ {\isachardoublequoteopen}A\ {\isasymnoteq}\ connected{\isacharunderscore}{\kern0pt}components{\isachardoublequoteclose}\isanewline
\ \ \isacommand{then}\isamarkupfalse%
\ \isacommand{obtain}\isamarkupfalse%
\ C\ \isakeyword{where}\ C{\isacharunderscore}{\kern0pt}conn{\isacharunderscore}{\kern0pt}comp{\isacharcolon}{\kern0pt}\ {\isachardoublequoteopen}C\ {\isasymin}\ connected{\isacharunderscore}{\kern0pt}components\ {\isachardoublequoteclose}\ {\isachardoublequoteopen}C\ {\isasymnotin}\ A{\isachardoublequoteclose}\ \isacommand{using}\isamarkupfalse%
\ A{\isacharunderscore}{\kern0pt}subset{\isacharunderscore}{\kern0pt}conn{\isacharunderscore}{\kern0pt}comps\ \isacommand{by}\isamarkupfalse%
\ blast\isanewline
\ \ \isacommand{then}\isamarkupfalse%
\ \isacommand{have}\isamarkupfalse%
\ {\isachardoublequoteopen}C\ {\isacharequal}{\kern0pt}\ {\isacharbraceleft}{\kern0pt}{\isacharbraceright}{\kern0pt}{\isachardoublequoteclose}\ \isacommand{using}\isamarkupfalse%
\ A{\isacharunderscore}{\kern0pt}subset{\isacharunderscore}{\kern0pt}conn{\isacharunderscore}{\kern0pt}comps\ Un{\isacharunderscore}{\kern0pt}A\ connected{\isacharunderscore}{\kern0pt}components{\isacharunderscore}{\kern0pt}partition{\isacharunderscore}{\kern0pt}on{\isacharunderscore}{\kern0pt}V\ \isacommand{unfolding}\isamarkupfalse%
\ partition{\isacharunderscore}{\kern0pt}on{\isacharunderscore}{\kern0pt}def\isanewline
\ \ \ \ \isacommand{by}\isamarkupfalse%
\ {\isacharparenleft}{\kern0pt}auto{\isacharcomma}{\kern0pt}\ smt\ {\isacharparenleft}{\kern0pt}verit{\isacharcomma}{\kern0pt}\ best{\isacharparenright}{\kern0pt}\ UnionE\ UnionI\ disjnt{\isacharunderscore}{\kern0pt}iff\ pairwise{\isacharunderscore}{\kern0pt}def\ subset{\isacharunderscore}{\kern0pt}iff{\isacharparenright}{\kern0pt}\isanewline
\ \ \isacommand{then}\isamarkupfalse%
\ \isacommand{show}\isamarkupfalse%
\ False\ \isacommand{using}\isamarkupfalse%
\ connected{\isacharunderscore}{\kern0pt}components{\isacharunderscore}{\kern0pt}partition{\isacharunderscore}{\kern0pt}on{\isacharunderscore}{\kern0pt}V\ C{\isacharunderscore}{\kern0pt}conn{\isacharunderscore}{\kern0pt}comp\ \isacommand{unfolding}\isamarkupfalse%
\ partition{\isacharunderscore}{\kern0pt}on{\isacharunderscore}{\kern0pt}def\ \isacommand{by}\isamarkupfalse%
\ blast\isanewline
\isacommand{qed}\isamarkupfalse%
%
\endisatagproof
{\isafoldproof}%
%
\isadelimproof
\isanewline
%
\endisadelimproof
\isanewline
\isacommand{lemma}\isamarkupfalse%
\ {\isacharparenleft}{\kern0pt}\isakeyword{in}\ connected{\isacharunderscore}{\kern0pt}ulgraph{\isacharparenright}{\kern0pt}\ exists{\isacharunderscore}{\kern0pt}adj{\isacharunderscore}{\kern0pt}vert{\isacharunderscore}{\kern0pt}removed{\isacharcolon}{\kern0pt}\isanewline
\ \ \isakeyword{assumes}\ {\isachardoublequoteopen}v\ {\isasymin}\ V{\isachardoublequoteclose}\isanewline
\ \ \ \ \isakeyword{and}\ remove{\isacharunderscore}{\kern0pt}vertex{\isacharcolon}{\kern0pt}\ {\isachardoublequoteopen}remove{\isacharunderscore}{\kern0pt}vertex\ v\ {\isacharequal}{\kern0pt}\ {\isacharparenleft}{\kern0pt}V{\isacharprime}{\kern0pt}{\isacharcomma}{\kern0pt}E{\isacharprime}{\kern0pt}{\isacharparenright}{\kern0pt}{\isachardoublequoteclose}\isanewline
\ \ \ \ \isakeyword{and}\ conn{\isacharunderscore}{\kern0pt}component{\isacharcolon}{\kern0pt}\ {\isachardoublequoteopen}C\ {\isasymin}\ ulgraph{\isachardot}{\kern0pt}connected{\isacharunderscore}{\kern0pt}components\ V{\isacharprime}{\kern0pt}\ E{\isacharprime}{\kern0pt}{\isachardoublequoteclose}\isanewline
\ \ \isakeyword{shows}\ {\isachardoublequoteopen}{\isasymexists}u{\isasymin}C{\isachardot}{\kern0pt}\ vert{\isacharunderscore}{\kern0pt}adj\ v\ u{\isachardoublequoteclose}\isanewline
%
\isadelimproof
%
\endisadelimproof
%
\isatagproof
\isacommand{proof}\isamarkupfalse%
{\isacharminus}{\kern0pt}\isanewline
\ \ \isacommand{have}\isamarkupfalse%
\ V{\isacharprime}{\kern0pt}{\isacharcolon}{\kern0pt}\ {\isachardoublequoteopen}V{\isacharprime}{\kern0pt}\ {\isacharequal}{\kern0pt}\ V\ {\isacharminus}{\kern0pt}\ {\isacharbraceleft}{\kern0pt}v{\isacharbraceright}{\kern0pt}{\isachardoublequoteclose}\ \isakeyword{and}\ E{\isacharprime}{\kern0pt}{\isacharcolon}{\kern0pt}\ {\isachardoublequoteopen}E{\isacharprime}{\kern0pt}\ {\isacharequal}{\kern0pt}\ {\isacharbraceleft}{\kern0pt}e{\isasymin}E{\isachardot}{\kern0pt}\ v\ {\isasymnotin}\ e{\isacharbraceright}{\kern0pt}{\isachardoublequoteclose}\ \isacommand{using}\isamarkupfalse%
\ remove{\isacharunderscore}{\kern0pt}vertex\ \isacommand{unfolding}\isamarkupfalse%
\ remove{\isacharunderscore}{\kern0pt}vertex{\isacharunderscore}{\kern0pt}def\ incident{\isacharunderscore}{\kern0pt}def\ \isacommand{by}\isamarkupfalse%
\ auto\isanewline
\ \ \isacommand{interpret}\isamarkupfalse%
\ subg{\isacharcolon}{\kern0pt}\ subgraph\ {\isachardoublequoteopen}V\ {\isacharminus}{\kern0pt}\ {\isacharbraceleft}{\kern0pt}v{\isacharbraceright}{\kern0pt}{\isachardoublequoteclose}\ {\isachardoublequoteopen}{\isacharbraceleft}{\kern0pt}e{\isasymin}E{\isachardot}{\kern0pt}\ v\ {\isasymnotin}\ e{\isacharbraceright}{\kern0pt}{\isachardoublequoteclose}\ V\ E\ \isacommand{using}\isamarkupfalse%
\ subgraph{\isacharunderscore}{\kern0pt}remove{\isacharunderscore}{\kern0pt}vertex\ remove{\isacharunderscore}{\kern0pt}vertex\ V{\isacharprime}{\kern0pt}\ E{\isacharprime}{\kern0pt}\ \isacommand{by}\isamarkupfalse%
\ metis\isanewline
\ \ \isacommand{interpret}\isamarkupfalse%
\ g{\isacharprime}{\kern0pt}{\isacharcolon}{\kern0pt}\ ulgraph\ {\isachardoublequoteopen}V\ {\isacharminus}{\kern0pt}\ {\isacharbraceleft}{\kern0pt}v{\isacharbraceright}{\kern0pt}{\isachardoublequoteclose}\ {\isachardoublequoteopen}{\isacharbraceleft}{\kern0pt}e{\isasymin}E{\isachardot}{\kern0pt}\ v\ {\isasymnotin}\ e{\isacharbraceright}{\kern0pt}{\isachardoublequoteclose}\ \isacommand{using}\isamarkupfalse%
\ subg{\isachardot}{\kern0pt}is{\isacharunderscore}{\kern0pt}subgraph{\isacharunderscore}{\kern0pt}ulgraph\ ulgraph{\isacharunderscore}{\kern0pt}axioms\ \isacommand{by}\isamarkupfalse%
\ blast\isanewline
\ \ \isacommand{obtain}\isamarkupfalse%
\ c\ \isakeyword{where}\ {\isachardoublequoteopen}c\ {\isasymin}\ C{\isachardoublequoteclose}\ \isacommand{using}\isamarkupfalse%
\ g{\isacharprime}{\kern0pt}{\isachardot}{\kern0pt}connected{\isacharunderscore}{\kern0pt}component{\isacharunderscore}{\kern0pt}non{\isacharunderscore}{\kern0pt}empty\ conn{\isacharunderscore}{\kern0pt}component\ V{\isacharprime}{\kern0pt}\ E{\isacharprime}{\kern0pt}\ \isacommand{by}\isamarkupfalse%
\ blast\isanewline
\ \ \isacommand{then}\isamarkupfalse%
\ \isacommand{have}\isamarkupfalse%
\ {\isachardoublequoteopen}c\ {\isasymin}\ V{\isacharprime}{\kern0pt}{\isachardoublequoteclose}\ \isacommand{using}\isamarkupfalse%
\ g{\isacharprime}{\kern0pt}{\isachardot}{\kern0pt}connected{\isacharunderscore}{\kern0pt}component{\isacharunderscore}{\kern0pt}wf\ conn{\isacharunderscore}{\kern0pt}component\ V{\isacharprime}{\kern0pt}\ E{\isacharprime}{\kern0pt}\ \isacommand{by}\isamarkupfalse%
\ blast\isanewline
\ \ \isacommand{then}\isamarkupfalse%
\ \isacommand{have}\isamarkupfalse%
\ {\isachardoublequoteopen}c\ {\isasymin}\ V{\isachardoublequoteclose}\ \isacommand{using}\isamarkupfalse%
\ subg{\isachardot}{\kern0pt}verts{\isacharunderscore}{\kern0pt}ss\ V{\isacharprime}{\kern0pt}\ \isacommand{by}\isamarkupfalse%
\ blast\isanewline
\ \ \isacommand{then}\isamarkupfalse%
\ \isacommand{obtain}\isamarkupfalse%
\ p\ \isakeyword{where}\ conn{\isacharunderscore}{\kern0pt}path{\isacharcolon}{\kern0pt}\ {\isachardoublequoteopen}connecting{\isacharunderscore}{\kern0pt}path\ v\ c\ p{\isachardoublequoteclose}\ \isacommand{using}\isamarkupfalse%
\ {\isacartoucheopen}v{\isasymin}V{\isacartoucheclose}\ vertices{\isacharunderscore}{\kern0pt}connected{\isacharunderscore}{\kern0pt}path\ \isacommand{by}\isamarkupfalse%
\ blast\ \ \isanewline
\ \ \isacommand{have}\isamarkupfalse%
\ {\isachardoublequoteopen}v\ {\isasymnoteq}\ c{\isachardoublequoteclose}\ \isacommand{using}\isamarkupfalse%
\ {\isacartoucheopen}c{\isasymin}V{\isacharprime}{\kern0pt}{\isacartoucheclose}\ remove{\isacharunderscore}{\kern0pt}vertex\ \isacommand{unfolding}\isamarkupfalse%
\ remove{\isacharunderscore}{\kern0pt}vertex{\isacharunderscore}{\kern0pt}def\ \isacommand{by}\isamarkupfalse%
\ blast\isanewline
\ \ \isacommand{then}\isamarkupfalse%
\ \isacommand{obtain}\isamarkupfalse%
\ u\ p{\isacharprime}{\kern0pt}\ \isakeyword{where}\ p{\isacharcolon}{\kern0pt}\ {\isachardoublequoteopen}p\ {\isacharequal}{\kern0pt}\ v\ {\isacharhash}{\kern0pt}\ u\ {\isacharhash}{\kern0pt}\ p{\isacharprime}{\kern0pt}{\isachardoublequoteclose}\ \isacommand{using}\isamarkupfalse%
\ conn{\isacharunderscore}{\kern0pt}path\isanewline
\ \ \ \ \isacommand{by}\isamarkupfalse%
\ {\isacharparenleft}{\kern0pt}metis\ connecting{\isacharunderscore}{\kern0pt}path{\isacharunderscore}{\kern0pt}def\ is{\isacharunderscore}{\kern0pt}gen{\isacharunderscore}{\kern0pt}path{\isacharunderscore}{\kern0pt}def\ is{\isacharunderscore}{\kern0pt}walk{\isacharunderscore}{\kern0pt}def\ last{\isachardot}{\kern0pt}simps\ list{\isachardot}{\kern0pt}exhaust{\isacharunderscore}{\kern0pt}sel{\isacharparenright}{\kern0pt}\isanewline
\ \ \isacommand{then}\isamarkupfalse%
\ \isacommand{have}\isamarkupfalse%
\ conn{\isacharunderscore}{\kern0pt}path{\isacharunderscore}{\kern0pt}uc{\isacharcolon}{\kern0pt}\ {\isachardoublequoteopen}connecting{\isacharunderscore}{\kern0pt}path\ u\ c\ {\isacharparenleft}{\kern0pt}u{\isacharhash}{\kern0pt}p{\isacharprime}{\kern0pt}{\isacharparenright}{\kern0pt}{\isachardoublequoteclose}\ \isacommand{using}\isamarkupfalse%
\ conn{\isacharunderscore}{\kern0pt}path\ connecting{\isacharunderscore}{\kern0pt}path{\isacharunderscore}{\kern0pt}tl\ \isacommand{unfolding}\isamarkupfalse%
\ p\ \isacommand{by}\isamarkupfalse%
\ blast\isanewline
\ \ \isacommand{have}\isamarkupfalse%
\ v{\isacharunderscore}{\kern0pt}notin{\isacharunderscore}{\kern0pt}p{\isacharprime}{\kern0pt}{\isacharcolon}{\kern0pt}\ {\isachardoublequoteopen}v\ {\isasymnotin}\ set\ {\isacharparenleft}{\kern0pt}u{\isacharhash}{\kern0pt}p{\isacharprime}{\kern0pt}{\isacharparenright}{\kern0pt}{\isachardoublequoteclose}\ \isacommand{using}\isamarkupfalse%
\ conn{\isacharunderscore}{\kern0pt}path\ {\isacartoucheopen}v{\isasymnoteq}c{\isacartoucheclose}\ \isacommand{unfolding}\isamarkupfalse%
\ p\ connecting{\isacharunderscore}{\kern0pt}path{\isacharunderscore}{\kern0pt}def\ is{\isacharunderscore}{\kern0pt}gen{\isacharunderscore}{\kern0pt}path{\isacharunderscore}{\kern0pt}def\ \isacommand{by}\isamarkupfalse%
\ auto\isanewline
\ \ \isacommand{then}\isamarkupfalse%
\ \isacommand{have}\isamarkupfalse%
\ {\isachardoublequoteopen}g{\isacharprime}{\kern0pt}{\isachardot}{\kern0pt}connecting{\isacharunderscore}{\kern0pt}path\ u\ c\ {\isacharparenleft}{\kern0pt}u{\isacharhash}{\kern0pt}p{\isacharprime}{\kern0pt}{\isacharparenright}{\kern0pt}{\isachardoublequoteclose}\ \isacommand{using}\isamarkupfalse%
\ conn{\isacharunderscore}{\kern0pt}path{\isacharunderscore}{\kern0pt}uc\ v{\isacharunderscore}{\kern0pt}notin{\isacharunderscore}{\kern0pt}p{\isacharprime}{\kern0pt}\ walk{\isacharunderscore}{\kern0pt}edges{\isacharunderscore}{\kern0pt}in{\isacharunderscore}{\kern0pt}verts\isanewline
\ \ \ \ \isacommand{unfolding}\isamarkupfalse%
\ g{\isacharprime}{\kern0pt}{\isachardot}{\kern0pt}connecting{\isacharunderscore}{\kern0pt}path{\isacharunderscore}{\kern0pt}def\ connecting{\isacharunderscore}{\kern0pt}path{\isacharunderscore}{\kern0pt}def\ g{\isacharprime}{\kern0pt}{\isachardot}{\kern0pt}is{\isacharunderscore}{\kern0pt}gen{\isacharunderscore}{\kern0pt}path{\isacharunderscore}{\kern0pt}def\ is{\isacharunderscore}{\kern0pt}gen{\isacharunderscore}{\kern0pt}path{\isacharunderscore}{\kern0pt}def\ g{\isacharprime}{\kern0pt}{\isachardot}{\kern0pt}is{\isacharunderscore}{\kern0pt}walk{\isacharunderscore}{\kern0pt}def\ is{\isacharunderscore}{\kern0pt}walk{\isacharunderscore}{\kern0pt}def\isanewline
\ \ \ \ \isacommand{by}\isamarkupfalse%
\ blast\ \isanewline
\ \ \isacommand{then}\isamarkupfalse%
\ \isacommand{have}\isamarkupfalse%
\ {\isachardoublequoteopen}g{\isacharprime}{\kern0pt}{\isachardot}{\kern0pt}vert{\isacharunderscore}{\kern0pt}connected\ u\ c{\isachardoublequoteclose}\ \isacommand{unfolding}\isamarkupfalse%
\ g{\isacharprime}{\kern0pt}{\isachardot}{\kern0pt}vert{\isacharunderscore}{\kern0pt}connected{\isacharunderscore}{\kern0pt}def\ \isacommand{by}\isamarkupfalse%
\ blast\isanewline
\ \ \isacommand{then}\isamarkupfalse%
\ \isacommand{have}\isamarkupfalse%
\ {\isachardoublequoteopen}u\ {\isasymin}\ C{\isachardoublequoteclose}\ \isacommand{using}\isamarkupfalse%
\ {\isacartoucheopen}c{\isasymin}C{\isacartoucheclose}\ conn{\isacharunderscore}{\kern0pt}component\ g{\isacharprime}{\kern0pt}{\isachardot}{\kern0pt}vert{\isacharunderscore}{\kern0pt}connected{\isacharunderscore}{\kern0pt}connected{\isacharunderscore}{\kern0pt}component\ g{\isacharprime}{\kern0pt}{\isachardot}{\kern0pt}vert{\isacharunderscore}{\kern0pt}connected{\isacharunderscore}{\kern0pt}rev\ \isacommand{unfolding}\isamarkupfalse%
\ V{\isacharprime}{\kern0pt}\ E{\isacharprime}{\kern0pt}\ \isacommand{by}\isamarkupfalse%
\ blast\isanewline
\ \ \isacommand{have}\isamarkupfalse%
\ {\isachardoublequoteopen}vert{\isacharunderscore}{\kern0pt}adj\ v\ u{\isachardoublequoteclose}\ \isacommand{using}\isamarkupfalse%
\ conn{\isacharunderscore}{\kern0pt}path\ \isacommand{unfolding}\isamarkupfalse%
\ p\ connecting{\isacharunderscore}{\kern0pt}path{\isacharunderscore}{\kern0pt}def\ is{\isacharunderscore}{\kern0pt}gen{\isacharunderscore}{\kern0pt}path{\isacharunderscore}{\kern0pt}def\ is{\isacharunderscore}{\kern0pt}walk{\isacharunderscore}{\kern0pt}def\ vert{\isacharunderscore}{\kern0pt}adj{\isacharunderscore}{\kern0pt}def\ \isacommand{by}\isamarkupfalse%
\ auto\isanewline
\ \ \isacommand{then}\isamarkupfalse%
\ \isacommand{show}\isamarkupfalse%
\ {\isacharquery}{\kern0pt}thesis\ \isacommand{using}\isamarkupfalse%
\ {\isacartoucheopen}u{\isasymin}C{\isacartoucheclose}\ \isacommand{by}\isamarkupfalse%
\ blast\isanewline
\isacommand{qed}\isamarkupfalse%
%
\endisatagproof
{\isafoldproof}%
%
\isadelimproof
%
\endisadelimproof
%
\isadelimdocument
%
\endisadelimdocument
%
\isatagdocument
%
\isamarkupsubsection{Trees%
}
\isamarkuptrue%
%
\endisatagdocument
{\isafolddocument}%
%
\isadelimdocument
%
\endisadelimdocument
\isacommand{locale}\isamarkupfalse%
\ tree\ {\isacharequal}{\kern0pt}\ fin{\isacharunderscore}{\kern0pt}connected{\isacharunderscore}{\kern0pt}ulgraph\ {\isacharplus}{\kern0pt}\isanewline
\ \ \isakeyword{assumes}\ no{\isacharunderscore}{\kern0pt}cycles{\isacharcolon}{\kern0pt}\ {\isachardoublequoteopen}{\isasymnot}\ is{\isacharunderscore}{\kern0pt}cycle{\isadigit{2}}\ c{\isachardoublequoteclose}\isanewline
\isakeyword{begin}\isanewline
\isanewline
\isacommand{sublocale}\isamarkupfalse%
\ fin{\isacharunderscore}{\kern0pt}connected{\isacharunderscore}{\kern0pt}sgraph\isanewline
%
\isadelimproof
\ \ %
\endisadelimproof
%
\isatagproof
\isacommand{using}\isamarkupfalse%
\ alt{\isacharunderscore}{\kern0pt}edge{\isacharunderscore}{\kern0pt}size\ no{\isacharunderscore}{\kern0pt}cycles\ loop{\isacharunderscore}{\kern0pt}is{\isacharunderscore}{\kern0pt}cycle{\isadigit{2}}\ card{\isacharunderscore}{\kern0pt}{\isadigit{1}}{\isacharunderscore}{\kern0pt}singletonE\ connected\isanewline
\ \ \isacommand{by}\isamarkupfalse%
\ {\isacharparenleft}{\kern0pt}unfold{\isacharunderscore}{\kern0pt}locales{\isacharcomma}{\kern0pt}\ metis{\isacharcomma}{\kern0pt}\ simp{\isacharparenright}{\kern0pt}%
\endisatagproof
{\isafoldproof}%
%
\isadelimproof
\isanewline
%
\endisadelimproof
\isanewline
\isacommand{end}\isamarkupfalse%
\isanewline
\isanewline
\isacommand{locale}\isamarkupfalse%
\ spanning{\isacharunderscore}{\kern0pt}tree\ {\isacharequal}{\kern0pt}\ fin{\isacharunderscore}{\kern0pt}ulgraph\ V\ E\ {\isacharplus}{\kern0pt}\ T{\isacharcolon}{\kern0pt}\ tree\ V\ T\ \isakeyword{for}\ V\ E\ T\ {\isacharplus}{\kern0pt}\isanewline
\ \ \isakeyword{assumes}\ subgraph{\isacharcolon}{\kern0pt}\ {\isachardoublequoteopen}T\ {\isasymsubseteq}\ E{\isachardoublequoteclose}\isanewline
\isanewline
\isacommand{lemma}\isamarkupfalse%
\ {\isacharparenleft}{\kern0pt}\isakeyword{in}\ fin{\isacharunderscore}{\kern0pt}connected{\isacharunderscore}{\kern0pt}ulgraph{\isacharparenright}{\kern0pt}\ has{\isacharunderscore}{\kern0pt}spanning{\isacharunderscore}{\kern0pt}tree{\isacharcolon}{\kern0pt}\ {\isachardoublequoteopen}{\isasymexists}T{\isachardot}{\kern0pt}\ spanning{\isacharunderscore}{\kern0pt}tree\ V\ E\ T{\isachardoublequoteclose}\isanewline
%
\isadelimproof
\ \ %
\endisadelimproof
%
\isatagproof
\isacommand{using}\isamarkupfalse%
\ fin{\isacharunderscore}{\kern0pt}connected{\isacharunderscore}{\kern0pt}ulgraph{\isacharunderscore}{\kern0pt}axioms\isanewline
\isacommand{proof}\isamarkupfalse%
\ {\isacharparenleft}{\kern0pt}induction\ {\isachardoublequoteopen}card\ E{\isachardoublequoteclose}\ arbitrary{\isacharcolon}{\kern0pt}\ E{\isacharparenright}{\kern0pt}\isanewline
\ \ \isacommand{case}\isamarkupfalse%
\ {\isadigit{0}}\isanewline
\ \ \isacommand{then}\isamarkupfalse%
\ \isacommand{interpret}\isamarkupfalse%
\ g{\isacharcolon}{\kern0pt}\ fin{\isacharunderscore}{\kern0pt}connected{\isacharunderscore}{\kern0pt}ulgraph\ V\ edges\ \isacommand{by}\isamarkupfalse%
\ blast\isanewline
\ \ \isacommand{have}\isamarkupfalse%
\ edges{\isacharcolon}{\kern0pt}\ {\isachardoublequoteopen}edges\ {\isacharequal}{\kern0pt}\ {\isacharbraceleft}{\kern0pt}{\isacharbraceright}{\kern0pt}{\isachardoublequoteclose}\ \isacommand{using}\isamarkupfalse%
\ g{\isachardot}{\kern0pt}fin{\isacharunderscore}{\kern0pt}edges\ {\isadigit{0}}\ \isacommand{by}\isamarkupfalse%
\ simp\isanewline
\ \ \isacommand{then}\isamarkupfalse%
\ \isacommand{obtain}\isamarkupfalse%
\ v\ \isakeyword{where}\ V{\isacharcolon}{\kern0pt}\ {\isachardoublequoteopen}V\ {\isacharequal}{\kern0pt}\ {\isacharbraceleft}{\kern0pt}v{\isacharbraceright}{\kern0pt}{\isachardoublequoteclose}\ \isacommand{using}\isamarkupfalse%
\ g{\isachardot}{\kern0pt}V{\isacharunderscore}{\kern0pt}E{\isacharunderscore}{\kern0pt}empty\ \isacommand{by}\isamarkupfalse%
\ blast\isanewline
\ \ \isacommand{interpret}\isamarkupfalse%
\ g{\isacharprime}{\kern0pt}{\isacharcolon}{\kern0pt}\ fin{\isacharunderscore}{\kern0pt}connected{\isacharunderscore}{\kern0pt}sgraph\ V\ edges\ \isacommand{using}\isamarkupfalse%
\ g{\isachardot}{\kern0pt}connected\ edges\ \isacommand{by}\isamarkupfalse%
\ {\isacharparenleft}{\kern0pt}unfold{\isacharunderscore}{\kern0pt}locales{\isacharcomma}{\kern0pt}\ auto{\isacharparenright}{\kern0pt}\isanewline
\ \ \isacommand{interpret}\isamarkupfalse%
\ t{\isacharcolon}{\kern0pt}\ tree\ V\ edges\ \isacommand{using}\isamarkupfalse%
\ g{\isachardot}{\kern0pt}length{\isacharunderscore}{\kern0pt}cycle{\isacharunderscore}{\kern0pt}card{\isacharunderscore}{\kern0pt}V\ g{\isacharprime}{\kern0pt}{\isachardot}{\kern0pt}cycle{\isadigit{2}}{\isacharunderscore}{\kern0pt}min{\isacharunderscore}{\kern0pt}length\ g{\isachardot}{\kern0pt}is{\isacharunderscore}{\kern0pt}cycle{\isadigit{2}}{\isacharunderscore}{\kern0pt}def\ V\ \isacommand{by}\isamarkupfalse%
\ {\isacharparenleft}{\kern0pt}unfold{\isacharunderscore}{\kern0pt}locales{\isacharcomma}{\kern0pt}\ fastforce{\isacharparenright}{\kern0pt}\isanewline
\ \ \isacommand{have}\isamarkupfalse%
\ {\isachardoublequoteopen}spanning{\isacharunderscore}{\kern0pt}tree\ V\ edges\ edges{\isachardoublequoteclose}\ \isacommand{by}\isamarkupfalse%
\ {\isacharparenleft}{\kern0pt}unfold{\isacharunderscore}{\kern0pt}locales{\isacharcomma}{\kern0pt}\ auto{\isacharparenright}{\kern0pt}\ \isanewline
\ \ \isacommand{then}\isamarkupfalse%
\ \isacommand{show}\isamarkupfalse%
\ {\isacharquery}{\kern0pt}case\ \isacommand{by}\isamarkupfalse%
\ blast\isanewline
\isacommand{next}\isamarkupfalse%
\isanewline
\ \ \isacommand{case}\isamarkupfalse%
\ {\isacharparenleft}{\kern0pt}Suc\ m{\isacharparenright}{\kern0pt}\isanewline
\ \ \isacommand{then}\isamarkupfalse%
\ \isacommand{interpret}\isamarkupfalse%
\ g{\isacharcolon}{\kern0pt}\ fin{\isacharunderscore}{\kern0pt}connected{\isacharunderscore}{\kern0pt}ulgraph\ V\ edges\ \isacommand{by}\isamarkupfalse%
\ blast\isanewline
\ \ \isacommand{show}\isamarkupfalse%
\ {\isacharquery}{\kern0pt}case\isanewline
\ \ \isacommand{proof}\isamarkupfalse%
\ {\isacharparenleft}{\kern0pt}cases\ {\isachardoublequoteopen}{\isasymforall}c{\isachardot}{\kern0pt}\ {\isasymnot}g{\isachardot}{\kern0pt}is{\isacharunderscore}{\kern0pt}cycle{\isadigit{2}}\ c{\isachardoublequoteclose}{\isacharparenright}{\kern0pt}\isanewline
\ \ \ \ \isacommand{case}\isamarkupfalse%
\ True\isanewline
\ \ \ \ \isacommand{then}\isamarkupfalse%
\ \isacommand{have}\isamarkupfalse%
\ {\isachardoublequoteopen}spanning{\isacharunderscore}{\kern0pt}tree\ V\ edges\ edges{\isachardoublequoteclose}\ \isacommand{by}\isamarkupfalse%
\ {\isacharparenleft}{\kern0pt}unfold{\isacharunderscore}{\kern0pt}locales{\isacharcomma}{\kern0pt}\ auto{\isacharparenright}{\kern0pt}\isanewline
\ \ \ \ \isacommand{then}\isamarkupfalse%
\ \isacommand{show}\isamarkupfalse%
\ {\isacharquery}{\kern0pt}thesis\ \isacommand{by}\isamarkupfalse%
\ blast\isanewline
\ \ \isacommand{next}\isamarkupfalse%
\isanewline
\ \ \ \ \isacommand{case}\isamarkupfalse%
\ False\isanewline
\ \ \ \ \isacommand{then}\isamarkupfalse%
\ \isacommand{obtain}\isamarkupfalse%
\ c\ \isakeyword{where}\ cycle{\isacharcolon}{\kern0pt}\ {\isachardoublequoteopen}g{\isachardot}{\kern0pt}is{\isacharunderscore}{\kern0pt}cycle{\isadigit{2}}\ c{\isachardoublequoteclose}\ \isacommand{by}\isamarkupfalse%
\ blast\isanewline
\ \ \ \ \isacommand{then}\isamarkupfalse%
\ \isacommand{have}\isamarkupfalse%
\ {\isachardoublequoteopen}length\ c\ {\isasymge}\ {\isadigit{2}}{\isachardoublequoteclose}\ \isacommand{unfolding}\isamarkupfalse%
\ g{\isachardot}{\kern0pt}is{\isacharunderscore}{\kern0pt}cycle{\isadigit{2}}{\isacharunderscore}{\kern0pt}def\ g{\isachardot}{\kern0pt}is{\isacharunderscore}{\kern0pt}cycle{\isacharunderscore}{\kern0pt}alt\ walk{\isacharunderscore}{\kern0pt}length{\isacharunderscore}{\kern0pt}conv\ \isacommand{by}\isamarkupfalse%
\ auto\isanewline
\ \ \ \ \isacommand{then}\isamarkupfalse%
\ \isacommand{obtain}\isamarkupfalse%
\ u\ v\ xs\ \isakeyword{where}\ c{\isacharcolon}{\kern0pt}\ {\isachardoublequoteopen}c\ {\isacharequal}{\kern0pt}\ u{\isacharhash}{\kern0pt}v{\isacharhash}{\kern0pt}xs{\isachardoublequoteclose}\ \isacommand{by}\isamarkupfalse%
\ {\isacharparenleft}{\kern0pt}metis\ Suc{\isacharunderscore}{\kern0pt}le{\isacharunderscore}{\kern0pt}length{\isacharunderscore}{\kern0pt}iff\ numeral{\isacharunderscore}{\kern0pt}{\isadigit{2}}{\isacharunderscore}{\kern0pt}eq{\isacharunderscore}{\kern0pt}{\isadigit{2}}{\isacharparenright}{\kern0pt}\isanewline
\ \ \ \ \isacommand{then}\isamarkupfalse%
\ \isacommand{have}\isamarkupfalse%
\ g{\isacharprime}{\kern0pt}{\isacharcolon}{\kern0pt}\ {\isachardoublequoteopen}fin{\isacharunderscore}{\kern0pt}connected{\isacharunderscore}{\kern0pt}ulgraph\ V\ {\isacharparenleft}{\kern0pt}edges\ {\isacharminus}{\kern0pt}\ {\isacharbraceleft}{\kern0pt}{\isacharbraceleft}{\kern0pt}u{\isacharcomma}{\kern0pt}v{\isacharbraceright}{\kern0pt}{\isacharbraceright}{\kern0pt}{\isacharparenright}{\kern0pt}{\isachardoublequoteclose}\ \isacommand{using}\isamarkupfalse%
\ finV\ g{\isachardot}{\kern0pt}connected{\isacharunderscore}{\kern0pt}remove{\isacharunderscore}{\kern0pt}cycle{\isacharunderscore}{\kern0pt}edges\isanewline
\ \ \ \ \ \ \isacommand{by}\isamarkupfalse%
\ {\isacharparenleft}{\kern0pt}metis\ connected{\isacharunderscore}{\kern0pt}ulgraph{\isacharunderscore}{\kern0pt}def\ cycle\ fin{\isacharunderscore}{\kern0pt}connected{\isacharunderscore}{\kern0pt}ulgraph{\isacharunderscore}{\kern0pt}def\ fin{\isacharunderscore}{\kern0pt}graph{\isacharunderscore}{\kern0pt}system{\isachardot}{\kern0pt}intro\ fin{\isacharunderscore}{\kern0pt}graph{\isacharunderscore}{\kern0pt}system{\isacharunderscore}{\kern0pt}axioms{\isachardot}{\kern0pt}intro\ fin{\isacharunderscore}{\kern0pt}ulgraph{\isachardot}{\kern0pt}intro\ ulgraph{\isacharunderscore}{\kern0pt}def{\isacharparenright}{\kern0pt}\isanewline
\ \ \ \ \isacommand{have}\isamarkupfalse%
\ {\isachardoublequoteopen}{\isacharbraceleft}{\kern0pt}u{\isacharcomma}{\kern0pt}v{\isacharbraceright}{\kern0pt}\ {\isasymin}\ edges{\isachardoublequoteclose}\ \isacommand{using}\isamarkupfalse%
\ cycle\ \isacommand{unfolding}\isamarkupfalse%
\ c\ g{\isachardot}{\kern0pt}is{\isacharunderscore}{\kern0pt}cycle{\isadigit{2}}{\isacharunderscore}{\kern0pt}def\ g{\isachardot}{\kern0pt}is{\isacharunderscore}{\kern0pt}cycle{\isacharunderscore}{\kern0pt}alt\ g{\isachardot}{\kern0pt}is{\isacharunderscore}{\kern0pt}walk{\isacharunderscore}{\kern0pt}def\ \isacommand{by}\isamarkupfalse%
\ auto\isanewline
\ \ \ \ \isacommand{then}\isamarkupfalse%
\ \isacommand{obtain}\isamarkupfalse%
\ T\ \isakeyword{where}\ {\isachardoublequoteopen}spanning{\isacharunderscore}{\kern0pt}tree\ V\ {\isacharparenleft}{\kern0pt}edges\ {\isacharminus}{\kern0pt}\ {\isacharbraceleft}{\kern0pt}{\isacharbraceleft}{\kern0pt}u{\isacharcomma}{\kern0pt}v{\isacharbraceright}{\kern0pt}{\isacharbraceright}{\kern0pt}{\isacharparenright}{\kern0pt}\ T{\isachardoublequoteclose}\ \isacommand{using}\isamarkupfalse%
\ Suc\ card{\isacharunderscore}{\kern0pt}Diff{\isacharunderscore}{\kern0pt}singleton\ g{\isacharprime}{\kern0pt}\ \isacommand{by}\isamarkupfalse%
\ fastforce\isanewline
\ \ \ \ \isacommand{then}\isamarkupfalse%
\ \isacommand{have}\isamarkupfalse%
\ {\isachardoublequoteopen}spanning{\isacharunderscore}{\kern0pt}tree\ V\ edges\ T{\isachardoublequoteclose}\ \isacommand{unfolding}\isamarkupfalse%
\ spanning{\isacharunderscore}{\kern0pt}tree{\isacharunderscore}{\kern0pt}def\ spanning{\isacharunderscore}{\kern0pt}tree{\isacharunderscore}{\kern0pt}axioms{\isacharunderscore}{\kern0pt}def\ \isacommand{using}\isamarkupfalse%
\ g{\isachardot}{\kern0pt}fin{\isacharunderscore}{\kern0pt}ulgraph{\isacharunderscore}{\kern0pt}axioms\ \isacommand{by}\isamarkupfalse%
\ blast\isanewline
\ \ \ \ \isacommand{then}\isamarkupfalse%
\ \isacommand{show}\isamarkupfalse%
\ {\isacharquery}{\kern0pt}thesis\ \isacommand{by}\isamarkupfalse%
\ blast\isanewline
\ \ \isacommand{qed}\isamarkupfalse%
\isanewline
\isacommand{qed}\isamarkupfalse%
%
\endisatagproof
{\isafoldproof}%
%
\isadelimproof
\isanewline
%
\endisadelimproof
\isanewline
\isacommand{context}\isamarkupfalse%
\ tree\isanewline
\isakeyword{begin}\isanewline
\isanewline
\isacommand{definition}\isamarkupfalse%
\ leaf\ {\isacharcolon}{\kern0pt}{\isacharcolon}{\kern0pt}\ {\isachardoublequoteopen}{\isacharprime}{\kern0pt}a\ {\isasymRightarrow}\ bool{\isachardoublequoteclose}\ \isakeyword{where}\isanewline
\ \ {\isachardoublequoteopen}leaf\ v\ {\isasymlongleftrightarrow}\ degree\ v\ {\isacharequal}{\kern0pt}\ {\isadigit{1}}{\isachardoublequoteclose}\isanewline
\isanewline
\isacommand{definition}\isamarkupfalse%
\ leaves\ {\isacharcolon}{\kern0pt}{\isacharcolon}{\kern0pt}\ {\isachardoublequoteopen}{\isacharprime}{\kern0pt}a\ set{\isachardoublequoteclose}\ \isakeyword{where}\isanewline
\ \ {\isachardoublequoteopen}leaves\ {\isacharequal}{\kern0pt}\ {\isacharbraceleft}{\kern0pt}v{\isachardot}{\kern0pt}\ leaf\ v{\isacharbraceright}{\kern0pt}{\isachardoublequoteclose}\isanewline
\isanewline
\isacommand{definition}\isamarkupfalse%
\ non{\isacharunderscore}{\kern0pt}trivial\ {\isacharcolon}{\kern0pt}{\isacharcolon}{\kern0pt}\ {\isachardoublequoteopen}bool{\isachardoublequoteclose}\ \isakeyword{where}\isanewline
\ \ {\isachardoublequoteopen}non{\isacharunderscore}{\kern0pt}trivial\ {\isasymlongleftrightarrow}\ card\ V\ {\isasymge}\ {\isadigit{2}}{\isachardoublequoteclose}\isanewline
\isanewline
\isacommand{lemma}\isamarkupfalse%
\ obtain{\isacharunderscore}{\kern0pt}{\isadigit{2}}{\isacharunderscore}{\kern0pt}verts{\isacharcolon}{\kern0pt}\isanewline
\ \ \isakeyword{assumes}\ {\isachardoublequoteopen}non{\isacharunderscore}{\kern0pt}trivial{\isachardoublequoteclose}\isanewline
\ \ \isakeyword{obtains}\ u\ v\ \isakeyword{where}\ {\isachardoublequoteopen}u\ {\isasymin}\ V{\isachardoublequoteclose}\ {\isachardoublequoteopen}v\ {\isasymin}\ V{\isachardoublequoteclose}\ {\isachardoublequoteopen}u\ {\isasymnoteq}\ v{\isachardoublequoteclose}\isanewline
%
\isadelimproof
\ \ %
\endisadelimproof
%
\isatagproof
\isacommand{using}\isamarkupfalse%
\ assms\ \isacommand{unfolding}\isamarkupfalse%
\ non{\isacharunderscore}{\kern0pt}trivial{\isacharunderscore}{\kern0pt}def\isanewline
\ \ \isacommand{by}\isamarkupfalse%
\ {\isacharparenleft}{\kern0pt}meson\ diameter{\isacharunderscore}{\kern0pt}obtains{\isacharunderscore}{\kern0pt}path{\isacharunderscore}{\kern0pt}vertices{\isacharparenright}{\kern0pt}%
\endisatagproof
{\isafoldproof}%
%
\isadelimproof
\isanewline
%
\endisadelimproof
\isanewline
\isacommand{lemma}\isamarkupfalse%
\ leaf{\isacharunderscore}{\kern0pt}in{\isacharunderscore}{\kern0pt}V{\isacharcolon}{\kern0pt}\ {\isachardoublequoteopen}leaf\ v\ {\isasymLongrightarrow}\ v\ {\isasymin}\ V{\isachardoublequoteclose}\isanewline
%
\isadelimproof
\ \ %
\endisadelimproof
%
\isatagproof
\isacommand{unfolding}\isamarkupfalse%
\ leaf{\isacharunderscore}{\kern0pt}def\ \isacommand{using}\isamarkupfalse%
\ degree{\isacharunderscore}{\kern0pt}none\ \isacommand{by}\isamarkupfalse%
\ force%
\endisatagproof
{\isafoldproof}%
%
\isadelimproof
\isanewline
%
\endisadelimproof
\isanewline
\isacommand{lemma}\isamarkupfalse%
\ exists{\isacharunderscore}{\kern0pt}leaf{\isacharcolon}{\kern0pt}\isanewline
\ \ \isakeyword{assumes}\ {\isachardoublequoteopen}non{\isacharunderscore}{\kern0pt}trivial{\isachardoublequoteclose}\isanewline
\ \ \isakeyword{shows}\ {\isachardoublequoteopen}{\isasymexists}v{\isachardot}{\kern0pt}\ leaf\ v{\isachardoublequoteclose}\isanewline
%
\isadelimproof
%
\endisadelimproof
%
\isatagproof
\isacommand{proof}\isamarkupfalse%
{\isacharminus}{\kern0pt}\isanewline
\ \ \isacommand{obtain}\isamarkupfalse%
\ p\ \isakeyword{where}\ is{\isacharunderscore}{\kern0pt}path{\isacharcolon}{\kern0pt}\ {\isachardoublequoteopen}is{\isacharunderscore}{\kern0pt}path\ p{\isachardoublequoteclose}\ \isakeyword{and}\ longest{\isacharunderscore}{\kern0pt}path{\isacharcolon}{\kern0pt}\ {\isachardoublequoteopen}{\isasymforall}s{\isachardot}{\kern0pt}\ is{\isacharunderscore}{\kern0pt}path\ s\ {\isasymlongrightarrow}\ length\ s\ {\isasymle}\ length\ p{\isachardoublequoteclose}\isanewline
\ \ \ \ \isacommand{using}\isamarkupfalse%
\ obtain{\isacharunderscore}{\kern0pt}longest{\isacharunderscore}{\kern0pt}path\ \isanewline
\ \ \ \ \isacommand{by}\isamarkupfalse%
\ {\isacharparenleft}{\kern0pt}metis\ One{\isacharunderscore}{\kern0pt}nat{\isacharunderscore}{\kern0pt}def\ assms\ connected\ connected{\isacharunderscore}{\kern0pt}sgraph{\isacharunderscore}{\kern0pt}axioms\ connected{\isacharunderscore}{\kern0pt}sgraph{\isacharunderscore}{\kern0pt}def\ degree{\isacharunderscore}{\kern0pt}{\isadigit{0}}{\isacharunderscore}{\kern0pt}not{\isacharunderscore}{\kern0pt}connected\isanewline
\ \ \ \ \ \ \ \ is{\isacharunderscore}{\kern0pt}connected{\isacharunderscore}{\kern0pt}setD\ is{\isacharunderscore}{\kern0pt}edge{\isacharunderscore}{\kern0pt}or{\isacharunderscore}{\kern0pt}loop\ is{\isacharunderscore}{\kern0pt}isolated{\isacharunderscore}{\kern0pt}vertex{\isacharunderscore}{\kern0pt}def\ is{\isacharunderscore}{\kern0pt}isolated{\isacharunderscore}{\kern0pt}vertex{\isacharunderscore}{\kern0pt}degree{\isadigit{0}}\ is{\isacharunderscore}{\kern0pt}loop{\isacharunderscore}{\kern0pt}def\isanewline
\ \ \ \ \ \ \ \ n{\isacharunderscore}{\kern0pt}not{\isacharunderscore}{\kern0pt}Suc{\isacharunderscore}{\kern0pt}n\ numeral{\isacharunderscore}{\kern0pt}{\isadigit{2}}{\isacharunderscore}{\kern0pt}eq{\isacharunderscore}{\kern0pt}{\isadigit{2}}\ obtain{\isacharunderscore}{\kern0pt}{\isadigit{2}}{\isacharunderscore}{\kern0pt}verts\ sgraph{\isachardot}{\kern0pt}two{\isacharunderscore}{\kern0pt}edges\ vert{\isacharunderscore}{\kern0pt}adj{\isacharunderscore}{\kern0pt}def{\isacharparenright}{\kern0pt}\isanewline
\ \ \isacommand{then}\isamarkupfalse%
\ \isacommand{obtain}\isamarkupfalse%
\ l\ v\ xs\ \isakeyword{where}\ p{\isacharcolon}{\kern0pt}\ {\isachardoublequoteopen}p\ {\isacharequal}{\kern0pt}\ l{\isacharhash}{\kern0pt}v{\isacharhash}{\kern0pt}xs{\isachardoublequoteclose}\isanewline
\ \ \ \ \isacommand{by}\isamarkupfalse%
\ {\isacharparenleft}{\kern0pt}metis\ is{\isacharunderscore}{\kern0pt}open{\isacharunderscore}{\kern0pt}walk{\isacharunderscore}{\kern0pt}def\ is{\isacharunderscore}{\kern0pt}path{\isacharunderscore}{\kern0pt}def\ is{\isacharunderscore}{\kern0pt}walk{\isacharunderscore}{\kern0pt}not{\isacharunderscore}{\kern0pt}empty{\isadigit{2}}\ last{\isacharunderscore}{\kern0pt}ConsL\ list{\isachardot}{\kern0pt}exhaust{\isacharunderscore}{\kern0pt}sel{\isacharparenright}{\kern0pt}\isanewline
\ \ \isacommand{then}\isamarkupfalse%
\ \isacommand{have}\isamarkupfalse%
\ lv{\isacharunderscore}{\kern0pt}incident{\isacharcolon}{\kern0pt}\ {\isachardoublequoteopen}{\isacharbraceleft}{\kern0pt}l{\isacharcomma}{\kern0pt}v{\isacharbraceright}{\kern0pt}\ {\isasymin}\ incident{\isacharunderscore}{\kern0pt}edges\ l{\isachardoublequoteclose}\ \isacommand{using}\isamarkupfalse%
\ is{\isacharunderscore}{\kern0pt}path\isanewline
\ \ \ \ \isacommand{unfolding}\isamarkupfalse%
\ incident{\isacharunderscore}{\kern0pt}edges{\isacharunderscore}{\kern0pt}def\ incident{\isacharunderscore}{\kern0pt}def\ is{\isacharunderscore}{\kern0pt}path{\isacharunderscore}{\kern0pt}def\ is{\isacharunderscore}{\kern0pt}open{\isacharunderscore}{\kern0pt}walk{\isacharunderscore}{\kern0pt}def\ is{\isacharunderscore}{\kern0pt}walk{\isacharunderscore}{\kern0pt}def\ \isacommand{by}\isamarkupfalse%
\ simp\isanewline
\ \ \isacommand{have}\isamarkupfalse%
\ {\isachardoublequoteopen}{\isasymAnd}e{\isachardot}{\kern0pt}\ e{\isasymin}E\ {\isasymLongrightarrow}\ e\ {\isasymnoteq}\ {\isacharbraceleft}{\kern0pt}l{\isacharcomma}{\kern0pt}v{\isacharbraceright}{\kern0pt}\ {\isasymLongrightarrow}\ e\ {\isasymnotin}\ incident{\isacharunderscore}{\kern0pt}edges\ l{\isachardoublequoteclose}\isanewline
\ \ \isacommand{proof}\isamarkupfalse%
\isanewline
\ \ \ \ \isacommand{fix}\isamarkupfalse%
\ e\isanewline
\ \ \ \ \isacommand{assume}\isamarkupfalse%
\ e{\isacharunderscore}{\kern0pt}in{\isacharunderscore}{\kern0pt}E{\isacharcolon}{\kern0pt}\ {\isachardoublequoteopen}e\ {\isasymin}\ E{\isachardoublequoteclose}\isanewline
\ \ \ \ \ \ \isakeyword{and}\ not{\isacharunderscore}{\kern0pt}lv{\isacharcolon}{\kern0pt}\ {\isachardoublequoteopen}e\ {\isasymnoteq}\ {\isacharbraceleft}{\kern0pt}l{\isacharcomma}{\kern0pt}v{\isacharbraceright}{\kern0pt}{\isachardoublequoteclose}\isanewline
\ \ \ \ \ \ \isakeyword{and}\ incident{\isacharcolon}{\kern0pt}\ {\isachardoublequoteopen}e\ {\isasymin}\ incident{\isacharunderscore}{\kern0pt}edges\ l{\isachardoublequoteclose}\isanewline
\ \ \ \ \isacommand{obtain}\isamarkupfalse%
\ u\ \isakeyword{where}\ e{\isacharcolon}{\kern0pt}\ {\isachardoublequoteopen}e\ {\isacharequal}{\kern0pt}\ {\isacharbraceleft}{\kern0pt}l{\isacharcomma}{\kern0pt}u{\isacharbraceright}{\kern0pt}{\isachardoublequoteclose}\ \isacommand{using}\isamarkupfalse%
\ e{\isacharunderscore}{\kern0pt}in{\isacharunderscore}{\kern0pt}E\ obtain{\isacharunderscore}{\kern0pt}edge{\isacharunderscore}{\kern0pt}pair{\isacharunderscore}{\kern0pt}adj\ incident\isanewline
\ \ \ \ \ \ \isacommand{unfolding}\isamarkupfalse%
\ incident{\isacharunderscore}{\kern0pt}edges{\isacharunderscore}{\kern0pt}def\ incident{\isacharunderscore}{\kern0pt}def\ \isacommand{by}\isamarkupfalse%
\ auto\isanewline
\ \ \ \ \isacommand{then}\isamarkupfalse%
\ \isacommand{have}\isamarkupfalse%
\ {\isachardoublequoteopen}u\ {\isasymnoteq}\ l{\isachardoublequoteclose}\ \isacommand{using}\isamarkupfalse%
\ e{\isacharunderscore}{\kern0pt}in{\isacharunderscore}{\kern0pt}E\ edge{\isacharunderscore}{\kern0pt}vertices{\isacharunderscore}{\kern0pt}not{\isacharunderscore}{\kern0pt}equal\ \isacommand{by}\isamarkupfalse%
\ blast\isanewline
\ \ \ \ \isacommand{have}\isamarkupfalse%
\ {\isachardoublequoteopen}u\ {\isasymnoteq}\ v{\isachardoublequoteclose}\ \isacommand{using}\isamarkupfalse%
\ e\ not{\isacharunderscore}{\kern0pt}lv\ \isacommand{by}\isamarkupfalse%
\ auto\isanewline
\ \ \ \ \isacommand{have}\isamarkupfalse%
\ u{\isacharunderscore}{\kern0pt}in{\isacharunderscore}{\kern0pt}V{\isacharcolon}{\kern0pt}\ {\isachardoublequoteopen}u\ {\isasymin}\ V{\isachardoublequoteclose}\ \isacommand{using}\isamarkupfalse%
\ e{\isacharunderscore}{\kern0pt}in{\isacharunderscore}{\kern0pt}E\ e\ wellformed\ \isacommand{by}\isamarkupfalse%
\ blast\isanewline
\ \ \ \ \isacommand{then}\isamarkupfalse%
\ \isacommand{show}\isamarkupfalse%
\ False\isanewline
\ \ \ \ \isacommand{proof}\isamarkupfalse%
\ {\isacharparenleft}{\kern0pt}cases\ {\isachardoublequoteopen}u\ {\isasymin}\ set\ p{\isachardoublequoteclose}{\isacharparenright}{\kern0pt}\isanewline
\ \ \ \ \ \ \isacommand{case}\isamarkupfalse%
\ True\isanewline
\ \ \ \ \ \ \isacommand{then}\isamarkupfalse%
\ \isacommand{have}\isamarkupfalse%
\ {\isachardoublequoteopen}u\ {\isasymin}\ set\ xs{\isachardoublequoteclose}\ \isacommand{using}\isamarkupfalse%
\ {\isacartoucheopen}u{\isasymnoteq}l{\isacartoucheclose}\ {\isacartoucheopen}u{\isasymnoteq}v{\isacartoucheclose}\ p\ \isacommand{by}\isamarkupfalse%
\ simp\isanewline
\ \ \ \ \ \ \isacommand{then}\isamarkupfalse%
\ \isacommand{obtain}\isamarkupfalse%
\ ys\ zs\ \isakeyword{where}\ {\isachardoublequoteopen}xs\ {\isacharequal}{\kern0pt}\ ys{\isacharat}{\kern0pt}u{\isacharhash}{\kern0pt}zs{\isachardoublequoteclose}\ \isacommand{by}\isamarkupfalse%
\ {\isacharparenleft}{\kern0pt}meson\ split{\isacharunderscore}{\kern0pt}list{\isacharparenright}{\kern0pt}\isanewline
\ \ \ \ \ \ \isacommand{then}\isamarkupfalse%
\ \isacommand{have}\isamarkupfalse%
\ {\isachardoublequoteopen}is{\isacharunderscore}{\kern0pt}cycle{\isadigit{2}}\ {\isacharparenleft}{\kern0pt}u{\isacharhash}{\kern0pt}l{\isacharhash}{\kern0pt}v{\isacharhash}{\kern0pt}ys{\isacharat}{\kern0pt}{\isacharbrackleft}{\kern0pt}u{\isacharbrackright}{\kern0pt}{\isacharparenright}{\kern0pt}{\isachardoublequoteclose}\isanewline
\ \ \ \ \ \ \ \ \isacommand{using}\isamarkupfalse%
\ is{\isacharunderscore}{\kern0pt}path\ {\isacartoucheopen}u{\isasymnoteq}l{\isacartoucheclose}\ {\isacartoucheopen}u{\isasymnoteq}v{\isacartoucheclose}\ e{\isacharunderscore}{\kern0pt}in{\isacharunderscore}{\kern0pt}E\ distinct{\isacharunderscore}{\kern0pt}edgesI\ walk{\isacharunderscore}{\kern0pt}edges{\isacharunderscore}{\kern0pt}append{\isacharunderscore}{\kern0pt}ss{\isadigit{2}}\ walk{\isacharunderscore}{\kern0pt}edges{\isacharunderscore}{\kern0pt}in{\isacharunderscore}{\kern0pt}verts\isanewline
\ \ \ \ \ \ \ \ \isacommand{unfolding}\isamarkupfalse%
\ is{\isacharunderscore}{\kern0pt}cycle{\isadigit{2}}{\isacharunderscore}{\kern0pt}def\ is{\isacharunderscore}{\kern0pt}cycle{\isacharunderscore}{\kern0pt}def\ p\ is{\isacharunderscore}{\kern0pt}path{\isacharunderscore}{\kern0pt}def\ is{\isacharunderscore}{\kern0pt}closed{\isacharunderscore}{\kern0pt}walk{\isacharunderscore}{\kern0pt}def\ is{\isacharunderscore}{\kern0pt}open{\isacharunderscore}{\kern0pt}walk{\isacharunderscore}{\kern0pt}def\ is{\isacharunderscore}{\kern0pt}walk{\isacharunderscore}{\kern0pt}def\ e\ walk{\isacharunderscore}{\kern0pt}length{\isacharunderscore}{\kern0pt}conv\isanewline
\ \ \ \ \ \ \ \ \isacommand{by}\isamarkupfalse%
\ {\isacharparenleft}{\kern0pt}auto{\isacharcomma}{\kern0pt}\ metis\ insert{\isacharunderscore}{\kern0pt}commute{\isacharcomma}{\kern0pt}\ fastforce{\isacharplus}{\kern0pt}{\isacharparenright}{\kern0pt}\isanewline
\ \ \ \ \ \ \isacommand{then}\isamarkupfalse%
\ \isacommand{show}\isamarkupfalse%
\ {\isacharquery}{\kern0pt}thesis\ \isacommand{using}\isamarkupfalse%
\ no{\isacharunderscore}{\kern0pt}cycles\ \isacommand{by}\isamarkupfalse%
\ blast\isanewline
\ \ \ \ \isacommand{next}\isamarkupfalse%
\isanewline
\ \ \ \ \ \ \isacommand{case}\isamarkupfalse%
\ False\isanewline
\ \ \ \ \ \ \isacommand{then}\isamarkupfalse%
\ \isacommand{have}\isamarkupfalse%
\ {\isachardoublequoteopen}is{\isacharunderscore}{\kern0pt}path\ {\isacharparenleft}{\kern0pt}u{\isacharhash}{\kern0pt}p{\isacharparenright}{\kern0pt}{\isachardoublequoteclose}\ \isacommand{using}\isamarkupfalse%
\ is{\isacharunderscore}{\kern0pt}path\ u{\isacharunderscore}{\kern0pt}in{\isacharunderscore}{\kern0pt}V\ e{\isacharunderscore}{\kern0pt}in{\isacharunderscore}{\kern0pt}E\isanewline
\ \ \ \ \ \ \ \ \isacommand{unfolding}\isamarkupfalse%
\ is{\isacharunderscore}{\kern0pt}path{\isacharunderscore}{\kern0pt}def\ is{\isacharunderscore}{\kern0pt}open{\isacharunderscore}{\kern0pt}walk{\isacharunderscore}{\kern0pt}def\ is{\isacharunderscore}{\kern0pt}walk{\isacharunderscore}{\kern0pt}def\ e\ p\ \isacommand{by}\isamarkupfalse%
\ {\isacharparenleft}{\kern0pt}auto{\isacharcomma}{\kern0pt}\ {\isacharparenleft}{\kern0pt}metis\ insert{\isacharunderscore}{\kern0pt}commute{\isacharparenright}{\kern0pt}{\isacharplus}{\kern0pt}{\isacharparenright}{\kern0pt}\isanewline
\ \ \ \ \ \ \isacommand{then}\isamarkupfalse%
\ \isacommand{show}\isamarkupfalse%
\ False\ \isacommand{using}\isamarkupfalse%
\ longest{\isacharunderscore}{\kern0pt}path\ \isacommand{by}\isamarkupfalse%
\ auto\isanewline
\ \ \ \ \isacommand{qed}\isamarkupfalse%
\isanewline
\ \ \isacommand{qed}\isamarkupfalse%
\isanewline
\ \ \isacommand{then}\isamarkupfalse%
\ \isacommand{have}\isamarkupfalse%
\ {\isachardoublequoteopen}incident{\isacharunderscore}{\kern0pt}edges\ l\ {\isacharequal}{\kern0pt}\ {\isacharbraceleft}{\kern0pt}{\isacharbraceleft}{\kern0pt}l{\isacharcomma}{\kern0pt}v{\isacharbraceright}{\kern0pt}{\isacharbraceright}{\kern0pt}{\isachardoublequoteclose}\ \isacommand{using}\isamarkupfalse%
\ lv{\isacharunderscore}{\kern0pt}incident\ \isacommand{unfolding}\isamarkupfalse%
\ incident{\isacharunderscore}{\kern0pt}edges{\isacharunderscore}{\kern0pt}def\ \isacommand{by}\isamarkupfalse%
\ blast\isanewline
\ \ \isacommand{then}\isamarkupfalse%
\ \isacommand{have}\isamarkupfalse%
\ {\isachardoublequoteopen}leaf\ l{\isachardoublequoteclose}\ \isacommand{unfolding}\isamarkupfalse%
\ leaf{\isacharunderscore}{\kern0pt}def\ alt{\isacharunderscore}{\kern0pt}degree{\isacharunderscore}{\kern0pt}def\ \isacommand{by}\isamarkupfalse%
\ simp\isanewline
\ \ \isacommand{then}\isamarkupfalse%
\ \isacommand{show}\isamarkupfalse%
\ {\isacharquery}{\kern0pt}thesis\ \isacommand{{\isachardot}{\kern0pt}{\isachardot}{\kern0pt}}\isamarkupfalse%
\isanewline
\isacommand{qed}\isamarkupfalse%
%
\endisatagproof
{\isafoldproof}%
%
\isadelimproof
\isanewline
%
\endisadelimproof
\isanewline
\isacommand{lemma}\isamarkupfalse%
\ tree{\isacharunderscore}{\kern0pt}remove{\isacharunderscore}{\kern0pt}leaf{\isacharcolon}{\kern0pt}\isanewline
\ \ \isakeyword{assumes}\ leaf{\isacharcolon}{\kern0pt}\ {\isachardoublequoteopen}leaf\ l{\isachardoublequoteclose}\isanewline
\ \ \ \ \isakeyword{and}\ remove{\isacharunderscore}{\kern0pt}vertex{\isacharcolon}{\kern0pt}\ {\isachardoublequoteopen}remove{\isacharunderscore}{\kern0pt}vertex\ l\ {\isacharequal}{\kern0pt}\ {\isacharparenleft}{\kern0pt}V{\isacharprime}{\kern0pt}{\isacharcomma}{\kern0pt}E{\isacharprime}{\kern0pt}{\isacharparenright}{\kern0pt}{\isachardoublequoteclose}\isanewline
\ \ \isakeyword{shows}\ {\isachardoublequoteopen}tree\ V{\isacharprime}{\kern0pt}\ E{\isacharprime}{\kern0pt}{\isachardoublequoteclose}\isanewline
%
\isadelimproof
%
\endisadelimproof
%
\isatagproof
\isacommand{proof}\isamarkupfalse%
{\isacharminus}{\kern0pt}\isanewline
\ \ \isacommand{interpret}\isamarkupfalse%
\ g{\isacharprime}{\kern0pt}{\isacharcolon}{\kern0pt}\ ulgraph\ V{\isacharprime}{\kern0pt}\ E{\isacharprime}{\kern0pt}\ \isacommand{using}\isamarkupfalse%
\ remove{\isacharunderscore}{\kern0pt}vertex\ wellformed\ edge{\isacharunderscore}{\kern0pt}size\ \isacommand{unfolding}\isamarkupfalse%
\ remove{\isacharunderscore}{\kern0pt}vertex{\isacharunderscore}{\kern0pt}def\ incident{\isacharunderscore}{\kern0pt}def\isanewline
\ \ \ \ \isacommand{by}\isamarkupfalse%
\ {\isacharparenleft}{\kern0pt}unfold{\isacharunderscore}{\kern0pt}locales{\isacharcomma}{\kern0pt}\ auto{\isacharparenright}{\kern0pt}\isanewline
\ \ \isacommand{interpret}\isamarkupfalse%
\ subg{\isacharcolon}{\kern0pt}\ ulsubgraph\ V{\isacharprime}{\kern0pt}\ E{\isacharprime}{\kern0pt}\ V\ E\ \isacommand{using}\isamarkupfalse%
\ subgraph{\isacharunderscore}{\kern0pt}remove{\isacharunderscore}{\kern0pt}vertex\ ulgraph{\isacharunderscore}{\kern0pt}axioms\ remove{\isacharunderscore}{\kern0pt}vertex\isanewline
\ \ \ \ \isacommand{unfolding}\isamarkupfalse%
\ ulsubgraph{\isacharunderscore}{\kern0pt}def\ \isacommand{by}\isamarkupfalse%
\ blast\isanewline
\ \ \isacommand{have}\isamarkupfalse%
\ V{\isacharprime}{\kern0pt}{\isacharcolon}{\kern0pt}\ {\isachardoublequoteopen}V{\isacharprime}{\kern0pt}\ {\isacharequal}{\kern0pt}\ V\ {\isacharminus}{\kern0pt}\ {\isacharbraceleft}{\kern0pt}l{\isacharbraceright}{\kern0pt}{\isachardoublequoteclose}\ \isacommand{using}\isamarkupfalse%
\ remove{\isacharunderscore}{\kern0pt}vertex\ \isacommand{unfolding}\isamarkupfalse%
\ remove{\isacharunderscore}{\kern0pt}vertex{\isacharunderscore}{\kern0pt}def\ \isacommand{by}\isamarkupfalse%
\ blast\isanewline
\ \ \isacommand{have}\isamarkupfalse%
\ E{\isacharprime}{\kern0pt}{\isacharcolon}{\kern0pt}\ {\isachardoublequoteopen}E{\isacharprime}{\kern0pt}\ {\isacharequal}{\kern0pt}\ {\isacharbraceleft}{\kern0pt}e{\isasymin}E{\isachardot}{\kern0pt}\ l\ {\isasymnotin}\ e{\isacharbraceright}{\kern0pt}{\isachardoublequoteclose}\ \isacommand{using}\isamarkupfalse%
\ remove{\isacharunderscore}{\kern0pt}vertex\ \isacommand{unfolding}\isamarkupfalse%
\ remove{\isacharunderscore}{\kern0pt}vertex{\isacharunderscore}{\kern0pt}def\ incident{\isacharunderscore}{\kern0pt}def\ \isacommand{by}\isamarkupfalse%
\ blast\isanewline
\ \ \isacommand{have}\isamarkupfalse%
\ {\isachardoublequoteopen}{\isasymexists}v{\isasymin}V{\isachardot}{\kern0pt}\ v\ {\isasymnoteq}\ l{\isachardoublequoteclose}\ \isacommand{using}\isamarkupfalse%
\ leaf\ \isacommand{unfolding}\isamarkupfalse%
\ leaf{\isacharunderscore}{\kern0pt}def\isanewline
\ \ \ \ \isacommand{by}\isamarkupfalse%
\ {\isacharparenleft}{\kern0pt}metis\ One{\isacharunderscore}{\kern0pt}nat{\isacharunderscore}{\kern0pt}def\ is{\isacharunderscore}{\kern0pt}independent{\isacharunderscore}{\kern0pt}alt\ is{\isacharunderscore}{\kern0pt}isolated{\isacharunderscore}{\kern0pt}vertex{\isacharunderscore}{\kern0pt}def\ is{\isacharunderscore}{\kern0pt}isolated{\isacharunderscore}{\kern0pt}vertex{\isacharunderscore}{\kern0pt}degree{\isadigit{0}}\isanewline
\ \ \ \ \ \ \ \ n{\isacharunderscore}{\kern0pt}not{\isacharunderscore}{\kern0pt}Suc{\isacharunderscore}{\kern0pt}n\ radius{\isacharunderscore}{\kern0pt}obtains\ singletonI\ singleton{\isacharunderscore}{\kern0pt}independent{\isacharunderscore}{\kern0pt}set{\isacharparenright}{\kern0pt}\isanewline
\ \ \isacommand{then}\isamarkupfalse%
\ \isacommand{have}\isamarkupfalse%
\ {\isachardoublequoteopen}V{\isacharprime}{\kern0pt}\ {\isasymnoteq}\ {\isacharbraceleft}{\kern0pt}{\isacharbraceright}{\kern0pt}{\isachardoublequoteclose}\ \isacommand{using}\isamarkupfalse%
\ remove{\isacharunderscore}{\kern0pt}vertex\ \isacommand{unfolding}\isamarkupfalse%
\ remove{\isacharunderscore}{\kern0pt}vertex{\isacharunderscore}{\kern0pt}def\ incident{\isacharunderscore}{\kern0pt}def\ \isacommand{by}\isamarkupfalse%
\ blast\isanewline
\ \ \isacommand{then}\isamarkupfalse%
\ \isacommand{have}\isamarkupfalse%
\ {\isachardoublequoteopen}g{\isacharprime}{\kern0pt}{\isachardot}{\kern0pt}is{\isacharunderscore}{\kern0pt}connected{\isacharunderscore}{\kern0pt}set\ V{\isacharprime}{\kern0pt}{\isachardoublequoteclose}\ \isacommand{using}\isamarkupfalse%
\ connected{\isacharunderscore}{\kern0pt}remove{\isacharunderscore}{\kern0pt}leaf\ leaf\ remove{\isacharunderscore}{\kern0pt}vertex\ \isacommand{unfolding}\isamarkupfalse%
\ leaf{\isacharunderscore}{\kern0pt}def\ \isacommand{by}\isamarkupfalse%
\ blast\isanewline
\ \ \isacommand{then}\isamarkupfalse%
\ \isacommand{show}\isamarkupfalse%
\ {\isacharquery}{\kern0pt}thesis\ \isacommand{using}\isamarkupfalse%
\ {\isacartoucheopen}V{\isacharprime}{\kern0pt}{\isasymnoteq}{\isacharbraceleft}{\kern0pt}{\isacharbraceright}{\kern0pt}{\isacartoucheclose}\ finV\ subg{\isachardot}{\kern0pt}is{\isacharunderscore}{\kern0pt}cycle{\isadigit{2}}\ V{\isacharprime}{\kern0pt}\ E{\isacharprime}{\kern0pt}\ no{\isacharunderscore}{\kern0pt}cycles\ \isacommand{by}\isamarkupfalse%
\ {\isacharparenleft}{\kern0pt}unfold{\isacharunderscore}{\kern0pt}locales{\isacharcomma}{\kern0pt}\ auto{\isacharparenright}{\kern0pt}\isanewline
\isacommand{qed}\isamarkupfalse%
%
\endisatagproof
{\isafoldproof}%
%
\isadelimproof
\isanewline
%
\endisadelimproof
\isanewline
\isacommand{end}\isamarkupfalse%
\isanewline
\isanewline
\isacommand{lemma}\isamarkupfalse%
\ tree{\isacharunderscore}{\kern0pt}induct\ {\isacharbrackleft}{\kern0pt}case{\isacharunderscore}{\kern0pt}names\ singolton\ insert{\isacharcomma}{\kern0pt}\ induct\ set{\isacharcolon}{\kern0pt}\ tree{\isacharbrackright}{\kern0pt}{\isacharcolon}{\kern0pt}\isanewline
\ \ \isakeyword{assumes}\ tree{\isacharcolon}{\kern0pt}\ {\isachardoublequoteopen}tree\ V\ E{\isachardoublequoteclose}\isanewline
\ \ \ \ \isakeyword{and}\ trivial{\isacharcolon}{\kern0pt}\ {\isachardoublequoteopen}{\isasymAnd}v{\isachardot}{\kern0pt}\ tree\ {\isacharbraceleft}{\kern0pt}v{\isacharbraceright}{\kern0pt}\ {\isacharbraceleft}{\kern0pt}{\isacharbraceright}{\kern0pt}\ {\isasymLongrightarrow}\ P\ {\isacharbraceleft}{\kern0pt}v{\isacharbraceright}{\kern0pt}\ {\isacharbraceleft}{\kern0pt}{\isacharbraceright}{\kern0pt}{\isachardoublequoteclose}\isanewline
\ \ \ \ \isakeyword{and}\ insert{\isacharcolon}{\kern0pt}\ {\isachardoublequoteopen}{\isasymAnd}l\ v\ V\ E{\isachardot}{\kern0pt}\ tree\ V\ E\ {\isasymLongrightarrow}\ P\ V\ E\ {\isasymLongrightarrow}\ l\ {\isasymnotin}\ V\ {\isasymLongrightarrow}\ v\ {\isasymin}\ V\ {\isasymLongrightarrow}\ {\isacharbraceleft}{\kern0pt}l{\isacharcomma}{\kern0pt}v{\isacharbraceright}{\kern0pt}\ {\isasymnotin}\ E\ {\isasymLongrightarrow}\ tree{\isachardot}{\kern0pt}leaf\ {\isacharparenleft}{\kern0pt}insert\ {\isacharbraceleft}{\kern0pt}l{\isacharcomma}{\kern0pt}v{\isacharbraceright}{\kern0pt}\ E{\isacharparenright}{\kern0pt}\ l\ {\isasymLongrightarrow}\ P\ {\isacharparenleft}{\kern0pt}insert\ l\ V{\isacharparenright}{\kern0pt}\ {\isacharparenleft}{\kern0pt}insert\ {\isacharbraceleft}{\kern0pt}l{\isacharcomma}{\kern0pt}v{\isacharbraceright}{\kern0pt}\ E{\isacharparenright}{\kern0pt}{\isachardoublequoteclose}\isanewline
\ \ \isakeyword{shows}\ {\isachardoublequoteopen}P\ V\ E{\isachardoublequoteclose}\isanewline
%
\isadelimproof
\ \ %
\endisadelimproof
%
\isatagproof
\isacommand{using}\isamarkupfalse%
\ tree\isanewline
\isacommand{proof}\isamarkupfalse%
\ {\isacharparenleft}{\kern0pt}induction\ {\isachardoublequoteopen}card\ V{\isachardoublequoteclose}\ arbitrary{\isacharcolon}{\kern0pt}\ V\ E{\isacharparenright}{\kern0pt}\isanewline
\ \ \isacommand{case}\isamarkupfalse%
\ {\isadigit{0}}\isanewline
\ \ \isacommand{then}\isamarkupfalse%
\ \isacommand{interpret}\isamarkupfalse%
\ tree\ V\ E\ \isacommand{by}\isamarkupfalse%
\ simp\isanewline
\ \ \isacommand{have}\isamarkupfalse%
\ {\isachardoublequoteopen}V\ {\isacharequal}{\kern0pt}\ {\isacharbraceleft}{\kern0pt}{\isacharbraceright}{\kern0pt}{\isachardoublequoteclose}\ \isacommand{using}\isamarkupfalse%
\ finV\ {\isadigit{0}}{\isacharparenleft}{\kern0pt}{\isadigit{1}}{\isacharparenright}{\kern0pt}\ \isacommand{by}\isamarkupfalse%
\ simp\isanewline
\ \ \isacommand{then}\isamarkupfalse%
\ \isacommand{show}\isamarkupfalse%
\ {\isacharquery}{\kern0pt}case\ \isacommand{using}\isamarkupfalse%
\ not{\isacharunderscore}{\kern0pt}empty\ \isacommand{by}\isamarkupfalse%
\ blast\isanewline
\isacommand{next}\isamarkupfalse%
\isanewline
\ \ \isacommand{case}\isamarkupfalse%
\ {\isacharparenleft}{\kern0pt}Suc\ n{\isacharparenright}{\kern0pt}\isanewline
\ \ \isacommand{then}\isamarkupfalse%
\ \isacommand{interpret}\isamarkupfalse%
\ t{\isacharcolon}{\kern0pt}\ tree\ V\ E\ \isacommand{by}\isamarkupfalse%
\ simp\isanewline
\ \ \isacommand{show}\isamarkupfalse%
\ {\isacharquery}{\kern0pt}case\isanewline
\ \ \isacommand{proof}\isamarkupfalse%
\ {\isacharparenleft}{\kern0pt}cases\ {\isachardoublequoteopen}card\ V\ {\isacharequal}{\kern0pt}\ {\isadigit{1}}{\isachardoublequoteclose}{\isacharparenright}{\kern0pt}\isanewline
\ \ \ \ \isacommand{case}\isamarkupfalse%
\ True\isanewline
\ \ \ \ \isacommand{then}\isamarkupfalse%
\ \isacommand{obtain}\isamarkupfalse%
\ v\ \isakeyword{where}\ V{\isacharcolon}{\kern0pt}\ {\isachardoublequoteopen}V\ {\isacharequal}{\kern0pt}\ {\isacharbraceleft}{\kern0pt}v{\isacharbraceright}{\kern0pt}{\isachardoublequoteclose}\ \isacommand{using}\isamarkupfalse%
\ card{\isacharunderscore}{\kern0pt}{\isadigit{1}}{\isacharunderscore}{\kern0pt}singletonE\ \isacommand{by}\isamarkupfalse%
\ blast\isanewline
\ \ \ \ \isacommand{then}\isamarkupfalse%
\ \isacommand{have}\isamarkupfalse%
\ {\isachardoublequoteopen}E\ {\isacharequal}{\kern0pt}\ {\isacharbraceleft}{\kern0pt}{\isacharbraceright}{\kern0pt}{\isachardoublequoteclose}\isanewline
\ \ \ \ \ \ \isacommand{using}\isamarkupfalse%
\ True\ subset{\isacharunderscore}{\kern0pt}antisym\ t{\isachardot}{\kern0pt}edge{\isacharunderscore}{\kern0pt}incident{\isacharunderscore}{\kern0pt}vert\ t{\isachardot}{\kern0pt}incident{\isacharunderscore}{\kern0pt}def\ t{\isachardot}{\kern0pt}singleton{\isacharunderscore}{\kern0pt}not{\isacharunderscore}{\kern0pt}edge\ t{\isachardot}{\kern0pt}wellformed\isanewline
\ \ \ \ \ \ \isacommand{by}\isamarkupfalse%
\ fastforce\isanewline
\ \ \ \ \isacommand{then}\isamarkupfalse%
\ \isacommand{show}\isamarkupfalse%
\ {\isacharquery}{\kern0pt}thesis\ \isacommand{using}\isamarkupfalse%
\ trivial\ t{\isachardot}{\kern0pt}tree{\isacharunderscore}{\kern0pt}axioms\ V\ \isacommand{by}\isamarkupfalse%
\ simp\isanewline
\ \ \isacommand{next}\isamarkupfalse%
\isanewline
\ \ \ \ \isacommand{case}\isamarkupfalse%
\ False\isanewline
\ \ \ \ \isacommand{thm}\isamarkupfalse%
\ graph{\isacharunderscore}{\kern0pt}system{\isachardot}{\kern0pt}incident{\isacharunderscore}{\kern0pt}edges{\isacharunderscore}{\kern0pt}def\isanewline
\ \ \ \ \isacommand{then}\isamarkupfalse%
\ \isacommand{have}\isamarkupfalse%
\ card{\isacharunderscore}{\kern0pt}V{\isacharcolon}{\kern0pt}\ {\isachardoublequoteopen}card\ V\ {\isasymge}\ {\isadigit{2}}{\isachardoublequoteclose}\ \isacommand{using}\isamarkupfalse%
\ Suc\ \isacommand{by}\isamarkupfalse%
\ simp\isanewline
\ \ \ \ \isacommand{then}\isamarkupfalse%
\ \isacommand{obtain}\isamarkupfalse%
\ l\ \isakeyword{where}\ leaf{\isacharcolon}{\kern0pt}\ {\isachardoublequoteopen}t{\isachardot}{\kern0pt}leaf\ l{\isachardoublequoteclose}\ \isacommand{using}\isamarkupfalse%
\ t{\isachardot}{\kern0pt}exists{\isacharunderscore}{\kern0pt}leaf\ t{\isachardot}{\kern0pt}non{\isacharunderscore}{\kern0pt}trivial{\isacharunderscore}{\kern0pt}def\ \isacommand{by}\isamarkupfalse%
\ blast\isanewline
\ \ \ \ \isacommand{then}\isamarkupfalse%
\ \isacommand{obtain}\isamarkupfalse%
\ e\ \isakeyword{where}\ inc{\isacharunderscore}{\kern0pt}edges{\isacharcolon}{\kern0pt}\ {\isachardoublequoteopen}t{\isachardot}{\kern0pt}incident{\isacharunderscore}{\kern0pt}edges\ l\ {\isacharequal}{\kern0pt}\ {\isacharbraceleft}{\kern0pt}e{\isacharbraceright}{\kern0pt}{\isachardoublequoteclose}\isanewline
\ \ \ \ \ \ \isacommand{unfolding}\isamarkupfalse%
\ t{\isachardot}{\kern0pt}leaf{\isacharunderscore}{\kern0pt}def\ t{\isachardot}{\kern0pt}alt{\isacharunderscore}{\kern0pt}degree{\isacharunderscore}{\kern0pt}def\ \isacommand{using}\isamarkupfalse%
\ card{\isacharunderscore}{\kern0pt}{\isadigit{1}}{\isacharunderscore}{\kern0pt}singletonE\ \isacommand{by}\isamarkupfalse%
\ blast\isanewline
\ \ \ \ \isacommand{then}\isamarkupfalse%
\ \isacommand{have}\isamarkupfalse%
\ e{\isacharunderscore}{\kern0pt}in{\isacharunderscore}{\kern0pt}E{\isacharcolon}{\kern0pt}\ {\isachardoublequoteopen}e\ {\isasymin}\ E{\isachardoublequoteclose}\ \isacommand{unfolding}\isamarkupfalse%
\ t{\isachardot}{\kern0pt}incident{\isacharunderscore}{\kern0pt}edges{\isacharunderscore}{\kern0pt}def\ \isacommand{by}\isamarkupfalse%
\ blast\isanewline
\ \ \ \ \isacommand{then}\isamarkupfalse%
\ \isacommand{obtain}\isamarkupfalse%
\ u\ \isakeyword{where}\ e{\isacharcolon}{\kern0pt}\ {\isachardoublequoteopen}e\ {\isacharequal}{\kern0pt}\ {\isacharbraceleft}{\kern0pt}l{\isacharcomma}{\kern0pt}u{\isacharbraceright}{\kern0pt}{\isachardoublequoteclose}\ \isacommand{using}\isamarkupfalse%
\ t{\isachardot}{\kern0pt}two{\isacharunderscore}{\kern0pt}edges\ card{\isacharunderscore}{\kern0pt}{\isadigit{2}}{\isacharunderscore}{\kern0pt}iff\ inc{\isacharunderscore}{\kern0pt}edges\ \isacommand{unfolding}\isamarkupfalse%
\ t{\isachardot}{\kern0pt}incident{\isacharunderscore}{\kern0pt}edges{\isacharunderscore}{\kern0pt}def\ t{\isachardot}{\kern0pt}incident{\isacharunderscore}{\kern0pt}def\isanewline
\ \ \ \ \ \ \isacommand{by}\isamarkupfalse%
\ {\isacharparenleft}{\kern0pt}metis\ {\isacharparenleft}{\kern0pt}no{\isacharunderscore}{\kern0pt}types{\isacharcomma}{\kern0pt}\ lifting{\isacharparenright}{\kern0pt}\ empty{\isacharunderscore}{\kern0pt}iff\ insert{\isacharunderscore}{\kern0pt}commute\ insert{\isacharunderscore}{\kern0pt}iff\ mem{\isacharunderscore}{\kern0pt}Collect{\isacharunderscore}{\kern0pt}eq{\isacharparenright}{\kern0pt}\isanewline
\ \ \ \ \isacommand{then}\isamarkupfalse%
\ \isacommand{have}\isamarkupfalse%
\ {\isachardoublequoteopen}l\ {\isasymnoteq}\ u{\isachardoublequoteclose}\ \isacommand{using}\isamarkupfalse%
\ e{\isacharunderscore}{\kern0pt}in{\isacharunderscore}{\kern0pt}E\ t{\isachardot}{\kern0pt}edge{\isacharunderscore}{\kern0pt}vertices{\isacharunderscore}{\kern0pt}not{\isacharunderscore}{\kern0pt}equal\ \isacommand{by}\isamarkupfalse%
\ blast\isanewline
\ \ \ \ \isacommand{have}\isamarkupfalse%
\ {\isachardoublequoteopen}u\ {\isasymin}\ V{\isachardoublequoteclose}\ \isacommand{using}\isamarkupfalse%
\ e\ e{\isacharunderscore}{\kern0pt}in{\isacharunderscore}{\kern0pt}E\ t{\isachardot}{\kern0pt}wellformed\ \isacommand{by}\isamarkupfalse%
\ blast\isanewline
\ \ \ \ \isacommand{let}\isamarkupfalse%
\ {\isacharquery}{\kern0pt}V{\isacharprime}{\kern0pt}\ {\isacharequal}{\kern0pt}\ {\isachardoublequoteopen}V\ {\isacharminus}{\kern0pt}\ {\isacharbraceleft}{\kern0pt}l{\isacharbraceright}{\kern0pt}{\isachardoublequoteclose}\isanewline
\ \ \ \ \isacommand{let}\isamarkupfalse%
\ {\isacharquery}{\kern0pt}E{\isacharprime}{\kern0pt}\ {\isacharequal}{\kern0pt}\ {\isachardoublequoteopen}E\ {\isacharminus}{\kern0pt}\ {\isacharbraceleft}{\kern0pt}{\isacharbraceleft}{\kern0pt}l{\isacharcomma}{\kern0pt}u{\isacharbraceright}{\kern0pt}{\isacharbraceright}{\kern0pt}{\isachardoublequoteclose}\isanewline
\ \ \ \ \isacommand{have}\isamarkupfalse%
\ remove{\isacharunderscore}{\kern0pt}vertex{\isacharcolon}{\kern0pt}\ {\isachardoublequoteopen}t{\isachardot}{\kern0pt}remove{\isacharunderscore}{\kern0pt}vertex\ l\ {\isacharequal}{\kern0pt}\ {\isacharparenleft}{\kern0pt}{\isacharquery}{\kern0pt}V{\isacharprime}{\kern0pt}{\isacharcomma}{\kern0pt}\ {\isacharquery}{\kern0pt}E{\isacharprime}{\kern0pt}{\isacharparenright}{\kern0pt}{\isachardoublequoteclose}\isanewline
\ \ \ \ \ \ \isacommand{using}\isamarkupfalse%
\ inc{\isacharunderscore}{\kern0pt}edges\ e\ \isacommand{unfolding}\isamarkupfalse%
\ t{\isachardot}{\kern0pt}remove{\isacharunderscore}{\kern0pt}vertex{\isacharunderscore}{\kern0pt}def\ t{\isachardot}{\kern0pt}incident{\isacharunderscore}{\kern0pt}edges{\isacharunderscore}{\kern0pt}def\ \isacommand{by}\isamarkupfalse%
\ blast\isanewline
\ \ \ \ \isacommand{then}\isamarkupfalse%
\ \isacommand{have}\isamarkupfalse%
\ t{\isacharprime}{\kern0pt}{\isacharcolon}{\kern0pt}\ {\isachardoublequoteopen}tree\ {\isacharquery}{\kern0pt}V{\isacharprime}{\kern0pt}\ {\isacharquery}{\kern0pt}E{\isacharprime}{\kern0pt}{\isachardoublequoteclose}\ \isacommand{using}\isamarkupfalse%
\ t{\isachardot}{\kern0pt}tree{\isacharunderscore}{\kern0pt}remove{\isacharunderscore}{\kern0pt}leaf\ leaf\ \isacommand{by}\isamarkupfalse%
\ blast\isanewline
\ \ \ \ \isacommand{have}\isamarkupfalse%
\ {\isachardoublequoteopen}l\ {\isasymin}\ V{\isachardoublequoteclose}\ \isacommand{using}\isamarkupfalse%
\ leaf\ t{\isachardot}{\kern0pt}leaf{\isacharunderscore}{\kern0pt}in{\isacharunderscore}{\kern0pt}V\ \isacommand{by}\isamarkupfalse%
\ blast\isanewline
\ \ \ \ \isacommand{then}\isamarkupfalse%
\ \isacommand{have}\isamarkupfalse%
\ P{\isacharprime}{\kern0pt}{\isacharcolon}{\kern0pt}\ {\isachardoublequoteopen}P\ {\isacharquery}{\kern0pt}V{\isacharprime}{\kern0pt}\ {\isacharquery}{\kern0pt}E{\isacharprime}{\kern0pt}{\isachardoublequoteclose}\ \isacommand{using}\isamarkupfalse%
\ Suc\ t{\isacharprime}{\kern0pt}\ \isacommand{by}\isamarkupfalse%
\ auto\isanewline
\ \ \ \ \isacommand{show}\isamarkupfalse%
\ {\isacharquery}{\kern0pt}thesis\ \isacommand{using}\isamarkupfalse%
\ insert{\isacharbrackleft}{\kern0pt}OF\ t{\isacharprime}{\kern0pt}\ P{\isacharprime}{\kern0pt}{\isacharbrackright}{\kern0pt}\ Suc\ leaf\ {\isacartoucheopen}u{\isasymin}V{\isacartoucheclose}\ {\isacartoucheopen}l{\isasymnoteq}u{\isacartoucheclose}\ {\isacartoucheopen}l\ {\isasymin}\ V{\isacartoucheclose}\ e\ e{\isacharunderscore}{\kern0pt}in{\isacharunderscore}{\kern0pt}E\ \isacommand{by}\isamarkupfalse%
\ {\isacharparenleft}{\kern0pt}auto{\isacharcomma}{\kern0pt}\ metis\ insert{\isacharunderscore}{\kern0pt}Diff{\isacharparenright}{\kern0pt}\isanewline
\ \ \isacommand{qed}\isamarkupfalse%
\isanewline
\isacommand{qed}\isamarkupfalse%
%
\endisatagproof
{\isafoldproof}%
%
\isadelimproof
\isanewline
%
\endisadelimproof
\isanewline
\isacommand{context}\isamarkupfalse%
\ tree\isanewline
\isakeyword{begin}\isanewline
\isanewline
\isacommand{lemma}\isamarkupfalse%
\ card{\isacharunderscore}{\kern0pt}V{\isacharunderscore}{\kern0pt}card{\isacharunderscore}{\kern0pt}E{\isacharcolon}{\kern0pt}\ {\isachardoublequoteopen}card\ V\ {\isacharequal}{\kern0pt}\ Suc\ {\isacharparenleft}{\kern0pt}card\ E{\isacharparenright}{\kern0pt}{\isachardoublequoteclose}\isanewline
%
\isadelimproof
\ \ %
\endisadelimproof
%
\isatagproof
\isacommand{using}\isamarkupfalse%
\ tree{\isacharunderscore}{\kern0pt}axioms\isanewline
\isacommand{proof}\isamarkupfalse%
\ {\isacharparenleft}{\kern0pt}induction\ V\ E{\isacharparenright}{\kern0pt}\isanewline
\ \ \isacommand{case}\isamarkupfalse%
\ {\isacharparenleft}{\kern0pt}singolton\ v{\isacharparenright}{\kern0pt}\isanewline
\ \ \isacommand{then}\isamarkupfalse%
\ \isacommand{show}\isamarkupfalse%
\ {\isacharquery}{\kern0pt}case\ \isacommand{by}\isamarkupfalse%
\ auto\isanewline
\isacommand{next}\isamarkupfalse%
\isanewline
\ \ \isacommand{case}\isamarkupfalse%
\ {\isacharparenleft}{\kern0pt}insert\ l\ v\ V{\isacharprime}{\kern0pt}\ E{\isacharprime}{\kern0pt}{\isacharparenright}{\kern0pt}\isanewline
\ \ \isacommand{then}\isamarkupfalse%
\ \isacommand{interpret}\isamarkupfalse%
\ t{\isacharprime}{\kern0pt}{\isacharcolon}{\kern0pt}\ tree\ V{\isacharprime}{\kern0pt}\ E{\isacharprime}{\kern0pt}\ \isacommand{by}\isamarkupfalse%
\ simp\isanewline
\ \ \isacommand{show}\isamarkupfalse%
\ {\isacharquery}{\kern0pt}case\ \isacommand{using}\isamarkupfalse%
\ t{\isacharprime}{\kern0pt}{\isachardot}{\kern0pt}finV\ t{\isacharprime}{\kern0pt}{\isachardot}{\kern0pt}fin{\isacharunderscore}{\kern0pt}edges\ insert\ \isacommand{by}\isamarkupfalse%
\ simp\isanewline
\isacommand{qed}\isamarkupfalse%
%
\endisatagproof
{\isafoldproof}%
%
\isadelimproof
\isanewline
%
\endisadelimproof
\isanewline
\isacommand{end}\isamarkupfalse%
\isanewline
\isanewline
\isacommand{lemma}\isamarkupfalse%
\ card{\isacharunderscore}{\kern0pt}E{\isacharunderscore}{\kern0pt}treeI{\isacharcolon}{\kern0pt}\isanewline
\ \ \isakeyword{assumes}\ fin{\isacharunderscore}{\kern0pt}conn{\isacharunderscore}{\kern0pt}sgraph{\isacharcolon}{\kern0pt}\ {\isachardoublequoteopen}fin{\isacharunderscore}{\kern0pt}connected{\isacharunderscore}{\kern0pt}ulgraph\ V\ E{\isachardoublequoteclose}\isanewline
\ \ \ \ \isakeyword{and}\ card{\isacharunderscore}{\kern0pt}V{\isacharunderscore}{\kern0pt}E{\isacharcolon}{\kern0pt}\ {\isachardoublequoteopen}card\ V\ {\isacharequal}{\kern0pt}\ Suc\ {\isacharparenleft}{\kern0pt}card\ E{\isacharparenright}{\kern0pt}{\isachardoublequoteclose}\isanewline
\ \ \isakeyword{shows}\ {\isachardoublequoteopen}tree\ V\ E{\isachardoublequoteclose}\isanewline
%
\isadelimproof
%
\endisadelimproof
%
\isatagproof
\isacommand{proof}\isamarkupfalse%
{\isacharminus}{\kern0pt}\isanewline
\ \ \isacommand{interpret}\isamarkupfalse%
\ G{\isacharcolon}{\kern0pt}\ fin{\isacharunderscore}{\kern0pt}connected{\isacharunderscore}{\kern0pt}ulgraph\ V\ E\ \isacommand{using}\isamarkupfalse%
\ fin{\isacharunderscore}{\kern0pt}conn{\isacharunderscore}{\kern0pt}sgraph\ \isacommand{{\isachardot}{\kern0pt}}\isamarkupfalse%
\isanewline
\ \ \isacommand{obtain}\isamarkupfalse%
\ T\ \isakeyword{where}\ T{\isacharcolon}{\kern0pt}\ {\isachardoublequoteopen}spanning{\isacharunderscore}{\kern0pt}tree\ V\ E\ T{\isachardoublequoteclose}\ \isacommand{using}\isamarkupfalse%
\ G{\isachardot}{\kern0pt}has{\isacharunderscore}{\kern0pt}spanning{\isacharunderscore}{\kern0pt}tree\ \isacommand{by}\isamarkupfalse%
\ blast\isanewline
\ \ \isacommand{show}\isamarkupfalse%
\ {\isacharquery}{\kern0pt}thesis\isanewline
\ \ \isacommand{proof}\isamarkupfalse%
\ {\isacharparenleft}{\kern0pt}cases\ {\isachardoublequoteopen}E\ {\isacharequal}{\kern0pt}\ T{\isachardoublequoteclose}{\isacharparenright}{\kern0pt}\isanewline
\ \ \ \ \isacommand{case}\isamarkupfalse%
\ True\isanewline
\ \ \ \ \isacommand{then}\isamarkupfalse%
\ \isacommand{show}\isamarkupfalse%
\ {\isacharquery}{\kern0pt}thesis\ \isacommand{using}\isamarkupfalse%
\ T\ \isacommand{unfolding}\isamarkupfalse%
\ spanning{\isacharunderscore}{\kern0pt}tree{\isacharunderscore}{\kern0pt}def\ \isacommand{by}\isamarkupfalse%
\ blast\isanewline
\ \ \isacommand{next}\isamarkupfalse%
\isanewline
\ \ \ \ \isacommand{case}\isamarkupfalse%
\ False\isanewline
\ \ \ \ \isacommand{then}\isamarkupfalse%
\ \isacommand{have}\isamarkupfalse%
\ {\isachardoublequoteopen}card\ E\ {\isachargreater}{\kern0pt}\ card\ T{\isachardoublequoteclose}\ \isacommand{using}\isamarkupfalse%
\ T\ G{\isachardot}{\kern0pt}fin{\isacharunderscore}{\kern0pt}edges\ \isacommand{unfolding}\isamarkupfalse%
\ spanning{\isacharunderscore}{\kern0pt}tree{\isacharunderscore}{\kern0pt}def\ spanning{\isacharunderscore}{\kern0pt}tree{\isacharunderscore}{\kern0pt}axioms{\isacharunderscore}{\kern0pt}def\isanewline
\ \ \ \ \ \ \isacommand{by}\isamarkupfalse%
\ {\isacharparenleft}{\kern0pt}simp\ add{\isacharcolon}{\kern0pt}\ psubsetI\ psubset{\isacharunderscore}{\kern0pt}card{\isacharunderscore}{\kern0pt}mono{\isacharparenright}{\kern0pt}\isanewline
\ \ \ \ \isacommand{then}\isamarkupfalse%
\ \isacommand{show}\isamarkupfalse%
\ {\isacharquery}{\kern0pt}thesis\ \isacommand{using}\isamarkupfalse%
\ tree{\isachardot}{\kern0pt}card{\isacharunderscore}{\kern0pt}V{\isacharunderscore}{\kern0pt}card{\isacharunderscore}{\kern0pt}E\ T\ card{\isacharunderscore}{\kern0pt}V{\isacharunderscore}{\kern0pt}E\ \isacommand{unfolding}\isamarkupfalse%
\ spanning{\isacharunderscore}{\kern0pt}tree{\isacharunderscore}{\kern0pt}def\ \isacommand{by}\isamarkupfalse%
\ fastforce\isanewline
\ \ \isacommand{qed}\isamarkupfalse%
\isanewline
\isacommand{qed}\isamarkupfalse%
%
\endisatagproof
{\isafoldproof}%
%
\isadelimproof
\isanewline
%
\endisadelimproof
\isanewline
\isacommand{context}\isamarkupfalse%
\ tree\isanewline
\isakeyword{begin}\isanewline
\isanewline
\isacommand{lemma}\isamarkupfalse%
\ add{\isacharunderscore}{\kern0pt}vertex{\isacharunderscore}{\kern0pt}tree{\isacharcolon}{\kern0pt}\isanewline
\ \ \isakeyword{assumes}\ {\isachardoublequoteopen}v\ {\isasymnotin}\ V{\isachardoublequoteclose}\isanewline
\ \ \ \ \isakeyword{and}\ \ {\isachardoublequoteopen}w\ {\isasymin}\ V{\isachardoublequoteclose}\isanewline
\ \ \isakeyword{shows}\ {\isachardoublequoteopen}tree\ {\isacharparenleft}{\kern0pt}insert\ v\ V{\isacharparenright}{\kern0pt}\ {\isacharparenleft}{\kern0pt}insert\ {\isacharbraceleft}{\kern0pt}v{\isacharcomma}{\kern0pt}w{\isacharbraceright}{\kern0pt}\ E{\isacharparenright}{\kern0pt}{\isachardoublequoteclose}\isanewline
%
\isadelimproof
%
\endisadelimproof
%
\isatagproof
\isacommand{proof}\isamarkupfalse%
\ {\isacharminus}{\kern0pt}\isanewline
\ \ \isacommand{let}\isamarkupfalse%
\ {\isacharquery}{\kern0pt}V{\isacharprime}{\kern0pt}\ {\isacharequal}{\kern0pt}\ {\isachardoublequoteopen}insert\ v\ V{\isachardoublequoteclose}\ \isakeyword{and}\ {\isacharquery}{\kern0pt}E{\isacharprime}{\kern0pt}\ {\isacharequal}{\kern0pt}\ {\isachardoublequoteopen}insert\ {\isacharbraceleft}{\kern0pt}v{\isacharcomma}{\kern0pt}w{\isacharbraceright}{\kern0pt}\ E{\isachardoublequoteclose}\isanewline
\isanewline
\ \ \isacommand{have}\isamarkupfalse%
\ cardV{\isacharcolon}{\kern0pt}\ {\isachardoublequoteopen}card\ {\isacharbraceleft}{\kern0pt}v{\isacharcomma}{\kern0pt}w{\isacharbraceright}{\kern0pt}\ {\isacharequal}{\kern0pt}\ {\isadigit{2}}{\isachardoublequoteclose}\ \isacommand{using}\isamarkupfalse%
\ card{\isacharunderscore}{\kern0pt}{\isadigit{2}}{\isacharunderscore}{\kern0pt}iff\ assms\ \isacommand{by}\isamarkupfalse%
\ auto\isanewline
\ \ \isacommand{then}\isamarkupfalse%
\ \isacommand{interpret}\isamarkupfalse%
\ t{\isacharprime}{\kern0pt}{\isacharcolon}{\kern0pt}\ ulgraph\ {\isacharquery}{\kern0pt}V{\isacharprime}{\kern0pt}\ {\isacharquery}{\kern0pt}E{\isacharprime}{\kern0pt}\isanewline
\ \ \ \ \isacommand{using}\isamarkupfalse%
\ wellformed\ assms\ two{\isacharunderscore}{\kern0pt}edges\ \isacommand{by}\isamarkupfalse%
\ {\isacharparenleft}{\kern0pt}unfold{\isacharunderscore}{\kern0pt}locales{\isacharcomma}{\kern0pt}\ auto{\isacharparenright}{\kern0pt}\isanewline
\isanewline
\ \ \isacommand{interpret}\isamarkupfalse%
\ subg{\isacharcolon}{\kern0pt}\ ulsubgraph\ V\ E\ {\isacharquery}{\kern0pt}V{\isacharprime}{\kern0pt}\ {\isacharquery}{\kern0pt}E{\isacharprime}{\kern0pt}\ \isacommand{by}\isamarkupfalse%
\ {\isacharparenleft}{\kern0pt}unfold{\isacharunderscore}{\kern0pt}locales{\isacharcomma}{\kern0pt}\ auto{\isacharparenright}{\kern0pt}\isanewline
\isanewline
\ \ \isacommand{have}\isamarkupfalse%
\ connected{\isacharcolon}{\kern0pt}\ {\isachardoublequoteopen}t{\isacharprime}{\kern0pt}{\isachardot}{\kern0pt}is{\isacharunderscore}{\kern0pt}connected{\isacharunderscore}{\kern0pt}set\ {\isacharquery}{\kern0pt}V{\isacharprime}{\kern0pt}{\isachardoublequoteclose}\isanewline
\ \ \ \ \isacommand{unfolding}\isamarkupfalse%
\ t{\isacharprime}{\kern0pt}{\isachardot}{\kern0pt}is{\isacharunderscore}{\kern0pt}connected{\isacharunderscore}{\kern0pt}set{\isacharunderscore}{\kern0pt}def\isanewline
\ \ \ \ \isacommand{using}\isamarkupfalse%
\ subg{\isachardot}{\kern0pt}vert{\isacharunderscore}{\kern0pt}connected\ t{\isacharprime}{\kern0pt}{\isachardot}{\kern0pt}vert{\isacharunderscore}{\kern0pt}connected{\isacharunderscore}{\kern0pt}neighbors\ t{\isacharprime}{\kern0pt}{\isachardot}{\kern0pt}vert{\isacharunderscore}{\kern0pt}connected{\isacharunderscore}{\kern0pt}trans\isanewline
\ \ \ \ \ \ t{\isacharprime}{\kern0pt}{\isachardot}{\kern0pt}vert{\isacharunderscore}{\kern0pt}connected{\isacharunderscore}{\kern0pt}id\ vertices{\isacharunderscore}{\kern0pt}connected\ t{\isacharprime}{\kern0pt}{\isachardot}{\kern0pt}ulgraph{\isacharunderscore}{\kern0pt}axioms\ ulgraph{\isacharunderscore}{\kern0pt}axioms\ assms\ t{\isacharprime}{\kern0pt}{\isachardot}{\kern0pt}vert{\isacharunderscore}{\kern0pt}connected{\isacharunderscore}{\kern0pt}rev\isanewline
\ \ \ \ \isacommand{by}\isamarkupfalse%
\ {\isacharparenleft}{\kern0pt}auto{\isacharcomma}{\kern0pt}\ metis{\isacharplus}{\kern0pt}{\isacharparenright}{\kern0pt}\isanewline
\isanewline
\ \ \isacommand{then}\isamarkupfalse%
\ \isacommand{have}\isamarkupfalse%
\ fin{\isacharunderscore}{\kern0pt}connected{\isacharunderscore}{\kern0pt}ulgraph{\isacharcolon}{\kern0pt}\ {\isachardoublequoteopen}fin{\isacharunderscore}{\kern0pt}connected{\isacharunderscore}{\kern0pt}ulgraph\ {\isacharquery}{\kern0pt}V{\isacharprime}{\kern0pt}\ {\isacharquery}{\kern0pt}E{\isacharprime}{\kern0pt}{\isachardoublequoteclose}\ \isacommand{using}\isamarkupfalse%
\ finV\ \isacommand{by}\isamarkupfalse%
\ {\isacharparenleft}{\kern0pt}unfold{\isacharunderscore}{\kern0pt}locales{\isacharcomma}{\kern0pt}\ auto{\isacharparenright}{\kern0pt}\isanewline
\isanewline
\ \ \isacommand{from}\isamarkupfalse%
\ assms\ \isacommand{have}\isamarkupfalse%
\ {\isachardoublequoteopen}{\isacharbraceleft}{\kern0pt}v{\isacharcomma}{\kern0pt}w{\isacharbraceright}{\kern0pt}\ {\isasymnotin}\ E{\isachardoublequoteclose}\ \isacommand{using}\isamarkupfalse%
\ wellformed{\isacharunderscore}{\kern0pt}alt{\isacharunderscore}{\kern0pt}fst\ \isacommand{by}\isamarkupfalse%
\ auto\isanewline
\ \ \isacommand{then}\isamarkupfalse%
\ \isacommand{have}\isamarkupfalse%
\ {\isachardoublequoteopen}card\ {\isacharquery}{\kern0pt}E{\isacharprime}{\kern0pt}\ {\isacharequal}{\kern0pt}\ Suc\ {\isacharparenleft}{\kern0pt}card\ E{\isacharparenright}{\kern0pt}{\isachardoublequoteclose}\ \isacommand{using}\isamarkupfalse%
\ fin{\isacharunderscore}{\kern0pt}edges\ card{\isacharunderscore}{\kern0pt}insert{\isacharunderscore}{\kern0pt}if\ \isacommand{by}\isamarkupfalse%
\ auto\isanewline
\ \ \isacommand{then}\isamarkupfalse%
\ \isacommand{have}\isamarkupfalse%
\ {\isachardoublequoteopen}card\ {\isacharquery}{\kern0pt}V{\isacharprime}{\kern0pt}\ {\isacharequal}{\kern0pt}\ Suc\ {\isacharparenleft}{\kern0pt}card\ {\isacharquery}{\kern0pt}E{\isacharprime}{\kern0pt}{\isacharparenright}{\kern0pt}{\isachardoublequoteclose}\ \isacommand{using}\isamarkupfalse%
\ card{\isacharunderscore}{\kern0pt}V{\isacharunderscore}{\kern0pt}card{\isacharunderscore}{\kern0pt}E\ assms\ wellformed{\isacharunderscore}{\kern0pt}alt{\isacharunderscore}{\kern0pt}fst\ finV\ card{\isacharunderscore}{\kern0pt}insert{\isacharunderscore}{\kern0pt}if\ \isacommand{by}\isamarkupfalse%
\ auto\isanewline
\isanewline
\ \ \isacommand{then}\isamarkupfalse%
\ \isacommand{show}\isamarkupfalse%
\ {\isacharquery}{\kern0pt}thesis\ \isacommand{using}\isamarkupfalse%
\ card{\isacharunderscore}{\kern0pt}E{\isacharunderscore}{\kern0pt}treeI\ fin{\isacharunderscore}{\kern0pt}connected{\isacharunderscore}{\kern0pt}ulgraph\ \isacommand{by}\isamarkupfalse%
\ auto\isanewline
\isacommand{qed}\isamarkupfalse%
%
\endisatagproof
{\isafoldproof}%
%
\isadelimproof
\isanewline
%
\endisadelimproof
\isanewline
\isacommand{lemma}\isamarkupfalse%
\ tree{\isacharunderscore}{\kern0pt}connected{\isacharunderscore}{\kern0pt}set{\isacharcolon}{\kern0pt}\isanewline
\ \ \isakeyword{assumes}\ non{\isacharunderscore}{\kern0pt}empty{\isacharcolon}{\kern0pt}\ {\isachardoublequoteopen}V{\isacharprime}{\kern0pt}\ {\isasymnoteq}\ {\isacharbraceleft}{\kern0pt}{\isacharbraceright}{\kern0pt}{\isachardoublequoteclose}\isanewline
\ \ \ \ \isakeyword{and}\ subg{\isacharcolon}{\kern0pt}\ {\isachardoublequoteopen}V{\isacharprime}{\kern0pt}\ {\isasymsubseteq}\ V{\isachardoublequoteclose}\isanewline
\ \ \ \ \isakeyword{and}\ connected{\isacharunderscore}{\kern0pt}V{\isacharprime}{\kern0pt}{\isacharcolon}{\kern0pt}\ {\isachardoublequoteopen}ulgraph{\isachardot}{\kern0pt}is{\isacharunderscore}{\kern0pt}connected{\isacharunderscore}{\kern0pt}set\ V{\isacharprime}{\kern0pt}\ {\isacharparenleft}{\kern0pt}induced{\isacharunderscore}{\kern0pt}edges\ V{\isacharprime}{\kern0pt}{\isacharparenright}{\kern0pt}\ V{\isacharprime}{\kern0pt}{\isachardoublequoteclose}\isanewline
\ \ \isakeyword{shows}\ {\isachardoublequoteopen}tree\ V{\isacharprime}{\kern0pt}\ {\isacharparenleft}{\kern0pt}induced{\isacharunderscore}{\kern0pt}edges\ V{\isacharprime}{\kern0pt}{\isacharparenright}{\kern0pt}{\isachardoublequoteclose}\isanewline
%
\isadelimproof
%
\endisadelimproof
%
\isatagproof
\isacommand{proof}\isamarkupfalse%
{\isacharminus}{\kern0pt}\isanewline
\ \ \isacommand{interpret}\isamarkupfalse%
\ subg{\isacharcolon}{\kern0pt}\ subgraph\ V{\isacharprime}{\kern0pt}\ {\isachardoublequoteopen}induced{\isacharunderscore}{\kern0pt}edges\ V{\isacharprime}{\kern0pt}{\isachardoublequoteclose}\ V\ E\ \isacommand{using}\isamarkupfalse%
\ induced{\isacharunderscore}{\kern0pt}is{\isacharunderscore}{\kern0pt}subgraph\ subg\ \isacommand{by}\isamarkupfalse%
\ simp\isanewline
\ \ \isacommand{interpret}\isamarkupfalse%
\ g{\isacharprime}{\kern0pt}{\isacharcolon}{\kern0pt}\ ulgraph\ V{\isacharprime}{\kern0pt}\ {\isachardoublequoteopen}induced{\isacharunderscore}{\kern0pt}edges\ V{\isacharprime}{\kern0pt}{\isachardoublequoteclose}\ \isacommand{using}\isamarkupfalse%
\ subg{\isachardot}{\kern0pt}is{\isacharunderscore}{\kern0pt}subgraph{\isacharunderscore}{\kern0pt}ulgraph\ ulgraph{\isacharunderscore}{\kern0pt}axioms\ \isacommand{by}\isamarkupfalse%
\ blast\isanewline
\ \ \isacommand{interpret}\isamarkupfalse%
\ subg{\isacharcolon}{\kern0pt}\ ulsubgraph\ V{\isacharprime}{\kern0pt}\ {\isachardoublequoteopen}induced{\isacharunderscore}{\kern0pt}edges\ V{\isacharprime}{\kern0pt}{\isachardoublequoteclose}\ V\ E\ \isacommand{by}\isamarkupfalse%
\ unfold{\isacharunderscore}{\kern0pt}locales\isanewline
\ \ \isacommand{show}\isamarkupfalse%
\ {\isacharquery}{\kern0pt}thesis\ \isacommand{using}\isamarkupfalse%
\ connected{\isacharunderscore}{\kern0pt}V{\isacharprime}{\kern0pt}\ subg{\isachardot}{\kern0pt}is{\isacharunderscore}{\kern0pt}cycle{\isadigit{2}}\ no{\isacharunderscore}{\kern0pt}cycles\ finV\ subg\ non{\isacharunderscore}{\kern0pt}empty\ rev{\isacharunderscore}{\kern0pt}finite{\isacharunderscore}{\kern0pt}subset\ \isacommand{by}\isamarkupfalse%
\ {\isacharparenleft}{\kern0pt}unfold{\isacharunderscore}{\kern0pt}locales{\isacharparenright}{\kern0pt}\ {\isacharparenleft}{\kern0pt}auto{\isacharcomma}{\kern0pt}\ blast{\isacharparenright}{\kern0pt}\isanewline
\isacommand{qed}\isamarkupfalse%
%
\endisatagproof
{\isafoldproof}%
%
\isadelimproof
\isanewline
%
\endisadelimproof
\isanewline
\isacommand{lemma}\isamarkupfalse%
\ unique{\isacharunderscore}{\kern0pt}adj{\isacharunderscore}{\kern0pt}vert{\isacharunderscore}{\kern0pt}removed{\isacharcolon}{\kern0pt}\isanewline
\ \ \isakeyword{assumes}\ {\isachardoublequoteopen}v\ {\isasymin}\ V{\isachardoublequoteclose}\isanewline
\ \ \ \ \isakeyword{and}\ remove{\isacharunderscore}{\kern0pt}vertex{\isacharcolon}{\kern0pt}\ {\isachardoublequoteopen}remove{\isacharunderscore}{\kern0pt}vertex\ v\ {\isacharequal}{\kern0pt}\ {\isacharparenleft}{\kern0pt}V{\isacharprime}{\kern0pt}{\isacharcomma}{\kern0pt}E{\isacharprime}{\kern0pt}{\isacharparenright}{\kern0pt}{\isachardoublequoteclose}\isanewline
\ \ \ \ \isakeyword{and}\ conn{\isacharunderscore}{\kern0pt}component{\isacharcolon}{\kern0pt}\ {\isachardoublequoteopen}C\ {\isasymin}\ ulgraph{\isachardot}{\kern0pt}connected{\isacharunderscore}{\kern0pt}components\ V{\isacharprime}{\kern0pt}\ E{\isacharprime}{\kern0pt}{\isachardoublequoteclose}\isanewline
\ \ \isakeyword{shows}\ {\isachardoublequoteopen}{\isasymexists}{\isacharbang}{\kern0pt}u{\isasymin}C{\isachardot}{\kern0pt}\ vert{\isacharunderscore}{\kern0pt}adj\ v\ u{\isachardoublequoteclose}\isanewline
%
\isadelimproof
%
\endisadelimproof
%
\isatagproof
\isacommand{proof}\isamarkupfalse%
{\isacharminus}{\kern0pt}\isanewline
\ \ \isacommand{interpret}\isamarkupfalse%
\ subg{\isacharcolon}{\kern0pt}\ ulsubgraph\ V{\isacharprime}{\kern0pt}\ E{\isacharprime}{\kern0pt}\ V\ E\ \isacommand{using}\isamarkupfalse%
\ remove{\isacharunderscore}{\kern0pt}vertex\ subgraph{\isacharunderscore}{\kern0pt}remove{\isacharunderscore}{\kern0pt}vertex\ ulgraph{\isacharunderscore}{\kern0pt}axioms\ ulsubgraph{\isachardot}{\kern0pt}intro\ \isacommand{by}\isamarkupfalse%
\ metis\isanewline
\ \ \isacommand{interpret}\isamarkupfalse%
\ g{\isacharprime}{\kern0pt}{\isacharcolon}{\kern0pt}\ ulgraph\ V{\isacharprime}{\kern0pt}\ E{\isacharprime}{\kern0pt}\ \isacommand{using}\isamarkupfalse%
\ subg{\isachardot}{\kern0pt}is{\isacharunderscore}{\kern0pt}subgraph{\isacharunderscore}{\kern0pt}ulgraph\ ulgraph{\isacharunderscore}{\kern0pt}axioms\ \isacommand{by}\isamarkupfalse%
\ simp\isanewline
\ \ \isacommand{obtain}\isamarkupfalse%
\ u\ \isakeyword{where}\ {\isachardoublequoteopen}u\ {\isasymin}\ C{\isachardoublequoteclose}\ \isakeyword{and}\ adj{\isacharunderscore}{\kern0pt}vu{\isacharcolon}{\kern0pt}\ {\isachardoublequoteopen}vert{\isacharunderscore}{\kern0pt}adj\ v\ u{\isachardoublequoteclose}\ \isacommand{using}\isamarkupfalse%
\ exists{\isacharunderscore}{\kern0pt}adj{\isacharunderscore}{\kern0pt}vert{\isacharunderscore}{\kern0pt}removed\ \isacommand{using}\isamarkupfalse%
\ assms\ \isacommand{by}\isamarkupfalse%
\ blast\isanewline
\ \ \isacommand{have}\isamarkupfalse%
\ {\isachardoublequoteopen}w\ {\isacharequal}{\kern0pt}\ u{\isachardoublequoteclose}\ \isakeyword{if}\ {\isachardoublequoteopen}w\ {\isasymin}\ C{\isachardoublequoteclose}\ \isakeyword{and}\ adj{\isacharunderscore}{\kern0pt}vw{\isacharcolon}{\kern0pt}\ {\isachardoublequoteopen}vert{\isacharunderscore}{\kern0pt}adj\ v\ w{\isachardoublequoteclose}\ \isakeyword{for}\ w\isanewline
\ \ \isacommand{proof}\isamarkupfalse%
\ {\isacharparenleft}{\kern0pt}rule\ ccontr{\isacharparenright}{\kern0pt}\isanewline
\ \ \ \ \isacommand{assume}\isamarkupfalse%
\ {\isachardoublequoteopen}w\ {\isasymnoteq}\ u{\isachardoublequoteclose}\isanewline
\ \ \ \ \isacommand{obtain}\isamarkupfalse%
\ p\ \isakeyword{where}\ g{\isacharprime}{\kern0pt}{\isacharunderscore}{\kern0pt}conn{\isacharunderscore}{\kern0pt}path{\isacharcolon}{\kern0pt}\ {\isachardoublequoteopen}g{\isacharprime}{\kern0pt}{\isachardot}{\kern0pt}connecting{\isacharunderscore}{\kern0pt}path\ w\ u\ p{\isachardoublequoteclose}\ \isacommand{using}\isamarkupfalse%
\ {\isacartoucheopen}u{\isasymin}C{\isacartoucheclose}\ {\isacartoucheopen}w{\isasymin}C{\isacartoucheclose}\ conn{\isacharunderscore}{\kern0pt}component\isanewline
\ \ \ \ \ \ \ \ g{\isacharprime}{\kern0pt}{\isachardot}{\kern0pt}connected{\isacharunderscore}{\kern0pt}component{\isacharunderscore}{\kern0pt}connected\ g{\isacharprime}{\kern0pt}{\isachardot}{\kern0pt}is{\isacharunderscore}{\kern0pt}connected{\isacharunderscore}{\kern0pt}setD\ g{\isacharprime}{\kern0pt}{\isachardot}{\kern0pt}vert{\isacharunderscore}{\kern0pt}connected{\isacharunderscore}{\kern0pt}def\ \isacommand{by}\isamarkupfalse%
\ blast\isanewline
\ \ \ \ \isacommand{then}\isamarkupfalse%
\ \isacommand{have}\isamarkupfalse%
\ v{\isacharunderscore}{\kern0pt}notin{\isacharunderscore}{\kern0pt}p{\isacharcolon}{\kern0pt}\ {\isachardoublequoteopen}v\ {\isasymnotin}\ set\ p{\isachardoublequoteclose}\ \isacommand{using}\isamarkupfalse%
\ remove{\isacharunderscore}{\kern0pt}vertex\ \isacommand{unfolding}\isamarkupfalse%
\ g{\isacharprime}{\kern0pt}{\isachardot}{\kern0pt}connecting{\isacharunderscore}{\kern0pt}path{\isacharunderscore}{\kern0pt}def\ g{\isacharprime}{\kern0pt}{\isachardot}{\kern0pt}is{\isacharunderscore}{\kern0pt}gen{\isacharunderscore}{\kern0pt}path{\isacharunderscore}{\kern0pt}def\ g{\isacharprime}{\kern0pt}{\isachardot}{\kern0pt}is{\isacharunderscore}{\kern0pt}walk{\isacharunderscore}{\kern0pt}def\ remove{\isacharunderscore}{\kern0pt}vertex{\isacharunderscore}{\kern0pt}def\ \isacommand{by}\isamarkupfalse%
\ blast\isanewline
\ \ \ \ \isacommand{have}\isamarkupfalse%
\ conn{\isacharunderscore}{\kern0pt}path{\isacharcolon}{\kern0pt}\ {\isachardoublequoteopen}connecting{\isacharunderscore}{\kern0pt}path\ w\ u\ p{\isachardoublequoteclose}\ \isacommand{using}\isamarkupfalse%
\ g{\isacharprime}{\kern0pt}{\isacharunderscore}{\kern0pt}conn{\isacharunderscore}{\kern0pt}path\ subg{\isachardot}{\kern0pt}connecting{\isacharunderscore}{\kern0pt}path\ \isacommand{by}\isamarkupfalse%
\ simp\isanewline
\ \ \ \ \isacommand{then}\isamarkupfalse%
\ \isacommand{obtain}\isamarkupfalse%
\ p{\isacharprime}{\kern0pt}\ \isakeyword{where}\ p{\isacharcolon}{\kern0pt}\ {\isachardoublequoteopen}p\ {\isacharequal}{\kern0pt}\ w\ {\isacharhash}{\kern0pt}\ p{\isacharprime}{\kern0pt}\ {\isacharat}{\kern0pt}\ {\isacharbrackleft}{\kern0pt}u{\isacharbrackright}{\kern0pt}{\isachardoublequoteclose}\ \isacommand{unfolding}\isamarkupfalse%
\ connecting{\isacharunderscore}{\kern0pt}path{\isacharunderscore}{\kern0pt}def\ \isacommand{using}\isamarkupfalse%
\ {\isacartoucheopen}w{\isasymnoteq}u{\isacartoucheclose}\isanewline
\ \ \ \ \ \ \isacommand{by}\isamarkupfalse%
\ {\isacharparenleft}{\kern0pt}metis\ hd{\isacharunderscore}{\kern0pt}Cons{\isacharunderscore}{\kern0pt}tl\ last{\isachardot}{\kern0pt}simps\ last{\isacharunderscore}{\kern0pt}rev\ rev{\isacharunderscore}{\kern0pt}is{\isacharunderscore}{\kern0pt}Nil{\isacharunderscore}{\kern0pt}conv\ snoc{\isacharunderscore}{\kern0pt}eq{\isacharunderscore}{\kern0pt}iff{\isacharunderscore}{\kern0pt}butlast{\isacharparenright}{\kern0pt}\isanewline
\ \ \ \ \isacommand{then}\isamarkupfalse%
\ \isacommand{have}\isamarkupfalse%
\ {\isachardoublequoteopen}walk{\isacharunderscore}{\kern0pt}edges\ {\isacharparenleft}{\kern0pt}v{\isacharhash}{\kern0pt}p{\isacharat}{\kern0pt}{\isacharbrackleft}{\kern0pt}v{\isacharbrackright}{\kern0pt}{\isacharparenright}{\kern0pt}\ {\isacharequal}{\kern0pt}\ {\isacharbraceleft}{\kern0pt}v{\isacharcomma}{\kern0pt}w{\isacharbraceright}{\kern0pt}\ {\isacharhash}{\kern0pt}\ walk{\isacharunderscore}{\kern0pt}edges\ {\isacharparenleft}{\kern0pt}{\isacharparenleft}{\kern0pt}w\ {\isacharhash}{\kern0pt}\ p{\isacharprime}{\kern0pt}{\isacharparenright}{\kern0pt}\ {\isacharat}{\kern0pt}\ {\isacharbrackleft}{\kern0pt}u{\isacharcomma}{\kern0pt}v{\isacharbrackright}{\kern0pt}{\isacharparenright}{\kern0pt}{\isachardoublequoteclose}\ \isacommand{by}\isamarkupfalse%
\ simp\isanewline
\ \ \ \ \isacommand{also}\isamarkupfalse%
\ \isacommand{have}\isamarkupfalse%
\ {\isachardoublequoteopen}{\isasymdots}\ {\isacharequal}{\kern0pt}\ {\isacharbraceleft}{\kern0pt}v{\isacharcomma}{\kern0pt}w{\isacharbraceright}{\kern0pt}\ {\isacharhash}{\kern0pt}\ walk{\isacharunderscore}{\kern0pt}edges\ p\ {\isacharat}{\kern0pt}\ {\isacharbrackleft}{\kern0pt}{\isacharbraceleft}{\kern0pt}u{\isacharcomma}{\kern0pt}v{\isacharbraceright}{\kern0pt}{\isacharbrackright}{\kern0pt}{\isachardoublequoteclose}\ \isacommand{unfolding}\isamarkupfalse%
\ p\ \isacommand{using}\isamarkupfalse%
\ walk{\isacharunderscore}{\kern0pt}edges{\isacharunderscore}{\kern0pt}app\ \isacommand{by}\isamarkupfalse%
\ {\isacharparenleft}{\kern0pt}metis\ Cons{\isacharunderscore}{\kern0pt}eq{\isacharunderscore}{\kern0pt}appendI{\isacharparenright}{\kern0pt}\isanewline
\ \ \ \ \isacommand{finally}\isamarkupfalse%
\ \isacommand{have}\isamarkupfalse%
\ walk{\isacharunderscore}{\kern0pt}edges{\isacharcolon}{\kern0pt}\ {\isachardoublequoteopen}walk{\isacharunderscore}{\kern0pt}edges\ {\isacharparenleft}{\kern0pt}v{\isacharhash}{\kern0pt}p{\isacharat}{\kern0pt}{\isacharbrackleft}{\kern0pt}v{\isacharbrackright}{\kern0pt}{\isacharparenright}{\kern0pt}\ {\isacharequal}{\kern0pt}\ {\isacharbraceleft}{\kern0pt}v{\isacharcomma}{\kern0pt}w{\isacharbraceright}{\kern0pt}\ {\isacharhash}{\kern0pt}\ walk{\isacharunderscore}{\kern0pt}edges\ p\ {\isacharat}{\kern0pt}\ {\isacharbrackleft}{\kern0pt}{\isacharbraceleft}{\kern0pt}v{\isacharcomma}{\kern0pt}u{\isacharbraceright}{\kern0pt}{\isacharbrackright}{\kern0pt}{\isachardoublequoteclose}\ \isacommand{by}\isamarkupfalse%
\ {\isacharparenleft}{\kern0pt}simp\ add{\isacharcolon}{\kern0pt}\ insert{\isacharunderscore}{\kern0pt}commute{\isacharparenright}{\kern0pt}\isanewline
\ \ \ \ \isacommand{then}\isamarkupfalse%
\ \isacommand{have}\isamarkupfalse%
\ {\isachardoublequoteopen}is{\isacharunderscore}{\kern0pt}cycle\ {\isacharparenleft}{\kern0pt}v{\isacharhash}{\kern0pt}p{\isacharat}{\kern0pt}{\isacharbrackleft}{\kern0pt}v{\isacharbrackright}{\kern0pt}{\isacharparenright}{\kern0pt}{\isachardoublequoteclose}\ \isacommand{using}\isamarkupfalse%
\ conn{\isacharunderscore}{\kern0pt}path\ adj{\isacharunderscore}{\kern0pt}vu\ adj{\isacharunderscore}{\kern0pt}vw\ {\isacartoucheopen}w{\isasymnoteq}u{\isacartoucheclose}\ {\isacartoucheopen}v{\isasymin}V{\isacartoucheclose}\ g{\isacharprime}{\kern0pt}{\isachardot}{\kern0pt}walk{\isacharunderscore}{\kern0pt}length{\isacharunderscore}{\kern0pt}conv\ singleton{\isacharunderscore}{\kern0pt}not{\isacharunderscore}{\kern0pt}edge\ v{\isacharunderscore}{\kern0pt}notin{\isacharunderscore}{\kern0pt}p\isanewline
\ \ \ \ \ \ \isacommand{unfolding}\isamarkupfalse%
\ connecting{\isacharunderscore}{\kern0pt}path{\isacharunderscore}{\kern0pt}def\ is{\isacharunderscore}{\kern0pt}cycle{\isacharunderscore}{\kern0pt}def\ is{\isacharunderscore}{\kern0pt}gen{\isacharunderscore}{\kern0pt}path{\isacharunderscore}{\kern0pt}def\ is{\isacharunderscore}{\kern0pt}closed{\isacharunderscore}{\kern0pt}walk{\isacharunderscore}{\kern0pt}def\ is{\isacharunderscore}{\kern0pt}walk{\isacharunderscore}{\kern0pt}def\ p\ vert{\isacharunderscore}{\kern0pt}adj{\isacharunderscore}{\kern0pt}def\ \isacommand{by}\isamarkupfalse%
\ auto\isanewline
\ \ \ \ \isacommand{then}\isamarkupfalse%
\ \isacommand{have}\isamarkupfalse%
\ {\isachardoublequoteopen}is{\isacharunderscore}{\kern0pt}cycle{\isadigit{2}}\ {\isacharparenleft}{\kern0pt}v{\isacharhash}{\kern0pt}p{\isacharat}{\kern0pt}{\isacharbrackleft}{\kern0pt}v{\isacharbrackright}{\kern0pt}{\isacharparenright}{\kern0pt}{\isachardoublequoteclose}\ \isacommand{using}\isamarkupfalse%
\ {\isacartoucheopen}w{\isasymnoteq}u{\isacartoucheclose}\ v{\isacharunderscore}{\kern0pt}notin{\isacharunderscore}{\kern0pt}p\ walk{\isacharunderscore}{\kern0pt}edges{\isacharunderscore}{\kern0pt}in{\isacharunderscore}{\kern0pt}verts\ \isacommand{unfolding}\isamarkupfalse%
\ is{\isacharunderscore}{\kern0pt}cycle{\isadigit{2}}{\isacharunderscore}{\kern0pt}def\ walk{\isacharunderscore}{\kern0pt}edges\isanewline
\ \ \ \ \ \ \isacommand{by}\isamarkupfalse%
\ {\isacharparenleft}{\kern0pt}auto\ simp{\isacharcolon}{\kern0pt}\ doubleton{\isacharunderscore}{\kern0pt}eq{\isacharunderscore}{\kern0pt}iff\ is{\isacharunderscore}{\kern0pt}cycle{\isacharunderscore}{\kern0pt}alt\ distinct{\isacharunderscore}{\kern0pt}edgesI{\isacharparenright}{\kern0pt}\isanewline
\ \ \ \ \isacommand{then}\isamarkupfalse%
\ \isacommand{show}\isamarkupfalse%
\ False\ \isacommand{using}\isamarkupfalse%
\ no{\isacharunderscore}{\kern0pt}cycles\ \isacommand{by}\isamarkupfalse%
\ blast\isanewline
\ \ \isacommand{qed}\isamarkupfalse%
\isanewline
\ \ \isacommand{then}\isamarkupfalse%
\ \isacommand{show}\isamarkupfalse%
\ {\isacharquery}{\kern0pt}thesis\ \isacommand{using}\isamarkupfalse%
\ {\isacartoucheopen}u{\isasymin}C{\isacartoucheclose}\ adj{\isacharunderscore}{\kern0pt}vu\ \isacommand{by}\isamarkupfalse%
\ blast\isanewline
\isacommand{qed}\isamarkupfalse%
%
\endisatagproof
{\isafoldproof}%
%
\isadelimproof
\isanewline
%
\endisadelimproof
\isanewline
\isacommand{lemma}\isamarkupfalse%
\ non{\isacharunderscore}{\kern0pt}trivial{\isacharunderscore}{\kern0pt}card{\isacharunderscore}{\kern0pt}E{\isacharcolon}{\kern0pt}\ {\isachardoublequoteopen}non{\isacharunderscore}{\kern0pt}trivial\ {\isasymLongrightarrow}\ card\ E\ {\isasymge}\ {\isadigit{1}}{\isachardoublequoteclose}\isanewline
%
\isadelimproof
\ \ %
\endisadelimproof
%
\isatagproof
\isacommand{using}\isamarkupfalse%
\ card{\isacharunderscore}{\kern0pt}V{\isacharunderscore}{\kern0pt}card{\isacharunderscore}{\kern0pt}E\ \isacommand{unfolding}\isamarkupfalse%
\ non{\isacharunderscore}{\kern0pt}trivial{\isacharunderscore}{\kern0pt}def\ \isacommand{by}\isamarkupfalse%
\ simp%
\endisatagproof
{\isafoldproof}%
%
\isadelimproof
\isanewline
%
\endisadelimproof
\isanewline
\isacommand{lemma}\isamarkupfalse%
\ V{\isacharunderscore}{\kern0pt}Union{\isacharunderscore}{\kern0pt}E{\isacharcolon}{\kern0pt}\ {\isachardoublequoteopen}non{\isacharunderscore}{\kern0pt}trivial\ {\isasymLongrightarrow}\ V\ {\isacharequal}{\kern0pt}\ {\isasymUnion}E{\isachardoublequoteclose}\isanewline
%
\isadelimproof
\ \ %
\endisadelimproof
%
\isatagproof
\isacommand{using}\isamarkupfalse%
\ tree{\isacharunderscore}{\kern0pt}axioms\isanewline
\isacommand{proof}\isamarkupfalse%
\ {\isacharparenleft}{\kern0pt}induction\ V\ E{\isacharparenright}{\kern0pt}\isanewline
\ \ \isacommand{case}\isamarkupfalse%
\ {\isacharparenleft}{\kern0pt}singolton\ v{\isacharparenright}{\kern0pt}\ \isanewline
\ \ \isacommand{then}\isamarkupfalse%
\ \isacommand{interpret}\isamarkupfalse%
\ t{\isacharcolon}{\kern0pt}\ tree\ {\isachardoublequoteopen}{\isacharbraceleft}{\kern0pt}v{\isacharbraceright}{\kern0pt}{\isachardoublequoteclose}\ {\isachardoublequoteopen}{\isacharbraceleft}{\kern0pt}{\isacharbraceright}{\kern0pt}{\isachardoublequoteclose}\ \isacommand{by}\isamarkupfalse%
\ simp\isanewline
\ \ \isacommand{show}\isamarkupfalse%
\ {\isacharquery}{\kern0pt}case\ \isacommand{using}\isamarkupfalse%
\ singolton\ \isacommand{unfolding}\isamarkupfalse%
\ t{\isachardot}{\kern0pt}non{\isacharunderscore}{\kern0pt}trivial{\isacharunderscore}{\kern0pt}def\ \isacommand{by}\isamarkupfalse%
\ simp\isanewline
\isacommand{next}\isamarkupfalse%
\isanewline
\ \ \isacommand{case}\isamarkupfalse%
\ {\isacharparenleft}{\kern0pt}insert\ l\ v\ V{\isacharprime}{\kern0pt}\ E{\isacharprime}{\kern0pt}{\isacharparenright}{\kern0pt}\isanewline
\ \ \isacommand{then}\isamarkupfalse%
\ \isacommand{interpret}\isamarkupfalse%
\ t{\isacharcolon}{\kern0pt}\ tree\ V{\isacharprime}{\kern0pt}\ E{\isacharprime}{\kern0pt}\ \isacommand{by}\isamarkupfalse%
\ simp\isanewline
\ \ \isacommand{show}\isamarkupfalse%
\ {\isacharquery}{\kern0pt}case\isanewline
\ \ \isacommand{proof}\isamarkupfalse%
\ {\isacharparenleft}{\kern0pt}cases\ {\isachardoublequoteopen}card\ V{\isacharprime}{\kern0pt}\ {\isacharequal}{\kern0pt}\ {\isadigit{1}}{\isachardoublequoteclose}{\isacharparenright}{\kern0pt}\isanewline
\ \ \ \ \isacommand{case}\isamarkupfalse%
\ True\isanewline
\ \ \ \ \isacommand{then}\isamarkupfalse%
\ \isacommand{have}\isamarkupfalse%
\ V{\isacharcolon}{\kern0pt}\ {\isachardoublequoteopen}V{\isacharprime}{\kern0pt}\ {\isacharequal}{\kern0pt}\ {\isacharbraceleft}{\kern0pt}v{\isacharbraceright}{\kern0pt}{\isachardoublequoteclose}\ \isacommand{using}\isamarkupfalse%
\ insert{\isacharparenleft}{\kern0pt}{\isadigit{3}}{\isacharparenright}{\kern0pt}\ card{\isacharunderscore}{\kern0pt}{\isadigit{1}}{\isacharunderscore}{\kern0pt}singletonE\ \isacommand{by}\isamarkupfalse%
\ blast\isanewline
\ \ \ \ \isacommand{then}\isamarkupfalse%
\ \isacommand{have}\isamarkupfalse%
\ E{\isacharcolon}{\kern0pt}\ {\isachardoublequoteopen}E{\isacharprime}{\kern0pt}\ {\isacharequal}{\kern0pt}\ {\isacharbraceleft}{\kern0pt}{\isacharbraceright}{\kern0pt}{\isachardoublequoteclose}\ \isacommand{using}\isamarkupfalse%
\ t{\isachardot}{\kern0pt}fin{\isacharunderscore}{\kern0pt}edges\ t{\isachardot}{\kern0pt}card{\isacharunderscore}{\kern0pt}V{\isacharunderscore}{\kern0pt}card{\isacharunderscore}{\kern0pt}E\ \isacommand{by}\isamarkupfalse%
\ fastforce\isanewline
\ \ \ \ \isacommand{then}\isamarkupfalse%
\ \isacommand{show}\isamarkupfalse%
\ {\isacharquery}{\kern0pt}thesis\ \isacommand{unfolding}\isamarkupfalse%
\ E\ V\ \isacommand{by}\isamarkupfalse%
\ simp\isanewline
\ \ \isacommand{next}\isamarkupfalse%
\isanewline
\ \ \ \ \isacommand{case}\isamarkupfalse%
\ False\isanewline
\ \ \ \ \isacommand{then}\isamarkupfalse%
\ \isacommand{have}\isamarkupfalse%
\ {\isachardoublequoteopen}t{\isachardot}{\kern0pt}non{\isacharunderscore}{\kern0pt}trivial{\isachardoublequoteclose}\ \isacommand{using}\isamarkupfalse%
\ t{\isachardot}{\kern0pt}card{\isacharunderscore}{\kern0pt}V{\isacharunderscore}{\kern0pt}card{\isacharunderscore}{\kern0pt}E\ \isacommand{unfolding}\isamarkupfalse%
\ t{\isachardot}{\kern0pt}non{\isacharunderscore}{\kern0pt}trivial{\isacharunderscore}{\kern0pt}def\ \isacommand{by}\isamarkupfalse%
\ simp\isanewline
\ \ \ \ \isacommand{then}\isamarkupfalse%
\ \isacommand{show}\isamarkupfalse%
\ {\isacharquery}{\kern0pt}thesis\ \isacommand{using}\isamarkupfalse%
\ insert\ \isacommand{by}\isamarkupfalse%
\ blast\isanewline
\ \ \isacommand{qed}\isamarkupfalse%
\isanewline
\isacommand{qed}\isamarkupfalse%
%
\endisatagproof
{\isafoldproof}%
%
\isadelimproof
\isanewline
%
\endisadelimproof
\isanewline
\isacommand{end}\isamarkupfalse%
\isanewline
\isanewline
\isacommand{lemma}\isamarkupfalse%
\ singleton{\isacharunderscore}{\kern0pt}tree{\isacharcolon}{\kern0pt}\ {\isachardoublequoteopen}tree\ {\isacharbraceleft}{\kern0pt}v{\isacharbraceright}{\kern0pt}\ {\isacharbraceleft}{\kern0pt}{\isacharbraceright}{\kern0pt}{\isachardoublequoteclose}\isanewline
%
\isadelimproof
%
\endisadelimproof
%
\isatagproof
\isacommand{proof}\isamarkupfalse%
{\isacharminus}{\kern0pt}\isanewline
\ \ \isacommand{interpret}\isamarkupfalse%
\ g{\isacharcolon}{\kern0pt}\ fin{\isacharunderscore}{\kern0pt}ulgraph\ {\isachardoublequoteopen}{\isacharbraceleft}{\kern0pt}v{\isacharbraceright}{\kern0pt}{\isachardoublequoteclose}\ {\isachardoublequoteopen}{\isacharbraceleft}{\kern0pt}{\isacharbraceright}{\kern0pt}{\isachardoublequoteclose}\ \isacommand{by}\isamarkupfalse%
\ {\isacharparenleft}{\kern0pt}unfold{\isacharunderscore}{\kern0pt}locales{\isacharcomma}{\kern0pt}\ auto{\isacharparenright}{\kern0pt}\isanewline
\ \ \isacommand{show}\isamarkupfalse%
\ {\isacharquery}{\kern0pt}thesis\ \isacommand{using}\isamarkupfalse%
\ g{\isachardot}{\kern0pt}is{\isacharunderscore}{\kern0pt}walk{\isacharunderscore}{\kern0pt}def\ g{\isachardot}{\kern0pt}walk{\isacharunderscore}{\kern0pt}length{\isacharunderscore}{\kern0pt}def\ \isacommand{by}\isamarkupfalse%
\ {\isacharparenleft}{\kern0pt}unfold{\isacharunderscore}{\kern0pt}locales{\isacharcomma}{\kern0pt}\ auto\ simp{\isacharcolon}{\kern0pt}\ g{\isachardot}{\kern0pt}is{\isacharunderscore}{\kern0pt}connected{\isacharunderscore}{\kern0pt}set{\isacharunderscore}{\kern0pt}singleton\ g{\isachardot}{\kern0pt}is{\isacharunderscore}{\kern0pt}cycle{\isadigit{2}}{\isacharunderscore}{\kern0pt}def\ g{\isachardot}{\kern0pt}is{\isacharunderscore}{\kern0pt}cycle{\isacharunderscore}{\kern0pt}alt{\isacharparenright}{\kern0pt}\isanewline
\isacommand{qed}\isamarkupfalse%
%
\endisatagproof
{\isafoldproof}%
%
\isadelimproof
\isanewline
%
\endisadelimproof
\isanewline
\isacommand{locale}\isamarkupfalse%
\ graph{\isacharunderscore}{\kern0pt}isomorphism\ {\isacharequal}{\kern0pt}\isanewline
\ \ G{\isacharcolon}{\kern0pt}\ graph{\isacharunderscore}{\kern0pt}system\ V\isactrlsub G\ E\isactrlsub G\ \isakeyword{for}\ V\isactrlsub G\ E\isactrlsub G\ {\isacharplus}{\kern0pt}\isanewline
\ \ \isakeyword{fixes}\ V\isactrlsub H\ E\isactrlsub H\ f\isanewline
\ \ \isakeyword{assumes}\ bij{\isacharunderscore}{\kern0pt}f{\isacharcolon}{\kern0pt}\ {\isachardoublequoteopen}bij{\isacharunderscore}{\kern0pt}betw\ f\ V\isactrlsub G\ V\isactrlsub H{\isachardoublequoteclose}\isanewline
\ \ \isakeyword{and}\ edge{\isacharunderscore}{\kern0pt}preserving{\isacharcolon}{\kern0pt}\ {\isachardoublequoteopen}{\isacharparenleft}{\kern0pt}{\isacharparenleft}{\kern0pt}{\isacharbackquote}{\kern0pt}{\isacharparenright}{\kern0pt}\ f{\isacharparenright}{\kern0pt}\ {\isacharbackquote}{\kern0pt}\ E\isactrlsub G\ {\isacharequal}{\kern0pt}\ E\isactrlsub H{\isachardoublequoteclose}\isanewline
\isakeyword{begin}\isanewline
\isanewline
\isacommand{lemma}\isamarkupfalse%
\ inj{\isacharunderscore}{\kern0pt}f{\isacharcolon}{\kern0pt}\ {\isachardoublequoteopen}inj{\isacharunderscore}{\kern0pt}on\ f\ V\isactrlsub G{\isachardoublequoteclose}\isanewline
%
\isadelimproof
\ \ %
\endisadelimproof
%
\isatagproof
\isacommand{using}\isamarkupfalse%
\ bij{\isacharunderscore}{\kern0pt}f\ \isacommand{unfolding}\isamarkupfalse%
\ bij{\isacharunderscore}{\kern0pt}betw{\isacharunderscore}{\kern0pt}def\ \isacommand{by}\isamarkupfalse%
\ blast%
\endisatagproof
{\isafoldproof}%
%
\isadelimproof
\isanewline
%
\endisadelimproof
\isanewline
\isacommand{lemma}\isamarkupfalse%
\ V\isactrlsub H{\isacharunderscore}{\kern0pt}def{\isacharcolon}{\kern0pt}\ {\isachardoublequoteopen}V\isactrlsub H\ {\isacharequal}{\kern0pt}\ f\ {\isacharbackquote}{\kern0pt}\ V\isactrlsub G{\isachardoublequoteclose}\isanewline
%
\isadelimproof
\ \ %
\endisadelimproof
%
\isatagproof
\isacommand{using}\isamarkupfalse%
\ bij{\isacharunderscore}{\kern0pt}f\ \isacommand{unfolding}\isamarkupfalse%
\ bij{\isacharunderscore}{\kern0pt}betw{\isacharunderscore}{\kern0pt}def\ \isacommand{by}\isamarkupfalse%
\ blast%
\endisatagproof
{\isafoldproof}%
%
\isadelimproof
\isanewline
%
\endisadelimproof
\isanewline
\isacommand{definition}\isamarkupfalse%
\ {\isachardoublequoteopen}inv{\isacharunderscore}{\kern0pt}iso\ {\isasymequiv}\ the{\isacharunderscore}{\kern0pt}inv{\isacharunderscore}{\kern0pt}into\ V\isactrlsub G\ f{\isachardoublequoteclose}\isanewline
\isanewline
\isacommand{lemma}\isamarkupfalse%
\ graph{\isacharunderscore}{\kern0pt}system{\isacharunderscore}{\kern0pt}H{\isacharcolon}{\kern0pt}\ {\isachardoublequoteopen}graph{\isacharunderscore}{\kern0pt}system\ V\isactrlsub H\ E\isactrlsub H{\isachardoublequoteclose}\isanewline
%
\isadelimproof
\ \ %
\endisadelimproof
%
\isatagproof
\isacommand{using}\isamarkupfalse%
\ G{\isachardot}{\kern0pt}wellformed\ edge{\isacharunderscore}{\kern0pt}preserving\ bij{\isacharunderscore}{\kern0pt}f\ bij{\isacharunderscore}{\kern0pt}betw{\isacharunderscore}{\kern0pt}imp{\isacharunderscore}{\kern0pt}surj{\isacharunderscore}{\kern0pt}on\ \isacommand{by}\isamarkupfalse%
\ unfold{\isacharunderscore}{\kern0pt}locales\ blast%
\endisatagproof
{\isafoldproof}%
%
\isadelimproof
\isanewline
%
\endisadelimproof
\isanewline
\isacommand{interpretation}\isamarkupfalse%
\ H{\isacharcolon}{\kern0pt}\ graph{\isacharunderscore}{\kern0pt}system\ V\isactrlsub H\ E\isactrlsub H%
\isadelimproof
\ %
\endisadelimproof
%
\isatagproof
\isacommand{using}\isamarkupfalse%
\ graph{\isacharunderscore}{\kern0pt}system{\isacharunderscore}{\kern0pt}H\ \isacommand{{\isachardot}{\kern0pt}}\isamarkupfalse%
%
\endisatagproof
{\isafoldproof}%
%
\isadelimproof
%
\endisadelimproof
\isanewline
\isanewline
\isacommand{lemma}\isamarkupfalse%
\ graph{\isacharunderscore}{\kern0pt}isomorphism{\isacharunderscore}{\kern0pt}inv{\isacharcolon}{\kern0pt}\ {\isachardoublequoteopen}graph{\isacharunderscore}{\kern0pt}isomorphism\ V\isactrlsub H\ E\isactrlsub H\ V\isactrlsub G\ E\isactrlsub G\ inv{\isacharunderscore}{\kern0pt}iso{\isachardoublequoteclose}\isanewline
%
\isadelimproof
%
\endisadelimproof
%
\isatagproof
\isacommand{proof}\isamarkupfalse%
\ {\isacharparenleft}{\kern0pt}unfold{\isacharunderscore}{\kern0pt}locales{\isacharparenright}{\kern0pt}\isanewline
\ \ \isacommand{show}\isamarkupfalse%
\ {\isachardoublequoteopen}bij{\isacharunderscore}{\kern0pt}betw\ inv{\isacharunderscore}{\kern0pt}iso\ V\isactrlsub H\ V\isactrlsub G{\isachardoublequoteclose}\ \isacommand{unfolding}\isamarkupfalse%
\ inv{\isacharunderscore}{\kern0pt}iso{\isacharunderscore}{\kern0pt}def\ \isacommand{using}\isamarkupfalse%
\ bij{\isacharunderscore}{\kern0pt}betw{\isacharunderscore}{\kern0pt}the{\isacharunderscore}{\kern0pt}inv{\isacharunderscore}{\kern0pt}into\ bij{\isacharunderscore}{\kern0pt}f\ \isacommand{by}\isamarkupfalse%
\ blast\isanewline
\isacommand{next}\isamarkupfalse%
\isanewline
\ \ \isacommand{have}\isamarkupfalse%
\ {\isachardoublequoteopen}{\isasymforall}v{\isasymin}V\isactrlsub G{\isachardot}{\kern0pt}\ the{\isacharunderscore}{\kern0pt}inv{\isacharunderscore}{\kern0pt}into\ V\isactrlsub G\ f\ {\isacharparenleft}{\kern0pt}f\ v{\isacharparenright}{\kern0pt}\ {\isacharequal}{\kern0pt}\ v{\isachardoublequoteclose}\ \isacommand{using}\isamarkupfalse%
\ bij{\isacharunderscore}{\kern0pt}f\ \isacommand{by}\isamarkupfalse%
\ {\isacharparenleft}{\kern0pt}simp\ add{\isacharcolon}{\kern0pt}\ bij{\isacharunderscore}{\kern0pt}betw{\isacharunderscore}{\kern0pt}imp{\isacharunderscore}{\kern0pt}inj{\isacharunderscore}{\kern0pt}on\ the{\isacharunderscore}{\kern0pt}inv{\isacharunderscore}{\kern0pt}into{\isacharunderscore}{\kern0pt}f{\isacharunderscore}{\kern0pt}f{\isacharparenright}{\kern0pt}\isanewline
\ \ \isacommand{then}\isamarkupfalse%
\ \isacommand{have}\isamarkupfalse%
\ {\isachardoublequoteopen}{\isasymforall}e{\isasymin}E\isactrlsub G{\isachardot}{\kern0pt}\ {\isacharparenleft}{\kern0pt}{\isasymlambda}v{\isachardot}{\kern0pt}\ the{\isacharunderscore}{\kern0pt}inv{\isacharunderscore}{\kern0pt}into\ V\isactrlsub G\ f\ {\isacharparenleft}{\kern0pt}f\ v{\isacharparenright}{\kern0pt}{\isacharparenright}{\kern0pt}\ {\isacharbackquote}{\kern0pt}\ e\ {\isacharequal}{\kern0pt}\ e{\isachardoublequoteclose}\ \isacommand{using}\isamarkupfalse%
\ G{\isachardot}{\kern0pt}wellformed\isanewline
\ \ \ \ \isacommand{by}\isamarkupfalse%
\ {\isacharparenleft}{\kern0pt}simp\ add{\isacharcolon}{\kern0pt}\ subset{\isacharunderscore}{\kern0pt}iff{\isacharparenright}{\kern0pt}\isanewline
\ \ \isacommand{then}\isamarkupfalse%
\ \isacommand{show}\isamarkupfalse%
\ {\isachardoublequoteopen}{\isacharparenleft}{\kern0pt}{\isacharparenleft}{\kern0pt}{\isacharbackquote}{\kern0pt}{\isacharparenright}{\kern0pt}\ inv{\isacharunderscore}{\kern0pt}iso{\isacharparenright}{\kern0pt}{\isacharbackquote}{\kern0pt}\ E\isactrlsub H\ {\isacharequal}{\kern0pt}\ E\isactrlsub G{\isachardoublequoteclose}\ \isacommand{unfolding}\isamarkupfalse%
\ inv{\isacharunderscore}{\kern0pt}iso{\isacharunderscore}{\kern0pt}def\ \isacommand{by}\isamarkupfalse%
\ {\isacharparenleft}{\kern0pt}simp\ add{\isacharcolon}{\kern0pt}\ edge{\isacharunderscore}{\kern0pt}preserving{\isacharbrackleft}{\kern0pt}symmetric{\isacharbrackright}{\kern0pt}\ image{\isacharunderscore}{\kern0pt}comp{\isacharparenright}{\kern0pt}\isanewline
\isacommand{qed}\isamarkupfalse%
%
\endisatagproof
{\isafoldproof}%
%
\isadelimproof
\isanewline
%
\endisadelimproof
\isanewline
\isacommand{interpretation}\isamarkupfalse%
\ inv{\isacharunderscore}{\kern0pt}iso{\isacharcolon}{\kern0pt}\ graph{\isacharunderscore}{\kern0pt}isomorphism\ V\isactrlsub H\ E\isactrlsub H\ V\isactrlsub G\ E\isactrlsub G\ inv{\isacharunderscore}{\kern0pt}iso%
\isadelimproof
\ %
\endisadelimproof
%
\isatagproof
\isacommand{using}\isamarkupfalse%
\ graph{\isacharunderscore}{\kern0pt}isomorphism{\isacharunderscore}{\kern0pt}inv\ \isacommand{{\isachardot}{\kern0pt}}\isamarkupfalse%
%
\endisatagproof
{\isafoldproof}%
%
\isadelimproof
%
\endisadelimproof
\isanewline
\isanewline
\isacommand{end}\isamarkupfalse%
\isanewline
\isanewline
\isacommand{fun}\isamarkupfalse%
\ graph{\isacharunderscore}{\kern0pt}isomorph\ {\isacharcolon}{\kern0pt}{\isacharcolon}{\kern0pt}\ {\isachardoublequoteopen}{\isacharprime}{\kern0pt}a\ pregraph\ {\isasymRightarrow}\ {\isacharprime}{\kern0pt}b\ pregraph\ {\isasymRightarrow}\ bool{\isachardoublequoteclose}\ {\isacharparenleft}{\kern0pt}\isakeyword{infix}\ {\isachardoublequoteopen}{\isasymsimeq}{\isachardoublequoteclose}\ {\isadigit{5}}{\isadigit{0}}{\isacharparenright}{\kern0pt}\ \isakeyword{where}\isanewline
\ \ {\isachardoublequoteopen}{\isacharparenleft}{\kern0pt}V\isactrlsub G{\isacharcomma}{\kern0pt}E\isactrlsub G{\isacharparenright}{\kern0pt}\ {\isasymsimeq}\ {\isacharparenleft}{\kern0pt}V\isactrlsub H{\isacharcomma}{\kern0pt}E\isactrlsub H{\isacharparenright}{\kern0pt}\ {\isasymlongleftrightarrow}\ {\isacharparenleft}{\kern0pt}{\isasymexists}f{\isachardot}{\kern0pt}\ graph{\isacharunderscore}{\kern0pt}isomorphism\ V\isactrlsub G\ E\isactrlsub G\ V\isactrlsub H\ E\isactrlsub H\ f{\isacharparenright}{\kern0pt}{\isachardoublequoteclose}\isanewline
\isanewline
\isacommand{lemma}\isamarkupfalse%
\ {\isacharparenleft}{\kern0pt}\isakeyword{in}\ graph{\isacharunderscore}{\kern0pt}system{\isacharparenright}{\kern0pt}\ graph{\isacharunderscore}{\kern0pt}isomorphism{\isacharunderscore}{\kern0pt}id{\isacharcolon}{\kern0pt}\ {\isachardoublequoteopen}graph{\isacharunderscore}{\kern0pt}isomorphism\ V\ E\ V\ E\ id{\isachardoublequoteclose}\isanewline
%
\isadelimproof
\ \ %
\endisadelimproof
%
\isatagproof
\isacommand{by}\isamarkupfalse%
\ unfold{\isacharunderscore}{\kern0pt}locales\ auto%
\endisatagproof
{\isafoldproof}%
%
\isadelimproof
\isanewline
%
\endisadelimproof
\isanewline
\isacommand{lemma}\isamarkupfalse%
\ {\isacharparenleft}{\kern0pt}\isakeyword{in}\ graph{\isacharunderscore}{\kern0pt}system{\isacharparenright}{\kern0pt}\ graph{\isacharunderscore}{\kern0pt}isomorph{\isacharunderscore}{\kern0pt}refl{\isacharcolon}{\kern0pt}\ {\isachardoublequoteopen}{\isacharparenleft}{\kern0pt}V{\isacharcomma}{\kern0pt}E{\isacharparenright}{\kern0pt}\ {\isasymsimeq}\ {\isacharparenleft}{\kern0pt}V{\isacharcomma}{\kern0pt}E{\isacharparenright}{\kern0pt}{\isachardoublequoteclose}\isanewline
%
\isadelimproof
\ \ %
\endisadelimproof
%
\isatagproof
\isacommand{using}\isamarkupfalse%
\ graph{\isacharunderscore}{\kern0pt}isomorphism{\isacharunderscore}{\kern0pt}id\ \isacommand{by}\isamarkupfalse%
\ auto%
\endisatagproof
{\isafoldproof}%
%
\isadelimproof
\isanewline
%
\endisadelimproof
\isanewline
\isacommand{lemma}\isamarkupfalse%
\ graph{\isacharunderscore}{\kern0pt}isomorph{\isacharunderscore}{\kern0pt}sym{\isacharcolon}{\kern0pt}\ {\isachardoublequoteopen}symp\ {\isacharparenleft}{\kern0pt}{\isasymsimeq}{\isacharparenright}{\kern0pt}{\isachardoublequoteclose}\isanewline
%
\isadelimproof
\ \ %
\endisadelimproof
%
\isatagproof
\isacommand{using}\isamarkupfalse%
\ graph{\isacharunderscore}{\kern0pt}isomorphism{\isachardot}{\kern0pt}graph{\isacharunderscore}{\kern0pt}isomorphism{\isacharunderscore}{\kern0pt}inv\ \isacommand{unfolding}\isamarkupfalse%
\ symp{\isacharunderscore}{\kern0pt}def\ \isacommand{by}\isamarkupfalse%
\ fastforce%
\endisatagproof
{\isafoldproof}%
%
\isadelimproof
\isanewline
%
\endisadelimproof
\isanewline
\isacommand{lemma}\isamarkupfalse%
\ graph{\isacharunderscore}{\kern0pt}isomorphism{\isacharunderscore}{\kern0pt}trans{\isacharcolon}{\kern0pt}\ {\isachardoublequoteopen}graph{\isacharunderscore}{\kern0pt}isomorphism\ V\isactrlsub G\ E\isactrlsub G\ V\isactrlsub H\ E\isactrlsub H\ f\ {\isasymLongrightarrow}\ graph{\isacharunderscore}{\kern0pt}isomorphism\ V\isactrlsub H\ E\isactrlsub H\ V\isactrlsub F\ E\isactrlsub F\ g\ {\isasymLongrightarrow}\ graph{\isacharunderscore}{\kern0pt}isomorphism\ V\isactrlsub G\ E\isactrlsub G\ V\isactrlsub F\ E\isactrlsub F\ {\isacharparenleft}{\kern0pt}g\ o\ f{\isacharparenright}{\kern0pt}{\isachardoublequoteclose}\isanewline
%
\isadelimproof
\ \ %
\endisadelimproof
%
\isatagproof
\isacommand{unfolding}\isamarkupfalse%
\ graph{\isacharunderscore}{\kern0pt}isomorphism{\isacharunderscore}{\kern0pt}def\ graph{\isacharunderscore}{\kern0pt}isomorphism{\isacharunderscore}{\kern0pt}axioms{\isacharunderscore}{\kern0pt}def\ \isacommand{using}\isamarkupfalse%
\ bij{\isacharunderscore}{\kern0pt}betw{\isacharunderscore}{\kern0pt}trans\ \isacommand{by}\isamarkupfalse%
\ {\isacharparenleft}{\kern0pt}auto{\isacharcomma}{\kern0pt}\ blast{\isacharparenright}{\kern0pt}%
\endisatagproof
{\isafoldproof}%
%
\isadelimproof
\isanewline
%
\endisadelimproof
\isanewline
\isacommand{lemma}\isamarkupfalse%
\ graph{\isacharunderscore}{\kern0pt}isomorph{\isacharunderscore}{\kern0pt}trans{\isacharcolon}{\kern0pt}\ {\isachardoublequoteopen}transp\ {\isacharparenleft}{\kern0pt}{\isasymsimeq}{\isacharparenright}{\kern0pt}{\isachardoublequoteclose}\isanewline
%
\isadelimproof
\ \ %
\endisadelimproof
%
\isatagproof
\isacommand{using}\isamarkupfalse%
\ graph{\isacharunderscore}{\kern0pt}isomorphism{\isacharunderscore}{\kern0pt}trans\ \isacommand{unfolding}\isamarkupfalse%
\ transp{\isacharunderscore}{\kern0pt}def\ \isacommand{by}\isamarkupfalse%
\ fastforce%
\endisatagproof
{\isafoldproof}%
%
\isadelimproof
\isanewline
%
\endisadelimproof
%
\isadelimtheory
\isanewline
%
\endisadelimtheory
%
\isatagtheory
\isacommand{end}\isamarkupfalse%
%
\endisatagtheory
{\isafoldtheory}%
%
\isadelimtheory
%
\endisadelimtheory
%
\end{isabellebody}%
\endinput
%:%file=Tree_Graph.tex%:%
%:%11=1%:%
%:%27=3%:%
%:%28=3%:%
%:%29=4%:%
%:%30=5%:%
%:%44=7%:%
%:%54=9%:%
%:%55=9%:%
%:%56=10%:%
%:%57=11%:%
%:%58=12%:%
%:%59=12%:%
%:%60=13%:%
%:%61=14%:%
%:%62=15%:%
%:%63=15%:%
%:%66=16%:%
%:%70=16%:%
%:%71=16%:%
%:%72=16%:%
%:%73=16%:%
%:%78=16%:%
%:%81=17%:%
%:%82=18%:%
%:%83=18%:%
%:%86=19%:%
%:%90=19%:%
%:%91=19%:%
%:%92=19%:%
%:%97=19%:%
%:%100=20%:%
%:%101=21%:%
%:%102=21%:%
%:%105=22%:%
%:%109=22%:%
%:%110=22%:%
%:%111=22%:%
%:%112=22%:%
%:%117=22%:%
%:%120=23%:%
%:%121=24%:%
%:%122=24%:%
%:%125=25%:%
%:%129=25%:%
%:%130=25%:%
%:%131=25%:%
%:%132=25%:%
%:%137=25%:%
%:%140=26%:%
%:%141=27%:%
%:%142=27%:%
%:%145=28%:%
%:%149=28%:%
%:%150=28%:%
%:%151=28%:%
%:%156=28%:%
%:%159=29%:%
%:%160=30%:%
%:%161=30%:%
%:%164=31%:%
%:%168=31%:%
%:%169=31%:%
%:%170=31%:%
%:%171=31%:%
%:%176=31%:%
%:%179=32%:%
%:%180=33%:%
%:%181=34%:%
%:%182=34%:%
%:%185=35%:%
%:%189=35%:%
%:%190=35%:%
%:%191=35%:%
%:%196=35%:%
%:%199=36%:%
%:%200=37%:%
%:%201=37%:%
%:%202=38%:%
%:%209=40%:%
%:%219=42%:%
%:%220=42%:%
%:%223=43%:%
%:%227=43%:%
%:%228=43%:%
%:%229=43%:%
%:%230=43%:%
%:%235=43%:%
%:%238=44%:%
%:%239=45%:%
%:%240=45%:%
%:%243=46%:%
%:%247=46%:%
%:%248=46%:%
%:%249=47%:%
%:%250=47%:%
%:%251=48%:%
%:%252=48%:%
%:%253=49%:%
%:%254=49%:%
%:%255=49%:%
%:%256=49%:%
%:%257=50%:%
%:%258=50%:%
%:%259=50%:%
%:%260=50%:%
%:%261=51%:%
%:%262=51%:%
%:%263=52%:%
%:%264=52%:%
%:%265=53%:%
%:%266=53%:%
%:%267=53%:%
%:%268=53%:%
%:%269=54%:%
%:%270=54%:%
%:%271=54%:%
%:%272=54%:%
%:%273=55%:%
%:%274=55%:%
%:%275=56%:%
%:%276=56%:%
%:%277=57%:%
%:%278=57%:%
%:%279=58%:%
%:%280=58%:%
%:%281=58%:%
%:%282=58%:%
%:%283=58%:%
%:%284=59%:%
%:%285=59%:%
%:%286=59%:%
%:%287=59%:%
%:%288=59%:%
%:%289=60%:%
%:%290=60%:%
%:%291=60%:%
%:%292=60%:%
%:%293=60%:%
%:%294=61%:%
%:%295=61%:%
%:%296=61%:%
%:%297=61%:%
%:%298=61%:%
%:%299=61%:%
%:%300=62%:%
%:%301=62%:%
%:%302=62%:%
%:%303=62%:%
%:%304=62%:%
%:%305=63%:%
%:%306=63%:%
%:%307=63%:%
%:%308=63%:%
%:%309=63%:%
%:%310=63%:%
%:%311=64%:%
%:%312=64%:%
%:%313=64%:%
%:%314=64%:%
%:%315=65%:%
%:%316=65%:%
%:%317=65%:%
%:%318=66%:%
%:%319=66%:%
%:%320=67%:%
%:%321=67%:%
%:%322=67%:%
%:%323=67%:%
%:%324=67%:%
%:%325=68%:%
%:%326=68%:%
%:%327=68%:%
%:%328=68%:%
%:%329=68%:%
%:%330=69%:%
%:%331=69%:%
%:%332=69%:%
%:%333=69%:%
%:%334=70%:%
%:%335=70%:%
%:%336=71%:%
%:%337=71%:%
%:%338=72%:%
%:%339=72%:%
%:%340=72%:%
%:%341=72%:%
%:%342=73%:%
%:%343=73%:%
%:%344=73%:%
%:%345=73%:%
%:%346=73%:%
%:%347=74%:%
%:%348=74%:%
%:%349=74%:%
%:%350=74%:%
%:%351=74%:%
%:%352=74%:%
%:%353=75%:%
%:%354=75%:%
%:%355=76%:%
%:%356=76%:%
%:%357=76%:%
%:%358=76%:%
%:%359=77%:%
%:%360=77%:%
%:%361=78%:%
%:%362=78%:%
%:%363=78%:%
%:%364=78%:%
%:%365=79%:%
%:%366=79%:%
%:%367=79%:%
%:%368=80%:%
%:%369=80%:%
%:%370=80%:%
%:%371=80%:%
%:%372=81%:%
%:%373=81%:%
%:%374=82%:%
%:%375=82%:%
%:%376=82%:%
%:%377=82%:%
%:%378=83%:%
%:%379=83%:%
%:%380=83%:%
%:%381=84%:%
%:%382=84%:%
%:%383=85%:%
%:%384=85%:%
%:%385=85%:%
%:%386=86%:%
%:%387=86%:%
%:%388=87%:%
%:%389=87%:%
%:%390=87%:%
%:%391=87%:%
%:%392=88%:%
%:%393=88%:%
%:%394=88%:%
%:%395=89%:%
%:%396=89%:%
%:%397=90%:%
%:%398=90%:%
%:%399=90%:%
%:%400=91%:%
%:%401=91%:%
%:%402=91%:%
%:%403=91%:%
%:%404=92%:%
%:%405=92%:%
%:%406=92%:%
%:%407=93%:%
%:%408=93%:%
%:%409=93%:%
%:%410=94%:%
%:%411=94%:%
%:%412=94%:%
%:%413=95%:%
%:%414=95%:%
%:%415=95%:%
%:%416=96%:%
%:%417=96%:%
%:%418=96%:%
%:%419=96%:%
%:%420=96%:%
%:%421=97%:%
%:%422=97%:%
%:%423=97%:%
%:%424=97%:%
%:%425=97%:%
%:%426=98%:%
%:%427=98%:%
%:%428=98%:%
%:%429=98%:%
%:%430=99%:%
%:%431=99%:%
%:%432=100%:%
%:%447=103%:%
%:%457=105%:%
%:%458=105%:%
%:%461=106%:%
%:%465=106%:%
%:%466=106%:%
%:%467=106%:%
%:%472=106%:%
%:%475=107%:%
%:%476=108%:%
%:%477=108%:%
%:%480=109%:%
%:%484=109%:%
%:%485=109%:%
%:%490=109%:%
%:%493=110%:%
%:%494=111%:%
%:%495=111%:%
%:%498=112%:%
%:%502=112%:%
%:%503=112%:%
%:%504=112%:%
%:%505=112%:%
%:%510=112%:%
%:%513=113%:%
%:%514=114%:%
%:%515=114%:%
%:%516=115%:%
%:%519=116%:%
%:%523=116%:%
%:%524=116%:%
%:%538=118%:%
%:%548=120%:%
%:%549=120%:%
%:%552=121%:%
%:%556=121%:%
%:%557=121%:%
%:%558=121%:%
%:%559=121%:%
%:%564=121%:%
%:%567=122%:%
%:%568=123%:%
%:%569=123%:%
%:%572=124%:%
%:%576=124%:%
%:%577=124%:%
%:%582=124%:%
%:%585=125%:%
%:%586=126%:%
%:%587=126%:%
%:%590=127%:%
%:%594=127%:%
%:%595=127%:%
%:%596=128%:%
%:%601=128%:%
%:%604=129%:%
%:%605=130%:%
%:%606=130%:%
%:%609=131%:%
%:%613=131%:%
%:%614=131%:%
%:%619=131%:%
%:%622=132%:%
%:%623=133%:%
%:%624=133%:%
%:%627=134%:%
%:%631=134%:%
%:%632=134%:%
%:%633=134%:%
%:%634=134%:%
%:%635=135%:%
%:%636=135%:%
%:%641=135%:%
%:%644=136%:%
%:%645=137%:%
%:%646=137%:%
%:%649=138%:%
%:%653=138%:%
%:%654=138%:%
%:%655=138%:%
%:%656=138%:%
%:%661=138%:%
%:%664=139%:%
%:%665=140%:%
%:%666=140%:%
%:%669=141%:%
%:%673=141%:%
%:%674=141%:%
%:%675=141%:%
%:%676=141%:%
%:%681=141%:%
%:%684=142%:%
%:%685=143%:%
%:%686=143%:%
%:%687=144%:%
%:%688=145%:%
%:%689=146%:%
%:%696=147%:%
%:%697=147%:%
%:%698=148%:%
%:%699=148%:%
%:%700=149%:%
%:%701=149%:%
%:%702=149%:%
%:%703=149%:%
%:%704=149%:%
%:%705=150%:%
%:%706=150%:%
%:%707=150%:%
%:%708=150%:%
%:%709=150%:%
%:%710=151%:%
%:%711=151%:%
%:%712=151%:%
%:%713=151%:%
%:%714=151%:%
%:%715=151%:%
%:%716=152%:%
%:%717=152%:%
%:%718=153%:%
%:%719=153%:%
%:%720=153%:%
%:%721=154%:%
%:%722=154%:%
%:%723=154%:%
%:%724=154%:%
%:%725=155%:%
%:%740=157%:%
%:%750=159%:%
%:%751=159%:%
%:%752=160%:%
%:%753=161%:%
%:%754=162%:%
%:%755=162%:%
%:%756=163%:%
%:%757=164%:%
%:%758=165%:%
%:%759=165%:%
%:%762=166%:%
%:%766=166%:%
%:%767=166%:%
%:%768=166%:%
%:%769=166%:%
%:%774=166%:%
%:%777=167%:%
%:%778=168%:%
%:%779=168%:%
%:%780=169%:%
%:%781=170%:%
%:%782=170%:%
%:%783=171%:%
%:%784=172%:%
%:%791=173%:%
%:%792=173%:%
%:%793=174%:%
%:%794=174%:%
%:%795=175%:%
%:%796=175%:%
%:%797=176%:%
%:%798=176%:%
%:%799=176%:%
%:%800=176%:%
%:%801=176%:%
%:%802=176%:%
%:%803=177%:%
%:%809=177%:%
%:%812=178%:%
%:%813=179%:%
%:%814=179%:%
%:%817=180%:%
%:%821=180%:%
%:%822=180%:%
%:%823=180%:%
%:%824=180%:%
%:%829=180%:%
%:%832=181%:%
%:%833=182%:%
%:%834=182%:%
%:%837=183%:%
%:%841=183%:%
%:%842=183%:%
%:%843=183%:%
%:%844=183%:%
%:%849=183%:%
%:%852=184%:%
%:%853=185%:%
%:%854=185%:%
%:%857=186%:%
%:%861=186%:%
%:%862=186%:%
%:%863=186%:%
%:%877=188%:%
%:%887=190%:%
%:%888=190%:%
%:%889=191%:%
%:%890=192%:%
%:%891=193%:%
%:%892=194%:%
%:%893=194%:%
%:%896=195%:%
%:%900=195%:%
%:%901=195%:%
%:%902=195%:%
%:%907=195%:%
%:%910=196%:%
%:%911=197%:%
%:%912=197%:%
%:%915=198%:%
%:%919=198%:%
%:%920=198%:%
%:%921=198%:%
%:%922=198%:%
%:%927=198%:%
%:%930=199%:%
%:%931=200%:%
%:%932=200%:%
%:%935=201%:%
%:%939=201%:%
%:%940=201%:%
%:%941=201%:%
%:%942=201%:%
%:%947=201%:%
%:%950=202%:%
%:%951=203%:%
%:%952=203%:%
%:%955=204%:%
%:%959=204%:%
%:%960=204%:%
%:%961=204%:%
%:%962=204%:%
%:%967=204%:%
%:%970=205%:%
%:%971=206%:%
%:%972=206%:%
%:%975=207%:%
%:%979=207%:%
%:%980=207%:%
%:%981=207%:%
%:%982=207%:%
%:%987=207%:%
%:%990=208%:%
%:%991=209%:%
%:%992=209%:%
%:%995=210%:%
%:%999=210%:%
%:%1000=210%:%
%:%1001=210%:%
%:%1002=210%:%
%:%1007=210%:%
%:%1010=211%:%
%:%1011=212%:%
%:%1012=212%:%
%:%1015=213%:%
%:%1019=213%:%
%:%1020=213%:%
%:%1021=213%:%
%:%1022=213%:%
%:%1027=213%:%
%:%1030=214%:%
%:%1031=215%:%
%:%1032=215%:%
%:%1035=216%:%
%:%1039=216%:%
%:%1040=216%:%
%:%1041=216%:%
%:%1042=216%:%
%:%1047=216%:%
%:%1050=217%:%
%:%1051=218%:%
%:%1052=218%:%
%:%1055=219%:%
%:%1059=219%:%
%:%1060=219%:%
%:%1061=219%:%
%:%1062=219%:%
%:%1067=219%:%
%:%1070=220%:%
%:%1071=221%:%
%:%1072=221%:%
%:%1073=222%:%
%:%1074=223%:%
%:%1075=223%:%
%:%1078=224%:%
%:%1082=224%:%
%:%1083=224%:%
%:%1084=224%:%
%:%1085=224%:%
%:%1099=226%:%
%:%1109=228%:%
%:%1110=228%:%
%:%1111=229%:%
%:%1112=230%:%
%:%1119=231%:%
%:%1120=231%:%
%:%1121=232%:%
%:%1122=232%:%
%:%1123=233%:%
%:%1124=233%:%
%:%1125=234%:%
%:%1126=234%:%
%:%1127=234%:%
%:%1128=235%:%
%:%1129=235%:%
%:%1130=235%:%
%:%1131=235%:%
%:%1132=235%:%
%:%1133=236%:%
%:%1134=236%:%
%:%1135=236%:%
%:%1136=236%:%
%:%1137=236%:%
%:%1138=236%:%
%:%1139=236%:%
%:%1140=237%:%
%:%1141=237%:%
%:%1142=237%:%
%:%1143=237%:%
%:%1144=237%:%
%:%1145=238%:%
%:%1146=238%:%
%:%1147=239%:%
%:%1148=239%:%
%:%1149=239%:%
%:%1150=239%:%
%:%1151=239%:%
%:%1152=239%:%
%:%1153=240%:%
%:%1159=240%:%
%:%1162=241%:%
%:%1163=242%:%
%:%1164=242%:%
%:%1165=243%:%
%:%1166=244%:%
%:%1173=245%:%
%:%1174=245%:%
%:%1175=246%:%
%:%1176=246%:%
%:%1177=246%:%
%:%1178=246%:%
%:%1179=246%:%
%:%1180=247%:%
%:%1181=247%:%
%:%1182=247%:%
%:%1183=247%:%
%:%1184=247%:%
%:%1185=248%:%
%:%1191=248%:%
%:%1194=249%:%
%:%1195=250%:%
%:%1196=250%:%
%:%1197=251%:%
%:%1198=252%:%
%:%1199=253%:%
%:%1206=254%:%
%:%1207=254%:%
%:%1208=255%:%
%:%1209=255%:%
%:%1210=256%:%
%:%1211=256%:%
%:%1212=256%:%
%:%1213=256%:%
%:%1214=256%:%
%:%1215=256%:%
%:%1216=257%:%
%:%1217=257%:%
%:%1218=257%:%
%:%1219=257%:%
%:%1220=257%:%
%:%1221=257%:%
%:%1222=258%:%
%:%1223=258%:%
%:%1224=258%:%
%:%1225=258%:%
%:%1226=258%:%
%:%1227=258%:%
%:%1228=259%:%
%:%1229=259%:%
%:%1230=259%:%
%:%1231=259%:%
%:%1232=259%:%
%:%1233=260%:%
%:%1239=260%:%
%:%1242=261%:%
%:%1243=262%:%
%:%1244=262%:%
%:%1245=263%:%
%:%1246=264%:%
%:%1247=265%:%
%:%1254=266%:%
%:%1255=266%:%
%:%1256=267%:%
%:%1257=267%:%
%:%1258=268%:%
%:%1259=268%:%
%:%1260=268%:%
%:%1261=268%:%
%:%1262=268%:%
%:%1263=269%:%
%:%1264=269%:%
%:%1265=269%:%
%:%1266=269%:%
%:%1267=269%:%
%:%1268=270%:%
%:%1269=270%:%
%:%1270=270%:%
%:%1271=270%:%
%:%1272=270%:%
%:%1273=271%:%
%:%1274=271%:%
%:%1275=271%:%
%:%1276=271%:%
%:%1277=271%:%
%:%1278=272%:%
%:%1279=272%:%
%:%1280=273%:%
%:%1281=273%:%
%:%1282=273%:%
%:%1283=273%:%
%:%1284=273%:%
%:%1285=273%:%
%:%1286=274%:%
%:%1287=274%:%
%:%1288=274%:%
%:%1289=274%:%
%:%1290=274%:%
%:%1291=275%:%
%:%1292=275%:%
%:%1293=275%:%
%:%1294=275%:%
%:%1295=275%:%
%:%1296=276%:%
%:%1302=276%:%
%:%1305=277%:%
%:%1306=278%:%
%:%1307=278%:%
%:%1308=279%:%
%:%1309=280%:%
%:%1310=281%:%
%:%1317=282%:%
%:%1318=282%:%
%:%1319=283%:%
%:%1320=283%:%
%:%1321=283%:%
%:%1322=284%:%
%:%1323=284%:%
%:%1324=285%:%
%:%1325=285%:%
%:%1326=285%:%
%:%1327=285%:%
%:%1328=285%:%
%:%1329=286%:%
%:%1335=286%:%
%:%1338=287%:%
%:%1339=288%:%
%:%1340=288%:%
%:%1343=289%:%
%:%1347=289%:%
%:%1348=289%:%
%:%1349=289%:%
%:%1350=290%:%
%:%1351=290%:%
%:%1356=290%:%
%:%1359=291%:%
%:%1360=292%:%
%:%1361=292%:%
%:%1362=293%:%
%:%1363=294%:%
%:%1370=295%:%
%:%1371=295%:%
%:%1372=296%:%
%:%1373=296%:%
%:%1374=296%:%
%:%1375=297%:%
%:%1376=297%:%
%:%1377=297%:%
%:%1378=297%:%
%:%1379=298%:%
%:%1380=298%:%
%:%1381=298%:%
%:%1382=298%:%
%:%1383=299%:%
%:%1384=299%:%
%:%1385=299%:%
%:%1386=299%:%
%:%1387=299%:%
%:%1388=300%:%
%:%1389=300%:%
%:%1390=300%:%
%:%1391=301%:%
%:%1392=301%:%
%:%1393=302%:%
%:%1394=302%:%
%:%1395=303%:%
%:%1396=303%:%
%:%1397=304%:%
%:%1398=304%:%
%:%1399=304%:%
%:%1400=305%:%
%:%1401=305%:%
%:%1402=306%:%
%:%1403=306%:%
%:%1404=307%:%
%:%1405=307%:%
%:%1406=307%:%
%:%1407=307%:%
%:%1408=307%:%
%:%1409=308%:%
%:%1410=308%:%
%:%1411=309%:%
%:%1412=309%:%
%:%1413=310%:%
%:%1414=310%:%
%:%1415=310%:%
%:%1416=310%:%
%:%1417=310%:%
%:%1418=311%:%
%:%1419=311%:%
%:%1420=311%:%
%:%1421=311%:%
%:%1422=312%:%
%:%1423=312%:%
%:%1424=312%:%
%:%1425=312%:%
%:%1426=313%:%
%:%1427=313%:%
%:%1428=313%:%
%:%1429=313%:%
%:%1430=314%:%
%:%1431=314%:%
%:%1432=314%:%
%:%1433=314%:%
%:%1434=314%:%
%:%1435=315%:%
%:%1436=315%:%
%:%1437=315%:%
%:%1438=315%:%
%:%1439=316%:%
%:%1440=316%:%
%:%1441=317%:%
%:%1442=317%:%
%:%1443=317%:%
%:%1444=317%:%
%:%1445=317%:%
%:%1446=318%:%
%:%1447=318%:%
%:%1448=318%:%
%:%1449=318%:%
%:%1450=319%:%
%:%1451=319%:%
%:%1452=319%:%
%:%1453=319%:%
%:%1454=319%:%
%:%1455=320%:%
%:%1456=320%:%
%:%1457=320%:%
%:%1458=320%:%
%:%1459=321%:%
%:%1460=321%:%
%:%1461=321%:%
%:%1462=322%:%
%:%1463=322%:%
%:%1464=322%:%
%:%1465=322%:%
%:%1466=322%:%
%:%1467=323%:%
%:%1468=323%:%
%:%1469=324%:%
%:%1470=324%:%
%:%1471=325%:%
%:%1472=325%:%
%:%1473=326%:%
%:%1474=326%:%
%:%1475=326%:%
%:%1476=326%:%
%:%1477=327%:%
%:%1478=327%:%
%:%1479=327%:%
%:%1480=328%:%
%:%1481=328%:%
%:%1482=328%:%
%:%1483=328%:%
%:%1484=328%:%
%:%1485=329%:%
%:%1486=329%:%
%:%1487=330%:%
%:%1493=330%:%
%:%1496=331%:%
%:%1497=332%:%
%:%1498=332%:%
%:%1499=333%:%
%:%1500=334%:%
%:%1507=335%:%
%:%1508=335%:%
%:%1509=336%:%
%:%1510=336%:%
%:%1511=336%:%
%:%1512=336%:%
%:%1513=337%:%
%:%1514=337%:%
%:%1515=337%:%
%:%1516=337%:%
%:%1517=337%:%
%:%1518=338%:%
%:%1519=338%:%
%:%1520=338%:%
%:%1521=338%:%
%:%1522=338%:%
%:%1523=339%:%
%:%1524=339%:%
%:%1525=339%:%
%:%1526=339%:%
%:%1527=340%:%
%:%1528=340%:%
%:%1529=340%:%
%:%1530=341%:%
%:%1531=341%:%
%:%1532=341%:%
%:%1533=341%:%
%:%1534=341%:%
%:%1535=341%:%
%:%1536=342%:%
%:%1542=342%:%
%:%1545=343%:%
%:%1546=344%:%
%:%1547=344%:%
%:%1548=345%:%
%:%1549=346%:%
%:%1556=347%:%
%:%1557=347%:%
%:%1558=348%:%
%:%1559=348%:%
%:%1560=348%:%
%:%1561=348%:%
%:%1562=349%:%
%:%1563=349%:%
%:%1564=350%:%
%:%1565=350%:%
%:%1566=351%:%
%:%1567=351%:%
%:%1568=351%:%
%:%1569=351%:%
%:%1570=351%:%
%:%1571=352%:%
%:%1572=352%:%
%:%1573=352%:%
%:%1574=352%:%
%:%1575=353%:%
%:%1576=353%:%
%:%1577=354%:%
%:%1578=354%:%
%:%1579=354%:%
%:%1580=354%:%
%:%1581=354%:%
%:%1582=355%:%
%:%1588=355%:%
%:%1591=356%:%
%:%1592=357%:%
%:%1593=357%:%
%:%1594=358%:%
%:%1595=359%:%
%:%1596=360%:%
%:%1603=361%:%
%:%1604=361%:%
%:%1605=362%:%
%:%1606=362%:%
%:%1607=362%:%
%:%1608=363%:%
%:%1609=363%:%
%:%1610=363%:%
%:%1611=364%:%
%:%1612=364%:%
%:%1613=364%:%
%:%1614=364%:%
%:%1615=364%:%
%:%1616=365%:%
%:%1617=365%:%
%:%1618=365%:%
%:%1619=365%:%
%:%1620=365%:%
%:%1621=366%:%
%:%1622=366%:%
%:%1623=367%:%
%:%1624=367%:%
%:%1625=368%:%
%:%1626=368%:%
%:%1627=369%:%
%:%1628=369%:%
%:%1629=369%:%
%:%1630=369%:%
%:%1631=369%:%
%:%1632=369%:%
%:%1633=370%:%
%:%1634=370%:%
%:%1635=370%:%
%:%1636=370%:%
%:%1637=370%:%
%:%1638=371%:%
%:%1639=371%:%
%:%1640=372%:%
%:%1641=372%:%
%:%1642=373%:%
%:%1643=373%:%
%:%1644=374%:%
%:%1645=374%:%
%:%1646=374%:%
%:%1647=374%:%
%:%1648=374%:%
%:%1649=375%:%
%:%1650=375%:%
%:%1651=376%:%
%:%1652=376%:%
%:%1653=377%:%
%:%1654=377%:%
%:%1655=377%:%
%:%1656=377%:%
%:%1657=377%:%
%:%1658=377%:%
%:%1659=378%:%
%:%1660=378%:%
%:%1661=379%:%
%:%1662=379%:%
%:%1663=380%:%
%:%1664=380%:%
%:%1665=381%:%
%:%1666=381%:%
%:%1667=381%:%
%:%1668=381%:%
%:%1669=382%:%
%:%1670=382%:%
%:%1671=382%:%
%:%1672=382%:%
%:%1673=382%:%
%:%1674=383%:%
%:%1675=383%:%
%:%1676=383%:%
%:%1677=383%:%
%:%1678=383%:%
%:%1679=383%:%
%:%1680=384%:%
%:%1681=384%:%
%:%1682=384%:%
%:%1683=384%:%
%:%1684=385%:%
%:%1685=385%:%
%:%1686=385%:%
%:%1687=385%:%
%:%1688=385%:%
%:%1689=386%:%
%:%1690=386%:%
%:%1691=386%:%
%:%1692=387%:%
%:%1693=387%:%
%:%1694=388%:%
%:%1695=389%:%
%:%1696=389%:%
%:%1697=389%:%
%:%1698=389%:%
%:%1699=390%:%
%:%1700=390%:%
%:%1701=390%:%
%:%1702=391%:%
%:%1703=392%:%
%:%1704=392%:%
%:%1705=392%:%
%:%1706=392%:%
%:%1707=392%:%
%:%1708=393%:%
%:%1709=393%:%
%:%1710=393%:%
%:%1711=393%:%
%:%1712=394%:%
%:%1713=394%:%
%:%1714=394%:%
%:%1715=394%:%
%:%1716=394%:%
%:%1717=395%:%
%:%1718=395%:%
%:%1719=395%:%
%:%1720=395%:%
%:%1721=395%:%
%:%1722=396%:%
%:%1723=396%:%
%:%1724=396%:%
%:%1725=396%:%
%:%1726=397%:%
%:%1727=397%:%
%:%1728=397%:%
%:%1729=398%:%
%:%1730=399%:%
%:%1731=399%:%
%:%1732=399%:%
%:%1733=399%:%
%:%1734=399%:%
%:%1735=400%:%
%:%1736=400%:%
%:%1737=400%:%
%:%1738=400%:%
%:%1739=400%:%
%:%1740=401%:%
%:%1741=401%:%
%:%1742=401%:%
%:%1743=401%:%
%:%1744=402%:%
%:%1745=402%:%
%:%1746=402%:%
%:%1747=402%:%
%:%1748=403%:%
%:%1749=403%:%
%:%1750=404%:%
%:%1751=404%:%
%:%1752=404%:%
%:%1753=404%:%
%:%1754=404%:%
%:%1755=404%:%
%:%1756=405%:%
%:%1757=405%:%
%:%1758=405%:%
%:%1759=405%:%
%:%1760=405%:%
%:%1761=406%:%
%:%1762=406%:%
%:%1763=407%:%
%:%1764=407%:%
%:%1765=407%:%
%:%1766=407%:%
%:%1767=407%:%
%:%1768=407%:%
%:%1769=408%:%
%:%1770=408%:%
%:%1771=408%:%
%:%1772=408%:%
%:%1773=409%:%
%:%1774=409%:%
%:%1775=409%:%
%:%1776=410%:%
%:%1777=410%:%
%:%1778=410%:%
%:%1779=410%:%
%:%1780=410%:%
%:%1781=411%:%
%:%1782=411%:%
%:%1783=412%:%
%:%1784=412%:%
%:%1785=413%:%
%:%1786=413%:%
%:%1787=413%:%
%:%1788=413%:%
%:%1789=413%:%
%:%1790=414%:%
%:%1805=416%:%
%:%1815=418%:%
%:%1816=418%:%
%:%1817=419%:%
%:%1818=420%:%
%:%1819=421%:%
%:%1820=421%:%
%:%1821=422%:%
%:%1822=423%:%
%:%1823=423%:%
%:%1824=424%:%
%:%1825=425%:%
%:%1826=426%:%
%:%1827=426%:%
%:%1828=427%:%
%:%1829=428%:%
%:%1830=429%:%
%:%1831=429%:%
%:%1834=430%:%
%:%1838=430%:%
%:%1839=430%:%
%:%1840=430%:%
%:%1845=430%:%
%:%1848=431%:%
%:%1849=432%:%
%:%1850=432%:%
%:%1853=433%:%
%:%1857=433%:%
%:%1858=433%:%
%:%1859=433%:%
%:%1860=433%:%
%:%1865=433%:%
%:%1868=434%:%
%:%1869=435%:%
%:%1870=435%:%
%:%1873=436%:%
%:%1877=436%:%
%:%1878=436%:%
%:%1879=436%:%
%:%1880=436%:%
%:%1885=436%:%
%:%1888=437%:%
%:%1889=438%:%
%:%1890=438%:%
%:%1893=439%:%
%:%1897=439%:%
%:%1898=439%:%
%:%1899=439%:%
%:%1900=439%:%
%:%1905=439%:%
%:%1908=440%:%
%:%1909=441%:%
%:%1910=441%:%
%:%1913=442%:%
%:%1917=442%:%
%:%1918=442%:%
%:%1919=442%:%
%:%1920=442%:%
%:%1925=442%:%
%:%1928=443%:%
%:%1929=444%:%
%:%1930=444%:%
%:%1933=445%:%
%:%1937=445%:%
%:%1938=445%:%
%:%1939=445%:%
%:%1940=445%:%
%:%1945=445%:%
%:%1948=446%:%
%:%1949=447%:%
%:%1950=447%:%
%:%1953=448%:%
%:%1957=448%:%
%:%1958=448%:%
%:%1959=448%:%
%:%1960=448%:%
%:%1965=448%:%
%:%1968=449%:%
%:%1969=450%:%
%:%1970=450%:%
%:%1973=451%:%
%:%1977=451%:%
%:%1978=451%:%
%:%1983=451%:%
%:%1986=452%:%
%:%1987=453%:%
%:%1988=453%:%
%:%1991=454%:%
%:%1995=454%:%
%:%1996=454%:%
%:%1997=454%:%
%:%1998=454%:%
%:%2003=454%:%
%:%2006=455%:%
%:%2007=456%:%
%:%2008=456%:%
%:%2011=457%:%
%:%2015=457%:%
%:%2016=457%:%
%:%2017=457%:%
%:%2022=457%:%
%:%2025=458%:%
%:%2026=459%:%
%:%2027=459%:%
%:%2030=460%:%
%:%2034=460%:%
%:%2035=460%:%
%:%2036=460%:%
%:%2041=460%:%
%:%2044=461%:%
%:%2045=462%:%
%:%2046=462%:%
%:%2049=463%:%
%:%2053=463%:%
%:%2054=463%:%
%:%2055=463%:%
%:%2060=463%:%
%:%2063=464%:%
%:%2064=465%:%
%:%2065=465%:%
%:%2066=466%:%
%:%2067=467%:%
%:%2074=468%:%
%:%2075=468%:%
%:%2076=469%:%
%:%2077=469%:%
%:%2078=469%:%
%:%2079=469%:%
%:%2080=470%:%
%:%2081=470%:%
%:%2082=470%:%
%:%2083=470%:%
%:%2084=471%:%
%:%2085=471%:%
%:%2086=472%:%
%:%2087=472%:%
%:%2088=473%:%
%:%2089=473%:%
%:%2090=473%:%
%:%2091=474%:%
%:%2092=474%:%
%:%2093=474%:%
%:%2094=474%:%
%:%2095=474%:%
%:%2096=474%:%
%:%2097=475%:%
%:%2098=475%:%
%:%2099=475%:%
%:%2100=475%:%
%:%2101=475%:%
%:%2102=476%:%
%:%2103=476%:%
%:%2104=476%:%
%:%2105=476%:%
%:%2106=477%:%
%:%2107=478%:%
%:%2108=478%:%
%:%2109=478%:%
%:%2110=479%:%
%:%2111=479%:%
%:%2112=479%:%
%:%2113=480%:%
%:%2114=480%:%
%:%2115=480%:%
%:%2116=480%:%
%:%2117=481%:%
%:%2118=481%:%
%:%2119=481%:%
%:%2120=482%:%
%:%2121=482%:%
%:%2122=483%:%
%:%2123=483%:%
%:%2124=484%:%
%:%2125=484%:%
%:%2126=484%:%
%:%2127=485%:%
%:%2128=485%:%
%:%2129=485%:%
%:%2130=485%:%
%:%2131=485%:%
%:%2132=486%:%
%:%2133=486%:%
%:%2134=487%:%
%:%2135=487%:%
%:%2136=487%:%
%:%2137=487%:%
%:%2138=487%:%
%:%2139=488%:%
%:%2145=488%:%
%:%2148=489%:%
%:%2149=490%:%
%:%2150=490%:%
%:%2153=491%:%
%:%2157=491%:%
%:%2158=491%:%
%:%2159=491%:%
%:%2160=491%:%
%:%2165=491%:%
%:%2168=492%:%
%:%2169=493%:%
%:%2170=493%:%
%:%2171=494%:%
%:%2172=495%:%
%:%2179=496%:%
%:%2180=496%:%
%:%2181=497%:%
%:%2182=497%:%
%:%2183=497%:%
%:%2184=497%:%
%:%2185=498%:%
%:%2186=498%:%
%:%2187=498%:%
%:%2188=498%:%
%:%2189=499%:%
%:%2190=499%:%
%:%2191=499%:%
%:%2192=499%:%
%:%2193=500%:%
%:%2199=500%:%
%:%2202=501%:%
%:%2203=502%:%
%:%2204=502%:%
%:%2207=503%:%
%:%2211=503%:%
%:%2212=503%:%
%:%2213=503%:%
%:%2214=503%:%
%:%2219=503%:%
%:%2222=504%:%
%:%2223=505%:%
%:%2224=505%:%
%:%2227=506%:%
%:%2231=506%:%
%:%2232=506%:%
%:%2233=506%:%
%:%2234=506%:%
%:%2239=506%:%
%:%2242=507%:%
%:%2243=508%:%
%:%2244=508%:%
%:%2247=509%:%
%:%2251=509%:%
%:%2252=509%:%
%:%2253=509%:%
%:%2254=509%:%
%:%2259=509%:%
%:%2262=510%:%
%:%2263=511%:%
%:%2264=511%:%
%:%2271=512%:%
%:%2272=512%:%
%:%2273=513%:%
%:%2274=513%:%
%:%2275=514%:%
%:%2276=514%:%
%:%2277=515%:%
%:%2278=515%:%
%:%2279=515%:%
%:%2280=516%:%
%:%2281=516%:%
%:%2282=516%:%
%:%2283=516%:%
%:%2284=516%:%
%:%2285=517%:%
%:%2286=517%:%
%:%2287=517%:%
%:%2288=517%:%
%:%2289=517%:%
%:%2290=518%:%
%:%2291=518%:%
%:%2292=518%:%
%:%2293=518%:%
%:%2294=518%:%
%:%2295=518%:%
%:%2296=519%:%
%:%2297=519%:%
%:%2298=519%:%
%:%2299=519%:%
%:%2300=519%:%
%:%2301=520%:%
%:%2302=520%:%
%:%2303=521%:%
%:%2304=521%:%
%:%2305=521%:%
%:%2306=521%:%
%:%2307=521%:%
%:%2308=522%:%
%:%2314=522%:%
%:%2317=523%:%
%:%2318=524%:%
%:%2319=524%:%
%:%2320=525%:%
%:%2321=526%:%
%:%2328=527%:%
%:%2329=527%:%
%:%2330=528%:%
%:%2331=528%:%
%:%2332=528%:%
%:%2333=528%:%
%:%2334=529%:%
%:%2335=529%:%
%:%2336=529%:%
%:%2337=529%:%
%:%2338=529%:%
%:%2339=530%:%
%:%2345=530%:%
%:%2348=531%:%
%:%2349=532%:%
%:%2350=532%:%
%:%2351=533%:%
%:%2352=534%:%
%:%2359=535%:%
%:%2360=535%:%
%:%2361=536%:%
%:%2362=536%:%
%:%2363=537%:%
%:%2364=537%:%
%:%2365=537%:%
%:%2366=537%:%
%:%2367=537%:%
%:%2368=537%:%
%:%2369=538%:%
%:%2370=538%:%
%:%2371=538%:%
%:%2372=538%:%
%:%2373=538%:%
%:%2374=538%:%
%:%2375=539%:%
%:%2376=539%:%
%:%2377=540%:%
%:%2378=540%:%
%:%2379=541%:%
%:%2380=541%:%
%:%2381=541%:%
%:%2382=541%:%
%:%2383=542%:%
%:%2389=542%:%
%:%2392=543%:%
%:%2393=544%:%
%:%2394=544%:%
%:%2395=545%:%
%:%2396=546%:%
%:%2397=546%:%
%:%2400=547%:%
%:%2404=547%:%
%:%2405=547%:%
%:%2406=547%:%
%:%2411=547%:%
%:%2414=548%:%
%:%2415=549%:%
%:%2416=549%:%
%:%2419=550%:%
%:%2423=550%:%
%:%2424=550%:%
%:%2425=550%:%
%:%2426=550%:%
%:%2431=550%:%
%:%2434=551%:%
%:%2435=552%:%
%:%2436=552%:%
%:%2439=553%:%
%:%2443=553%:%
%:%2444=553%:%
%:%2445=553%:%
%:%2450=553%:%
%:%2453=554%:%
%:%2454=555%:%
%:%2455=555%:%
%:%2456=556%:%
%:%2457=557%:%
%:%2458=558%:%
%:%2465=559%:%
%:%2466=559%:%
%:%2467=560%:%
%:%2468=560%:%
%:%2469=560%:%
%:%2470=560%:%
%:%2471=561%:%
%:%2472=561%:%
%:%2473=561%:%
%:%2474=561%:%
%:%2475=562%:%
%:%2476=562%:%
%:%2477=563%:%
%:%2478=563%:%
%:%2479=563%:%
%:%2480=563%:%
%:%2481=564%:%
%:%2482=564%:%
%:%2483=564%:%
%:%2484=564%:%
%:%2485=564%:%
%:%2486=564%:%
%:%2487=565%:%
%:%2493=565%:%
%:%2496=566%:%
%:%2497=567%:%
%:%2498=567%:%
%:%2499=568%:%
%:%2500=569%:%
%:%2501=570%:%
%:%2502=571%:%
%:%2509=572%:%
%:%2510=572%:%
%:%2511=573%:%
%:%2512=573%:%
%:%2513=574%:%
%:%2514=574%:%
%:%2515=575%:%
%:%2516=575%:%
%:%2517=576%:%
%:%2518=576%:%
%:%2519=576%:%
%:%2520=576%:%
%:%2521=577%:%
%:%2522=577%:%
%:%2523=578%:%
%:%2524=578%:%
%:%2525=579%:%
%:%2526=579%:%
%:%2527=579%:%
%:%2528=580%:%
%:%2529=580%:%
%:%2530=581%:%
%:%2531=581%:%
%:%2532=582%:%
%:%2533=582%:%
%:%2534=582%:%
%:%2535=582%:%
%:%2536=582%:%
%:%2537=583%:%
%:%2538=583%:%
%:%2539=584%:%
%:%2540=584%:%
%:%2541=584%:%
%:%2542=585%:%
%:%2543=585%:%
%:%2544=585%:%
%:%2545=585%:%
%:%2546=585%:%
%:%2547=586%:%
%:%2548=586%:%
%:%2549=586%:%
%:%2550=586%:%
%:%2551=586%:%
%:%2552=587%:%
%:%2553=587%:%
%:%2554=587%:%
%:%2555=587%:%
%:%2556=587%:%
%:%2557=588%:%
%:%2558=588%:%
%:%2559=588%:%
%:%2560=588%:%
%:%2561=588%:%
%:%2562=589%:%
%:%2563=589%:%
%:%2564=589%:%
%:%2565=589%:%
%:%2566=589%:%
%:%2567=590%:%
%:%2568=590%:%
%:%2569=591%:%
%:%2570=591%:%
%:%2571=592%:%
%:%2572=592%:%
%:%2573=592%:%
%:%2574=592%:%
%:%2575=593%:%
%:%2576=593%:%
%:%2577=593%:%
%:%2578=593%:%
%:%2579=593%:%
%:%2580=594%:%
%:%2581=594%:%
%:%2582=595%:%
%:%2583=595%:%
%:%2584=596%:%
%:%2585=596%:%
%:%2586=596%:%
%:%2587=596%:%
%:%2588=596%:%
%:%2589=597%:%
%:%2590=597%:%
%:%2591=597%:%
%:%2592=597%:%
%:%2593=597%:%
%:%2594=597%:%
%:%2595=598%:%
%:%2596=598%:%
%:%2597=599%:%
%:%2598=599%:%
%:%2599=599%:%
%:%2600=599%:%
%:%2601=599%:%
%:%2602=599%:%
%:%2603=600%:%
%:%2604=600%:%
%:%2605=600%:%
%:%2606=600%:%
%:%2607=600%:%
%:%2608=601%:%
%:%2614=601%:%
%:%2617=602%:%
%:%2618=603%:%
%:%2619=603%:%
%:%2620=604%:%
%:%2621=605%:%
%:%2622=606%:%
%:%2629=607%:%
%:%2630=607%:%
%:%2631=608%:%
%:%2632=608%:%
%:%2633=609%:%
%:%2634=609%:%
%:%2635=609%:%
%:%2636=609%:%
%:%2637=609%:%
%:%2638=610%:%
%:%2639=610%:%
%:%2640=610%:%
%:%2641=610%:%
%:%2642=610%:%
%:%2643=611%:%
%:%2644=611%:%
%:%2645=612%:%
%:%2646=612%:%
%:%2647=612%:%
%:%2648=612%:%
%:%2649=612%:%
%:%2650=612%:%
%:%2651=613%:%
%:%2657=613%:%
%:%2660=614%:%
%:%2661=615%:%
%:%2662=615%:%
%:%2663=616%:%
%:%2664=617%:%
%:%2665=618%:%
%:%2666=619%:%
%:%2673=620%:%
%:%2674=620%:%
%:%2675=621%:%
%:%2676=621%:%
%:%2677=621%:%
%:%2678=621%:%
%:%2679=621%:%
%:%2680=622%:%
%:%2681=622%:%
%:%2682=622%:%
%:%2683=622%:%
%:%2684=623%:%
%:%2685=623%:%
%:%2686=623%:%
%:%2687=623%:%
%:%2688=624%:%
%:%2689=624%:%
%:%2690=624%:%
%:%2691=624%:%
%:%2692=625%:%
%:%2693=625%:%
%:%2694=625%:%
%:%2695=625%:%
%:%2696=625%:%
%:%2697=626%:%
%:%2698=626%:%
%:%2699=626%:%
%:%2700=626%:%
%:%2701=626%:%
%:%2702=627%:%
%:%2703=627%:%
%:%2704=627%:%
%:%2705=627%:%
%:%2706=627%:%
%:%2707=628%:%
%:%2708=628%:%
%:%2709=628%:%
%:%2710=628%:%
%:%2711=628%:%
%:%2712=629%:%
%:%2713=629%:%
%:%2714=629%:%
%:%2715=629%:%
%:%2716=630%:%
%:%2717=630%:%
%:%2718=631%:%
%:%2719=631%:%
%:%2720=631%:%
%:%2721=631%:%
%:%2722=631%:%
%:%2723=631%:%
%:%2724=632%:%
%:%2725=632%:%
%:%2726=632%:%
%:%2727=632%:%
%:%2728=632%:%
%:%2729=633%:%
%:%2730=633%:%
%:%2731=633%:%
%:%2732=633%:%
%:%2733=634%:%
%:%2734=634%:%
%:%2735=635%:%
%:%2736=635%:%
%:%2737=636%:%
%:%2738=636%:%
%:%2739=636%:%
%:%2740=636%:%
%:%2741=636%:%
%:%2742=637%:%
%:%2743=637%:%
%:%2744=637%:%
%:%2745=637%:%
%:%2746=637%:%
%:%2747=637%:%
%:%2748=638%:%
%:%2749=638%:%
%:%2750=638%:%
%:%2751=638%:%
%:%2752=638%:%
%:%2753=639%:%
%:%2754=639%:%
%:%2755=639%:%
%:%2756=639%:%
%:%2757=639%:%
%:%2758=640%:%
%:%2773=642%:%
%:%2783=644%:%
%:%2784=644%:%
%:%2785=645%:%
%:%2786=646%:%
%:%2787=647%:%
%:%2788=648%:%
%:%2789=648%:%
%:%2792=649%:%
%:%2796=649%:%
%:%2797=649%:%
%:%2798=650%:%
%:%2799=650%:%
%:%2804=650%:%
%:%2807=651%:%
%:%2808=652%:%
%:%2809=652%:%
%:%2810=653%:%
%:%2811=654%:%
%:%2812=654%:%
%:%2813=655%:%
%:%2814=656%:%
%:%2815=657%:%
%:%2816=657%:%
%:%2819=658%:%
%:%2823=658%:%
%:%2824=658%:%
%:%2825=659%:%
%:%2826=659%:%
%:%2827=660%:%
%:%2828=660%:%
%:%2829=661%:%
%:%2830=661%:%
%:%2831=661%:%
%:%2832=661%:%
%:%2833=662%:%
%:%2834=662%:%
%:%2835=662%:%
%:%2836=662%:%
%:%2837=663%:%
%:%2838=663%:%
%:%2839=663%:%
%:%2840=663%:%
%:%2841=663%:%
%:%2842=664%:%
%:%2843=664%:%
%:%2844=664%:%
%:%2845=664%:%
%:%2846=665%:%
%:%2847=665%:%
%:%2848=665%:%
%:%2849=665%:%
%:%2850=666%:%
%:%2851=666%:%
%:%2852=666%:%
%:%2853=667%:%
%:%2854=667%:%
%:%2855=667%:%
%:%2856=667%:%
%:%2857=668%:%
%:%2858=668%:%
%:%2859=669%:%
%:%2860=669%:%
%:%2861=670%:%
%:%2862=670%:%
%:%2863=670%:%
%:%2864=670%:%
%:%2865=671%:%
%:%2866=671%:%
%:%2867=672%:%
%:%2868=672%:%
%:%2869=673%:%
%:%2870=673%:%
%:%2871=674%:%
%:%2872=674%:%
%:%2873=674%:%
%:%2874=674%:%
%:%2875=675%:%
%:%2876=675%:%
%:%2877=675%:%
%:%2878=675%:%
%:%2879=676%:%
%:%2880=676%:%
%:%2881=677%:%
%:%2882=677%:%
%:%2883=678%:%
%:%2884=678%:%
%:%2885=678%:%
%:%2886=678%:%
%:%2887=679%:%
%:%2888=679%:%
%:%2889=679%:%
%:%2890=679%:%
%:%2891=679%:%
%:%2892=680%:%
%:%2893=680%:%
%:%2894=680%:%
%:%2895=680%:%
%:%2896=681%:%
%:%2897=681%:%
%:%2898=681%:%
%:%2899=681%:%
%:%2900=682%:%
%:%2901=682%:%
%:%2902=683%:%
%:%2903=683%:%
%:%2904=683%:%
%:%2905=683%:%
%:%2906=683%:%
%:%2907=684%:%
%:%2908=684%:%
%:%2909=684%:%
%:%2910=684%:%
%:%2911=684%:%
%:%2912=685%:%
%:%2913=685%:%
%:%2914=685%:%
%:%2915=685%:%
%:%2916=685%:%
%:%2917=685%:%
%:%2918=686%:%
%:%2919=686%:%
%:%2920=686%:%
%:%2921=686%:%
%:%2922=687%:%
%:%2923=687%:%
%:%2924=688%:%
%:%2930=688%:%
%:%2933=689%:%
%:%2934=690%:%
%:%2935=690%:%
%:%2936=691%:%
%:%2937=692%:%
%:%2938=693%:%
%:%2939=693%:%
%:%2940=694%:%
%:%2941=695%:%
%:%2942=696%:%
%:%2943=696%:%
%:%2944=697%:%
%:%2945=698%:%
%:%2946=699%:%
%:%2947=699%:%
%:%2948=700%:%
%:%2949=701%:%
%:%2950=702%:%
%:%2951=702%:%
%:%2952=703%:%
%:%2953=704%:%
%:%2956=705%:%
%:%2960=705%:%
%:%2961=705%:%
%:%2962=705%:%
%:%2963=706%:%
%:%2964=706%:%
%:%2969=706%:%
%:%2972=707%:%
%:%2973=708%:%
%:%2974=708%:%
%:%2977=709%:%
%:%2981=709%:%
%:%2982=709%:%
%:%2983=709%:%
%:%2984=709%:%
%:%2989=709%:%
%:%2992=710%:%
%:%2993=711%:%
%:%2994=711%:%
%:%2995=712%:%
%:%2996=713%:%
%:%3003=714%:%
%:%3004=714%:%
%:%3005=715%:%
%:%3006=715%:%
%:%3007=716%:%
%:%3008=716%:%
%:%3009=717%:%
%:%3010=717%:%
%:%3011=718%:%
%:%3012=719%:%
%:%3013=720%:%
%:%3014=720%:%
%:%3015=720%:%
%:%3016=721%:%
%:%3017=721%:%
%:%3018=722%:%
%:%3019=722%:%
%:%3020=722%:%
%:%3021=722%:%
%:%3022=723%:%
%:%3023=723%:%
%:%3024=723%:%
%:%3025=724%:%
%:%3026=724%:%
%:%3027=725%:%
%:%3028=725%:%
%:%3029=726%:%
%:%3030=726%:%
%:%3031=727%:%
%:%3032=727%:%
%:%3033=728%:%
%:%3034=729%:%
%:%3035=730%:%
%:%3036=730%:%
%:%3037=730%:%
%:%3038=731%:%
%:%3039=731%:%
%:%3040=731%:%
%:%3041=732%:%
%:%3042=732%:%
%:%3043=732%:%
%:%3044=732%:%
%:%3045=732%:%
%:%3046=733%:%
%:%3047=733%:%
%:%3048=733%:%
%:%3049=733%:%
%:%3050=734%:%
%:%3051=734%:%
%:%3052=734%:%
%:%3053=734%:%
%:%3054=735%:%
%:%3055=735%:%
%:%3056=735%:%
%:%3057=736%:%
%:%3058=736%:%
%:%3059=737%:%
%:%3060=737%:%
%:%3061=738%:%
%:%3062=738%:%
%:%3063=738%:%
%:%3064=738%:%
%:%3065=738%:%
%:%3066=739%:%
%:%3067=739%:%
%:%3068=739%:%
%:%3069=739%:%
%:%3070=740%:%
%:%3071=740%:%
%:%3072=740%:%
%:%3073=741%:%
%:%3074=741%:%
%:%3075=742%:%
%:%3076=742%:%
%:%3077=743%:%
%:%3078=743%:%
%:%3079=744%:%
%:%3080=744%:%
%:%3081=744%:%
%:%3082=744%:%
%:%3083=744%:%
%:%3084=745%:%
%:%3085=745%:%
%:%3086=746%:%
%:%3087=746%:%
%:%3088=747%:%
%:%3089=747%:%
%:%3090=747%:%
%:%3091=747%:%
%:%3092=748%:%
%:%3093=748%:%
%:%3094=748%:%
%:%3095=749%:%
%:%3096=749%:%
%:%3097=749%:%
%:%3098=749%:%
%:%3099=749%:%
%:%3100=750%:%
%:%3101=750%:%
%:%3102=751%:%
%:%3103=751%:%
%:%3104=752%:%
%:%3105=752%:%
%:%3106=752%:%
%:%3107=752%:%
%:%3108=752%:%
%:%3109=752%:%
%:%3110=753%:%
%:%3111=753%:%
%:%3112=753%:%
%:%3113=753%:%
%:%3114=753%:%
%:%3115=754%:%
%:%3116=754%:%
%:%3117=754%:%
%:%3118=754%:%
%:%3119=755%:%
%:%3125=755%:%
%:%3128=756%:%
%:%3129=757%:%
%:%3130=757%:%
%:%3131=758%:%
%:%3132=759%:%
%:%3133=760%:%
%:%3140=761%:%
%:%3141=761%:%
%:%3142=762%:%
%:%3143=762%:%
%:%3144=762%:%
%:%3145=762%:%
%:%3146=763%:%
%:%3147=763%:%
%:%3148=764%:%
%:%3149=764%:%
%:%3150=764%:%
%:%3151=765%:%
%:%3152=765%:%
%:%3153=765%:%
%:%3154=766%:%
%:%3155=766%:%
%:%3156=766%:%
%:%3157=766%:%
%:%3158=766%:%
%:%3159=767%:%
%:%3160=767%:%
%:%3161=767%:%
%:%3162=767%:%
%:%3163=767%:%
%:%3164=768%:%
%:%3165=768%:%
%:%3166=768%:%
%:%3167=768%:%
%:%3168=769%:%
%:%3169=769%:%
%:%3170=770%:%
%:%3171=771%:%
%:%3172=771%:%
%:%3173=771%:%
%:%3174=771%:%
%:%3175=771%:%
%:%3176=771%:%
%:%3177=772%:%
%:%3178=772%:%
%:%3179=772%:%
%:%3180=772%:%
%:%3181=772%:%
%:%3182=772%:%
%:%3183=773%:%
%:%3184=773%:%
%:%3185=773%:%
%:%3186=773%:%
%:%3187=773%:%
%:%3188=774%:%
%:%3194=774%:%
%:%3197=775%:%
%:%3198=776%:%
%:%3199=776%:%
%:%3200=777%:%
%:%3201=778%:%
%:%3202=778%:%
%:%3203=779%:%
%:%3204=780%:%
%:%3205=781%:%
%:%3206=782%:%
%:%3209=783%:%
%:%3213=783%:%
%:%3214=783%:%
%:%3215=784%:%
%:%3216=784%:%
%:%3217=785%:%
%:%3218=785%:%
%:%3219=786%:%
%:%3220=786%:%
%:%3221=786%:%
%:%3222=786%:%
%:%3223=787%:%
%:%3224=787%:%
%:%3225=787%:%
%:%3226=787%:%
%:%3227=788%:%
%:%3228=788%:%
%:%3229=788%:%
%:%3230=788%:%
%:%3231=788%:%
%:%3232=789%:%
%:%3233=789%:%
%:%3234=790%:%
%:%3235=790%:%
%:%3236=791%:%
%:%3237=791%:%
%:%3238=791%:%
%:%3239=791%:%
%:%3240=792%:%
%:%3241=792%:%
%:%3242=793%:%
%:%3243=793%:%
%:%3244=794%:%
%:%3245=794%:%
%:%3246=795%:%
%:%3247=795%:%
%:%3248=795%:%
%:%3249=795%:%
%:%3250=795%:%
%:%3251=796%:%
%:%3252=796%:%
%:%3253=796%:%
%:%3254=797%:%
%:%3255=797%:%
%:%3256=798%:%
%:%3257=798%:%
%:%3258=799%:%
%:%3259=799%:%
%:%3260=799%:%
%:%3261=799%:%
%:%3262=799%:%
%:%3263=800%:%
%:%3264=800%:%
%:%3265=801%:%
%:%3266=801%:%
%:%3267=802%:%
%:%3268=802%:%
%:%3269=803%:%
%:%3270=803%:%
%:%3271=803%:%
%:%3272=803%:%
%:%3273=803%:%
%:%3274=804%:%
%:%3275=804%:%
%:%3276=804%:%
%:%3277=804%:%
%:%3278=804%:%
%:%3279=805%:%
%:%3280=805%:%
%:%3281=805%:%
%:%3282=806%:%
%:%3283=806%:%
%:%3284=806%:%
%:%3285=806%:%
%:%3286=807%:%
%:%3287=807%:%
%:%3288=807%:%
%:%3289=807%:%
%:%3290=807%:%
%:%3291=808%:%
%:%3292=808%:%
%:%3293=808%:%
%:%3294=808%:%
%:%3295=808%:%
%:%3296=809%:%
%:%3297=809%:%
%:%3298=810%:%
%:%3299=810%:%
%:%3300=810%:%
%:%3301=810%:%
%:%3302=810%:%
%:%3303=811%:%
%:%3304=811%:%
%:%3305=811%:%
%:%3306=811%:%
%:%3307=812%:%
%:%3308=812%:%
%:%3309=813%:%
%:%3310=813%:%
%:%3311=814%:%
%:%3312=814%:%
%:%3313=815%:%
%:%3314=815%:%
%:%3315=815%:%
%:%3316=815%:%
%:%3317=816%:%
%:%3318=816%:%
%:%3319=816%:%
%:%3320=816%:%
%:%3321=816%:%
%:%3322=817%:%
%:%3323=817%:%
%:%3324=817%:%
%:%3325=817%:%
%:%3326=818%:%
%:%3327=818%:%
%:%3328=818%:%
%:%3329=818%:%
%:%3330=818%:%
%:%3331=819%:%
%:%3332=819%:%
%:%3333=819%:%
%:%3334=819%:%
%:%3335=820%:%
%:%3336=820%:%
%:%3337=821%:%
%:%3343=821%:%
%:%3346=822%:%
%:%3347=823%:%
%:%3348=823%:%
%:%3349=824%:%
%:%3350=825%:%
%:%3351=826%:%
%:%3352=826%:%
%:%3355=827%:%
%:%3359=827%:%
%:%3360=827%:%
%:%3361=828%:%
%:%3362=828%:%
%:%3363=829%:%
%:%3364=829%:%
%:%3365=830%:%
%:%3366=830%:%
%:%3367=830%:%
%:%3368=830%:%
%:%3369=831%:%
%:%3370=831%:%
%:%3371=832%:%
%:%3372=832%:%
%:%3373=833%:%
%:%3374=833%:%
%:%3375=833%:%
%:%3376=833%:%
%:%3377=834%:%
%:%3378=834%:%
%:%3379=834%:%
%:%3380=834%:%
%:%3381=835%:%
%:%3387=835%:%
%:%3390=836%:%
%:%3391=837%:%
%:%3392=837%:%
%:%3393=838%:%
%:%3394=839%:%
%:%3395=839%:%
%:%3396=840%:%
%:%3397=841%:%
%:%3398=842%:%
%:%3405=843%:%
%:%3406=843%:%
%:%3407=844%:%
%:%3408=844%:%
%:%3409=844%:%
%:%3410=844%:%
%:%3411=845%:%
%:%3412=845%:%
%:%3413=845%:%
%:%3414=845%:%
%:%3415=846%:%
%:%3416=846%:%
%:%3417=847%:%
%:%3418=847%:%
%:%3419=848%:%
%:%3420=848%:%
%:%3421=849%:%
%:%3422=849%:%
%:%3423=849%:%
%:%3424=849%:%
%:%3425=849%:%
%:%3426=849%:%
%:%3427=850%:%
%:%3428=850%:%
%:%3429=851%:%
%:%3430=851%:%
%:%3431=852%:%
%:%3432=852%:%
%:%3433=852%:%
%:%3434=852%:%
%:%3435=852%:%
%:%3436=853%:%
%:%3437=853%:%
%:%3438=854%:%
%:%3439=854%:%
%:%3440=854%:%
%:%3441=854%:%
%:%3442=854%:%
%:%3443=854%:%
%:%3444=855%:%
%:%3445=855%:%
%:%3446=856%:%
%:%3452=856%:%
%:%3455=857%:%
%:%3456=858%:%
%:%3457=858%:%
%:%3458=859%:%
%:%3459=860%:%
%:%3460=861%:%
%:%3461=861%:%
%:%3462=862%:%
%:%3463=863%:%
%:%3464=864%:%
%:%3471=865%:%
%:%3472=865%:%
%:%3473=866%:%
%:%3474=866%:%
%:%3475=867%:%
%:%3476=868%:%
%:%3477=868%:%
%:%3478=868%:%
%:%3479=868%:%
%:%3480=869%:%
%:%3481=869%:%
%:%3482=869%:%
%:%3483=870%:%
%:%3484=870%:%
%:%3485=870%:%
%:%3486=871%:%
%:%3487=872%:%
%:%3488=872%:%
%:%3489=872%:%
%:%3490=873%:%
%:%3491=874%:%
%:%3492=874%:%
%:%3493=875%:%
%:%3494=875%:%
%:%3495=876%:%
%:%3496=876%:%
%:%3497=877%:%
%:%3498=878%:%
%:%3499=878%:%
%:%3500=879%:%
%:%3501=880%:%
%:%3502=880%:%
%:%3503=880%:%
%:%3504=880%:%
%:%3505=880%:%
%:%3506=881%:%
%:%3507=882%:%
%:%3508=882%:%
%:%3509=882%:%
%:%3510=882%:%
%:%3511=882%:%
%:%3512=883%:%
%:%3513=883%:%
%:%3514=883%:%
%:%3515=883%:%
%:%3516=883%:%
%:%3517=884%:%
%:%3518=884%:%
%:%3519=884%:%
%:%3520=884%:%
%:%3521=884%:%
%:%3522=885%:%
%:%3523=886%:%
%:%3524=886%:%
%:%3525=886%:%
%:%3526=886%:%
%:%3527=886%:%
%:%3528=887%:%
%:%3534=887%:%
%:%3537=888%:%
%:%3538=889%:%
%:%3539=889%:%
%:%3540=890%:%
%:%3541=891%:%
%:%3542=892%:%
%:%3543=893%:%
%:%3550=894%:%
%:%3551=894%:%
%:%3552=895%:%
%:%3553=895%:%
%:%3554=895%:%
%:%3555=895%:%
%:%3556=896%:%
%:%3557=896%:%
%:%3558=896%:%
%:%3559=896%:%
%:%3560=897%:%
%:%3561=897%:%
%:%3562=897%:%
%:%3563=898%:%
%:%3564=898%:%
%:%3565=898%:%
%:%3566=898%:%
%:%3567=899%:%
%:%3573=899%:%
%:%3576=900%:%
%:%3577=901%:%
%:%3578=901%:%
%:%3579=902%:%
%:%3580=903%:%
%:%3581=904%:%
%:%3582=905%:%
%:%3589=906%:%
%:%3590=906%:%
%:%3591=907%:%
%:%3592=907%:%
%:%3593=907%:%
%:%3594=907%:%
%:%3595=908%:%
%:%3596=908%:%
%:%3597=908%:%
%:%3598=908%:%
%:%3599=909%:%
%:%3600=909%:%
%:%3601=909%:%
%:%3602=909%:%
%:%3603=909%:%
%:%3604=910%:%
%:%3605=910%:%
%:%3606=911%:%
%:%3607=911%:%
%:%3608=912%:%
%:%3609=912%:%
%:%3610=913%:%
%:%3611=913%:%
%:%3612=913%:%
%:%3613=914%:%
%:%3614=914%:%
%:%3615=915%:%
%:%3616=915%:%
%:%3617=915%:%
%:%3618=915%:%
%:%3619=915%:%
%:%3620=915%:%
%:%3621=916%:%
%:%3622=916%:%
%:%3623=916%:%
%:%3624=916%:%
%:%3625=917%:%
%:%3626=917%:%
%:%3627=917%:%
%:%3628=917%:%
%:%3629=917%:%
%:%3630=918%:%
%:%3631=918%:%
%:%3632=919%:%
%:%3633=919%:%
%:%3634=919%:%
%:%3635=919%:%
%:%3636=920%:%
%:%3637=920%:%
%:%3638=920%:%
%:%3639=920%:%
%:%3640=920%:%
%:%3641=920%:%
%:%3642=921%:%
%:%3643=921%:%
%:%3644=921%:%
%:%3645=921%:%
%:%3646=922%:%
%:%3647=922%:%
%:%3648=922%:%
%:%3649=922%:%
%:%3650=923%:%
%:%3651=923%:%
%:%3652=923%:%
%:%3653=924%:%
%:%3654=924%:%
%:%3655=924%:%
%:%3656=924%:%
%:%3657=924%:%
%:%3658=925%:%
%:%3659=925%:%
%:%3660=926%:%
%:%3661=926%:%
%:%3662=926%:%
%:%3663=926%:%
%:%3664=926%:%
%:%3665=927%:%
%:%3666=927%:%
%:%3667=928%:%
%:%3668=928%:%
%:%3669=928%:%
%:%3670=928%:%
%:%3671=928%:%
%:%3672=929%:%
%:%3678=929%:%
%:%3681=930%:%
%:%3682=931%:%
%:%3683=931%:%
%:%3686=932%:%
%:%3690=932%:%
%:%3691=932%:%
%:%3692=932%:%
%:%3693=932%:%
%:%3698=932%:%
%:%3701=933%:%
%:%3702=934%:%
%:%3703=934%:%
%:%3706=935%:%
%:%3710=935%:%
%:%3711=935%:%
%:%3712=936%:%
%:%3713=936%:%
%:%3714=937%:%
%:%3715=937%:%
%:%3716=938%:%
%:%3717=938%:%
%:%3718=938%:%
%:%3719=938%:%
%:%3720=939%:%
%:%3721=939%:%
%:%3722=939%:%
%:%3723=939%:%
%:%3724=939%:%
%:%3725=940%:%
%:%3726=940%:%
%:%3727=941%:%
%:%3728=941%:%
%:%3729=942%:%
%:%3730=942%:%
%:%3731=942%:%
%:%3732=942%:%
%:%3733=943%:%
%:%3734=943%:%
%:%3735=944%:%
%:%3736=944%:%
%:%3737=945%:%
%:%3738=945%:%
%:%3739=946%:%
%:%3740=946%:%
%:%3741=946%:%
%:%3742=946%:%
%:%3743=946%:%
%:%3744=947%:%
%:%3745=947%:%
%:%3746=947%:%
%:%3747=947%:%
%:%3748=947%:%
%:%3749=948%:%
%:%3750=948%:%
%:%3751=948%:%
%:%3752=948%:%
%:%3753=948%:%
%:%3754=949%:%
%:%3755=949%:%
%:%3756=950%:%
%:%3757=950%:%
%:%3758=951%:%
%:%3759=951%:%
%:%3760=951%:%
%:%3761=951%:%
%:%3762=951%:%
%:%3763=951%:%
%:%3764=952%:%
%:%3765=952%:%
%:%3766=952%:%
%:%3767=952%:%
%:%3768=952%:%
%:%3769=953%:%
%:%3770=953%:%
%:%3771=954%:%
%:%3777=954%:%
%:%3780=955%:%
%:%3781=956%:%
%:%3782=956%:%
%:%3783=957%:%
%:%3784=958%:%
%:%3785=958%:%
%:%3792=959%:%
%:%3793=959%:%
%:%3794=960%:%
%:%3795=960%:%
%:%3796=960%:%
%:%3797=961%:%
%:%3798=961%:%
%:%3799=961%:%
%:%3800=961%:%
%:%3801=962%:%
%:%3807=962%:%
%:%3810=963%:%
%:%3811=964%:%
%:%3812=964%:%
%:%3813=965%:%
%:%3814=966%:%
%:%3815=967%:%
%:%3816=968%:%
%:%3817=969%:%
%:%3818=970%:%
%:%3819=971%:%
%:%3820=971%:%
%:%3823=972%:%
%:%3827=972%:%
%:%3828=972%:%
%:%3829=972%:%
%:%3830=972%:%
%:%3835=972%:%
%:%3838=973%:%
%:%3839=974%:%
%:%3840=974%:%
%:%3843=975%:%
%:%3847=975%:%
%:%3848=975%:%
%:%3849=975%:%
%:%3850=975%:%
%:%3855=975%:%
%:%3858=976%:%
%:%3859=977%:%
%:%3860=977%:%
%:%3861=978%:%
%:%3862=979%:%
%:%3863=979%:%
%:%3866=980%:%
%:%3870=980%:%
%:%3871=980%:%
%:%3872=980%:%
%:%3877=980%:%
%:%3880=981%:%
%:%3881=982%:%
%:%3882=982%:%
%:%3884=982%:%
%:%3888=982%:%
%:%3889=982%:%
%:%3897=982%:%
%:%3898=983%:%
%:%3899=984%:%
%:%3900=984%:%
%:%3907=985%:%
%:%3908=985%:%
%:%3909=986%:%
%:%3910=986%:%
%:%3911=986%:%
%:%3912=986%:%
%:%3913=986%:%
%:%3914=987%:%
%:%3915=987%:%
%:%3916=988%:%
%:%3917=988%:%
%:%3918=988%:%
%:%3919=988%:%
%:%3920=989%:%
%:%3921=989%:%
%:%3922=989%:%
%:%3923=989%:%
%:%3924=990%:%
%:%3925=990%:%
%:%3926=991%:%
%:%3927=991%:%
%:%3928=991%:%
%:%3929=991%:%
%:%3930=991%:%
%:%3931=992%:%
%:%3937=992%:%
%:%3940=993%:%
%:%3941=994%:%
%:%3942=994%:%
%:%3944=994%:%
%:%3948=994%:%
%:%3949=994%:%
%:%3957=994%:%
%:%3958=995%:%
%:%3959=996%:%
%:%3960=996%:%
%:%3961=997%:%
%:%3962=998%:%
%:%3963=998%:%
%:%3964=999%:%
%:%3965=1000%:%
%:%3966=1001%:%
%:%3967=1001%:%
%:%3970=1002%:%
%:%3974=1002%:%
%:%3975=1002%:%
%:%3980=1002%:%
%:%3983=1003%:%
%:%3984=1004%:%
%:%3985=1004%:%
%:%3988=1005%:%
%:%3992=1005%:%
%:%3993=1005%:%
%:%3994=1005%:%
%:%3999=1005%:%
%:%4002=1006%:%
%:%4003=1007%:%
%:%4004=1007%:%
%:%4007=1008%:%
%:%4011=1008%:%
%:%4012=1008%:%
%:%4013=1008%:%
%:%4014=1008%:%
%:%4019=1008%:%
%:%4022=1009%:%
%:%4023=1010%:%
%:%4024=1010%:%
%:%4027=1011%:%
%:%4031=1011%:%
%:%4032=1011%:%
%:%4033=1011%:%
%:%4034=1011%:%
%:%4039=1011%:%
%:%4042=1012%:%
%:%4043=1013%:%
%:%4044=1013%:%
%:%4047=1014%:%
%:%4051=1014%:%
%:%4052=1014%:%
%:%4053=1014%:%
%:%4054=1014%:%
%:%4059=1014%:%
%:%4064=1015%:%
%:%4069=1016%:%

%
\begin{isabellebody}%
\setisabellecontext{Labeled{\isacharunderscore}{\kern0pt}Tree{\isacharunderscore}{\kern0pt}Enumeration}%
%
\isadelimdocument
%
\endisadelimdocument
%
\isatagdocument
%
\isamarkupsection{Labeled Trees%
}
\isamarkuptrue%
%
\endisatagdocument
{\isafolddocument}%
%
\isadelimdocument
%
\endisadelimdocument
%
\isadelimtheory
%
\endisadelimtheory
%
\isatagtheory
\isacommand{theory}\isamarkupfalse%
\ Labeled{\isacharunderscore}{\kern0pt}Tree{\isacharunderscore}{\kern0pt}Enumeration\isanewline
\ \ \isakeyword{imports}\ Tree{\isacharunderscore}{\kern0pt}Graph\ Combinatorial{\isacharunderscore}{\kern0pt}Enumeration{\isacharunderscore}{\kern0pt}Algorithms{\isachardot}{\kern0pt}n{\isacharunderscore}{\kern0pt}Sequences\isanewline
\isakeyword{begin}%
\endisatagtheory
{\isafoldtheory}%
%
\isadelimtheory
%
\endisadelimtheory
%
\isadelimdocument
%
\endisadelimdocument
%
\isatagdocument
%
\isamarkupsubsection{Definition%
}
\isamarkuptrue%
%
\endisatagdocument
{\isafolddocument}%
%
\isadelimdocument
%
\endisadelimdocument
\isacommand{definition}\isamarkupfalse%
\ labeled{\isacharunderscore}{\kern0pt}trees\ {\isacharcolon}{\kern0pt}{\isacharcolon}{\kern0pt}\ {\isachardoublequoteopen}{\isacharprime}{\kern0pt}a\ set\ {\isasymRightarrow}\ {\isacharprime}{\kern0pt}a\ pregraph\ set{\isachardoublequoteclose}\ \isakeyword{where}\isanewline
\ \ {\isachardoublequoteopen}labeled{\isacharunderscore}{\kern0pt}trees\ V\ {\isacharequal}{\kern0pt}\ {\isacharbraceleft}{\kern0pt}{\isacharparenleft}{\kern0pt}V{\isacharcomma}{\kern0pt}E{\isacharparenright}{\kern0pt}\ {\isacharbar}{\kern0pt}\ E{\isachardot}{\kern0pt}\ tree\ V\ E{\isacharbraceright}{\kern0pt}{\isachardoublequoteclose}%
\isadelimdocument
%
\endisadelimdocument
%
\isatagdocument
%
\isamarkupsubsection{Algorithm%
}
\isamarkuptrue%
%
\endisatagdocument
{\isafolddocument}%
%
\isadelimdocument
%
\endisadelimdocument
%
\begin{isamarkuptext}%
Prüfer sequence to tree%
\end{isamarkuptext}\isamarkuptrue%
\isacommand{definition}\isamarkupfalse%
\ prufer{\isacharunderscore}{\kern0pt}sequences\ {\isacharcolon}{\kern0pt}{\isacharcolon}{\kern0pt}\ {\isachardoublequoteopen}{\isacharprime}{\kern0pt}a\ list\ {\isasymRightarrow}\ {\isacharprime}{\kern0pt}a\ list\ set{\isachardoublequoteclose}\ \isakeyword{where}\isanewline
\ \ {\isachardoublequoteopen}prufer{\isacharunderscore}{\kern0pt}sequences\ verts\ {\isacharequal}{\kern0pt}\ n{\isacharunderscore}{\kern0pt}sequences\ {\isacharparenleft}{\kern0pt}set\ verts{\isacharparenright}{\kern0pt}\ {\isacharparenleft}{\kern0pt}length\ verts\ {\isacharminus}{\kern0pt}\ {\isadigit{2}}{\isacharparenright}{\kern0pt}{\isachardoublequoteclose}\isanewline
\isanewline
\isacommand{fun}\isamarkupfalse%
\ prufer{\isacharunderscore}{\kern0pt}seq{\isacharunderscore}{\kern0pt}to{\isacharunderscore}{\kern0pt}tree{\isacharunderscore}{\kern0pt}edges\ {\isacharcolon}{\kern0pt}{\isacharcolon}{\kern0pt}\ {\isachardoublequoteopen}{\isacharprime}{\kern0pt}a\ list\ {\isasymRightarrow}\ {\isacharprime}{\kern0pt}a\ list\ {\isasymRightarrow}\ {\isacharparenleft}{\kern0pt}{\isacharprime}{\kern0pt}a\ {\isasymtimes}\ {\isacharprime}{\kern0pt}a{\isacharparenright}{\kern0pt}\ list{\isachardoublequoteclose}\ \isakeyword{where}\isanewline
\ \ {\isachardoublequoteopen}prufer{\isacharunderscore}{\kern0pt}seq{\isacharunderscore}{\kern0pt}to{\isacharunderscore}{\kern0pt}tree{\isacharunderscore}{\kern0pt}edges\ {\isacharbrackleft}{\kern0pt}v{\isacharcomma}{\kern0pt}w{\isacharbrackright}{\kern0pt}\ {\isacharbrackleft}{\kern0pt}{\isacharbrackright}{\kern0pt}\ {\isacharequal}{\kern0pt}\ {\isacharbrackleft}{\kern0pt}{\isacharparenleft}{\kern0pt}v{\isacharcomma}{\kern0pt}w{\isacharparenright}{\kern0pt}{\isacharbrackright}{\kern0pt}{\isachardoublequoteclose}\isanewline
{\isacharbar}{\kern0pt}\ {\isachardoublequoteopen}prufer{\isacharunderscore}{\kern0pt}seq{\isacharunderscore}{\kern0pt}to{\isacharunderscore}{\kern0pt}tree{\isacharunderscore}{\kern0pt}edges\ verts\ {\isacharparenleft}{\kern0pt}a{\isacharhash}{\kern0pt}seq{\isacharparenright}{\kern0pt}\ {\isacharequal}{\kern0pt}\isanewline
\ \ \ \ {\isacharparenleft}{\kern0pt}case\ find\ {\isacharparenleft}{\kern0pt}{\isasymlambda}x{\isachardot}{\kern0pt}\ x\ {\isasymnotin}\ set\ {\isacharparenleft}{\kern0pt}a{\isacharhash}{\kern0pt}seq{\isacharparenright}{\kern0pt}{\isacharparenright}{\kern0pt}\ verts\ of\isanewline
\ \ \ \ \ \ Some\ b\ {\isasymRightarrow}\ {\isacharparenleft}{\kern0pt}a{\isacharcomma}{\kern0pt}b{\isacharparenright}{\kern0pt}\ {\isacharhash}{\kern0pt}\ prufer{\isacharunderscore}{\kern0pt}seq{\isacharunderscore}{\kern0pt}to{\isacharunderscore}{\kern0pt}tree{\isacharunderscore}{\kern0pt}edges\ {\isacharparenleft}{\kern0pt}remove{\isadigit{1}}\ b\ verts{\isacharparenright}{\kern0pt}\ seq{\isacharparenright}{\kern0pt}{\isachardoublequoteclose}\isanewline
\isanewline
\isacommand{definition}\isamarkupfalse%
\ edges{\isacharunderscore}{\kern0pt}of{\isacharunderscore}{\kern0pt}edge{\isacharunderscore}{\kern0pt}list\ {\isacharcolon}{\kern0pt}{\isacharcolon}{\kern0pt}\ {\isachardoublequoteopen}{\isacharparenleft}{\kern0pt}{\isacharprime}{\kern0pt}a\ {\isasymtimes}\ {\isacharprime}{\kern0pt}a{\isacharparenright}{\kern0pt}\ list\ {\isasymRightarrow}\ {\isacharprime}{\kern0pt}a\ edge\ set{\isachardoublequoteclose}\ \isakeyword{where}\isanewline
\ \ {\isachardoublequoteopen}edges{\isacharunderscore}{\kern0pt}of{\isacharunderscore}{\kern0pt}edge{\isacharunderscore}{\kern0pt}list\ edge{\isacharunderscore}{\kern0pt}list\ {\isasymequiv}\ mk{\isacharunderscore}{\kern0pt}edge\ {\isacharbackquote}{\kern0pt}\ set\ edge{\isacharunderscore}{\kern0pt}list{\isachardoublequoteclose}\isanewline
\isanewline
\isacommand{definition}\isamarkupfalse%
\ prufer{\isacharunderscore}{\kern0pt}seq{\isacharunderscore}{\kern0pt}to{\isacharunderscore}{\kern0pt}tree\ {\isacharcolon}{\kern0pt}{\isacharcolon}{\kern0pt}\ {\isachardoublequoteopen}{\isacharprime}{\kern0pt}a\ list\ {\isasymRightarrow}\ {\isacharprime}{\kern0pt}a\ list\ {\isasymRightarrow}\ {\isacharprime}{\kern0pt}a\ pregraph{\isachardoublequoteclose}\ \isakeyword{where}\isanewline
\ \ {\isachardoublequoteopen}prufer{\isacharunderscore}{\kern0pt}seq{\isacharunderscore}{\kern0pt}to{\isacharunderscore}{\kern0pt}tree\ verts\ seq\ {\isacharequal}{\kern0pt}\ {\isacharparenleft}{\kern0pt}set\ verts{\isacharcomma}{\kern0pt}\ edges{\isacharunderscore}{\kern0pt}of{\isacharunderscore}{\kern0pt}edge{\isacharunderscore}{\kern0pt}list\ {\isacharparenleft}{\kern0pt}prufer{\isacharunderscore}{\kern0pt}seq{\isacharunderscore}{\kern0pt}to{\isacharunderscore}{\kern0pt}tree{\isacharunderscore}{\kern0pt}edges\ verts\ seq{\isacharparenright}{\kern0pt}{\isacharparenright}{\kern0pt}{\isachardoublequoteclose}\isanewline
\isanewline
\isacommand{definition}\isamarkupfalse%
\ labeled{\isacharunderscore}{\kern0pt}tree{\isacharunderscore}{\kern0pt}enum\ {\isacharcolon}{\kern0pt}{\isacharcolon}{\kern0pt}\ {\isachardoublequoteopen}{\isacharprime}{\kern0pt}a\ list\ {\isasymRightarrow}\ {\isacharprime}{\kern0pt}a\ pregraph\ list{\isachardoublequoteclose}\ \isakeyword{where}\isanewline
\ \ {\isachardoublequoteopen}labeled{\isacharunderscore}{\kern0pt}tree{\isacharunderscore}{\kern0pt}enum\ verts\ {\isacharequal}{\kern0pt}\ map\ {\isacharparenleft}{\kern0pt}prufer{\isacharunderscore}{\kern0pt}seq{\isacharunderscore}{\kern0pt}to{\isacharunderscore}{\kern0pt}tree\ verts{\isacharparenright}{\kern0pt}\ {\isacharparenleft}{\kern0pt}n{\isacharunderscore}{\kern0pt}sequence{\isacharunderscore}{\kern0pt}enum\ verts\ {\isacharparenleft}{\kern0pt}length\ verts\ {\isacharminus}{\kern0pt}\ {\isadigit{2}}{\isacharparenright}{\kern0pt}{\isacharparenright}{\kern0pt}{\isachardoublequoteclose}%
\isadelimdocument
%
\endisadelimdocument
%
\isatagdocument
%
\isamarkupsubsection{Correctness%
}
\isamarkuptrue%
%
\endisatagdocument
{\isafolddocument}%
%
\isadelimdocument
%
\endisadelimdocument
%
\begin{isamarkuptext}%
Tree to Prüfer sequence%
\end{isamarkuptext}\isamarkuptrue%
\isacommand{definition}\isamarkupfalse%
\ incident{\isacharunderscore}{\kern0pt}edges\ {\isacharcolon}{\kern0pt}{\isacharcolon}{\kern0pt}\ {\isachardoublequoteopen}{\isacharprime}{\kern0pt}a\ {\isasymRightarrow}\ {\isacharparenleft}{\kern0pt}{\isacharprime}{\kern0pt}a\ {\isasymtimes}\ {\isacharprime}{\kern0pt}a{\isacharparenright}{\kern0pt}\ list\ {\isasymRightarrow}\ {\isacharparenleft}{\kern0pt}{\isacharprime}{\kern0pt}a\ {\isasymtimes}\ {\isacharprime}{\kern0pt}a{\isacharparenright}{\kern0pt}\ list{\isachardoublequoteclose}\ \isakeyword{where}\isanewline
\ \ {\isachardoublequoteopen}incident{\isacharunderscore}{\kern0pt}edges\ v\ edge{\isacharunderscore}{\kern0pt}list\ {\isacharequal}{\kern0pt}\ filter\ {\isacharparenleft}{\kern0pt}{\isasymlambda}{\isacharparenleft}{\kern0pt}u{\isacharcomma}{\kern0pt}w{\isacharparenright}{\kern0pt}{\isachardot}{\kern0pt}\ u\ {\isacharequal}{\kern0pt}\ v\ {\isasymor}\ w\ {\isacharequal}{\kern0pt}\ v{\isacharparenright}{\kern0pt}\ edge{\isacharunderscore}{\kern0pt}list{\isachardoublequoteclose}\isanewline
\isanewline
\isacommand{abbreviation}\isamarkupfalse%
\ {\isachardoublequoteopen}degree\ v\ edge{\isacharunderscore}{\kern0pt}list\ {\isasymequiv}\ length\ {\isacharparenleft}{\kern0pt}incident{\isacharunderscore}{\kern0pt}edges\ v\ edge{\isacharunderscore}{\kern0pt}list{\isacharparenright}{\kern0pt}{\isachardoublequoteclose}\isanewline
\isanewline
\isacommand{fun}\isamarkupfalse%
\ neighbor\ {\isacharcolon}{\kern0pt}{\isacharcolon}{\kern0pt}\ {\isachardoublequoteopen}{\isacharprime}{\kern0pt}a\ {\isasymRightarrow}\ {\isacharparenleft}{\kern0pt}{\isacharprime}{\kern0pt}a\ {\isasymtimes}\ {\isacharprime}{\kern0pt}a{\isacharparenright}{\kern0pt}\ list\ {\isasymRightarrow}\ {\isacharprime}{\kern0pt}a{\isachardoublequoteclose}\ \isakeyword{where}\isanewline
\ \ {\isachardoublequoteopen}neighbor\ v\ {\isacharbrackleft}{\kern0pt}{\isacharbrackright}{\kern0pt}\ {\isacharequal}{\kern0pt}\ undefined{\isachardoublequoteclose}\isanewline
{\isacharbar}{\kern0pt}\ {\isachardoublequoteopen}neighbor\ v\ {\isacharparenleft}{\kern0pt}{\isacharparenleft}{\kern0pt}u{\isacharcomma}{\kern0pt}w{\isacharparenright}{\kern0pt}{\isacharhash}{\kern0pt}edges{\isacharparenright}{\kern0pt}\ {\isacharequal}{\kern0pt}\ {\isacharparenleft}{\kern0pt}if\ v\ {\isacharequal}{\kern0pt}\ u\ then\ w\ else\ if\ v\ {\isacharequal}{\kern0pt}\ w\ then\ u\ else\ neighbor\ v\ edges{\isacharparenright}{\kern0pt}{\isachardoublequoteclose}\isanewline
\isanewline
\isacommand{definition}\isamarkupfalse%
\ remove{\isacharunderscore}{\kern0pt}vertex\ {\isacharcolon}{\kern0pt}{\isacharcolon}{\kern0pt}\ {\isachardoublequoteopen}{\isacharprime}{\kern0pt}a\ {\isasymRightarrow}\ {\isacharparenleft}{\kern0pt}{\isacharprime}{\kern0pt}a\ {\isasymtimes}\ {\isacharprime}{\kern0pt}a{\isacharparenright}{\kern0pt}\ list\ {\isasymRightarrow}\ {\isacharparenleft}{\kern0pt}{\isacharprime}{\kern0pt}a\ {\isasymtimes}\ {\isacharprime}{\kern0pt}a{\isacharparenright}{\kern0pt}\ list{\isachardoublequoteclose}\ \isakeyword{where}\isanewline
\ \ {\isachardoublequoteopen}remove{\isacharunderscore}{\kern0pt}vertex\ v\ {\isacharequal}{\kern0pt}\ filter\ {\isacharparenleft}{\kern0pt}{\isasymlambda}{\isacharparenleft}{\kern0pt}u{\isacharcomma}{\kern0pt}w{\isacharparenright}{\kern0pt}{\isachardot}{\kern0pt}\ u\ {\isasymnoteq}\ v\ {\isasymand}\ w\ {\isasymnoteq}\ v{\isacharparenright}{\kern0pt}{\isachardoublequoteclose}\isanewline
\isanewline
\isacommand{lemma}\isamarkupfalse%
\ find{\isacharunderscore}{\kern0pt}in{\isacharunderscore}{\kern0pt}list{\isacharbrackleft}{\kern0pt}termination{\isacharunderscore}{\kern0pt}simp{\isacharbrackright}{\kern0pt}{\isacharcolon}{\kern0pt}\ {\isachardoublequoteopen}find\ P\ verts\ {\isacharequal}{\kern0pt}\ Some\ v\ {\isasymLongrightarrow}\ v\ {\isasymin}\ set\ verts{\isachardoublequoteclose}\isanewline
%
\isadelimproof
\ \ %
\endisadelimproof
%
\isatagproof
\isacommand{by}\isamarkupfalse%
\ {\isacharparenleft}{\kern0pt}metis\ find{\isacharunderscore}{\kern0pt}Some{\isacharunderscore}{\kern0pt}iff\ nth{\isacharunderscore}{\kern0pt}mem{\isacharparenright}{\kern0pt}%
\endisatagproof
{\isafoldproof}%
%
\isadelimproof
\isanewline
%
\endisadelimproof
\isanewline
\isacommand{lemma}\isamarkupfalse%
\ {\isacharbrackleft}{\kern0pt}termination{\isacharunderscore}{\kern0pt}simp{\isacharbrackright}{\kern0pt}{\isacharcolon}{\kern0pt}\ {\isachardoublequoteopen}find\ P\ verts\ {\isacharequal}{\kern0pt}\ Some\ v\ {\isasymLongrightarrow}\ length\ verts\ {\isacharminus}{\kern0pt}\ Suc\ {\isadigit{0}}\ {\isacharless}{\kern0pt}\ length\ verts{\isachardoublequoteclose}\isanewline
%
\isadelimproof
\ \ %
\endisadelimproof
%
\isatagproof
\isacommand{by}\isamarkupfalse%
\ {\isacharparenleft}{\kern0pt}meson\ diff{\isacharunderscore}{\kern0pt}Suc{\isacharunderscore}{\kern0pt}less\ length{\isacharunderscore}{\kern0pt}pos{\isacharunderscore}{\kern0pt}if{\isacharunderscore}{\kern0pt}in{\isacharunderscore}{\kern0pt}set\ find{\isacharunderscore}{\kern0pt}in{\isacharunderscore}{\kern0pt}list{\isacharparenright}{\kern0pt}%
\endisatagproof
{\isafoldproof}%
%
\isadelimproof
\isanewline
%
\endisadelimproof
\isanewline
\isacommand{fun}\isamarkupfalse%
\ tree{\isacharunderscore}{\kern0pt}to{\isacharunderscore}{\kern0pt}prufer{\isacharunderscore}{\kern0pt}seq\ {\isacharcolon}{\kern0pt}{\isacharcolon}{\kern0pt}\ {\isachardoublequoteopen}{\isacharprime}{\kern0pt}a\ list\ {\isasymRightarrow}\ {\isacharparenleft}{\kern0pt}{\isacharprime}{\kern0pt}a\ {\isasymtimes}\ {\isacharprime}{\kern0pt}a{\isacharparenright}{\kern0pt}\ list\ {\isasymRightarrow}\ {\isacharprime}{\kern0pt}a\ list{\isachardoublequoteclose}\ \isakeyword{where}\isanewline
\ \ {\isachardoublequoteopen}tree{\isacharunderscore}{\kern0pt}to{\isacharunderscore}{\kern0pt}prufer{\isacharunderscore}{\kern0pt}seq\ verts\ {\isacharbrackleft}{\kern0pt}{\isacharbrackright}{\kern0pt}\ {\isacharequal}{\kern0pt}\ undefined{\isachardoublequoteclose}\isanewline
{\isacharbar}{\kern0pt}\ {\isachardoublequoteopen}tree{\isacharunderscore}{\kern0pt}to{\isacharunderscore}{\kern0pt}prufer{\isacharunderscore}{\kern0pt}seq\ verts\ {\isacharbrackleft}{\kern0pt}{\isacharparenleft}{\kern0pt}u{\isacharcomma}{\kern0pt}w{\isacharparenright}{\kern0pt}{\isacharbrackright}{\kern0pt}\ {\isacharequal}{\kern0pt}\ {\isacharbrackleft}{\kern0pt}{\isacharbrackright}{\kern0pt}{\isachardoublequoteclose}\isanewline
{\isacharbar}{\kern0pt}\ {\isachardoublequoteopen}tree{\isacharunderscore}{\kern0pt}to{\isacharunderscore}{\kern0pt}prufer{\isacharunderscore}{\kern0pt}seq\ verts\ edges\ {\isacharequal}{\kern0pt}\isanewline
\ \ \ \ {\isacharparenleft}{\kern0pt}case\ find\ {\isacharparenleft}{\kern0pt}{\isasymlambda}v{\isachardot}{\kern0pt}\ degree\ v\ edges\ {\isacharequal}{\kern0pt}\ {\isadigit{1}}{\isacharparenright}{\kern0pt}\ verts\ of\isanewline
\ \ \ \ \ \ Some\ leaf\ {\isasymRightarrow}\ neighbor\ leaf\ edges\ {\isacharhash}{\kern0pt}\ tree{\isacharunderscore}{\kern0pt}to{\isacharunderscore}{\kern0pt}prufer{\isacharunderscore}{\kern0pt}seq\ {\isacharparenleft}{\kern0pt}remove{\isadigit{1}}\ leaf\ verts{\isacharparenright}{\kern0pt}\ {\isacharparenleft}{\kern0pt}remove{\isacharunderscore}{\kern0pt}vertex\ leaf\ edges{\isacharparenright}{\kern0pt}{\isacharparenright}{\kern0pt}{\isachardoublequoteclose}\isanewline
\isanewline
\isacommand{lemma}\isamarkupfalse%
\ remove{\isacharunderscore}{\kern0pt}vertex{\isacharcolon}{\kern0pt}\ {\isachardoublequoteopen}edges{\isacharunderscore}{\kern0pt}of{\isacharunderscore}{\kern0pt}edge{\isacharunderscore}{\kern0pt}list\ {\isacharparenleft}{\kern0pt}remove{\isacharunderscore}{\kern0pt}vertex\ v\ edge{\isacharunderscore}{\kern0pt}list{\isacharparenright}{\kern0pt}\ {\isacharequal}{\kern0pt}\ {\isacharbraceleft}{\kern0pt}e\ {\isasymin}\ edges{\isacharunderscore}{\kern0pt}of{\isacharunderscore}{\kern0pt}edge{\isacharunderscore}{\kern0pt}list\ edge{\isacharunderscore}{\kern0pt}list{\isachardot}{\kern0pt}\ v\ {\isasymnotin}\ e{\isacharbraceright}{\kern0pt}{\isachardoublequoteclose}\isanewline
%
\isadelimproof
\ \ %
\endisadelimproof
%
\isatagproof
\isacommand{unfolding}\isamarkupfalse%
\ remove{\isacharunderscore}{\kern0pt}vertex{\isacharunderscore}{\kern0pt}def\ \isacommand{by}\isamarkupfalse%
\ {\isacharparenleft}{\kern0pt}auto\ simp{\isacharcolon}{\kern0pt}\ edges{\isacharunderscore}{\kern0pt}of{\isacharunderscore}{\kern0pt}edge{\isacharunderscore}{\kern0pt}list{\isacharunderscore}{\kern0pt}def{\isacharparenright}{\kern0pt}%
\endisatagproof
{\isafoldproof}%
%
\isadelimproof
\isanewline
%
\endisadelimproof
\isanewline
\isacommand{lemma}\isamarkupfalse%
\ neighbor{\isacharunderscore}{\kern0pt}ne{\isacharcolon}{\kern0pt}\ {\isachardoublequoteopen}{\isasymforall}{\isacharparenleft}{\kern0pt}u{\isacharcomma}{\kern0pt}w{\isacharparenright}{\kern0pt}{\isasymin}set\ edge{\isacharunderscore}{\kern0pt}list{\isachardot}{\kern0pt}\ u\ {\isasymnoteq}\ w\ {\isasymLongrightarrow}\ degree\ v\ edge{\isacharunderscore}{\kern0pt}list\ {\isasymge}\ {\isadigit{1}}\ {\isasymLongrightarrow}\ neighbor\ v\ edge{\isacharunderscore}{\kern0pt}list\ {\isasymnoteq}\ v{\isachardoublequoteclose}\isanewline
%
\isadelimproof
\ \ %
\endisadelimproof
%
\isatagproof
\isacommand{unfolding}\isamarkupfalse%
\ incident{\isacharunderscore}{\kern0pt}edges{\isacharunderscore}{\kern0pt}def\ \isacommand{by}\isamarkupfalse%
\ {\isacharparenleft}{\kern0pt}induction\ edge{\isacharunderscore}{\kern0pt}list\ rule{\isacharcolon}{\kern0pt}\ neighbor{\isachardot}{\kern0pt}induct{\isacharparenright}{\kern0pt}\ auto%
\endisatagproof
{\isafoldproof}%
%
\isadelimproof
\isanewline
%
\endisadelimproof
\isanewline
\isacommand{lemma}\isamarkupfalse%
\ degree{\isacharunderscore}{\kern0pt}remove{\isacharunderscore}{\kern0pt}vertex{\isacharunderscore}{\kern0pt}{\isadigit{0}}{\isacharbrackleft}{\kern0pt}simp{\isacharbrackright}{\kern0pt}{\isacharcolon}{\kern0pt}\ {\isachardoublequoteopen}degree\ v\ {\isacharparenleft}{\kern0pt}remove{\isacharunderscore}{\kern0pt}vertex\ v\ edge{\isacharunderscore}{\kern0pt}list{\isacharparenright}{\kern0pt}\ {\isacharequal}{\kern0pt}\ {\isadigit{0}}{\isachardoublequoteclose}\isanewline
%
\isadelimproof
\ \ %
\endisadelimproof
%
\isatagproof
\isacommand{unfolding}\isamarkupfalse%
\ incident{\isacharunderscore}{\kern0pt}edges{\isacharunderscore}{\kern0pt}def\ remove{\isacharunderscore}{\kern0pt}vertex{\isacharunderscore}{\kern0pt}def\isanewline
\ \ \isacommand{by}\isamarkupfalse%
\ {\isacharparenleft}{\kern0pt}smt\ {\isacharparenleft}{\kern0pt}verit{\isacharcomma}{\kern0pt}\ best{\isacharparenright}{\kern0pt}\ filter{\isacharunderscore}{\kern0pt}False\ list{\isachardot}{\kern0pt}size{\isacharparenleft}{\kern0pt}{\isadigit{3}}{\isacharparenright}{\kern0pt}\ mem{\isacharunderscore}{\kern0pt}Collect{\isacharunderscore}{\kern0pt}eq\ set{\isacharunderscore}{\kern0pt}filter\ split{\isacharunderscore}{\kern0pt}def{\isacharparenright}{\kern0pt}%
\endisatagproof
{\isafoldproof}%
%
\isadelimproof
\isanewline
%
\endisadelimproof
\isanewline
\isacommand{lemma}\isamarkupfalse%
\ degree{\isacharunderscore}{\kern0pt}{\isadigit{0}}{\isacharunderscore}{\kern0pt}remove{\isacharunderscore}{\kern0pt}vertex{\isacharcolon}{\kern0pt}\isanewline
\ \ \isakeyword{assumes}\ degree{\isacharunderscore}{\kern0pt}{\isadigit{0}}{\isacharcolon}{\kern0pt}\ {\isachardoublequoteopen}degree\ v\ edge{\isacharunderscore}{\kern0pt}list\ {\isacharequal}{\kern0pt}\ {\isadigit{0}}{\isachardoublequoteclose}\isanewline
\ \ \isakeyword{shows}\ {\isachardoublequoteopen}remove{\isacharunderscore}{\kern0pt}vertex\ v\ edge{\isacharunderscore}{\kern0pt}list\ {\isacharequal}{\kern0pt}\ edge{\isacharunderscore}{\kern0pt}list{\isachardoublequoteclose}\isanewline
%
\isadelimproof
%
\endisadelimproof
%
\isatagproof
\isacommand{proof}\isamarkupfalse%
{\isacharminus}{\kern0pt}\isanewline
\ \ \isacommand{have}\isamarkupfalse%
\ {\isachardoublequoteopen}{\isasymforall}{\isacharparenleft}{\kern0pt}u{\isacharcomma}{\kern0pt}w{\isacharparenright}{\kern0pt}\ {\isasymin}\ set\ edge{\isacharunderscore}{\kern0pt}list{\isachardot}{\kern0pt}\ u\ {\isasymnoteq}\ v\ {\isasymand}\ w\ {\isasymnoteq}\ v{\isachardoublequoteclose}\ \isacommand{using}\isamarkupfalse%
\ degree{\isacharunderscore}{\kern0pt}{\isadigit{0}}\ \isacommand{unfolding}\isamarkupfalse%
\ incident{\isacharunderscore}{\kern0pt}edges{\isacharunderscore}{\kern0pt}def\isanewline
\ \ \ \ \isacommand{by}\isamarkupfalse%
\ {\isacharparenleft}{\kern0pt}simp\ add{\isacharcolon}{\kern0pt}\ filter{\isacharunderscore}{\kern0pt}empty{\isacharunderscore}{\kern0pt}conv\ split{\isacharunderscore}{\kern0pt}def{\isacharparenright}{\kern0pt}\isanewline
\ \ \isacommand{then}\isamarkupfalse%
\ \isacommand{show}\isamarkupfalse%
\ {\isacharquery}{\kern0pt}thesis\ \isacommand{unfolding}\isamarkupfalse%
\ remove{\isacharunderscore}{\kern0pt}vertex{\isacharunderscore}{\kern0pt}def\ \isacommand{by}\isamarkupfalse%
\ simp\isanewline
\isacommand{qed}\isamarkupfalse%
%
\endisatagproof
{\isafoldproof}%
%
\isadelimproof
\isanewline
%
\endisadelimproof
\isanewline
\isacommand{lemma}\isamarkupfalse%
\ degree{\isacharunderscore}{\kern0pt}length{\isacharunderscore}{\kern0pt}filter{\isacharcolon}{\kern0pt}\ {\isachardoublequoteopen}degree\ v\ edge{\isacharunderscore}{\kern0pt}list\ {\isacharequal}{\kern0pt}\ length\ {\isacharparenleft}{\kern0pt}filter\ {\isacharparenleft}{\kern0pt}{\isasymlambda}e{\isachardot}{\kern0pt}\ v\ {\isasymin}\ e{\isacharparenright}{\kern0pt}\ {\isacharparenleft}{\kern0pt}map\ mk{\isacharunderscore}{\kern0pt}edge\ edge{\isacharunderscore}{\kern0pt}list{\isacharparenright}{\kern0pt}{\isacharparenright}{\kern0pt}{\isachardoublequoteclose}\isanewline
%
\isadelimproof
%
\endisadelimproof
%
\isatagproof
\isacommand{proof}\isamarkupfalse%
{\isacharminus}{\kern0pt}\isanewline
\ \ \isacommand{have}\isamarkupfalse%
\ {\isachardoublequoteopen}{\isacharparenleft}{\kern0pt}{\isasymlambda}{\isacharparenleft}{\kern0pt}u{\isacharcomma}{\kern0pt}\ w{\isacharparenright}{\kern0pt}{\isachardot}{\kern0pt}\ u\ {\isacharequal}{\kern0pt}\ v\ {\isasymor}\ w\ {\isacharequal}{\kern0pt}\ v{\isacharparenright}{\kern0pt}\ {\isacharequal}{\kern0pt}\ {\isacharparenleft}{\kern0pt}{\isasymin}{\isacharparenright}{\kern0pt}\ v\ {\isasymcirc}\ mk{\isacharunderscore}{\kern0pt}edge{\isachardoublequoteclose}\ \isacommand{by}\isamarkupfalse%
\ auto\isanewline
\ \ \isacommand{then}\isamarkupfalse%
\ \isacommand{have}\isamarkupfalse%
\ {\isadigit{1}}{\isacharcolon}{\kern0pt}\ {\isachardoublequoteopen}map\ mk{\isacharunderscore}{\kern0pt}edge\ {\isacharparenleft}{\kern0pt}filter\ {\isacharparenleft}{\kern0pt}{\isasymlambda}{\isacharparenleft}{\kern0pt}u{\isacharcomma}{\kern0pt}\ w{\isacharparenright}{\kern0pt}{\isachardot}{\kern0pt}\ u\ {\isacharequal}{\kern0pt}\ v\ {\isasymor}\ w\ {\isacharequal}{\kern0pt}\ v{\isacharparenright}{\kern0pt}\ edge{\isacharunderscore}{\kern0pt}list{\isacharparenright}{\kern0pt}\ {\isacharequal}{\kern0pt}\ filter\ {\isacharparenleft}{\kern0pt}{\isacharparenleft}{\kern0pt}{\isasymin}{\isacharparenright}{\kern0pt}\ v{\isacharparenright}{\kern0pt}\ {\isacharparenleft}{\kern0pt}map\ mk{\isacharunderscore}{\kern0pt}edge\ edge{\isacharunderscore}{\kern0pt}list{\isacharparenright}{\kern0pt}{\isachardoublequoteclose}\ \isacommand{using}\isamarkupfalse%
\ filter{\isacharunderscore}{\kern0pt}map\ \isacommand{by}\isamarkupfalse%
\ metis\isanewline
\ \ \isacommand{have}\isamarkupfalse%
\ {\isachardoublequoteopen}length\ {\isacharparenleft}{\kern0pt}filter\ {\isacharparenleft}{\kern0pt}{\isasymlambda}{\isacharparenleft}{\kern0pt}u{\isacharcomma}{\kern0pt}\ w{\isacharparenright}{\kern0pt}{\isachardot}{\kern0pt}\ u\ {\isacharequal}{\kern0pt}\ v\ {\isasymor}\ w\ {\isacharequal}{\kern0pt}\ v{\isacharparenright}{\kern0pt}\ edge{\isacharunderscore}{\kern0pt}list{\isacharparenright}{\kern0pt}\ {\isacharequal}{\kern0pt}\ length\ {\isacharparenleft}{\kern0pt}map\ mk{\isacharunderscore}{\kern0pt}edge\ {\isacharparenleft}{\kern0pt}filter\ {\isacharparenleft}{\kern0pt}{\isasymlambda}{\isacharparenleft}{\kern0pt}u{\isacharcomma}{\kern0pt}\ w{\isacharparenright}{\kern0pt}{\isachardot}{\kern0pt}\ u\ {\isacharequal}{\kern0pt}\ v\ {\isasymor}\ w\ {\isacharequal}{\kern0pt}\ v{\isacharparenright}{\kern0pt}\ edge{\isacharunderscore}{\kern0pt}list{\isacharparenright}{\kern0pt}{\isacharparenright}{\kern0pt}{\isachardoublequoteclose}\ \isacommand{by}\isamarkupfalse%
\ simp\isanewline
\ \ \isacommand{then}\isamarkupfalse%
\ \isacommand{show}\isamarkupfalse%
\ {\isacharquery}{\kern0pt}thesis\ \isacommand{unfolding}\isamarkupfalse%
\ incident{\isacharunderscore}{\kern0pt}edges{\isacharunderscore}{\kern0pt}def\ \isacommand{using}\isamarkupfalse%
\ {\isadigit{1}}\ \isacommand{by}\isamarkupfalse%
\ argo\isanewline
\isacommand{qed}\isamarkupfalse%
%
\endisatagproof
{\isafoldproof}%
%
\isadelimproof
\isanewline
%
\endisadelimproof
\isanewline
\isacommand{lemma}\isamarkupfalse%
\ degree{\isacharunderscore}{\kern0pt}neighbor{\isacharunderscore}{\kern0pt}remove{\isacharunderscore}{\kern0pt}vertex{\isacharcolon}{\kern0pt}\ {\isachardoublequoteopen}degree\ v\ edge{\isacharunderscore}{\kern0pt}list\ {\isacharequal}{\kern0pt}\ {\isadigit{1}}\ {\isasymLongrightarrow}\ Suc\ {\isacharparenleft}{\kern0pt}degree\ {\isacharparenleft}{\kern0pt}neighbor\ v\ edge{\isacharunderscore}{\kern0pt}list{\isacharparenright}{\kern0pt}\ {\isacharparenleft}{\kern0pt}remove{\isacharunderscore}{\kern0pt}vertex\ v\ edge{\isacharunderscore}{\kern0pt}list{\isacharparenright}{\kern0pt}{\isacharparenright}{\kern0pt}\ {\isacharequal}{\kern0pt}\ degree\ {\isacharparenleft}{\kern0pt}neighbor\ v\ edge{\isacharunderscore}{\kern0pt}list{\isacharparenright}{\kern0pt}\ edge{\isacharunderscore}{\kern0pt}list{\isachardoublequoteclose}\isanewline
%
\isadelimproof
%
\endisadelimproof
%
\isatagproof
\isacommand{proof}\isamarkupfalse%
\ {\isacharparenleft}{\kern0pt}induction\ v\ edge{\isacharunderscore}{\kern0pt}list\ rule{\isacharcolon}{\kern0pt}\ neighbor{\isachardot}{\kern0pt}induct{\isacharparenright}{\kern0pt}\isanewline
\ \ \isacommand{case}\isamarkupfalse%
\ {\isacharparenleft}{\kern0pt}{\isadigit{1}}\ v{\isacharparenright}{\kern0pt}\isanewline
\ \ \isacommand{then}\isamarkupfalse%
\ \isacommand{show}\isamarkupfalse%
\ {\isacharquery}{\kern0pt}case\ \isacommand{unfolding}\isamarkupfalse%
\ incident{\isacharunderscore}{\kern0pt}edges{\isacharunderscore}{\kern0pt}def\ remove{\isacharunderscore}{\kern0pt}vertex{\isacharunderscore}{\kern0pt}def\ \isacommand{by}\isamarkupfalse%
\ simp\isanewline
\isacommand{next}\isamarkupfalse%
\isanewline
\ \ \isacommand{case}\isamarkupfalse%
\ {\isacharparenleft}{\kern0pt}{\isadigit{2}}\ v\ u\ w\ edges{\isacharparenright}{\kern0pt}\isanewline
\ \ \isacommand{assume}\isamarkupfalse%
\ degree{\isacharunderscore}{\kern0pt}{\isadigit{1}}{\isacharcolon}{\kern0pt}\ {\isachardoublequoteopen}degree\ v\ {\isacharparenleft}{\kern0pt}{\isacharparenleft}{\kern0pt}u{\isacharcomma}{\kern0pt}\ w{\isacharparenright}{\kern0pt}\ {\isacharhash}{\kern0pt}\ edges{\isacharparenright}{\kern0pt}\ {\isacharequal}{\kern0pt}\ {\isadigit{1}}{\isachardoublequoteclose}\isanewline
\ \ \isacommand{consider}\isamarkupfalse%
\ {\isachardoublequoteopen}u\ {\isacharequal}{\kern0pt}\ v\ {\isasymand}\ w\ {\isacharequal}{\kern0pt}\ v{\isachardoublequoteclose}\ {\isacharbar}{\kern0pt}\ {\isachardoublequoteopen}u\ {\isasymnoteq}\ v\ {\isasymand}\ w\ {\isacharequal}{\kern0pt}\ v{\isachardoublequoteclose}\ {\isacharbar}{\kern0pt}\ {\isachardoublequoteopen}u\ {\isacharequal}{\kern0pt}\ v\ {\isasymand}\ w\ {\isasymnoteq}\ v{\isachardoublequoteclose}\ {\isacharbar}{\kern0pt}\ {\isachardoublequoteopen}u\ {\isasymnoteq}\ v\ {\isasymand}\ w\ {\isasymnoteq}\ v{\isachardoublequoteclose}\ \isacommand{by}\isamarkupfalse%
\ blast\isanewline
\ \ \isacommand{then}\isamarkupfalse%
\ \isacommand{show}\isamarkupfalse%
\ {\isacharquery}{\kern0pt}case\isanewline
\ \ \isacommand{proof}\isamarkupfalse%
\ cases\isanewline
\ \ \ \ \isacommand{case}\isamarkupfalse%
\ {\isadigit{1}}\isanewline
\ \ \ \ \isacommand{then}\isamarkupfalse%
\ \isacommand{show}\isamarkupfalse%
\ {\isacharquery}{\kern0pt}thesis\ \isacommand{using}\isamarkupfalse%
\ {\isadigit{2}}\ \isacommand{by}\isamarkupfalse%
\ simp\isanewline
\ \ \isacommand{next}\isamarkupfalse%
\isanewline
\ \ \ \ \isacommand{case}\isamarkupfalse%
\ {\isadigit{2}}\isanewline
\ \ \ \ \isacommand{then}\isamarkupfalse%
\ \isacommand{have}\isamarkupfalse%
\ {\isachardoublequoteopen}degree\ v\ edges\ {\isacharequal}{\kern0pt}\ {\isadigit{0}}{\isachardoublequoteclose}\ \isacommand{using}\isamarkupfalse%
\ degree{\isacharunderscore}{\kern0pt}{\isadigit{1}}\ \isacommand{unfolding}\isamarkupfalse%
\ incident{\isacharunderscore}{\kern0pt}edges{\isacharunderscore}{\kern0pt}def\ \isacommand{by}\isamarkupfalse%
\ auto\isanewline
\ \ \ \ \isacommand{then}\isamarkupfalse%
\ \isacommand{show}\isamarkupfalse%
\ {\isacharquery}{\kern0pt}thesis\ \isacommand{using}\isamarkupfalse%
\ {\isadigit{2}}\ degree{\isacharunderscore}{\kern0pt}{\isadigit{0}}{\isacharunderscore}{\kern0pt}remove{\isacharunderscore}{\kern0pt}vertex\ \isacommand{unfolding}\isamarkupfalse%
\ remove{\isacharunderscore}{\kern0pt}vertex{\isacharunderscore}{\kern0pt}def\ incident{\isacharunderscore}{\kern0pt}edges{\isacharunderscore}{\kern0pt}def\ \isacommand{by}\isamarkupfalse%
\ fastforce\isanewline
\ \ \isacommand{next}\isamarkupfalse%
\isanewline
\ \ \ \ \isacommand{case}\isamarkupfalse%
\ {\isadigit{3}}\isanewline
\ \ \ \ \isacommand{then}\isamarkupfalse%
\ \isacommand{have}\isamarkupfalse%
\ {\isachardoublequoteopen}degree\ v\ edges\ {\isacharequal}{\kern0pt}\ {\isadigit{0}}{\isachardoublequoteclose}\ \isacommand{using}\isamarkupfalse%
\ degree{\isacharunderscore}{\kern0pt}{\isadigit{1}}\ \isacommand{unfolding}\isamarkupfalse%
\ incident{\isacharunderscore}{\kern0pt}edges{\isacharunderscore}{\kern0pt}def\ \isacommand{by}\isamarkupfalse%
\ auto\isanewline
\ \ \ \ \isacommand{then}\isamarkupfalse%
\ \isacommand{show}\isamarkupfalse%
\ {\isacharquery}{\kern0pt}thesis\ \isacommand{using}\isamarkupfalse%
\ {\isadigit{3}}\ degree{\isacharunderscore}{\kern0pt}{\isadigit{0}}{\isacharunderscore}{\kern0pt}remove{\isacharunderscore}{\kern0pt}vertex\ \isacommand{unfolding}\isamarkupfalse%
\ remove{\isacharunderscore}{\kern0pt}vertex{\isacharunderscore}{\kern0pt}def\ incident{\isacharunderscore}{\kern0pt}edges{\isacharunderscore}{\kern0pt}def\ \isacommand{by}\isamarkupfalse%
\ fastforce\isanewline
\ \ \isacommand{next}\isamarkupfalse%
\isanewline
\ \ \ \ \isacommand{case}\isamarkupfalse%
\ {\isadigit{4}}\isanewline
\ \ \ \ \isacommand{then}\isamarkupfalse%
\ \isacommand{have}\isamarkupfalse%
\ {\isachardoublequoteopen}degree\ v\ edges\ {\isacharequal}{\kern0pt}\ {\isadigit{1}}{\isachardoublequoteclose}\ \isacommand{using}\isamarkupfalse%
\ {\isadigit{2}}{\isacharparenleft}{\kern0pt}{\isadigit{2}}{\isacharparenright}{\kern0pt}\ \isacommand{unfolding}\isamarkupfalse%
\ incident{\isacharunderscore}{\kern0pt}edges{\isacharunderscore}{\kern0pt}def\ \isacommand{by}\isamarkupfalse%
\ auto\isanewline
\ \ \ \ \isacommand{then}\isamarkupfalse%
\ \isacommand{show}\isamarkupfalse%
\ {\isacharquery}{\kern0pt}thesis\ \isacommand{using}\isamarkupfalse%
\ {\isadigit{4}}\ {\isachardoublequoteopen}{\isadigit{2}}{\isachardot}{\kern0pt}IH{\isachardoublequoteclose}\ \isacommand{unfolding}\isamarkupfalse%
\ remove{\isacharunderscore}{\kern0pt}vertex{\isacharunderscore}{\kern0pt}def\ incident{\isacharunderscore}{\kern0pt}edges{\isacharunderscore}{\kern0pt}def\ \isacommand{by}\isamarkupfalse%
\ auto\isanewline
\ \ \isacommand{qed}\isamarkupfalse%
\isanewline
\isacommand{qed}\isamarkupfalse%
%
\endisatagproof
{\isafoldproof}%
%
\isadelimproof
\isanewline
%
\endisadelimproof
\isanewline
\isacommand{lemma}\isamarkupfalse%
\ distinct{\isacharunderscore}{\kern0pt}remove{\isacharunderscore}{\kern0pt}vertex{\isacharbrackleft}{\kern0pt}simp{\isacharbrackright}{\kern0pt}{\isacharcolon}{\kern0pt}\ {\isachardoublequoteopen}distinct\ {\isacharparenleft}{\kern0pt}map\ mk{\isacharunderscore}{\kern0pt}edge\ edge{\isacharunderscore}{\kern0pt}list{\isacharparenright}{\kern0pt}\ {\isasymLongrightarrow}\ distinct\ {\isacharparenleft}{\kern0pt}map\ mk{\isacharunderscore}{\kern0pt}edge\ {\isacharparenleft}{\kern0pt}remove{\isacharunderscore}{\kern0pt}vertex\ leaf\ edge{\isacharunderscore}{\kern0pt}list{\isacharparenright}{\kern0pt}{\isacharparenright}{\kern0pt}{\isachardoublequoteclose}\isanewline
%
\isadelimproof
\ \ %
\endisadelimproof
%
\isatagproof
\isacommand{unfolding}\isamarkupfalse%
\ remove{\isacharunderscore}{\kern0pt}vertex{\isacharunderscore}{\kern0pt}def\ \isacommand{using}\isamarkupfalse%
\ distinct{\isacharunderscore}{\kern0pt}map{\isacharunderscore}{\kern0pt}filter\ \isacommand{by}\isamarkupfalse%
\ fast%
\endisatagproof
{\isafoldproof}%
%
\isadelimproof
\isanewline
%
\endisadelimproof
\isanewline
\isacommand{lemma}\isamarkupfalse%
\ neighbor{\isacharunderscore}{\kern0pt}edge{\isacharunderscore}{\kern0pt}in{\isacharunderscore}{\kern0pt}edges{\isacharcolon}{\kern0pt}\ {\isachardoublequoteopen}degree\ v\ edge{\isacharunderscore}{\kern0pt}list\ {\isasymge}\ {\isadigit{1}}\ {\isasymLongrightarrow}\ {\isacharbraceleft}{\kern0pt}neighbor\ v\ edge{\isacharunderscore}{\kern0pt}list{\isacharcomma}{\kern0pt}\ v{\isacharbraceright}{\kern0pt}\ {\isasymin}\ edges{\isacharunderscore}{\kern0pt}of{\isacharunderscore}{\kern0pt}edge{\isacharunderscore}{\kern0pt}list\ edge{\isacharunderscore}{\kern0pt}list{\isachardoublequoteclose}\isanewline
%
\isadelimproof
\ \ %
\endisadelimproof
%
\isatagproof
\isacommand{unfolding}\isamarkupfalse%
\ incident{\isacharunderscore}{\kern0pt}edges{\isacharunderscore}{\kern0pt}def\ edges{\isacharunderscore}{\kern0pt}of{\isacharunderscore}{\kern0pt}edge{\isacharunderscore}{\kern0pt}list{\isacharunderscore}{\kern0pt}def\ \isacommand{by}\isamarkupfalse%
\ {\isacharparenleft}{\kern0pt}induction\ v\ edge{\isacharunderscore}{\kern0pt}list\ rule{\isacharcolon}{\kern0pt}\ neighbor{\isachardot}{\kern0pt}induct{\isacharparenright}{\kern0pt}\ auto%
\endisatagproof
{\isafoldproof}%
%
\isadelimproof
\isanewline
%
\endisadelimproof
\isanewline
\isacommand{lemma}\isamarkupfalse%
\ insert{\isacharunderscore}{\kern0pt}remove{\isacharunderscore}{\kern0pt}leaf{\isacharcolon}{\kern0pt}\isanewline
\ \ \isakeyword{assumes}\ degree{\isacharunderscore}{\kern0pt}leaf{\isacharcolon}{\kern0pt}\ {\isachardoublequoteopen}degree\ leaf\ edge{\isacharunderscore}{\kern0pt}list\ {\isacharequal}{\kern0pt}\ {\isadigit{1}}{\isachardoublequoteclose}\isanewline
\ \ \ \ \isakeyword{shows}\ {\isachardoublequoteopen}insert\ {\isacharbraceleft}{\kern0pt}neighbor\ leaf\ edge{\isacharunderscore}{\kern0pt}list{\isacharcomma}{\kern0pt}\ leaf{\isacharbraceright}{\kern0pt}\ {\isacharparenleft}{\kern0pt}edges{\isacharunderscore}{\kern0pt}of{\isacharunderscore}{\kern0pt}edge{\isacharunderscore}{\kern0pt}list\ {\isacharparenleft}{\kern0pt}remove{\isacharunderscore}{\kern0pt}vertex\ leaf\ edge{\isacharunderscore}{\kern0pt}list{\isacharparenright}{\kern0pt}{\isacharparenright}{\kern0pt}\ {\isacharequal}{\kern0pt}\ edges{\isacharunderscore}{\kern0pt}of{\isacharunderscore}{\kern0pt}edge{\isacharunderscore}{\kern0pt}list\ edge{\isacharunderscore}{\kern0pt}list{\isachardoublequoteclose}\isanewline
%
\isadelimproof
%
\endisadelimproof
%
\isatagproof
\isacommand{proof}\isamarkupfalse%
{\isacharminus}{\kern0pt}\isanewline
\ \ \isacommand{let}\isamarkupfalse%
\ {\isacharquery}{\kern0pt}leaf{\isacharunderscore}{\kern0pt}edges\ {\isacharequal}{\kern0pt}\ {\isachardoublequoteopen}filter\ {\isacharparenleft}{\kern0pt}{\isasymlambda}{\isacharparenleft}{\kern0pt}u{\isacharcomma}{\kern0pt}w{\isacharparenright}{\kern0pt}{\isachardot}{\kern0pt}\ u\ {\isacharequal}{\kern0pt}\ leaf\ {\isasymor}\ w\ {\isacharequal}{\kern0pt}\ leaf{\isacharparenright}{\kern0pt}\ edge{\isacharunderscore}{\kern0pt}list{\isachardoublequoteclose}\isanewline
\ \ \isacommand{have}\isamarkupfalse%
\ length{\isacharunderscore}{\kern0pt}leaf{\isacharunderscore}{\kern0pt}edges{\isacharcolon}{\kern0pt}\ {\isachardoublequoteopen}length\ {\isacharquery}{\kern0pt}leaf{\isacharunderscore}{\kern0pt}edges\ {\isacharequal}{\kern0pt}\ {\isadigit{1}}{\isachardoublequoteclose}\ \isacommand{using}\isamarkupfalse%
\ degree{\isacharunderscore}{\kern0pt}leaf\ \isacommand{unfolding}\isamarkupfalse%
\ incident{\isacharunderscore}{\kern0pt}edges{\isacharunderscore}{\kern0pt}def\ \isacommand{by}\isamarkupfalse%
\ simp\isanewline
\ \ \isacommand{have}\isamarkupfalse%
\ {\isachardoublequoteopen}{\isacharbraceleft}{\kern0pt}neighbor\ leaf\ edge{\isacharunderscore}{\kern0pt}list{\isacharcomma}{\kern0pt}\ leaf{\isacharbraceright}{\kern0pt}\ {\isasymin}\ edges{\isacharunderscore}{\kern0pt}of{\isacharunderscore}{\kern0pt}edge{\isacharunderscore}{\kern0pt}list\ edge{\isacharunderscore}{\kern0pt}list{\isachardoublequoteclose}\ \isacommand{using}\isamarkupfalse%
\ neighbor{\isacharunderscore}{\kern0pt}edge{\isacharunderscore}{\kern0pt}in{\isacharunderscore}{\kern0pt}edges\ degree{\isacharunderscore}{\kern0pt}leaf\ \isacommand{by}\isamarkupfalse%
\ force\isanewline
\ \ \isacommand{then}\isamarkupfalse%
\ \isacommand{have}\isamarkupfalse%
\ {\isachardoublequoteopen}{\isacharparenleft}{\kern0pt}neighbor\ leaf\ edge{\isacharunderscore}{\kern0pt}list{\isacharcomma}{\kern0pt}\ leaf{\isacharparenright}{\kern0pt}\ {\isasymin}\ set\ edge{\isacharunderscore}{\kern0pt}list\ {\isasymor}\ {\isacharparenleft}{\kern0pt}leaf{\isacharcomma}{\kern0pt}\ neighbor\ leaf\ edge{\isacharunderscore}{\kern0pt}list{\isacharparenright}{\kern0pt}\ {\isasymin}\ set\ edge{\isacharunderscore}{\kern0pt}list{\isachardoublequoteclose}\ \isacommand{by}\isamarkupfalse%
\ {\isacharparenleft}{\kern0pt}simp\ add{\isacharcolon}{\kern0pt}\ edges{\isacharunderscore}{\kern0pt}of{\isacharunderscore}{\kern0pt}edge{\isacharunderscore}{\kern0pt}list{\isacharunderscore}{\kern0pt}def\ in{\isacharunderscore}{\kern0pt}mk{\isacharunderscore}{\kern0pt}uedge{\isacharunderscore}{\kern0pt}img{\isacharunderscore}{\kern0pt}iff{\isacharparenright}{\kern0pt}\isanewline
\ \ \isacommand{then}\isamarkupfalse%
\ \isacommand{have}\isamarkupfalse%
\ {\isachardoublequoteopen}{\isacharparenleft}{\kern0pt}neighbor\ leaf\ edge{\isacharunderscore}{\kern0pt}list{\isacharcomma}{\kern0pt}\ leaf{\isacharparenright}{\kern0pt}\ {\isasymin}\ set\ {\isacharquery}{\kern0pt}leaf{\isacharunderscore}{\kern0pt}edges\ {\isasymor}\ {\isacharparenleft}{\kern0pt}leaf{\isacharcomma}{\kern0pt}\ neighbor\ leaf\ edge{\isacharunderscore}{\kern0pt}list{\isacharparenright}{\kern0pt}\ {\isasymin}\ set\ {\isacharquery}{\kern0pt}leaf{\isacharunderscore}{\kern0pt}edges{\isachardoublequoteclose}\ \isacommand{by}\isamarkupfalse%
\ simp\isanewline
\ \ \isacommand{then}\isamarkupfalse%
\ \isacommand{have}\isamarkupfalse%
\ {\isachardoublequoteopen}{\isacharquery}{\kern0pt}leaf{\isacharunderscore}{\kern0pt}edges\ {\isacharequal}{\kern0pt}\ {\isacharbrackleft}{\kern0pt}{\isacharparenleft}{\kern0pt}neighbor\ leaf\ edge{\isacharunderscore}{\kern0pt}list{\isacharcomma}{\kern0pt}\ leaf{\isacharparenright}{\kern0pt}{\isacharbrackright}{\kern0pt}\ {\isasymor}\ {\isacharquery}{\kern0pt}leaf{\isacharunderscore}{\kern0pt}edges\ {\isacharequal}{\kern0pt}\ {\isacharbrackleft}{\kern0pt}{\isacharparenleft}{\kern0pt}leaf{\isacharcomma}{\kern0pt}\ neighbor\ leaf\ edge{\isacharunderscore}{\kern0pt}list{\isacharparenright}{\kern0pt}{\isacharbrackright}{\kern0pt}{\isachardoublequoteclose}\ \isacommand{using}\isamarkupfalse%
\ length{\isacharunderscore}{\kern0pt}leaf{\isacharunderscore}{\kern0pt}edges\isanewline
\ \ \ \ \isacommand{by}\isamarkupfalse%
\ {\isacharparenleft}{\kern0pt}smt\ {\isacharparenleft}{\kern0pt}verit{\isacharparenright}{\kern0pt}\ One{\isacharunderscore}{\kern0pt}nat{\isacharunderscore}{\kern0pt}def\ empty{\isacharunderscore}{\kern0pt}iff\ empty{\isacharunderscore}{\kern0pt}set\ length{\isacharunderscore}{\kern0pt}{\isadigit{0}}{\isacharunderscore}{\kern0pt}conv\ length{\isacharunderscore}{\kern0pt}Suc{\isacharunderscore}{\kern0pt}conv\ list{\isachardot}{\kern0pt}inject\ list{\isachardot}{\kern0pt}set{\isacharunderscore}{\kern0pt}cases{\isacharparenright}{\kern0pt}\isanewline
\ \ \isacommand{then}\isamarkupfalse%
\ \isacommand{have}\isamarkupfalse%
\ leaf{\isacharunderscore}{\kern0pt}edges{\isacharcolon}{\kern0pt}\ {\isachardoublequoteopen}edges{\isacharunderscore}{\kern0pt}of{\isacharunderscore}{\kern0pt}edge{\isacharunderscore}{\kern0pt}list\ {\isacharquery}{\kern0pt}leaf{\isacharunderscore}{\kern0pt}edges\ {\isacharequal}{\kern0pt}\ {\isacharbraceleft}{\kern0pt}{\isacharbraceleft}{\kern0pt}neighbor\ leaf\ edge{\isacharunderscore}{\kern0pt}list{\isacharcomma}{\kern0pt}\ leaf{\isacharbraceright}{\kern0pt}{\isacharbraceright}{\kern0pt}{\isachardoublequoteclose}\ \isacommand{unfolding}\isamarkupfalse%
\ edges{\isacharunderscore}{\kern0pt}of{\isacharunderscore}{\kern0pt}edge{\isacharunderscore}{\kern0pt}list{\isacharunderscore}{\kern0pt}def\ \isacommand{by}\isamarkupfalse%
\ fastforce\isanewline
\isanewline
\ \ \isacommand{have}\isamarkupfalse%
\ {\isachardoublequoteopen}edges{\isacharunderscore}{\kern0pt}of{\isacharunderscore}{\kern0pt}edge{\isacharunderscore}{\kern0pt}list\ edge{\isacharunderscore}{\kern0pt}list\ {\isacharequal}{\kern0pt}\ edges{\isacharunderscore}{\kern0pt}of{\isacharunderscore}{\kern0pt}edge{\isacharunderscore}{\kern0pt}list\ {\isacharquery}{\kern0pt}leaf{\isacharunderscore}{\kern0pt}edges\ {\isasymunion}\ edges{\isacharunderscore}{\kern0pt}of{\isacharunderscore}{\kern0pt}edge{\isacharunderscore}{\kern0pt}list\ {\isacharparenleft}{\kern0pt}remove{\isacharunderscore}{\kern0pt}vertex\ leaf\ edge{\isacharunderscore}{\kern0pt}list{\isacharparenright}{\kern0pt}{\isachardoublequoteclose}\ \isacommand{unfolding}\isamarkupfalse%
\ remove{\isacharunderscore}{\kern0pt}vertex{\isacharunderscore}{\kern0pt}def\ edges{\isacharunderscore}{\kern0pt}of{\isacharunderscore}{\kern0pt}edge{\isacharunderscore}{\kern0pt}list{\isacharunderscore}{\kern0pt}def\ \isacommand{by}\isamarkupfalse%
\ auto\isanewline
\ \ \isacommand{then}\isamarkupfalse%
\ \isacommand{show}\isamarkupfalse%
\ {\isacharquery}{\kern0pt}thesis\ \isacommand{using}\isamarkupfalse%
\ leaf{\isacharunderscore}{\kern0pt}edges\ \isacommand{by}\isamarkupfalse%
\ auto\isanewline
\isacommand{qed}\isamarkupfalse%
%
\endisatagproof
{\isafoldproof}%
%
\isadelimproof
\isanewline
%
\endisadelimproof
\isanewline
\isacommand{lemma}\isamarkupfalse%
\ find{\isacharunderscore}{\kern0pt}Some{\isacharcolon}{\kern0pt}\ {\isachardoublequoteopen}find\ P\ xs\ {\isacharequal}{\kern0pt}\ Some\ x\ {\isasymLongrightarrow}\ P\ x{\isachardoublequoteclose}\isanewline
%
\isadelimproof
\ \ %
\endisadelimproof
%
\isatagproof
\isacommand{by}\isamarkupfalse%
\ {\isacharparenleft}{\kern0pt}metis\ find{\isacharunderscore}{\kern0pt}Some{\isacharunderscore}{\kern0pt}iff{\isacharparenright}{\kern0pt}%
\endisatagproof
{\isafoldproof}%
%
\isadelimproof
\isanewline
%
\endisadelimproof
\isanewline
\isacommand{definition}\isamarkupfalse%
\ verts{\isacharunderscore}{\kern0pt}of{\isacharunderscore}{\kern0pt}edges\ {\isacharcolon}{\kern0pt}{\isacharcolon}{\kern0pt}\ {\isachardoublequoteopen}{\isacharparenleft}{\kern0pt}{\isacharprime}{\kern0pt}a\ {\isasymtimes}\ {\isacharprime}{\kern0pt}a{\isacharparenright}{\kern0pt}\ list\ {\isasymRightarrow}\ {\isacharprime}{\kern0pt}a\ set{\isachardoublequoteclose}\ \isakeyword{where}\isanewline
\ \ {\isachardoublequoteopen}verts{\isacharunderscore}{\kern0pt}of{\isacharunderscore}{\kern0pt}edges\ edges\ {\isacharequal}{\kern0pt}\ {\isacharbraceleft}{\kern0pt}v\ {\isacharbar}{\kern0pt}\ v\ e{\isachardot}{\kern0pt}\ v\ {\isasymin}\ e\ {\isasymand}\ e\ {\isasymin}\ edges{\isacharunderscore}{\kern0pt}of{\isacharunderscore}{\kern0pt}edge{\isacharunderscore}{\kern0pt}list\ edges{\isacharbraceright}{\kern0pt}{\isachardoublequoteclose}\isanewline
\isanewline
\isanewline
\isacommand{locale}\isamarkupfalse%
\ prufer{\isacharunderscore}{\kern0pt}seq{\isacharunderscore}{\kern0pt}to{\isacharunderscore}{\kern0pt}tree{\isacharunderscore}{\kern0pt}context\ {\isacharequal}{\kern0pt}\isanewline
\ \ \isakeyword{fixes}\ verts\ {\isacharcolon}{\kern0pt}{\isacharcolon}{\kern0pt}\ {\isachardoublequoteopen}{\isacharprime}{\kern0pt}a\ list{\isachardoublequoteclose}\isanewline
\ \ \isakeyword{assumes}\ verts{\isacharunderscore}{\kern0pt}length{\isacharcolon}{\kern0pt}\ {\isachardoublequoteopen}length\ verts\ {\isasymge}\ {\isadigit{2}}{\isachardoublequoteclose}\isanewline
\ \ \ \ \isakeyword{and}\ distinct{\isacharunderscore}{\kern0pt}verts{\isacharcolon}{\kern0pt}\ {\isachardoublequoteopen}distinct\ verts{\isachardoublequoteclose}\isanewline
\isakeyword{begin}\isanewline
\isanewline
\isacommand{lemma}\isamarkupfalse%
\ card{\isacharunderscore}{\kern0pt}verts{\isacharcolon}{\kern0pt}\ {\isachardoublequoteopen}card\ {\isacharparenleft}{\kern0pt}set\ verts{\isacharparenright}{\kern0pt}\ {\isasymge}\ {\isadigit{2}}{\isachardoublequoteclose}\isanewline
%
\isadelimproof
\ \ %
\endisadelimproof
%
\isatagproof
\isacommand{using}\isamarkupfalse%
\ verts{\isacharunderscore}{\kern0pt}length\ distinct{\isacharunderscore}{\kern0pt}verts\ distinct{\isacharunderscore}{\kern0pt}card\ \isacommand{by}\isamarkupfalse%
\ fastforce%
\endisatagproof
{\isafoldproof}%
%
\isadelimproof
\isanewline
%
\endisadelimproof
\isanewline
\isacommand{lemma}\isamarkupfalse%
\ length{\isacharunderscore}{\kern0pt}gt{\isacharunderscore}{\kern0pt}find{\isacharunderscore}{\kern0pt}not{\isacharunderscore}{\kern0pt}in{\isacharunderscore}{\kern0pt}ys{\isacharcolon}{\kern0pt}\isanewline
\ \ \isakeyword{assumes}\ {\isachardoublequoteopen}length\ xs\ {\isachargreater}{\kern0pt}\ length\ ys{\isachardoublequoteclose}\isanewline
\ \ \ \ \isakeyword{and}\ {\isachardoublequoteopen}distinct\ xs{\isachardoublequoteclose}\isanewline
\ \ \isakeyword{shows}\ {\isachardoublequoteopen}{\isasymexists}x{\isachardot}{\kern0pt}\ find\ {\isacharparenleft}{\kern0pt}{\isasymlambda}x{\isachardot}{\kern0pt}\ x\ {\isasymnotin}\ set\ ys{\isacharparenright}{\kern0pt}\ xs\ {\isacharequal}{\kern0pt}\ Some\ x{\isachardoublequoteclose}\isanewline
%
\isadelimproof
%
\endisadelimproof
%
\isatagproof
\isacommand{proof}\isamarkupfalse%
{\isacharminus}{\kern0pt}\isanewline
\ \ \isacommand{have}\isamarkupfalse%
\ {\isachardoublequoteopen}card\ {\isacharparenleft}{\kern0pt}set\ xs{\isacharparenright}{\kern0pt}\ {\isachargreater}{\kern0pt}\ card\ {\isacharparenleft}{\kern0pt}set\ ys{\isacharparenright}{\kern0pt}{\isachardoublequoteclose}\isanewline
\ \ \ \ \isacommand{by}\isamarkupfalse%
\ {\isacharparenleft}{\kern0pt}metis\ assms\ card{\isacharunderscore}{\kern0pt}length\ distinct{\isacharunderscore}{\kern0pt}card\ le{\isacharunderscore}{\kern0pt}neq{\isacharunderscore}{\kern0pt}implies{\isacharunderscore}{\kern0pt}less\ order{\isacharunderscore}{\kern0pt}less{\isacharunderscore}{\kern0pt}trans{\isacharparenright}{\kern0pt}\isanewline
\ \ \isacommand{then}\isamarkupfalse%
\ \isacommand{have}\isamarkupfalse%
\ {\isachardoublequoteopen}{\isasymexists}x{\isasymin}set\ xs{\isachardot}{\kern0pt}\ x\ {\isasymnotin}\ set\ ys{\isachardoublequoteclose}\isanewline
\ \ \ \ \isacommand{by}\isamarkupfalse%
\ {\isacharparenleft}{\kern0pt}meson\ finite{\isacharunderscore}{\kern0pt}set\ card{\isacharunderscore}{\kern0pt}subset{\isacharunderscore}{\kern0pt}not{\isacharunderscore}{\kern0pt}gt{\isacharunderscore}{\kern0pt}card\ subsetI{\isacharparenright}{\kern0pt}\isanewline
\ \ \isacommand{then}\isamarkupfalse%
\ \isacommand{show}\isamarkupfalse%
\ {\isacharquery}{\kern0pt}thesis\ \isacommand{by}\isamarkupfalse%
\ {\isacharparenleft}{\kern0pt}metis\ find{\isacharunderscore}{\kern0pt}None{\isacharunderscore}{\kern0pt}iff{\isadigit{2}}\ not{\isacharunderscore}{\kern0pt}Some{\isacharunderscore}{\kern0pt}eq{\isacharparenright}{\kern0pt}\isanewline
\isacommand{qed}\isamarkupfalse%
%
\endisatagproof
{\isafoldproof}%
%
\isadelimproof
\isanewline
%
\endisadelimproof
\isanewline
\isacommand{lemma}\isamarkupfalse%
\ obtain{\isacharunderscore}{\kern0pt}b{\isacharunderscore}{\kern0pt}prufer{\isacharunderscore}{\kern0pt}seq{\isacharunderscore}{\kern0pt}to{\isacharunderscore}{\kern0pt}tree{\isacharunderscore}{\kern0pt}edges{\isacharcolon}{\kern0pt}\isanewline
\ \ \isakeyword{assumes}\ {\isachardoublequoteopen}{\isacharparenleft}{\kern0pt}a\ {\isacharhash}{\kern0pt}\ seq{\isacharparenright}{\kern0pt}\ {\isasymin}\ prufer{\isacharunderscore}{\kern0pt}sequences\ verts{\isachardoublequoteclose}\isanewline
\ \ \isakeyword{obtains}\ b\isanewline
\ \ \isakeyword{where}\ {\isachardoublequoteopen}find\ {\isacharparenleft}{\kern0pt}{\isasymlambda}x{\isachardot}{\kern0pt}\ x\ {\isasymnotin}\ set\ {\isacharparenleft}{\kern0pt}a\ {\isacharhash}{\kern0pt}\ seq{\isacharparenright}{\kern0pt}{\isacharparenright}{\kern0pt}\ verts\ {\isacharequal}{\kern0pt}\ Some\ b{\isachardoublequoteclose}\isanewline
\ \ \ \ \isakeyword{and}\ {\isachardoublequoteopen}b\ {\isasymin}\ set\ verts{\isachardoublequoteclose}\isanewline
\ \ \ \ \isakeyword{and}\ {\isachardoublequoteopen}b\ {\isasymnotin}\ set\ {\isacharparenleft}{\kern0pt}a\ {\isacharhash}{\kern0pt}\ seq{\isacharparenright}{\kern0pt}{\isachardoublequoteclose}\isanewline
\ \ \ \ \isakeyword{and}\ {\isachardoublequoteopen}seq\ {\isasymin}\ prufer{\isacharunderscore}{\kern0pt}sequences\ {\isacharparenleft}{\kern0pt}remove{\isadigit{1}}\ b\ verts{\isacharparenright}{\kern0pt}{\isachardoublequoteclose}\isanewline
\ \ \ \ \isakeyword{and}\ {\isachardoublequoteopen}length\ {\isacharparenleft}{\kern0pt}remove{\isadigit{1}}\ b\ verts{\isacharparenright}{\kern0pt}\ {\isasymge}\ {\isadigit{2}}{\isachardoublequoteclose}\isanewline
\ \ \ \ \isakeyword{and}\ {\isachardoublequoteopen}distinct\ {\isacharparenleft}{\kern0pt}remove{\isadigit{1}}\ b\ verts{\isacharparenright}{\kern0pt}{\isachardoublequoteclose}\isanewline
%
\isadelimproof
%
\endisadelimproof
%
\isatagproof
\isacommand{proof}\isamarkupfalse%
{\isacharminus}{\kern0pt}\isanewline
\ \ \isacommand{obtain}\isamarkupfalse%
\ b\ \isakeyword{where}\ b{\isacharunderscore}{\kern0pt}find{\isacharcolon}{\kern0pt}\ {\isachardoublequoteopen}find\ {\isacharparenleft}{\kern0pt}{\isasymlambda}x{\isachardot}{\kern0pt}\ x\ {\isasymnotin}\ set\ {\isacharparenleft}{\kern0pt}a{\isacharhash}{\kern0pt}seq{\isacharparenright}{\kern0pt}{\isacharparenright}{\kern0pt}\ verts\ {\isacharequal}{\kern0pt}\ Some\ b{\isachardoublequoteclose}\isanewline
\ \ \ \ \isacommand{using}\isamarkupfalse%
\ assms\ length{\isacharunderscore}{\kern0pt}gt{\isacharunderscore}{\kern0pt}find{\isacharunderscore}{\kern0pt}not{\isacharunderscore}{\kern0pt}in{\isacharunderscore}{\kern0pt}ys{\isacharbrackleft}{\kern0pt}of\ {\isachardoublequoteopen}a{\isacharhash}{\kern0pt}seq{\isachardoublequoteclose}\ verts{\isacharbrackright}{\kern0pt}\ distinct{\isacharunderscore}{\kern0pt}verts\isanewline
\ \ \ \ \isacommand{unfolding}\isamarkupfalse%
\ prufer{\isacharunderscore}{\kern0pt}sequences{\isacharunderscore}{\kern0pt}def\ n{\isacharunderscore}{\kern0pt}sequences{\isacharunderscore}{\kern0pt}def\isanewline
\ \ \ \ \isacommand{by}\isamarkupfalse%
\ fastforce\isanewline
\ \ \isacommand{have}\isamarkupfalse%
\ b{\isacharunderscore}{\kern0pt}in{\isacharunderscore}{\kern0pt}verts{\isacharcolon}{\kern0pt}\ {\isachardoublequoteopen}b\ {\isasymin}\ set\ verts{\isachardoublequoteclose}\ \isacommand{using}\isamarkupfalse%
\ b{\isacharunderscore}{\kern0pt}find\isanewline
\ \ \ \ \isacommand{by}\isamarkupfalse%
\ {\isacharparenleft}{\kern0pt}metis\ find{\isacharunderscore}{\kern0pt}Some{\isacharunderscore}{\kern0pt}iff\ nth{\isacharunderscore}{\kern0pt}mem{\isacharparenright}{\kern0pt}\isanewline
\ \ \isacommand{have}\isamarkupfalse%
\ b{\isacharunderscore}{\kern0pt}not{\isacharunderscore}{\kern0pt}in{\isacharunderscore}{\kern0pt}seq{\isacharcolon}{\kern0pt}\ {\isachardoublequoteopen}b\ {\isasymnotin}\ set\ {\isacharparenleft}{\kern0pt}a{\isacharhash}{\kern0pt}seq{\isacharparenright}{\kern0pt}{\isachardoublequoteclose}\ \isacommand{using}\isamarkupfalse%
\ b{\isacharunderscore}{\kern0pt}find\isanewline
\ \ \ \ \isacommand{by}\isamarkupfalse%
\ {\isacharparenleft}{\kern0pt}metis\ find{\isacharunderscore}{\kern0pt}Some{\isacharunderscore}{\kern0pt}iff{\isacharparenright}{\kern0pt}\isanewline
\ \ \isacommand{have}\isamarkupfalse%
\ seq{\isacharunderscore}{\kern0pt}prufer{\isacharunderscore}{\kern0pt}verts{\isacharprime}{\kern0pt}{\isacharcolon}{\kern0pt}\ {\isachardoublequoteopen}seq\ {\isasymin}\ prufer{\isacharunderscore}{\kern0pt}sequences\ {\isacharparenleft}{\kern0pt}remove{\isadigit{1}}\ b\ verts{\isacharparenright}{\kern0pt}{\isachardoublequoteclose}\isanewline
\ \ \ \ \isacommand{using}\isamarkupfalse%
\ assms\ b{\isacharunderscore}{\kern0pt}in{\isacharunderscore}{\kern0pt}verts\ set{\isacharunderscore}{\kern0pt}remove{\isadigit{1}}{\isacharunderscore}{\kern0pt}eq\ verts{\isacharunderscore}{\kern0pt}length\ b{\isacharunderscore}{\kern0pt}not{\isacharunderscore}{\kern0pt}in{\isacharunderscore}{\kern0pt}seq\ distinct{\isacharunderscore}{\kern0pt}verts\isanewline
\ \ \ \ \isacommand{unfolding}\isamarkupfalse%
\ prufer{\isacharunderscore}{\kern0pt}sequences{\isacharunderscore}{\kern0pt}def\ n{\isacharunderscore}{\kern0pt}sequences{\isacharunderscore}{\kern0pt}def\isanewline
\ \ \ \ \isacommand{by}\isamarkupfalse%
\ {\isacharparenleft}{\kern0pt}auto\ simp{\isacharcolon}{\kern0pt}\ length{\isacharunderscore}{\kern0pt}remove{\isadigit{1}}{\isacharparenright}{\kern0pt}\isanewline
\ \ \isacommand{have}\isamarkupfalse%
\ {\isachardoublequoteopen}length\ verts\ {\isasymge}\ {\isadigit{3}}{\isachardoublequoteclose}\ \isacommand{using}\isamarkupfalse%
\ assms\ \isacommand{unfolding}\isamarkupfalse%
\ prufer{\isacharunderscore}{\kern0pt}sequences{\isacharunderscore}{\kern0pt}def\ n{\isacharunderscore}{\kern0pt}sequences{\isacharunderscore}{\kern0pt}def\ \isacommand{by}\isamarkupfalse%
\ auto\isanewline
\ \ \isacommand{then}\isamarkupfalse%
\ \isacommand{have}\isamarkupfalse%
\ length{\isacharunderscore}{\kern0pt}verts{\isacharprime}{\kern0pt}{\isacharcolon}{\kern0pt}\ {\isachardoublequoteopen}length\ {\isacharparenleft}{\kern0pt}remove{\isadigit{1}}\ b\ verts{\isacharparenright}{\kern0pt}\ {\isasymge}\ {\isadigit{2}}{\isachardoublequoteclose}\ \isacommand{by}\isamarkupfalse%
\ {\isacharparenleft}{\kern0pt}auto\ simp{\isacharcolon}{\kern0pt}\ length{\isacharunderscore}{\kern0pt}remove{\isadigit{1}}{\isacharparenright}{\kern0pt}\isanewline
\ \ \isacommand{have}\isamarkupfalse%
\ distinct{\isacharcolon}{\kern0pt}\ {\isachardoublequoteopen}distinct\ {\isacharparenleft}{\kern0pt}remove{\isadigit{1}}\ b\ verts{\isacharparenright}{\kern0pt}{\isachardoublequoteclose}\ \isacommand{using}\isamarkupfalse%
\ distinct{\isacharunderscore}{\kern0pt}remove{\isadigit{1}}\ assms\ distinct{\isacharunderscore}{\kern0pt}verts\ \isacommand{by}\isamarkupfalse%
\ fast\isanewline
\ \ \isacommand{from}\isamarkupfalse%
\ b{\isacharunderscore}{\kern0pt}find\ b{\isacharunderscore}{\kern0pt}in{\isacharunderscore}{\kern0pt}verts\ b{\isacharunderscore}{\kern0pt}not{\isacharunderscore}{\kern0pt}in{\isacharunderscore}{\kern0pt}seq\ seq{\isacharunderscore}{\kern0pt}prufer{\isacharunderscore}{\kern0pt}verts{\isacharprime}{\kern0pt}\ length{\isacharunderscore}{\kern0pt}verts{\isacharprime}{\kern0pt}\ distinct\ \isacommand{show}\isamarkupfalse%
\ {\isacharquery}{\kern0pt}thesis\ \isacommand{{\isachardot}{\kern0pt}{\isachardot}{\kern0pt}}\isamarkupfalse%
\isanewline
\isacommand{qed}\isamarkupfalse%
%
\endisatagproof
{\isafoldproof}%
%
\isadelimproof
\isanewline
%
\endisadelimproof
\isanewline
\isacommand{lemma}\isamarkupfalse%
\ verts{\isacharunderscore}{\kern0pt}of{\isacharunderscore}{\kern0pt}edges{\isacharunderscore}{\kern0pt}prufer{\isacharunderscore}{\kern0pt}to{\isacharunderscore}{\kern0pt}tree{\isacharbrackleft}{\kern0pt}simp{\isacharbrackright}{\kern0pt}{\isacharcolon}{\kern0pt}\isanewline
\ \ {\isachardoublequoteopen}seq\ {\isasymin}\ prufer{\isacharunderscore}{\kern0pt}sequences\ verts\ {\isasymLongrightarrow}\isanewline
\ \ \ \ verts{\isacharunderscore}{\kern0pt}of{\isacharunderscore}{\kern0pt}edges\ {\isacharparenleft}{\kern0pt}prufer{\isacharunderscore}{\kern0pt}seq{\isacharunderscore}{\kern0pt}to{\isacharunderscore}{\kern0pt}tree{\isacharunderscore}{\kern0pt}edges\ verts\ seq{\isacharparenright}{\kern0pt}\ {\isacharequal}{\kern0pt}\ set\ verts{\isachardoublequoteclose}\isanewline
%
\isadelimproof
\ \ %
\endisadelimproof
%
\isatagproof
\isacommand{using}\isamarkupfalse%
\ verts{\isacharunderscore}{\kern0pt}length\ distinct{\isacharunderscore}{\kern0pt}verts\isanewline
\isacommand{proof}\isamarkupfalse%
\ {\isacharparenleft}{\kern0pt}induction\ verts\ seq\ rule{\isacharcolon}{\kern0pt}\ prufer{\isacharunderscore}{\kern0pt}seq{\isacharunderscore}{\kern0pt}to{\isacharunderscore}{\kern0pt}tree{\isacharunderscore}{\kern0pt}edges{\isachardot}{\kern0pt}induct{\isacharparenright}{\kern0pt}\isanewline
\ \ \isacommand{case}\isamarkupfalse%
\ {\isacharparenleft}{\kern0pt}{\isadigit{1}}\ v\ w{\isacharparenright}{\kern0pt}\isanewline
\ \ \isacommand{then}\isamarkupfalse%
\ \isacommand{show}\isamarkupfalse%
\ {\isacharquery}{\kern0pt}case\ \isacommand{unfolding}\isamarkupfalse%
\ verts{\isacharunderscore}{\kern0pt}of{\isacharunderscore}{\kern0pt}edges{\isacharunderscore}{\kern0pt}def\ edges{\isacharunderscore}{\kern0pt}of{\isacharunderscore}{\kern0pt}edge{\isacharunderscore}{\kern0pt}list{\isacharunderscore}{\kern0pt}def\ \isacommand{by}\isamarkupfalse%
\ auto\isanewline
\isacommand{next}\isamarkupfalse%
\isanewline
\ \ \isacommand{case}\isamarkupfalse%
\ {\isacharparenleft}{\kern0pt}{\isadigit{2}}\ verts\ a\ seq{\isacharparenright}{\kern0pt}\isanewline
\ \ \isacommand{then}\isamarkupfalse%
\ \isacommand{interpret}\isamarkupfalse%
\ contxt{\isacharcolon}{\kern0pt}\ prufer{\isacharunderscore}{\kern0pt}seq{\isacharunderscore}{\kern0pt}to{\isacharunderscore}{\kern0pt}tree{\isacharunderscore}{\kern0pt}context\ verts\ \isacommand{by}\isamarkupfalse%
\ unfold{\isacharunderscore}{\kern0pt}locales\isanewline
\ \ \isacommand{obtain}\isamarkupfalse%
\ b\isanewline
\ \ \ \ \isakeyword{where}\ b{\isacharunderscore}{\kern0pt}find{\isacharcolon}{\kern0pt}\ {\isachardoublequoteopen}find\ {\isacharparenleft}{\kern0pt}{\isasymlambda}x{\isachardot}{\kern0pt}\ x\ {\isasymnotin}\ set\ {\isacharparenleft}{\kern0pt}a\ {\isacharhash}{\kern0pt}\ seq{\isacharparenright}{\kern0pt}{\isacharparenright}{\kern0pt}\ verts\ {\isacharequal}{\kern0pt}\ Some\ b{\isachardoublequoteclose}\isanewline
\ \ \ \ \ \ \isakeyword{and}\ seq{\isacharunderscore}{\kern0pt}in{\isacharunderscore}{\kern0pt}verts{\isacharprime}{\kern0pt}{\isacharcolon}{\kern0pt}\ {\isachardoublequoteopen}seq\ {\isasymin}\ prufer{\isacharunderscore}{\kern0pt}sequences\ {\isacharparenleft}{\kern0pt}remove{\isadigit{1}}\ b\ verts{\isacharparenright}{\kern0pt}{\isachardoublequoteclose}\isanewline
\ \ \ \ \ \ \isakeyword{and}\ len{\isacharunderscore}{\kern0pt}verts{\isacharprime}{\kern0pt}{\isacharcolon}{\kern0pt}\ {\isachardoublequoteopen}{\isadigit{2}}\ {\isasymle}\ length\ {\isacharparenleft}{\kern0pt}remove{\isadigit{1}}\ b\ verts{\isacharparenright}{\kern0pt}{\isachardoublequoteclose}\isanewline
\ \ \ \ \ \ \isakeyword{and}\ distinct{\isacharunderscore}{\kern0pt}verts{\isacharprime}{\kern0pt}{\isacharcolon}{\kern0pt}\ {\isachardoublequoteopen}distinct\ {\isacharparenleft}{\kern0pt}remove{\isadigit{1}}\ b\ verts{\isacharparenright}{\kern0pt}{\isachardoublequoteclose}\isanewline
\ \ \ \ \ \ \isakeyword{and}\ b{\isacharunderscore}{\kern0pt}in{\isacharunderscore}{\kern0pt}verts{\isacharcolon}{\kern0pt}\ {\isachardoublequoteopen}b\ {\isasymin}\ set\ verts{\isachardoublequoteclose}\isanewline
\ \ \ \ \isacommand{using}\isamarkupfalse%
\ contxt{\isachardot}{\kern0pt}obtain{\isacharunderscore}{\kern0pt}b{\isacharunderscore}{\kern0pt}prufer{\isacharunderscore}{\kern0pt}seq{\isacharunderscore}{\kern0pt}to{\isacharunderscore}{\kern0pt}tree{\isacharunderscore}{\kern0pt}edges\ {\isadigit{2}}\ \isacommand{by}\isamarkupfalse%
\ metis\isanewline
\ \ \isacommand{then}\isamarkupfalse%
\ \isacommand{have}\isamarkupfalse%
\ {\isachardoublequoteopen}verts{\isacharunderscore}{\kern0pt}of{\isacharunderscore}{\kern0pt}edges\ {\isacharparenleft}{\kern0pt}prufer{\isacharunderscore}{\kern0pt}seq{\isacharunderscore}{\kern0pt}to{\isacharunderscore}{\kern0pt}tree{\isacharunderscore}{\kern0pt}edges\ verts\ {\isacharparenleft}{\kern0pt}a\ {\isacharhash}{\kern0pt}\ seq{\isacharparenright}{\kern0pt}{\isacharparenright}{\kern0pt}\isanewline
\ \ \ \ {\isacharequal}{\kern0pt}\ verts{\isacharunderscore}{\kern0pt}of{\isacharunderscore}{\kern0pt}edges\ {\isacharparenleft}{\kern0pt}{\isacharparenleft}{\kern0pt}a{\isacharcomma}{\kern0pt}b{\isacharparenright}{\kern0pt}\ {\isacharhash}{\kern0pt}\ prufer{\isacharunderscore}{\kern0pt}seq{\isacharunderscore}{\kern0pt}to{\isacharunderscore}{\kern0pt}tree{\isacharunderscore}{\kern0pt}edges\ {\isacharparenleft}{\kern0pt}remove{\isadigit{1}}\ b\ verts{\isacharparenright}{\kern0pt}\ seq{\isacharparenright}{\kern0pt}{\isachardoublequoteclose}\isanewline
\ \ \ \ \isacommand{by}\isamarkupfalse%
\ auto\isanewline
\ \ \isacommand{also}\isamarkupfalse%
\ \isacommand{have}\isamarkupfalse%
\ {\isachardoublequoteopen}{\isasymdots}\ {\isacharequal}{\kern0pt}\ {\isacharbraceleft}{\kern0pt}a{\isacharcomma}{\kern0pt}b{\isacharbraceright}{\kern0pt}\ {\isasymunion}\ verts{\isacharunderscore}{\kern0pt}of{\isacharunderscore}{\kern0pt}edges\ {\isacharparenleft}{\kern0pt}prufer{\isacharunderscore}{\kern0pt}seq{\isacharunderscore}{\kern0pt}to{\isacharunderscore}{\kern0pt}tree{\isacharunderscore}{\kern0pt}edges\ {\isacharparenleft}{\kern0pt}remove{\isadigit{1}}\ b\ verts{\isacharparenright}{\kern0pt}\ seq{\isacharparenright}{\kern0pt}{\isachardoublequoteclose}\isanewline
\ \ \ \ \isacommand{unfolding}\isamarkupfalse%
\ verts{\isacharunderscore}{\kern0pt}of{\isacharunderscore}{\kern0pt}edges{\isacharunderscore}{\kern0pt}def\ edges{\isacharunderscore}{\kern0pt}of{\isacharunderscore}{\kern0pt}edge{\isacharunderscore}{\kern0pt}list{\isacharunderscore}{\kern0pt}def\ \isacommand{by}\isamarkupfalse%
\ auto\isanewline
\ \ \isacommand{also}\isamarkupfalse%
\ \isacommand{have}\isamarkupfalse%
\ {\isachardoublequoteopen}{\isasymdots}\ {\isacharequal}{\kern0pt}\ {\isacharbraceleft}{\kern0pt}a{\isacharcomma}{\kern0pt}b{\isacharbraceright}{\kern0pt}\ {\isasymunion}\ {\isacharparenleft}{\kern0pt}set\ verts\ {\isacharminus}{\kern0pt}\ {\isacharbraceleft}{\kern0pt}b{\isacharbraceright}{\kern0pt}{\isacharparenright}{\kern0pt}{\isachardoublequoteclose}\ \isacommand{using}\isamarkupfalse%
\ {\isachardoublequoteopen}{\isadigit{2}}{\isachardot}{\kern0pt}IH{\isachardoublequoteclose}{\isacharbrackleft}{\kern0pt}OF\ b{\isacharunderscore}{\kern0pt}find\ seq{\isacharunderscore}{\kern0pt}in{\isacharunderscore}{\kern0pt}verts{\isacharprime}{\kern0pt}\ len{\isacharunderscore}{\kern0pt}verts{\isacharprime}{\kern0pt}\ distinct{\isacharunderscore}{\kern0pt}verts{\isacharprime}{\kern0pt}{\isacharbrackright}{\kern0pt}\ b{\isacharunderscore}{\kern0pt}in{\isacharunderscore}{\kern0pt}verts\ \isacommand{by}\isamarkupfalse%
\ fastforce\isanewline
\ \ \isacommand{also}\isamarkupfalse%
\ \isacommand{have}\isamarkupfalse%
\ {\isachardoublequoteopen}{\isasymdots}\ {\isacharequal}{\kern0pt}\ set\ verts{\isachardoublequoteclose}\ \isacommand{using}\isamarkupfalse%
\ {\isachardoublequoteopen}{\isadigit{2}}{\isachardot}{\kern0pt}prems{\isachardoublequoteclose}{\isacharparenleft}{\kern0pt}{\isadigit{1}}{\isacharparenright}{\kern0pt}\ b{\isacharunderscore}{\kern0pt}in{\isacharunderscore}{\kern0pt}verts\ \isacommand{unfolding}\isamarkupfalse%
\ prufer{\isacharunderscore}{\kern0pt}sequences{\isacharunderscore}{\kern0pt}def\ n{\isacharunderscore}{\kern0pt}sequences{\isacharunderscore}{\kern0pt}def\ \isacommand{by}\isamarkupfalse%
\ auto\isanewline
\ \ \isacommand{finally}\isamarkupfalse%
\ \isacommand{show}\isamarkupfalse%
\ {\isacharquery}{\kern0pt}case\ \isacommand{{\isachardot}{\kern0pt}}\isamarkupfalse%
\isanewline
\isacommand{qed}\isamarkupfalse%
\ {\isacharparenleft}{\kern0pt}auto\ simp{\isacharcolon}{\kern0pt}\ prufer{\isacharunderscore}{\kern0pt}sequences{\isacharunderscore}{\kern0pt}def\ n{\isacharunderscore}{\kern0pt}sequences{\isacharunderscore}{\kern0pt}def{\isacharparenright}{\kern0pt}%
\endisatagproof
{\isafoldproof}%
%
\isadelimproof
\isanewline
%
\endisadelimproof
\isanewline
\isacommand{lemma}\isamarkupfalse%
\ prufer{\isacharunderscore}{\kern0pt}seq{\isacharunderscore}{\kern0pt}to{\isacharunderscore}{\kern0pt}tree{\isacharunderscore}{\kern0pt}edges{\isacharunderscore}{\kern0pt}tree{\isacharcolon}{\kern0pt}\isanewline
\ \ \isakeyword{assumes}\ {\isachardoublequoteopen}seq\ {\isasymin}\ prufer{\isacharunderscore}{\kern0pt}sequences\ verts{\isachardoublequoteclose}\isanewline
\ \ \isakeyword{shows}\ {\isachardoublequoteopen}tree\ {\isacharparenleft}{\kern0pt}verts{\isacharunderscore}{\kern0pt}of{\isacharunderscore}{\kern0pt}edges\ {\isacharparenleft}{\kern0pt}prufer{\isacharunderscore}{\kern0pt}seq{\isacharunderscore}{\kern0pt}to{\isacharunderscore}{\kern0pt}tree{\isacharunderscore}{\kern0pt}edges\ verts\ seq{\isacharparenright}{\kern0pt}{\isacharparenright}{\kern0pt}\ {\isacharparenleft}{\kern0pt}edges{\isacharunderscore}{\kern0pt}of{\isacharunderscore}{\kern0pt}edge{\isacharunderscore}{\kern0pt}list\ {\isacharparenleft}{\kern0pt}prufer{\isacharunderscore}{\kern0pt}seq{\isacharunderscore}{\kern0pt}to{\isacharunderscore}{\kern0pt}tree{\isacharunderscore}{\kern0pt}edges\ verts\ seq{\isacharparenright}{\kern0pt}{\isacharparenright}{\kern0pt}{\isachardoublequoteclose}\isanewline
\ \ \ \ {\isacharparenleft}{\kern0pt}\isakeyword{is}\ {\isachardoublequoteopen}tree\ {\isacharparenleft}{\kern0pt}{\isacharquery}{\kern0pt}V\ verts\ seq{\isacharparenright}{\kern0pt}\ {\isacharparenleft}{\kern0pt}{\isacharquery}{\kern0pt}E\ verts\ seq{\isacharparenright}{\kern0pt}{\isachardoublequoteclose}{\isacharparenright}{\kern0pt}\isanewline
%
\isadelimproof
\ \ %
\endisadelimproof
%
\isatagproof
\isacommand{using}\isamarkupfalse%
\ assms\ verts{\isacharunderscore}{\kern0pt}length\ distinct{\isacharunderscore}{\kern0pt}verts\isanewline
\isacommand{proof}\isamarkupfalse%
{\isacharparenleft}{\kern0pt}induction\ verts\ seq\ rule{\isacharcolon}{\kern0pt}\ prufer{\isacharunderscore}{\kern0pt}seq{\isacharunderscore}{\kern0pt}to{\isacharunderscore}{\kern0pt}tree{\isacharunderscore}{\kern0pt}edges{\isachardot}{\kern0pt}induct{\isacharparenright}{\kern0pt}\isanewline
\ \ \isacommand{case}\isamarkupfalse%
\ {\isacharparenleft}{\kern0pt}{\isadigit{1}}\ v\ w{\isacharparenright}{\kern0pt}\isanewline
\ \ \isacommand{have}\isamarkupfalse%
\ {\isacharbrackleft}{\kern0pt}simp{\isacharbrackright}{\kern0pt}{\isacharcolon}{\kern0pt}\ {\isachardoublequoteopen}verts{\isacharunderscore}{\kern0pt}of{\isacharunderscore}{\kern0pt}edges\ {\isacharbrackleft}{\kern0pt}{\isacharparenleft}{\kern0pt}v{\isacharcomma}{\kern0pt}w{\isacharparenright}{\kern0pt}{\isacharbrackright}{\kern0pt}\ {\isacharequal}{\kern0pt}\ {\isacharbraceleft}{\kern0pt}v{\isacharcomma}{\kern0pt}w{\isacharbraceright}{\kern0pt}{\isachardoublequoteclose}\isanewline
\ \ \ \ \isacommand{unfolding}\isamarkupfalse%
\ verts{\isacharunderscore}{\kern0pt}of{\isacharunderscore}{\kern0pt}edges{\isacharunderscore}{\kern0pt}def\ edges{\isacharunderscore}{\kern0pt}of{\isacharunderscore}{\kern0pt}edge{\isacharunderscore}{\kern0pt}list{\isacharunderscore}{\kern0pt}def\ \isacommand{using}\isamarkupfalse%
\ {\isadigit{1}}\ \isacommand{by}\isamarkupfalse%
\ auto\isanewline
\isanewline
\ \ \isacommand{interpret}\isamarkupfalse%
\ ulgraph\ {\isachardoublequoteopen}{\isacharquery}{\kern0pt}V\ {\isacharbrackleft}{\kern0pt}v{\isacharcomma}{\kern0pt}w{\isacharbrackright}{\kern0pt}\ {\isacharbrackleft}{\kern0pt}{\isacharbrackright}{\kern0pt}{\isachardoublequoteclose}\ {\isachardoublequoteopen}{\isacharquery}{\kern0pt}E\ {\isacharbrackleft}{\kern0pt}v{\isacharcomma}{\kern0pt}w{\isacharbrackright}{\kern0pt}\ {\isacharbrackleft}{\kern0pt}{\isacharbrackright}{\kern0pt}{\isachardoublequoteclose}\isanewline
\ \ \ \ \isacommand{by}\isamarkupfalse%
\ {\isacharparenleft}{\kern0pt}unfold{\isacharunderscore}{\kern0pt}locales{\isacharcomma}{\kern0pt}\ auto\ simp{\isacharcolon}{\kern0pt}\ card{\isacharunderscore}{\kern0pt}insert{\isacharunderscore}{\kern0pt}if\ verts{\isacharunderscore}{\kern0pt}of{\isacharunderscore}{\kern0pt}edges{\isacharunderscore}{\kern0pt}def\ edges{\isacharunderscore}{\kern0pt}of{\isacharunderscore}{\kern0pt}edge{\isacharunderscore}{\kern0pt}list{\isacharunderscore}{\kern0pt}def{\isacharparenright}{\kern0pt}\isanewline
\ \ \isanewline
\ \ \isacommand{have}\isamarkupfalse%
\ {\isachardoublequoteopen}connecting{\isacharunderscore}{\kern0pt}path\ v\ w\ {\isacharbrackleft}{\kern0pt}v{\isacharcomma}{\kern0pt}w{\isacharbrackright}{\kern0pt}{\isachardoublequoteclose}\isanewline
\ \ \ \ \isacommand{unfolding}\isamarkupfalse%
\ connecting{\isacharunderscore}{\kern0pt}path{\isacharunderscore}{\kern0pt}def\ \ is{\isacharunderscore}{\kern0pt}gen{\isacharunderscore}{\kern0pt}path{\isacharunderscore}{\kern0pt}def\ is{\isacharunderscore}{\kern0pt}walk{\isacharunderscore}{\kern0pt}def\isanewline
\ \ \ \ \isacommand{by}\isamarkupfalse%
\ {\isacharparenleft}{\kern0pt}auto\ simp{\isacharcolon}{\kern0pt}\ verts{\isacharunderscore}{\kern0pt}of{\isacharunderscore}{\kern0pt}edges{\isacharunderscore}{\kern0pt}def\ edges{\isacharunderscore}{\kern0pt}of{\isacharunderscore}{\kern0pt}edge{\isacharunderscore}{\kern0pt}list{\isacharunderscore}{\kern0pt}def{\isacharparenright}{\kern0pt}\isanewline
\ \ \isacommand{then}\isamarkupfalse%
\ \isacommand{have}\isamarkupfalse%
\ {\isachardoublequoteopen}vert{\isacharunderscore}{\kern0pt}connected\ v\ w{\isachardoublequoteclose}\ {\isachardoublequoteopen}vert{\isacharunderscore}{\kern0pt}connected\ w\ v{\isachardoublequoteclose}\isanewline
\ \ \ \ \isacommand{unfolding}\isamarkupfalse%
\ vert{\isacharunderscore}{\kern0pt}connected{\isacharunderscore}{\kern0pt}def\ \isacommand{using}\isamarkupfalse%
\ connecting{\isacharunderscore}{\kern0pt}path{\isacharunderscore}{\kern0pt}rev\ \isacommand{by}\isamarkupfalse%
\ auto\isanewline
\ \ \isacommand{then}\isamarkupfalse%
\ \isacommand{have}\isamarkupfalse%
\ connected{\isacharcolon}{\kern0pt}\ {\isachardoublequoteopen}is{\isacharunderscore}{\kern0pt}connected{\isacharunderscore}{\kern0pt}set\ {\isacharparenleft}{\kern0pt}{\isacharquery}{\kern0pt}V\ {\isacharbrackleft}{\kern0pt}v{\isacharcomma}{\kern0pt}w{\isacharbrackright}{\kern0pt}\ {\isacharbrackleft}{\kern0pt}{\isacharbrackright}{\kern0pt}{\isacharparenright}{\kern0pt}{\isachardoublequoteclose}\isanewline
\ \ \ \ \isacommand{unfolding}\isamarkupfalse%
\ is{\isacharunderscore}{\kern0pt}connected{\isacharunderscore}{\kern0pt}set{\isacharunderscore}{\kern0pt}def\ \isacommand{using}\isamarkupfalse%
\ vert{\isacharunderscore}{\kern0pt}connected{\isacharunderscore}{\kern0pt}id\ \isacommand{by}\isamarkupfalse%
\ auto\isanewline
\ \ \isacommand{then}\isamarkupfalse%
\ \isacommand{have}\isamarkupfalse%
\ fin{\isacharunderscore}{\kern0pt}connected{\isacharunderscore}{\kern0pt}ulgraph{\isacharcolon}{\kern0pt}\ {\isachardoublequoteopen}fin{\isacharunderscore}{\kern0pt}connected{\isacharunderscore}{\kern0pt}ulgraph\ {\isacharparenleft}{\kern0pt}{\isacharquery}{\kern0pt}V\ {\isacharbrackleft}{\kern0pt}v{\isacharcomma}{\kern0pt}w{\isacharbrackright}{\kern0pt}\ {\isacharbrackleft}{\kern0pt}{\isacharbrackright}{\kern0pt}{\isacharparenright}{\kern0pt}\ {\isacharparenleft}{\kern0pt}{\isacharquery}{\kern0pt}E\ {\isacharbrackleft}{\kern0pt}v{\isacharcomma}{\kern0pt}w{\isacharbrackright}{\kern0pt}\ {\isacharbrackleft}{\kern0pt}{\isacharbrackright}{\kern0pt}{\isacharparenright}{\kern0pt}{\isachardoublequoteclose}\isanewline
\ \ \ \ \isacommand{using}\isamarkupfalse%
\ {\isadigit{1}}\ \isacommand{unfolding}\isamarkupfalse%
\ verts{\isacharunderscore}{\kern0pt}of{\isacharunderscore}{\kern0pt}edges{\isacharunderscore}{\kern0pt}def\ edges{\isacharunderscore}{\kern0pt}of{\isacharunderscore}{\kern0pt}edge{\isacharunderscore}{\kern0pt}list{\isacharunderscore}{\kern0pt}def\ \isacommand{by}\isamarkupfalse%
\ {\isacharparenleft}{\kern0pt}unfold{\isacharunderscore}{\kern0pt}locales{\isacharcomma}{\kern0pt}\ auto{\isacharparenright}{\kern0pt}\isanewline
\isanewline
\ \ \isacommand{then}\isamarkupfalse%
\ \isacommand{show}\isamarkupfalse%
\ {\isacharquery}{\kern0pt}case\ \isacommand{using}\isamarkupfalse%
\ fin{\isacharunderscore}{\kern0pt}connected{\isacharunderscore}{\kern0pt}ulgraph\ {\isadigit{1}}\ \isacommand{unfolding}\isamarkupfalse%
\ edges{\isacharunderscore}{\kern0pt}of{\isacharunderscore}{\kern0pt}edge{\isacharunderscore}{\kern0pt}list{\isacharunderscore}{\kern0pt}def\ \isacommand{by}\isamarkupfalse%
\ {\isacharparenleft}{\kern0pt}auto\ intro{\isacharcolon}{\kern0pt}\ card{\isacharunderscore}{\kern0pt}E{\isacharunderscore}{\kern0pt}treeI{\isacharparenright}{\kern0pt}\isanewline
\isacommand{next}\isamarkupfalse%
\isanewline
\ \ \isacommand{case}\isamarkupfalse%
\ {\isacharparenleft}{\kern0pt}{\isadigit{2}}\ verts\ a\ seq{\isacharparenright}{\kern0pt}\isanewline
\ \ \isacommand{then}\isamarkupfalse%
\ \isacommand{interpret}\isamarkupfalse%
\ contxt{\isacharcolon}{\kern0pt}\ prufer{\isacharunderscore}{\kern0pt}seq{\isacharunderscore}{\kern0pt}to{\isacharunderscore}{\kern0pt}tree{\isacharunderscore}{\kern0pt}context\ verts\ \isacommand{by}\isamarkupfalse%
\ unfold{\isacharunderscore}{\kern0pt}locales\isanewline
\ \ \isacommand{obtain}\isamarkupfalse%
\ b\isanewline
\ \ \ \ \isakeyword{where}\ b{\isacharunderscore}{\kern0pt}find{\isacharcolon}{\kern0pt}\ {\isachardoublequoteopen}find\ {\isacharparenleft}{\kern0pt}{\isasymlambda}x{\isachardot}{\kern0pt}\ x\ {\isasymnotin}\ set\ {\isacharparenleft}{\kern0pt}a\ {\isacharhash}{\kern0pt}\ seq{\isacharparenright}{\kern0pt}{\isacharparenright}{\kern0pt}\ verts\ {\isacharequal}{\kern0pt}\ Some\ b{\isachardoublequoteclose}\isanewline
\ \ \ \ \ \ \isakeyword{and}\ b{\isacharunderscore}{\kern0pt}in{\isacharunderscore}{\kern0pt}verts{\isacharcolon}{\kern0pt}\ {\isachardoublequoteopen}b\ {\isasymin}\ set\ verts{\isachardoublequoteclose}\isanewline
\ \ \ \ \ \ \isakeyword{and}\ b{\isacharunderscore}{\kern0pt}notin{\isacharunderscore}{\kern0pt}seq{\isacharcolon}{\kern0pt}\ {\isachardoublequoteopen}b\ {\isasymnotin}\ set\ {\isacharparenleft}{\kern0pt}a\ {\isacharhash}{\kern0pt}\ seq{\isacharparenright}{\kern0pt}{\isachardoublequoteclose}\isanewline
\ \ \ \ \ \ \isakeyword{and}\ seq{\isacharunderscore}{\kern0pt}pruf{\isacharunderscore}{\kern0pt}verts{\isacharprime}{\kern0pt}{\isacharcolon}{\kern0pt}\ {\isachardoublequoteopen}seq\ {\isasymin}\ prufer{\isacharunderscore}{\kern0pt}sequences\ {\isacharparenleft}{\kern0pt}remove{\isadigit{1}}\ b\ verts{\isacharparenright}{\kern0pt}{\isachardoublequoteclose}\isanewline
\ \ \ \ \ \ \isakeyword{and}\ length{\isacharunderscore}{\kern0pt}verts{\isacharprime}{\kern0pt}{\isacharcolon}{\kern0pt}\ {\isachardoublequoteopen}length\ {\isacharparenleft}{\kern0pt}remove{\isadigit{1}}\ b\ verts{\isacharparenright}{\kern0pt}\ {\isasymge}\ {\isadigit{2}}{\isachardoublequoteclose}\isanewline
\ \ \ \ \ \ \isakeyword{and}\ distinct{\isacharunderscore}{\kern0pt}verts{\isacharprime}{\kern0pt}{\isacharcolon}{\kern0pt}\ {\isachardoublequoteopen}distinct\ {\isacharparenleft}{\kern0pt}remove{\isadigit{1}}\ b\ verts{\isacharparenright}{\kern0pt}{\isachardoublequoteclose}\isanewline
\ \ \ \ \isacommand{using}\isamarkupfalse%
\ contxt{\isachardot}{\kern0pt}obtain{\isacharunderscore}{\kern0pt}b{\isacharunderscore}{\kern0pt}prufer{\isacharunderscore}{\kern0pt}seq{\isacharunderscore}{\kern0pt}to{\isacharunderscore}{\kern0pt}tree{\isacharunderscore}{\kern0pt}edges\ {\isadigit{2}}\ \isacommand{by}\isamarkupfalse%
\ metis\isanewline
\ \ \isacommand{then}\isamarkupfalse%
\ \isacommand{interpret}\isamarkupfalse%
\ tree{\isacharprime}{\kern0pt}{\isacharcolon}{\kern0pt}\ tree\ {\isachardoublequoteopen}{\isacharquery}{\kern0pt}V\ {\isacharparenleft}{\kern0pt}remove{\isadigit{1}}\ b\ verts{\isacharparenright}{\kern0pt}\ seq{\isachardoublequoteclose}\ {\isachardoublequoteopen}{\isacharquery}{\kern0pt}E\ {\isacharparenleft}{\kern0pt}remove{\isadigit{1}}\ b\ verts{\isacharparenright}{\kern0pt}\ seq{\isachardoublequoteclose}\isanewline
\ \ \ \ \isacommand{using}\isamarkupfalse%
\ {\isadigit{2}}\ seq{\isacharunderscore}{\kern0pt}pruf{\isacharunderscore}{\kern0pt}verts{\isacharprime}{\kern0pt}\ distinct{\isacharunderscore}{\kern0pt}remove{\isadigit{1}}\ b{\isacharunderscore}{\kern0pt}find\ b{\isacharunderscore}{\kern0pt}in{\isacharunderscore}{\kern0pt}verts\ \isacommand{by}\isamarkupfalse%
\ auto\isanewline
\isanewline
\ \ \isacommand{interpret}\isamarkupfalse%
\ contxt{\isacharprime}{\kern0pt}{\isacharcolon}{\kern0pt}\ prufer{\isacharunderscore}{\kern0pt}seq{\isacharunderscore}{\kern0pt}to{\isacharunderscore}{\kern0pt}tree{\isacharunderscore}{\kern0pt}context\ {\isachardoublequoteopen}remove{\isadigit{1}}\ b\ verts{\isachardoublequoteclose}\ \isacommand{using}\isamarkupfalse%
\ length{\isacharunderscore}{\kern0pt}verts{\isacharprime}{\kern0pt}\ distinct{\isacharunderscore}{\kern0pt}verts{\isacharprime}{\kern0pt}\ \isacommand{by}\isamarkupfalse%
\ unfold{\isacharunderscore}{\kern0pt}locales\isanewline
\isanewline
\ \ \isacommand{have}\isamarkupfalse%
\ V{\isacharprime}{\kern0pt}{\isacharbrackleft}{\kern0pt}simp{\isacharbrackright}{\kern0pt}{\isacharcolon}{\kern0pt}\ {\isachardoublequoteopen}{\isacharquery}{\kern0pt}V\ {\isacharparenleft}{\kern0pt}remove{\isadigit{1}}\ b\ verts{\isacharparenright}{\kern0pt}\ seq\ {\isacharequal}{\kern0pt}\ set\ verts\ {\isacharminus}{\kern0pt}\ {\isacharbraceleft}{\kern0pt}b{\isacharbraceright}{\kern0pt}{\isachardoublequoteclose}\isanewline
\ \ \ \ \isacommand{using}\isamarkupfalse%
\ contxt{\isacharprime}{\kern0pt}{\isachardot}{\kern0pt}verts{\isacharunderscore}{\kern0pt}of{\isacharunderscore}{\kern0pt}edges{\isacharunderscore}{\kern0pt}prufer{\isacharunderscore}{\kern0pt}to{\isacharunderscore}{\kern0pt}tree\ seq{\isacharunderscore}{\kern0pt}pruf{\isacharunderscore}{\kern0pt}verts{\isacharprime}{\kern0pt}\ set{\isacharunderscore}{\kern0pt}remove{\isadigit{1}}{\isacharunderscore}{\kern0pt}eq\ {\isadigit{2}}{\isacharparenleft}{\kern0pt}{\isadigit{4}}{\isacharparenright}{\kern0pt}\ \isacommand{by}\isamarkupfalse%
\ metis\isanewline
\ \ \isacommand{have}\isamarkupfalse%
\ V{\isacharunderscore}{\kern0pt}V{\isacharprime}{\kern0pt}{\isacharcolon}{\kern0pt}\ {\isachardoublequoteopen}{\isacharquery}{\kern0pt}V\ verts\ {\isacharparenleft}{\kern0pt}a\ {\isacharhash}{\kern0pt}\ seq{\isacharparenright}{\kern0pt}\ {\isacharequal}{\kern0pt}\ insert\ b\ {\isacharparenleft}{\kern0pt}{\isacharquery}{\kern0pt}V\ {\isacharparenleft}{\kern0pt}remove{\isadigit{1}}\ b\ verts{\isacharparenright}{\kern0pt}\ seq{\isacharparenright}{\kern0pt}{\isachardoublequoteclose}\isanewline
\ \ \ \ \isacommand{using}\isamarkupfalse%
\ contxt{\isachardot}{\kern0pt}verts{\isacharunderscore}{\kern0pt}of{\isacharunderscore}{\kern0pt}edges{\isacharunderscore}{\kern0pt}prufer{\isacharunderscore}{\kern0pt}to{\isacharunderscore}{\kern0pt}tree\ {\isadigit{2}}\ V{\isacharprime}{\kern0pt}\ b{\isacharunderscore}{\kern0pt}in{\isacharunderscore}{\kern0pt}verts\ \isacommand{by}\isamarkupfalse%
\ blast\isanewline
\ \ \isacommand{have}\isamarkupfalse%
\ edges{\isacharcolon}{\kern0pt}\ {\isachardoublequoteopen}{\isacharquery}{\kern0pt}E\ verts\ {\isacharparenleft}{\kern0pt}a\ {\isacharhash}{\kern0pt}\ seq{\isacharparenright}{\kern0pt}\ {\isacharequal}{\kern0pt}\ insert\ {\isacharbraceleft}{\kern0pt}a{\isacharcomma}{\kern0pt}b{\isacharbraceright}{\kern0pt}\ {\isacharparenleft}{\kern0pt}{\isacharquery}{\kern0pt}E\ {\isacharparenleft}{\kern0pt}remove{\isadigit{1}}\ b\ verts{\isacharparenright}{\kern0pt}\ seq{\isacharparenright}{\kern0pt}{\isachardoublequoteclose}\isanewline
\ \ \ \ \isacommand{unfolding}\isamarkupfalse%
\ edges{\isacharunderscore}{\kern0pt}of{\isacharunderscore}{\kern0pt}edge{\isacharunderscore}{\kern0pt}list{\isacharunderscore}{\kern0pt}def\ \isacommand{using}\isamarkupfalse%
\ b{\isacharunderscore}{\kern0pt}find\ \isacommand{by}\isamarkupfalse%
\ simp\isanewline
\ \ \isacommand{have}\isamarkupfalse%
\ b{\isacharunderscore}{\kern0pt}notin{\isacharunderscore}{\kern0pt}V{\isacharprime}{\kern0pt}{\isacharcolon}{\kern0pt}\ {\isachardoublequoteopen}b\ {\isasymnotin}\ {\isacharquery}{\kern0pt}V\ {\isacharparenleft}{\kern0pt}remove{\isadigit{1}}\ b\ verts{\isacharparenright}{\kern0pt}\ seq{\isachardoublequoteclose}\ \isacommand{using}\isamarkupfalse%
\ V{\isacharprime}{\kern0pt}\ \isacommand{by}\isamarkupfalse%
\ blast\isanewline
\ \ \isacommand{have}\isamarkupfalse%
\ a{\isacharunderscore}{\kern0pt}in{\isacharunderscore}{\kern0pt}V{\isacharprime}{\kern0pt}{\isacharcolon}{\kern0pt}\ {\isachardoublequoteopen}a\ {\isasymin}\ {\isacharquery}{\kern0pt}V\ {\isacharparenleft}{\kern0pt}remove{\isadigit{1}}\ b\ verts{\isacharparenright}{\kern0pt}\ seq{\isachardoublequoteclose}\isanewline
\ \ \ \ \isacommand{using}\isamarkupfalse%
\ V{\isacharprime}{\kern0pt}\ b{\isacharunderscore}{\kern0pt}notin{\isacharunderscore}{\kern0pt}seq\ {\isadigit{2}}{\isacharparenleft}{\kern0pt}{\isadigit{2}}{\isacharparenright}{\kern0pt}\ \isacommand{unfolding}\isamarkupfalse%
\ prufer{\isacharunderscore}{\kern0pt}sequences{\isacharunderscore}{\kern0pt}def\ n{\isacharunderscore}{\kern0pt}sequences{\isacharunderscore}{\kern0pt}def\ \isacommand{by}\isamarkupfalse%
\ auto\isanewline
\isanewline
\ \ \isacommand{show}\isamarkupfalse%
\ {\isacharquery}{\kern0pt}case\ \isacommand{using}\isamarkupfalse%
\ V{\isacharunderscore}{\kern0pt}V{\isacharprime}{\kern0pt}\ edges\ tree{\isacharprime}{\kern0pt}{\isachardot}{\kern0pt}add{\isacharunderscore}{\kern0pt}vertex{\isacharunderscore}{\kern0pt}tree{\isacharbrackleft}{\kern0pt}OF\ b{\isacharunderscore}{\kern0pt}notin{\isacharunderscore}{\kern0pt}V{\isacharprime}{\kern0pt}\ a{\isacharunderscore}{\kern0pt}in{\isacharunderscore}{\kern0pt}V{\isacharprime}{\kern0pt}{\isacharbrackright}{\kern0pt}\ insert{\isacharunderscore}{\kern0pt}commute\ \isacommand{by}\isamarkupfalse%
\ metis\isanewline
\isacommand{qed}\isamarkupfalse%
\ {\isacharparenleft}{\kern0pt}auto\ simp{\isacharcolon}{\kern0pt}\ prufer{\isacharunderscore}{\kern0pt}sequences{\isacharunderscore}{\kern0pt}def\ n{\isacharunderscore}{\kern0pt}sequences{\isacharunderscore}{\kern0pt}def{\isacharparenright}{\kern0pt}%
\endisatagproof
{\isafoldproof}%
%
\isadelimproof
\isanewline
%
\endisadelimproof
\isanewline
\isacommand{lemma}\isamarkupfalse%
\ prufer{\isacharunderscore}{\kern0pt}seq{\isacharunderscore}{\kern0pt}to{\isacharunderscore}{\kern0pt}tree{\isacharunderscore}{\kern0pt}tree{\isacharcolon}{\kern0pt}\ {\isachardoublequoteopen}seq\ {\isasymin}\ prufer{\isacharunderscore}{\kern0pt}sequences\ verts\ {\isasymLongrightarrow}\ {\isacharparenleft}{\kern0pt}V{\isacharcomma}{\kern0pt}E{\isacharparenright}{\kern0pt}\ {\isacharequal}{\kern0pt}\ prufer{\isacharunderscore}{\kern0pt}seq{\isacharunderscore}{\kern0pt}to{\isacharunderscore}{\kern0pt}tree\ verts\ seq\ {\isasymLongrightarrow}\ tree\ V\ E{\isachardoublequoteclose}\isanewline
%
\isadelimproof
\ \ %
\endisadelimproof
%
\isatagproof
\isacommand{unfolding}\isamarkupfalse%
\ prufer{\isacharunderscore}{\kern0pt}seq{\isacharunderscore}{\kern0pt}to{\isacharunderscore}{\kern0pt}tree{\isacharunderscore}{\kern0pt}def\ \isacommand{using}\isamarkupfalse%
\ prufer{\isacharunderscore}{\kern0pt}seq{\isacharunderscore}{\kern0pt}to{\isacharunderscore}{\kern0pt}tree{\isacharunderscore}{\kern0pt}edges{\isacharunderscore}{\kern0pt}tree\ verts{\isacharunderscore}{\kern0pt}of{\isacharunderscore}{\kern0pt}edges{\isacharunderscore}{\kern0pt}prufer{\isacharunderscore}{\kern0pt}to{\isacharunderscore}{\kern0pt}tree\ \isacommand{by}\isamarkupfalse%
\ auto%
\endisatagproof
{\isafoldproof}%
%
\isadelimproof
\isanewline
%
\endisadelimproof
\isanewline
\isacommand{lemma}\isamarkupfalse%
\ labeled{\isacharunderscore}{\kern0pt}tree{\isacharunderscore}{\kern0pt}enum{\isacharunderscore}{\kern0pt}tree{\isacharcolon}{\kern0pt}\ {\isachardoublequoteopen}{\isacharparenleft}{\kern0pt}V{\isacharcomma}{\kern0pt}E{\isacharparenright}{\kern0pt}\ {\isasymin}\ set\ {\isacharparenleft}{\kern0pt}labeled{\isacharunderscore}{\kern0pt}tree{\isacharunderscore}{\kern0pt}enum\ verts{\isacharparenright}{\kern0pt}\ {\isasymLongrightarrow}\ tree\ V\ E{\isachardoublequoteclose}\isanewline
%
\isadelimproof
\ \ %
\endisadelimproof
%
\isatagproof
\isacommand{using}\isamarkupfalse%
\ prufer{\isacharunderscore}{\kern0pt}seq{\isacharunderscore}{\kern0pt}to{\isacharunderscore}{\kern0pt}tree{\isacharunderscore}{\kern0pt}tree\ n{\isacharunderscore}{\kern0pt}sequence{\isacharunderscore}{\kern0pt}enum{\isacharunderscore}{\kern0pt}correct\ \isacommand{unfolding}\isamarkupfalse%
\ labeled{\isacharunderscore}{\kern0pt}tree{\isacharunderscore}{\kern0pt}enum{\isacharunderscore}{\kern0pt}def\ prufer{\isacharunderscore}{\kern0pt}sequences{\isacharunderscore}{\kern0pt}def\ \isacommand{by}\isamarkupfalse%
\ fastforce%
\endisatagproof
{\isafoldproof}%
%
\isadelimproof
\isanewline
%
\endisadelimproof
\isanewline
\isanewline
\isacommand{lemma}\isamarkupfalse%
\ prufer{\isacharunderscore}{\kern0pt}seq{\isacharunderscore}{\kern0pt}to{\isacharunderscore}{\kern0pt}tree{\isacharunderscore}{\kern0pt}edges{\isacharunderscore}{\kern0pt}wf{\isacharcolon}{\kern0pt}\isanewline
\ \ \isakeyword{assumes}\ pruf{\isacharunderscore}{\kern0pt}seq{\isacharcolon}{\kern0pt}\ {\isachardoublequoteopen}seq\ {\isasymin}\ prufer{\isacharunderscore}{\kern0pt}sequences\ verts{\isachardoublequoteclose}\isanewline
\ \ \ \ \isakeyword{and}\ edge{\isacharcolon}{\kern0pt}\ {\isachardoublequoteopen}e\ {\isasymin}\ edges{\isacharunderscore}{\kern0pt}of{\isacharunderscore}{\kern0pt}edge{\isacharunderscore}{\kern0pt}list\ {\isacharparenleft}{\kern0pt}prufer{\isacharunderscore}{\kern0pt}seq{\isacharunderscore}{\kern0pt}to{\isacharunderscore}{\kern0pt}tree{\isacharunderscore}{\kern0pt}edges\ verts\ seq{\isacharparenright}{\kern0pt}{\isachardoublequoteclose}\isanewline
\ \ \isakeyword{shows}\ {\isachardoublequoteopen}e\ {\isasymsubseteq}\ set\ verts{\isachardoublequoteclose}\isanewline
%
\isadelimproof
\ \ %
\endisadelimproof
%
\isatagproof
\isacommand{using}\isamarkupfalse%
\ prufer{\isacharunderscore}{\kern0pt}seq{\isacharunderscore}{\kern0pt}to{\isacharunderscore}{\kern0pt}tree{\isacharunderscore}{\kern0pt}context{\isacharunderscore}{\kern0pt}axioms\ assms\isanewline
\isacommand{proof}\isamarkupfalse%
\ {\isacharparenleft}{\kern0pt}induction\ seq\ arbitrary{\isacharcolon}{\kern0pt}\ verts{\isacharparenright}{\kern0pt}\isanewline
\ \ \isacommand{case}\isamarkupfalse%
\ Nil\isanewline
\ \ \isacommand{then}\isamarkupfalse%
\ \isacommand{interpret}\isamarkupfalse%
\ prufer{\isacharunderscore}{\kern0pt}seq{\isacharunderscore}{\kern0pt}to{\isacharunderscore}{\kern0pt}tree{\isacharunderscore}{\kern0pt}context\ verts\ \isacommand{by}\isamarkupfalse%
\ simp\isanewline
\ \ \isacommand{obtain}\isamarkupfalse%
\ u\ v\ \isakeyword{where}\ {\isachardoublequoteopen}verts\ {\isacharequal}{\kern0pt}\ {\isacharbrackleft}{\kern0pt}u{\isacharcomma}{\kern0pt}v{\isacharbrackright}{\kern0pt}{\isachardoublequoteclose}\ \isacommand{using}\isamarkupfalse%
\ Nil\ verts{\isacharunderscore}{\kern0pt}length\ \isacommand{unfolding}\isamarkupfalse%
\ prufer{\isacharunderscore}{\kern0pt}sequences{\isacharunderscore}{\kern0pt}def\ n{\isacharunderscore}{\kern0pt}sequences{\isacharunderscore}{\kern0pt}def\ \isacommand{apply}\isamarkupfalse%
\ auto\isanewline
\ \ \ \ \isacommand{by}\isamarkupfalse%
\ {\isacharparenleft}{\kern0pt}metis\ {\isacharparenleft}{\kern0pt}no{\isacharunderscore}{\kern0pt}types{\isacharcomma}{\kern0pt}\ opaque{\isacharunderscore}{\kern0pt}lifting{\isacharparenright}{\kern0pt}\ One{\isacharunderscore}{\kern0pt}nat{\isacharunderscore}{\kern0pt}def\ Suc{\isacharunderscore}{\kern0pt}{\isadigit{1}}\ length{\isacharunderscore}{\kern0pt}{\isadigit{0}}{\isacharunderscore}{\kern0pt}conv\ length{\isacharunderscore}{\kern0pt}Suc{\isacharunderscore}{\kern0pt}conv{\isacharparenright}{\kern0pt}\isanewline
\ \ \isacommand{then}\isamarkupfalse%
\ \isacommand{show}\isamarkupfalse%
\ {\isacharquery}{\kern0pt}case\ \isacommand{using}\isamarkupfalse%
\ Nil\ \isacommand{unfolding}\isamarkupfalse%
\ edges{\isacharunderscore}{\kern0pt}of{\isacharunderscore}{\kern0pt}edge{\isacharunderscore}{\kern0pt}list{\isacharunderscore}{\kern0pt}def\ \isacommand{by}\isamarkupfalse%
\ simp\isanewline
\isacommand{next}\isamarkupfalse%
\isanewline
\ \ \isacommand{case}\isamarkupfalse%
\ {\isacharparenleft}{\kern0pt}Cons\ a\ seq{\isacharparenright}{\kern0pt}\isanewline
\ \ \isacommand{then}\isamarkupfalse%
\ \isacommand{interpret}\isamarkupfalse%
\ prufer{\isacharunderscore}{\kern0pt}seq{\isacharunderscore}{\kern0pt}to{\isacharunderscore}{\kern0pt}tree{\isacharunderscore}{\kern0pt}context\ verts\ \isacommand{by}\isamarkupfalse%
\ simp\isanewline
\ \ \isacommand{obtain}\isamarkupfalse%
\ leaf\ \isakeyword{where}\ find{\isacharunderscore}{\kern0pt}leaf{\isacharcolon}{\kern0pt}\ {\isachardoublequoteopen}find\ {\isacharparenleft}{\kern0pt}{\isasymlambda}v{\isachardot}{\kern0pt}\ v\ {\isasymnotin}\ set\ {\isacharparenleft}{\kern0pt}a{\isacharhash}{\kern0pt}seq{\isacharparenright}{\kern0pt}{\isacharparenright}{\kern0pt}\ verts\ {\isacharequal}{\kern0pt}\ Some\ leaf{\isachardoublequoteclose}\isanewline
\ \ \ \ \isakeyword{and}\ pruf{\isacharunderscore}{\kern0pt}seq{\isacharprime}{\kern0pt}{\isacharcolon}{\kern0pt}\ {\isachardoublequoteopen}seq\ {\isasymin}\ prufer{\isacharunderscore}{\kern0pt}sequences\ {\isacharparenleft}{\kern0pt}remove{\isadigit{1}}\ leaf\ verts{\isacharparenright}{\kern0pt}{\isachardoublequoteclose}\isanewline
\ \ \ \ \isakeyword{and}\ leaf{\isacharunderscore}{\kern0pt}in{\isacharunderscore}{\kern0pt}verts{\isacharcolon}{\kern0pt}\ {\isachardoublequoteopen}leaf\ {\isasymin}\ set\ verts{\isachardoublequoteclose}\isanewline
\ \ \ \ \isakeyword{and}\ {\isachardoublequoteopen}length\ {\isacharparenleft}{\kern0pt}remove{\isadigit{1}}\ leaf\ verts{\isacharparenright}{\kern0pt}\ {\isasymge}\ {\isadigit{2}}{\isachardoublequoteclose}\isanewline
\ \ \ \ \isakeyword{and}\ {\isachardoublequoteopen}distinct\ {\isacharparenleft}{\kern0pt}remove{\isadigit{1}}\ leaf\ verts{\isacharparenright}{\kern0pt}{\isachardoublequoteclose}\ \isacommand{using}\isamarkupfalse%
\ Cons\ obtain{\isacharunderscore}{\kern0pt}b{\isacharunderscore}{\kern0pt}prufer{\isacharunderscore}{\kern0pt}seq{\isacharunderscore}{\kern0pt}to{\isacharunderscore}{\kern0pt}tree{\isacharunderscore}{\kern0pt}edges\ \isacommand{by}\isamarkupfalse%
\ blast\isanewline
\ \ \isacommand{then}\isamarkupfalse%
\ \isacommand{have}\isamarkupfalse%
\ contxt{\isacharprime}{\kern0pt}{\isacharcolon}{\kern0pt}\ {\isachardoublequoteopen}prufer{\isacharunderscore}{\kern0pt}seq{\isacharunderscore}{\kern0pt}to{\isacharunderscore}{\kern0pt}tree{\isacharunderscore}{\kern0pt}context\ {\isacharparenleft}{\kern0pt}remove{\isadigit{1}}\ leaf\ verts{\isacharparenright}{\kern0pt}{\isachardoublequoteclose}\ \isacommand{by}\isamarkupfalse%
\ {\isacharparenleft}{\kern0pt}unfold{\isacharunderscore}{\kern0pt}locales{\isacharcomma}{\kern0pt}\ simp{\isacharparenright}{\kern0pt}\isanewline
\ \ \isacommand{have}\isamarkupfalse%
\ a{\isacharunderscore}{\kern0pt}in{\isacharunderscore}{\kern0pt}verts{\isacharcolon}{\kern0pt}\ {\isachardoublequoteopen}a\ {\isasymin}\ set\ verts{\isachardoublequoteclose}\ \isacommand{using}\isamarkupfalse%
\ Cons{\isacharparenleft}{\kern0pt}{\isadigit{3}}{\isacharparenright}{\kern0pt}\ \isacommand{unfolding}\isamarkupfalse%
\ prufer{\isacharunderscore}{\kern0pt}sequences{\isacharunderscore}{\kern0pt}def\ n{\isacharunderscore}{\kern0pt}sequences{\isacharunderscore}{\kern0pt}def\ \isacommand{by}\isamarkupfalse%
\ simp\isanewline
\ \ \isacommand{show}\isamarkupfalse%
\ {\isacharquery}{\kern0pt}case\ \isacommand{using}\isamarkupfalse%
\ Cons{\isacharparenleft}{\kern0pt}{\isadigit{4}}{\isacharparenright}{\kern0pt}\ Cons{\isachardot}{\kern0pt}IH{\isacharbrackleft}{\kern0pt}OF\ contxt{\isacharprime}{\kern0pt}\ pruf{\isacharunderscore}{\kern0pt}seq{\isacharprime}{\kern0pt}{\isacharbrackright}{\kern0pt}\ find{\isacharunderscore}{\kern0pt}leaf\ a{\isacharunderscore}{\kern0pt}in{\isacharunderscore}{\kern0pt}verts\ leaf{\isacharunderscore}{\kern0pt}in{\isacharunderscore}{\kern0pt}verts\isanewline
\ \ \ \ \isacommand{unfolding}\isamarkupfalse%
\ edges{\isacharunderscore}{\kern0pt}of{\isacharunderscore}{\kern0pt}edge{\isacharunderscore}{\kern0pt}list{\isacharunderscore}{\kern0pt}def\ \isacommand{by}\isamarkupfalse%
\ {\isacharparenleft}{\kern0pt}auto{\isacharcomma}{\kern0pt}\ {\isacharparenleft}{\kern0pt}meson\ in{\isacharunderscore}{\kern0pt}mk{\isacharunderscore}{\kern0pt}uedge{\isacharunderscore}{\kern0pt}img{\isacharunderscore}{\kern0pt}iff\ notin{\isacharunderscore}{\kern0pt}set{\isacharunderscore}{\kern0pt}remove{\isadigit{1}}{\isacharparenright}{\kern0pt}{\isacharplus}{\kern0pt}{\isacharparenright}{\kern0pt}\isanewline
\isacommand{qed}\isamarkupfalse%
%
\endisatagproof
{\isafoldproof}%
%
\isadelimproof
\isanewline
%
\endisadelimproof
\isanewline
\isacommand{lemma}\isamarkupfalse%
\ distinct{\isacharunderscore}{\kern0pt}prufer{\isacharunderscore}{\kern0pt}seq{\isacharunderscore}{\kern0pt}to{\isacharunderscore}{\kern0pt}tree{\isacharcolon}{\kern0pt}\ {\isachardoublequoteopen}seq\ {\isasymin}\ prufer{\isacharunderscore}{\kern0pt}sequences\ verts\ {\isasymLongrightarrow}\ distinct\ {\isacharparenleft}{\kern0pt}map\ mk{\isacharunderscore}{\kern0pt}edge\ {\isacharparenleft}{\kern0pt}prufer{\isacharunderscore}{\kern0pt}seq{\isacharunderscore}{\kern0pt}to{\isacharunderscore}{\kern0pt}tree{\isacharunderscore}{\kern0pt}edges\ verts\ seq{\isacharparenright}{\kern0pt}{\isacharparenright}{\kern0pt}{\isachardoublequoteclose}\isanewline
%
\isadelimproof
\ \ %
\endisadelimproof
%
\isatagproof
\isacommand{using}\isamarkupfalse%
\ prufer{\isacharunderscore}{\kern0pt}seq{\isacharunderscore}{\kern0pt}to{\isacharunderscore}{\kern0pt}tree{\isacharunderscore}{\kern0pt}context{\isacharunderscore}{\kern0pt}axioms\isanewline
\isacommand{proof}\isamarkupfalse%
\ {\isacharparenleft}{\kern0pt}induction\ seq\ arbitrary{\isacharcolon}{\kern0pt}\ verts{\isacharparenright}{\kern0pt}\isanewline
\ \ \isacommand{case}\isamarkupfalse%
\ Nil\isanewline
\ \ \isacommand{then}\isamarkupfalse%
\ \isacommand{interpret}\isamarkupfalse%
\ prufer{\isacharunderscore}{\kern0pt}seq{\isacharunderscore}{\kern0pt}to{\isacharunderscore}{\kern0pt}tree{\isacharunderscore}{\kern0pt}context\ verts\ \isacommand{by}\isamarkupfalse%
\ simp\isanewline
\ \ \isacommand{obtain}\isamarkupfalse%
\ u\ v\ \isakeyword{where}\ {\isachardoublequoteopen}verts\ {\isacharequal}{\kern0pt}\ {\isacharbrackleft}{\kern0pt}u{\isacharcomma}{\kern0pt}v{\isacharbrackright}{\kern0pt}{\isachardoublequoteclose}\ \isacommand{using}\isamarkupfalse%
\ Nil\ verts{\isacharunderscore}{\kern0pt}length\ \isacommand{unfolding}\isamarkupfalse%
\ prufer{\isacharunderscore}{\kern0pt}sequences{\isacharunderscore}{\kern0pt}def\ n{\isacharunderscore}{\kern0pt}sequences{\isacharunderscore}{\kern0pt}def\ \isacommand{apply}\isamarkupfalse%
\ auto\isanewline
\ \ \ \ \isacommand{by}\isamarkupfalse%
\ {\isacharparenleft}{\kern0pt}metis\ {\isacharparenleft}{\kern0pt}no{\isacharunderscore}{\kern0pt}types{\isacharcomma}{\kern0pt}\ opaque{\isacharunderscore}{\kern0pt}lifting{\isacharparenright}{\kern0pt}\ One{\isacharunderscore}{\kern0pt}nat{\isacharunderscore}{\kern0pt}def\ Suc{\isacharunderscore}{\kern0pt}{\isadigit{1}}\ length{\isacharunderscore}{\kern0pt}{\isadigit{0}}{\isacharunderscore}{\kern0pt}conv\ length{\isacharunderscore}{\kern0pt}Suc{\isacharunderscore}{\kern0pt}conv{\isacharparenright}{\kern0pt}\isanewline
\ \ \isacommand{then}\isamarkupfalse%
\ \isacommand{show}\isamarkupfalse%
\ {\isacharquery}{\kern0pt}case\ \isacommand{by}\isamarkupfalse%
\ auto\isanewline
\isacommand{next}\isamarkupfalse%
\isanewline
\ \ \isacommand{case}\isamarkupfalse%
\ {\isacharparenleft}{\kern0pt}Cons\ a\ seq{\isacharparenright}{\kern0pt}\isanewline
\ \ \isacommand{then}\isamarkupfalse%
\ \isacommand{interpret}\isamarkupfalse%
\ prufer{\isacharunderscore}{\kern0pt}seq{\isacharunderscore}{\kern0pt}to{\isacharunderscore}{\kern0pt}tree{\isacharunderscore}{\kern0pt}context\ verts\ \isacommand{by}\isamarkupfalse%
\ simp\isanewline
\ \ \isacommand{obtain}\isamarkupfalse%
\ leaf\ \isakeyword{where}\ find{\isacharunderscore}{\kern0pt}leaf{\isacharcolon}{\kern0pt}\ {\isachardoublequoteopen}find\ {\isacharparenleft}{\kern0pt}{\isasymlambda}v{\isachardot}{\kern0pt}\ v\ {\isasymnotin}\ set\ {\isacharparenleft}{\kern0pt}a{\isacharhash}{\kern0pt}seq{\isacharparenright}{\kern0pt}{\isacharparenright}{\kern0pt}\ verts\ {\isacharequal}{\kern0pt}\ Some\ leaf{\isachardoublequoteclose}\isanewline
\ \ \ \ \isakeyword{and}\ pruf{\isacharunderscore}{\kern0pt}seq{\isacharprime}{\kern0pt}{\isacharcolon}{\kern0pt}\ {\isachardoublequoteopen}seq\ {\isasymin}\ prufer{\isacharunderscore}{\kern0pt}sequences\ {\isacharparenleft}{\kern0pt}remove{\isadigit{1}}\ leaf\ verts{\isacharparenright}{\kern0pt}{\isachardoublequoteclose}\isanewline
\ \ \ \ \isakeyword{and}\ {\isachardoublequoteopen}length\ {\isacharparenleft}{\kern0pt}remove{\isadigit{1}}\ leaf\ verts{\isacharparenright}{\kern0pt}\ {\isasymge}\ {\isadigit{2}}{\isachardoublequoteclose}\isanewline
\ \ \ \ \isakeyword{and}\ {\isachardoublequoteopen}distinct\ {\isacharparenleft}{\kern0pt}remove{\isadigit{1}}\ leaf\ verts{\isacharparenright}{\kern0pt}{\isachardoublequoteclose}\ \isacommand{using}\isamarkupfalse%
\ Cons\ obtain{\isacharunderscore}{\kern0pt}b{\isacharunderscore}{\kern0pt}prufer{\isacharunderscore}{\kern0pt}seq{\isacharunderscore}{\kern0pt}to{\isacharunderscore}{\kern0pt}tree{\isacharunderscore}{\kern0pt}edges\ \isacommand{by}\isamarkupfalse%
\ blast\isanewline
\ \ \isacommand{then}\isamarkupfalse%
\ \isacommand{interpret}\isamarkupfalse%
\ contxt{\isacharprime}{\kern0pt}{\isacharcolon}{\kern0pt}\ prufer{\isacharunderscore}{\kern0pt}seq{\isacharunderscore}{\kern0pt}to{\isacharunderscore}{\kern0pt}tree{\isacharunderscore}{\kern0pt}context\ {\isachardoublequoteopen}remove{\isadigit{1}}\ leaf\ verts{\isachardoublequoteclose}\ \isacommand{by}\isamarkupfalse%
\ {\isacharparenleft}{\kern0pt}unfold{\isacharunderscore}{\kern0pt}locales{\isacharcomma}{\kern0pt}\ simp{\isacharparenright}{\kern0pt}\isanewline
\ \ \isacommand{have}\isamarkupfalse%
\ {\isachardoublequoteopen}leaf\ {\isasymnotin}\ set\ {\isacharparenleft}{\kern0pt}remove{\isadigit{1}}\ leaf\ verts{\isacharparenright}{\kern0pt}{\isachardoublequoteclose}\ \isacommand{using}\isamarkupfalse%
\ distinct{\isacharunderscore}{\kern0pt}verts\ set{\isacharunderscore}{\kern0pt}remove{\isadigit{1}}{\isacharunderscore}{\kern0pt}eq\ \isacommand{by}\isamarkupfalse%
\ simp\isanewline
\ \ \isacommand{then}\isamarkupfalse%
\ \isacommand{have}\isamarkupfalse%
\ {\isachardoublequoteopen}{\isacharbraceleft}{\kern0pt}a{\isacharcomma}{\kern0pt}\ leaf{\isacharbraceright}{\kern0pt}\ {\isasymnotin}\ edges{\isacharunderscore}{\kern0pt}of{\isacharunderscore}{\kern0pt}edge{\isacharunderscore}{\kern0pt}list\ {\isacharparenleft}{\kern0pt}prufer{\isacharunderscore}{\kern0pt}seq{\isacharunderscore}{\kern0pt}to{\isacharunderscore}{\kern0pt}tree{\isacharunderscore}{\kern0pt}edges\ {\isacharparenleft}{\kern0pt}remove{\isadigit{1}}\ leaf\ verts{\isacharparenright}{\kern0pt}\ seq{\isacharparenright}{\kern0pt}{\isachardoublequoteclose}\isanewline
\ \ \ \ \isacommand{using}\isamarkupfalse%
\ contxt{\isacharprime}{\kern0pt}{\isachardot}{\kern0pt}prufer{\isacharunderscore}{\kern0pt}seq{\isacharunderscore}{\kern0pt}to{\isacharunderscore}{\kern0pt}tree{\isacharunderscore}{\kern0pt}edges{\isacharunderscore}{\kern0pt}wf\ pruf{\isacharunderscore}{\kern0pt}seq{\isacharprime}{\kern0pt}\ \isacommand{by}\isamarkupfalse%
\ blast\isanewline
\ \ \isacommand{then}\isamarkupfalse%
\ \isacommand{show}\isamarkupfalse%
\ {\isacharquery}{\kern0pt}case\ \isacommand{using}\isamarkupfalse%
\ find{\isacharunderscore}{\kern0pt}leaf\ Cons\ pruf{\isacharunderscore}{\kern0pt}seq{\isacharprime}{\kern0pt}\ contxt{\isacharprime}{\kern0pt}{\isachardot}{\kern0pt}prufer{\isacharunderscore}{\kern0pt}seq{\isacharunderscore}{\kern0pt}to{\isacharunderscore}{\kern0pt}tree{\isacharunderscore}{\kern0pt}context{\isacharunderscore}{\kern0pt}axioms\isanewline
\ \ \ \ \isacommand{unfolding}\isamarkupfalse%
\ edges{\isacharunderscore}{\kern0pt}of{\isacharunderscore}{\kern0pt}edge{\isacharunderscore}{\kern0pt}list{\isacharunderscore}{\kern0pt}def\ \isacommand{by}\isamarkupfalse%
\ simp\isanewline
\isacommand{qed}\isamarkupfalse%
%
\endisatagproof
{\isafoldproof}%
%
\isadelimproof
\isanewline
%
\endisadelimproof
\isanewline
\isacommand{end}\isamarkupfalse%
\isanewline
\isanewline
\isacommand{locale}\isamarkupfalse%
\ tree{\isacharunderscore}{\kern0pt}to{\isacharunderscore}{\kern0pt}prufer{\isacharunderscore}{\kern0pt}seq{\isacharunderscore}{\kern0pt}context\ {\isacharequal}{\kern0pt}\isanewline
\ \ \isakeyword{fixes}\ verts\ {\isacharcolon}{\kern0pt}{\isacharcolon}{\kern0pt}\ {\isachardoublequoteopen}{\isacharprime}{\kern0pt}a\ list{\isachardoublequoteclose}\isanewline
\ \ \ \ \isakeyword{and}\ edge{\isacharunderscore}{\kern0pt}list\ {\isacharcolon}{\kern0pt}{\isacharcolon}{\kern0pt}\ {\isachardoublequoteopen}{\isacharparenleft}{\kern0pt}{\isacharprime}{\kern0pt}a\ {\isasymtimes}\ {\isacharprime}{\kern0pt}a{\isacharparenright}{\kern0pt}\ list{\isachardoublequoteclose}\isanewline
\ \ \isakeyword{assumes}\ distinct{\isacharunderscore}{\kern0pt}verts{\isacharcolon}{\kern0pt}\ {\isachardoublequoteopen}distinct\ verts{\isachardoublequoteclose}\isanewline
\ \ \ \ \isakeyword{and}\ card{\isacharunderscore}{\kern0pt}V{\isacharcolon}{\kern0pt}\ {\isachardoublequoteopen}card\ {\isacharparenleft}{\kern0pt}set\ verts{\isacharparenright}{\kern0pt}\ {\isasymge}\ {\isadigit{2}}{\isachardoublequoteclose}\isanewline
\ \ \ \ \isakeyword{and}\ tree{\isacharcolon}{\kern0pt}\ {\isachardoublequoteopen}tree\ {\isacharparenleft}{\kern0pt}set\ verts{\isacharparenright}{\kern0pt}\ {\isacharparenleft}{\kern0pt}edges{\isacharunderscore}{\kern0pt}of{\isacharunderscore}{\kern0pt}edge{\isacharunderscore}{\kern0pt}list\ edge{\isacharunderscore}{\kern0pt}list{\isacharparenright}{\kern0pt}{\isachardoublequoteclose}\isanewline
\ \ \ \ \isakeyword{and}\ distinct{\isacharunderscore}{\kern0pt}edges{\isacharcolon}{\kern0pt}\ {\isachardoublequoteopen}distinct\ {\isacharparenleft}{\kern0pt}map\ mk{\isacharunderscore}{\kern0pt}edge\ edge{\isacharunderscore}{\kern0pt}list{\isacharparenright}{\kern0pt}{\isachardoublequoteclose}\isanewline
\isakeyword{begin}\isanewline
\isanewline
\isacommand{sublocale}\isamarkupfalse%
\ t{\isacharcolon}{\kern0pt}\ tree\ {\isachardoublequoteopen}set\ verts{\isachardoublequoteclose}\ {\isachardoublequoteopen}edges{\isacharunderscore}{\kern0pt}of{\isacharunderscore}{\kern0pt}edge{\isacharunderscore}{\kern0pt}list\ edge{\isacharunderscore}{\kern0pt}list{\isachardoublequoteclose}%
\isadelimproof
\ %
\endisadelimproof
%
\isatagproof
\isacommand{using}\isamarkupfalse%
\ tree\ \isacommand{{\isachardot}{\kern0pt}}\isamarkupfalse%
%
\endisatagproof
{\isafoldproof}%
%
\isadelimproof
%
\endisadelimproof
\isanewline
\isanewline
\isacommand{lemma}\isamarkupfalse%
\ non{\isacharunderscore}{\kern0pt}trivial{\isacharcolon}{\kern0pt}\ {\isachardoublequoteopen}t{\isachardot}{\kern0pt}non{\isacharunderscore}{\kern0pt}trivial{\isachardoublequoteclose}\isanewline
%
\isadelimproof
\ \ %
\endisadelimproof
%
\isatagproof
\isacommand{using}\isamarkupfalse%
\ card{\isacharunderscore}{\kern0pt}V\ \isacommand{unfolding}\isamarkupfalse%
\ t{\isachardot}{\kern0pt}non{\isacharunderscore}{\kern0pt}trivial{\isacharunderscore}{\kern0pt}def\ \isacommand{{\isachardot}{\kern0pt}}\isamarkupfalse%
%
\endisatagproof
{\isafoldproof}%
%
\isadelimproof
\isanewline
%
\endisadelimproof
\isanewline
\isacommand{lemma}\isamarkupfalse%
\ length{\isacharunderscore}{\kern0pt}verts{\isacharcolon}{\kern0pt}\ {\isachardoublequoteopen}length\ verts\ {\isasymge}\ {\isadigit{2}}{\isachardoublequoteclose}\isanewline
%
\isadelimproof
\ \ %
\endisadelimproof
%
\isatagproof
\isacommand{using}\isamarkupfalse%
\ card{\isacharunderscore}{\kern0pt}V\ distinct{\isacharunderscore}{\kern0pt}verts\ distinct{\isacharunderscore}{\kern0pt}card\ \isacommand{by}\isamarkupfalse%
\ fastforce%
\endisatagproof
{\isafoldproof}%
%
\isadelimproof
\isanewline
%
\endisadelimproof
\isanewline
\isacommand{sublocale}\isamarkupfalse%
\ prufer{\isacharunderscore}{\kern0pt}seq{\isacharunderscore}{\kern0pt}to{\isacharunderscore}{\kern0pt}tree{\isacharunderscore}{\kern0pt}context\ verts%
\isadelimproof
\ %
\endisadelimproof
%
\isatagproof
\isacommand{using}\isamarkupfalse%
\ length{\isacharunderscore}{\kern0pt}verts\ distinct{\isacharunderscore}{\kern0pt}verts\ prufer{\isacharunderscore}{\kern0pt}seq{\isacharunderscore}{\kern0pt}to{\isacharunderscore}{\kern0pt}tree{\isacharunderscore}{\kern0pt}context{\isachardot}{\kern0pt}intro\ \isacommand{by}\isamarkupfalse%
\ blast%
\endisatagproof
{\isafoldproof}%
%
\isadelimproof
%
\endisadelimproof
\isanewline
\isanewline
\isacommand{lemma}\isamarkupfalse%
\ edge{\isacharunderscore}{\kern0pt}ne{\isacharcolon}{\kern0pt}\ {\isachardoublequoteopen}{\isacharparenleft}{\kern0pt}u{\isacharcomma}{\kern0pt}v{\isacharparenright}{\kern0pt}\ {\isasymin}\ set\ edge{\isacharunderscore}{\kern0pt}list\ {\isasymLongrightarrow}\ u\ {\isasymnoteq}\ v{\isachardoublequoteclose}\isanewline
%
\isadelimproof
\ \ %
\endisadelimproof
%
\isatagproof
\isacommand{using}\isamarkupfalse%
\ t{\isachardot}{\kern0pt}two{\isacharunderscore}{\kern0pt}edges\ tree\ \isacommand{unfolding}\isamarkupfalse%
\ edges{\isacharunderscore}{\kern0pt}of{\isacharunderscore}{\kern0pt}edge{\isacharunderscore}{\kern0pt}list{\isacharunderscore}{\kern0pt}def\ \isacommand{by}\isamarkupfalse%
\ fastforce%
\endisatagproof
{\isafoldproof}%
%
\isadelimproof
\isanewline
%
\endisadelimproof
\isanewline
\isacommand{lemma}\isamarkupfalse%
\ distinct{\isacharunderscore}{\kern0pt}edge{\isacharunderscore}{\kern0pt}list{\isacharcolon}{\kern0pt}\ {\isachardoublequoteopen}distinct\ edge{\isacharunderscore}{\kern0pt}list{\isachardoublequoteclose}\isanewline
%
\isadelimproof
\ \ %
\endisadelimproof
%
\isatagproof
\isacommand{using}\isamarkupfalse%
\ distinct{\isacharunderscore}{\kern0pt}edges\ \isacommand{by}\isamarkupfalse%
\ {\isacharparenleft}{\kern0pt}simp\ add{\isacharcolon}{\kern0pt}\ distinct{\isacharunderscore}{\kern0pt}map{\isacharparenright}{\kern0pt}%
\endisatagproof
{\isafoldproof}%
%
\isadelimproof
\isanewline
%
\endisadelimproof
\isanewline
\isacommand{lemma}\isamarkupfalse%
\ length{\isacharunderscore}{\kern0pt}varts{\isacharunderscore}{\kern0pt}edge{\isacharunderscore}{\kern0pt}list{\isacharcolon}{\kern0pt}\ {\isachardoublequoteopen}length\ verts\ {\isacharequal}{\kern0pt}\ Suc\ {\isacharparenleft}{\kern0pt}length\ edge{\isacharunderscore}{\kern0pt}list{\isacharparenright}{\kern0pt}{\isachardoublequoteclose}\isanewline
%
\isadelimproof
\ \ %
\endisadelimproof
%
\isatagproof
\isacommand{using}\isamarkupfalse%
\ distinct{\isacharunderscore}{\kern0pt}verts\ t{\isachardot}{\kern0pt}card{\isacharunderscore}{\kern0pt}V{\isacharunderscore}{\kern0pt}card{\isacharunderscore}{\kern0pt}E\ distinct{\isacharunderscore}{\kern0pt}card\ distinct{\isacharunderscore}{\kern0pt}edges\ edges{\isacharunderscore}{\kern0pt}of{\isacharunderscore}{\kern0pt}edge{\isacharunderscore}{\kern0pt}list{\isacharunderscore}{\kern0pt}def\ length{\isacharunderscore}{\kern0pt}map\ list{\isachardot}{\kern0pt}set{\isacharunderscore}{\kern0pt}map\ \isacommand{by}\isamarkupfalse%
\ metis%
\endisatagproof
{\isafoldproof}%
%
\isadelimproof
\isanewline
%
\endisadelimproof
\isanewline
\isacommand{lemma}\isamarkupfalse%
\ incident{\isacharunderscore}{\kern0pt}edges{\isacharunderscore}{\kern0pt}correct{\isacharcolon}{\kern0pt}\ {\isachardoublequoteopen}edges{\isacharunderscore}{\kern0pt}of{\isacharunderscore}{\kern0pt}edge{\isacharunderscore}{\kern0pt}list\ {\isacharparenleft}{\kern0pt}incident{\isacharunderscore}{\kern0pt}edges\ v\ edge{\isacharunderscore}{\kern0pt}list{\isacharparenright}{\kern0pt}\ {\isacharequal}{\kern0pt}\ t{\isachardot}{\kern0pt}incident{\isacharunderscore}{\kern0pt}edges\ v{\isachardoublequoteclose}\isanewline
%
\isadelimproof
\ \ %
\endisadelimproof
%
\isatagproof
\isacommand{unfolding}\isamarkupfalse%
\ t{\isachardot}{\kern0pt}incident{\isacharunderscore}{\kern0pt}edges{\isacharunderscore}{\kern0pt}def\ t{\isachardot}{\kern0pt}incident{\isacharunderscore}{\kern0pt}def\ \isacommand{by}\isamarkupfalse%
\ {\isacharparenleft}{\kern0pt}auto\ simp{\isacharcolon}{\kern0pt}\ edges{\isacharunderscore}{\kern0pt}of{\isacharunderscore}{\kern0pt}edge{\isacharunderscore}{\kern0pt}list{\isacharunderscore}{\kern0pt}def\ incident{\isacharunderscore}{\kern0pt}edges{\isacharunderscore}{\kern0pt}def{\isacharparenright}{\kern0pt}%
\endisatagproof
{\isafoldproof}%
%
\isadelimproof
\isanewline
%
\endisadelimproof
\isanewline
\isacommand{lemma}\isamarkupfalse%
\ degree{\isacharunderscore}{\kern0pt}correct{\isacharcolon}{\kern0pt}\ {\isachardoublequoteopen}degree\ v\ edge{\isacharunderscore}{\kern0pt}list\ {\isacharequal}{\kern0pt}\ t{\isachardot}{\kern0pt}degree\ v{\isachardoublequoteclose}\isanewline
%
\isadelimproof
%
\endisadelimproof
%
\isatagproof
\isacommand{proof}\isamarkupfalse%
{\isacharminus}{\kern0pt}\isanewline
\ \ \isacommand{have}\isamarkupfalse%
\ distinct{\isacharunderscore}{\kern0pt}incident{\isacharunderscore}{\kern0pt}edges{\isacharcolon}{\kern0pt}\ {\isachardoublequoteopen}distinct\ {\isacharparenleft}{\kern0pt}map\ mk{\isacharunderscore}{\kern0pt}edge\ {\isacharparenleft}{\kern0pt}incident{\isacharunderscore}{\kern0pt}edges\ v\ edge{\isacharunderscore}{\kern0pt}list{\isacharparenright}{\kern0pt}{\isacharparenright}{\kern0pt}{\isachardoublequoteclose}\ \isacommand{unfolding}\isamarkupfalse%
\ incident{\isacharunderscore}{\kern0pt}edges{\isacharunderscore}{\kern0pt}def\ \isacommand{using}\isamarkupfalse%
\ distinct{\isacharunderscore}{\kern0pt}map{\isacharunderscore}{\kern0pt}filter\ distinct{\isacharunderscore}{\kern0pt}edges\ \isacommand{by}\isamarkupfalse%
\ blast\isanewline
\ \ \isacommand{have}\isamarkupfalse%
\ {\isachardoublequoteopen}degree\ v\ edge{\isacharunderscore}{\kern0pt}list\ {\isacharequal}{\kern0pt}\ length\ {\isacharparenleft}{\kern0pt}map\ mk{\isacharunderscore}{\kern0pt}edge\ {\isacharparenleft}{\kern0pt}incident{\isacharunderscore}{\kern0pt}edges\ v\ edge{\isacharunderscore}{\kern0pt}list{\isacharparenright}{\kern0pt}{\isacharparenright}{\kern0pt}{\isachardoublequoteclose}\ \isacommand{using}\isamarkupfalse%
\ distinct{\isacharunderscore}{\kern0pt}edges\ \isacommand{by}\isamarkupfalse%
\ simp\isanewline
\ \ \isacommand{also}\isamarkupfalse%
\ \isacommand{have}\isamarkupfalse%
\ {\isachardoublequoteopen}{\isasymdots}\ {\isacharequal}{\kern0pt}\ card\ {\isacharparenleft}{\kern0pt}edges{\isacharunderscore}{\kern0pt}of{\isacharunderscore}{\kern0pt}edge{\isacharunderscore}{\kern0pt}list\ {\isacharparenleft}{\kern0pt}incident{\isacharunderscore}{\kern0pt}edges\ v\ edge{\isacharunderscore}{\kern0pt}list{\isacharparenright}{\kern0pt}{\isacharparenright}{\kern0pt}{\isachardoublequoteclose}\ \isacommand{unfolding}\isamarkupfalse%
\ edges{\isacharunderscore}{\kern0pt}of{\isacharunderscore}{\kern0pt}edge{\isacharunderscore}{\kern0pt}list{\isacharunderscore}{\kern0pt}def\ \isacommand{using}\isamarkupfalse%
\ distinct{\isacharunderscore}{\kern0pt}incident{\isacharunderscore}{\kern0pt}edges\ distinct{\isacharunderscore}{\kern0pt}card\ \isacommand{by}\isamarkupfalse%
\ fastforce\isanewline
\ \ \isacommand{also}\isamarkupfalse%
\ \isacommand{have}\isamarkupfalse%
\ {\isachardoublequoteopen}{\isasymdots}\ {\isacharequal}{\kern0pt}\ card\ {\isacharparenleft}{\kern0pt}t{\isachardot}{\kern0pt}incident{\isacharunderscore}{\kern0pt}edges\ v{\isacharparenright}{\kern0pt}{\isachardoublequoteclose}\ \isacommand{using}\isamarkupfalse%
\ incident{\isacharunderscore}{\kern0pt}edges{\isacharunderscore}{\kern0pt}correct\ \isacommand{by}\isamarkupfalse%
\ simp\isanewline
\ \ \isacommand{finally}\isamarkupfalse%
\ \isacommand{show}\isamarkupfalse%
\ {\isacharquery}{\kern0pt}thesis\ \isacommand{by}\isamarkupfalse%
\ simp\isanewline
\isacommand{qed}\isamarkupfalse%
%
\endisatagproof
{\isafoldproof}%
%
\isadelimproof
\isanewline
%
\endisadelimproof
\isanewline
\isacommand{lemma}\isamarkupfalse%
\ obtain{\isacharunderscore}{\kern0pt}leaf{\isacharunderscore}{\kern0pt}tree{\isacharunderscore}{\kern0pt}to{\isacharunderscore}{\kern0pt}prufer{\isacharunderscore}{\kern0pt}seq{\isacharcolon}{\kern0pt}\isanewline
\ \ \isakeyword{assumes}\ length{\isacharunderscore}{\kern0pt}edge{\isacharunderscore}{\kern0pt}list{\isacharcolon}{\kern0pt}\ {\isachardoublequoteopen}length\ edge{\isacharunderscore}{\kern0pt}list\ {\isasymge}\ {\isadigit{2}}{\isachardoublequoteclose}\isanewline
\ \ \isakeyword{obtains}\ leaf\isanewline
\ \ \isakeyword{where}\ {\isachardoublequoteopen}find\ {\isacharparenleft}{\kern0pt}{\isasymlambda}v{\isachardot}{\kern0pt}\ degree\ v\ edge{\isacharunderscore}{\kern0pt}list\ {\isacharequal}{\kern0pt}\ {\isadigit{1}}{\isacharparenright}{\kern0pt}\ verts\ {\isacharequal}{\kern0pt}\ Some\ leaf{\isachardoublequoteclose}\isanewline
\ \ \ \ \isakeyword{and}\ {\isachardoublequoteopen}t{\isachardot}{\kern0pt}leaf\ leaf{\isachardoublequoteclose}\isanewline
\ \ \ \ \isakeyword{and}\ {\isachardoublequoteopen}leaf\ {\isasymin}\ set\ verts{\isachardoublequoteclose}\isanewline
\ \ \ \ \isakeyword{and}\ {\isachardoublequoteopen}tree{\isacharunderscore}{\kern0pt}to{\isacharunderscore}{\kern0pt}prufer{\isacharunderscore}{\kern0pt}seq{\isacharunderscore}{\kern0pt}context\ {\isacharparenleft}{\kern0pt}remove{\isadigit{1}}\ leaf\ verts{\isacharparenright}{\kern0pt}\ {\isacharparenleft}{\kern0pt}remove{\isacharunderscore}{\kern0pt}vertex\ leaf\ edge{\isacharunderscore}{\kern0pt}list{\isacharparenright}{\kern0pt}{\isachardoublequoteclose}\isanewline
%
\isadelimproof
%
\endisadelimproof
%
\isatagproof
\isacommand{proof}\isamarkupfalse%
{\isacharminus}{\kern0pt}\isanewline
\ \ \isacommand{obtain}\isamarkupfalse%
\ leaf\ \isakeyword{where}\ leaf{\isacharunderscore}{\kern0pt}find{\isacharcolon}{\kern0pt}\ {\isachardoublequoteopen}find\ {\isacharparenleft}{\kern0pt}{\isasymlambda}v{\isachardot}{\kern0pt}\ degree\ v\ edge{\isacharunderscore}{\kern0pt}list\ {\isacharequal}{\kern0pt}\ {\isadigit{1}}{\isacharparenright}{\kern0pt}\ verts\ {\isacharequal}{\kern0pt}\ Some\ leaf{\isachardoublequoteclose}\isanewline
\ \ \ \ \isacommand{using}\isamarkupfalse%
\ find{\isacharunderscore}{\kern0pt}None{\isacharunderscore}{\kern0pt}iff{\isadigit{2}}\ t{\isachardot}{\kern0pt}leaf{\isacharunderscore}{\kern0pt}in{\isacharunderscore}{\kern0pt}V\ degree{\isacharunderscore}{\kern0pt}correct\ t{\isachardot}{\kern0pt}leaf{\isacharunderscore}{\kern0pt}def\ t{\isachardot}{\kern0pt}exists{\isacharunderscore}{\kern0pt}leaf\ non{\isacharunderscore}{\kern0pt}trivial\ \isacommand{by}\isamarkupfalse%
\ fastforce\isanewline
\ \ \isacommand{then}\isamarkupfalse%
\ \isacommand{have}\isamarkupfalse%
\ {\isachardoublequoteopen}degree\ leaf\ edge{\isacharunderscore}{\kern0pt}list\ {\isacharequal}{\kern0pt}\ {\isadigit{1}}{\isachardoublequoteclose}\isanewline
\ \ \ \ \isacommand{by}\isamarkupfalse%
\ {\isacharparenleft}{\kern0pt}metis\ {\isacharparenleft}{\kern0pt}mono{\isacharunderscore}{\kern0pt}tags{\isacharcomma}{\kern0pt}\ lifting{\isacharparenright}{\kern0pt}\ find{\isacharunderscore}{\kern0pt}Some{\isacharunderscore}{\kern0pt}iff{\isacharparenright}{\kern0pt}\isanewline
\ \ \isacommand{then}\isamarkupfalse%
\ \isacommand{have}\isamarkupfalse%
\ leaf{\isacharcolon}{\kern0pt}\ {\isachardoublequoteopen}t{\isachardot}{\kern0pt}leaf\ leaf{\isachardoublequoteclose}\ \isacommand{using}\isamarkupfalse%
\ degree{\isacharunderscore}{\kern0pt}correct\ t{\isachardot}{\kern0pt}leaf{\isacharunderscore}{\kern0pt}def\ \isacommand{by}\isamarkupfalse%
\ auto\isanewline
\ \ \isacommand{have}\isamarkupfalse%
\ in{\isacharunderscore}{\kern0pt}verts{\isacharcolon}{\kern0pt}\ {\isachardoublequoteopen}leaf\ {\isasymin}\ set\ verts{\isachardoublequoteclose}\ \isacommand{by}\isamarkupfalse%
\ {\isacharparenleft}{\kern0pt}simp\ add{\isacharcolon}{\kern0pt}\ leaf\ t{\isachardot}{\kern0pt}leaf{\isacharunderscore}{\kern0pt}in{\isacharunderscore}{\kern0pt}V{\isacharparenright}{\kern0pt}\isanewline
\ \ \isacommand{let}\isamarkupfalse%
\ {\isacharquery}{\kern0pt}verts{\isacharprime}{\kern0pt}\ {\isacharequal}{\kern0pt}\ {\isachardoublequoteopen}remove{\isadigit{1}}\ leaf\ verts{\isachardoublequoteclose}\isanewline
\ \ \isacommand{let}\isamarkupfalse%
\ {\isacharquery}{\kern0pt}edge{\isacharunderscore}{\kern0pt}list{\isacharprime}{\kern0pt}\ {\isacharequal}{\kern0pt}\ {\isachardoublequoteopen}remove{\isacharunderscore}{\kern0pt}vertex\ leaf\ edge{\isacharunderscore}{\kern0pt}list{\isachardoublequoteclose}\isanewline
\ \ \isacommand{have}\isamarkupfalse%
\ distinct{\isacharunderscore}{\kern0pt}verts{\isacharprime}{\kern0pt}{\isacharcolon}{\kern0pt}\ {\isachardoublequoteopen}distinct\ {\isacharquery}{\kern0pt}verts{\isacharprime}{\kern0pt}{\isachardoublequoteclose}\ \isacommand{using}\isamarkupfalse%
\ distinct{\isacharunderscore}{\kern0pt}verts\ distinct{\isacharunderscore}{\kern0pt}remove{\isadigit{1}}\ \isacommand{by}\isamarkupfalse%
\ auto\isanewline
\ \ \isacommand{have}\isamarkupfalse%
\ {\isachardoublequoteopen}card\ {\isacharparenleft}{\kern0pt}edges{\isacharunderscore}{\kern0pt}of{\isacharunderscore}{\kern0pt}edge{\isacharunderscore}{\kern0pt}list\ edge{\isacharunderscore}{\kern0pt}list{\isacharparenright}{\kern0pt}\ {\isasymge}\ {\isadigit{2}}{\isachardoublequoteclose}\ \isacommand{unfolding}\isamarkupfalse%
\ edges{\isacharunderscore}{\kern0pt}of{\isacharunderscore}{\kern0pt}edge{\isacharunderscore}{\kern0pt}list{\isacharunderscore}{\kern0pt}def\ \isacommand{using}\isamarkupfalse%
\ length{\isacharunderscore}{\kern0pt}edge{\isacharunderscore}{\kern0pt}list\ distinct{\isacharunderscore}{\kern0pt}edges\ distinct{\isacharunderscore}{\kern0pt}card\ \isacommand{by}\isamarkupfalse%
\ fastforce\isanewline
\ \ \isacommand{then}\isamarkupfalse%
\ \isacommand{have}\isamarkupfalse%
\ {\isachardoublequoteopen}card\ {\isacharparenleft}{\kern0pt}set\ verts{\isacharparenright}{\kern0pt}\ {\isasymge}\ {\isadigit{3}}{\isachardoublequoteclose}\ \isacommand{using}\isamarkupfalse%
\ t{\isachardot}{\kern0pt}card{\isacharunderscore}{\kern0pt}V{\isacharunderscore}{\kern0pt}card{\isacharunderscore}{\kern0pt}E\ \isacommand{by}\isamarkupfalse%
\ simp\isanewline
\ \ \isacommand{then}\isamarkupfalse%
\ \isacommand{have}\isamarkupfalse%
\ card{\isacharunderscore}{\kern0pt}verts{\isacharprime}{\kern0pt}{\isacharcolon}{\kern0pt}\ {\isachardoublequoteopen}card\ {\isacharparenleft}{\kern0pt}set\ {\isacharquery}{\kern0pt}verts{\isacharprime}{\kern0pt}{\isacharparenright}{\kern0pt}\ {\isasymge}\ {\isadigit{2}}{\isachardoublequoteclose}\ \isacommand{by}\isamarkupfalse%
\ {\isacharparenleft}{\kern0pt}simp\ add{\isacharcolon}{\kern0pt}\ distinct{\isacharunderscore}{\kern0pt}verts\ in{\isacharunderscore}{\kern0pt}verts{\isacharparenright}{\kern0pt}\isanewline
\ \ \isacommand{then}\isamarkupfalse%
\ \isacommand{interpret}\isamarkupfalse%
\ t{\isacharprime}{\kern0pt}{\isacharcolon}{\kern0pt}\ tree\ {\isachardoublequoteopen}set\ {\isacharquery}{\kern0pt}verts{\isacharprime}{\kern0pt}{\isachardoublequoteclose}\ {\isachardoublequoteopen}edges{\isacharunderscore}{\kern0pt}of{\isacharunderscore}{\kern0pt}edge{\isacharunderscore}{\kern0pt}list\ {\isacharquery}{\kern0pt}edge{\isacharunderscore}{\kern0pt}list{\isacharprime}{\kern0pt}{\isachardoublequoteclose}\isanewline
\ \ \ \ \isacommand{using}\isamarkupfalse%
\ t{\isachardot}{\kern0pt}tree{\isacharunderscore}{\kern0pt}remove{\isacharunderscore}{\kern0pt}leaf\ leaf\ tree\ distinct{\isacharunderscore}{\kern0pt}verts\ \isacommand{by}\isamarkupfalse%
\ {\isacharparenleft}{\kern0pt}auto\ simp{\isacharcolon}{\kern0pt}\ remove{\isacharunderscore}{\kern0pt}vertex\ t{\isachardot}{\kern0pt}remove{\isacharunderscore}{\kern0pt}vertex{\isacharunderscore}{\kern0pt}def\ t{\isachardot}{\kern0pt}incident{\isacharunderscore}{\kern0pt}def{\isacharparenright}{\kern0pt}\isanewline
\ \ \isacommand{have}\isamarkupfalse%
\ distinct{\isacharunderscore}{\kern0pt}edges{\isacharprime}{\kern0pt}{\isacharcolon}{\kern0pt}\ {\isachardoublequoteopen}distinct\ {\isacharparenleft}{\kern0pt}map\ mk{\isacharunderscore}{\kern0pt}edge\ {\isacharquery}{\kern0pt}edge{\isacharunderscore}{\kern0pt}list{\isacharprime}{\kern0pt}{\isacharparenright}{\kern0pt}{\isachardoublequoteclose}\ \isacommand{using}\isamarkupfalse%
\ distinct{\isacharunderscore}{\kern0pt}edges\ distinct{\isacharunderscore}{\kern0pt}remove{\isacharunderscore}{\kern0pt}vertex\ \isacommand{by}\isamarkupfalse%
\ simp\isanewline
\ \ \isacommand{then}\isamarkupfalse%
\ \isacommand{have}\isamarkupfalse%
\ {\isachardoublequoteopen}tree{\isacharunderscore}{\kern0pt}to{\isacharunderscore}{\kern0pt}prufer{\isacharunderscore}{\kern0pt}seq{\isacharunderscore}{\kern0pt}context\ {\isacharquery}{\kern0pt}verts{\isacharprime}{\kern0pt}\ {\isacharquery}{\kern0pt}edge{\isacharunderscore}{\kern0pt}list{\isacharprime}{\kern0pt}{\isachardoublequoteclose}\ \isacommand{using}\isamarkupfalse%
\ distinct{\isacharunderscore}{\kern0pt}verts{\isacharprime}{\kern0pt}\ card{\isacharunderscore}{\kern0pt}verts{\isacharprime}{\kern0pt}\ \isacommand{by}\isamarkupfalse%
\ {\isacharparenleft}{\kern0pt}unfold{\isacharunderscore}{\kern0pt}locales{\isacharcomma}{\kern0pt}\ auto{\isacharparenright}{\kern0pt}\isanewline
\ \ \isacommand{then}\isamarkupfalse%
\ \isacommand{show}\isamarkupfalse%
\ {\isacharquery}{\kern0pt}thesis\ \isacommand{using}\isamarkupfalse%
\ that\ leaf{\isacharunderscore}{\kern0pt}find\ leaf\ in{\isacharunderscore}{\kern0pt}verts\ \isacommand{by}\isamarkupfalse%
\ auto\isanewline
\isacommand{qed}\isamarkupfalse%
%
\endisatagproof
{\isafoldproof}%
%
\isadelimproof
\isanewline
%
\endisadelimproof
\isanewline
\isacommand{lemma}\isamarkupfalse%
\ length{\isacharunderscore}{\kern0pt}edge{\isacharunderscore}{\kern0pt}list{\isacharcolon}{\kern0pt}\ {\isachardoublequoteopen}length\ edge{\isacharunderscore}{\kern0pt}list\ {\isasymge}\ {\isadigit{1}}{\isachardoublequoteclose}\isanewline
%
\isadelimproof
%
\endisadelimproof
%
\isatagproof
\isacommand{proof}\isamarkupfalse%
{\isacharminus}{\kern0pt}\isanewline
\ \ \isacommand{have}\isamarkupfalse%
\ {\isachardoublequoteopen}length\ edge{\isacharunderscore}{\kern0pt}list\ {\isacharequal}{\kern0pt}\ card\ {\isacharparenleft}{\kern0pt}edges{\isacharunderscore}{\kern0pt}of{\isacharunderscore}{\kern0pt}edge{\isacharunderscore}{\kern0pt}list\ edge{\isacharunderscore}{\kern0pt}list{\isacharparenright}{\kern0pt}{\isachardoublequoteclose}\ \isacommand{unfolding}\isamarkupfalse%
\ edges{\isacharunderscore}{\kern0pt}of{\isacharunderscore}{\kern0pt}edge{\isacharunderscore}{\kern0pt}list{\isacharunderscore}{\kern0pt}def\ \isacommand{using}\isamarkupfalse%
\ distinct{\isacharunderscore}{\kern0pt}edges\ distinct{\isacharunderscore}{\kern0pt}card\ \isacommand{by}\isamarkupfalse%
\ force\isanewline
\ \ \isacommand{then}\isamarkupfalse%
\ \isacommand{show}\isamarkupfalse%
\ {\isacharquery}{\kern0pt}thesis\ \isacommand{using}\isamarkupfalse%
\ t{\isachardot}{\kern0pt}card{\isacharunderscore}{\kern0pt}V{\isacharunderscore}{\kern0pt}card{\isacharunderscore}{\kern0pt}E\ length{\isacharunderscore}{\kern0pt}verts\ distinct{\isacharunderscore}{\kern0pt}verts\ distinct{\isacharunderscore}{\kern0pt}card\ \isacommand{by}\isamarkupfalse%
\ fastforce\isanewline
\isacommand{qed}\isamarkupfalse%
%
\endisatagproof
{\isafoldproof}%
%
\isadelimproof
\isanewline
%
\endisadelimproof
\isanewline
\isacommand{lemma}\isamarkupfalse%
\ pruf{\isacharunderscore}{\kern0pt}seq{\isacharunderscore}{\kern0pt}tree{\isacharunderscore}{\kern0pt}to{\isacharunderscore}{\kern0pt}prufer{\isacharunderscore}{\kern0pt}seq{\isacharcolon}{\kern0pt}\ {\isachardoublequoteopen}tree{\isacharunderscore}{\kern0pt}to{\isacharunderscore}{\kern0pt}prufer{\isacharunderscore}{\kern0pt}seq\ verts\ edge{\isacharunderscore}{\kern0pt}list\ {\isasymin}\ prufer{\isacharunderscore}{\kern0pt}sequences\ verts{\isachardoublequoteclose}\isanewline
%
\isadelimproof
\ \ %
\endisadelimproof
%
\isatagproof
\isacommand{using}\isamarkupfalse%
\ tree{\isacharunderscore}{\kern0pt}to{\isacharunderscore}{\kern0pt}prufer{\isacharunderscore}{\kern0pt}seq{\isacharunderscore}{\kern0pt}context{\isacharunderscore}{\kern0pt}axioms\isanewline
\isacommand{proof}\isamarkupfalse%
\ {\isacharparenleft}{\kern0pt}induction\ verts\ edge{\isacharunderscore}{\kern0pt}list\ rule{\isacharcolon}{\kern0pt}\ tree{\isacharunderscore}{\kern0pt}to{\isacharunderscore}{\kern0pt}prufer{\isacharunderscore}{\kern0pt}seq{\isachardot}{\kern0pt}induct{\isacharparenright}{\kern0pt}\isanewline
\ \ \isacommand{case}\isamarkupfalse%
\ {\isacharparenleft}{\kern0pt}{\isadigit{1}}\ verts{\isacharparenright}{\kern0pt}\isanewline
\ \ \isacommand{then}\isamarkupfalse%
\ \isacommand{interpret}\isamarkupfalse%
\ contxt{\isacharcolon}{\kern0pt}\ tree{\isacharunderscore}{\kern0pt}to{\isacharunderscore}{\kern0pt}prufer{\isacharunderscore}{\kern0pt}seq{\isacharunderscore}{\kern0pt}context\ verts\ {\isachardoublequoteopen}{\isacharbrackleft}{\kern0pt}{\isacharbrackright}{\kern0pt}{\isachardoublequoteclose}\isanewline
\ \ \ \ \isacommand{using}\isamarkupfalse%
\ tree{\isacharunderscore}{\kern0pt}to{\isacharunderscore}{\kern0pt}prufer{\isacharunderscore}{\kern0pt}seq{\isacharunderscore}{\kern0pt}context{\isachardot}{\kern0pt}intro\ \isacommand{by}\isamarkupfalse%
\ blast\isanewline
\ \ \isacommand{show}\isamarkupfalse%
\ {\isacharquery}{\kern0pt}case\ \isacommand{using}\isamarkupfalse%
\ contxt{\isachardot}{\kern0pt}length{\isacharunderscore}{\kern0pt}edge{\isacharunderscore}{\kern0pt}list\ \isacommand{by}\isamarkupfalse%
\ auto\isanewline
\isacommand{next}\isamarkupfalse%
\isanewline
\ \ \isacommand{case}\isamarkupfalse%
\ {\isacharparenleft}{\kern0pt}{\isadigit{2}}\ verts\ u\ w{\isacharparenright}{\kern0pt}\isanewline
\ \ \isacommand{then}\isamarkupfalse%
\ \isacommand{interpret}\isamarkupfalse%
\ contxt{\isacharcolon}{\kern0pt}\ tree{\isacharunderscore}{\kern0pt}to{\isacharunderscore}{\kern0pt}prufer{\isacharunderscore}{\kern0pt}seq{\isacharunderscore}{\kern0pt}context\ verts\ {\isachardoublequoteopen}{\isacharbrackleft}{\kern0pt}{\isacharparenleft}{\kern0pt}u{\isacharcomma}{\kern0pt}w{\isacharparenright}{\kern0pt}{\isacharbrackright}{\kern0pt}{\isachardoublequoteclose}\isanewline
\ \ \ \ \isacommand{using}\isamarkupfalse%
\ tree{\isacharunderscore}{\kern0pt}to{\isacharunderscore}{\kern0pt}prufer{\isacharunderscore}{\kern0pt}seq{\isacharunderscore}{\kern0pt}context{\isachardot}{\kern0pt}intro\ \isacommand{by}\isamarkupfalse%
\ blast\isanewline
\ \ \isacommand{show}\isamarkupfalse%
\ {\isacharquery}{\kern0pt}case\ \isacommand{using}\isamarkupfalse%
\ contxt{\isachardot}{\kern0pt}length{\isacharunderscore}{\kern0pt}varts{\isacharunderscore}{\kern0pt}edge{\isacharunderscore}{\kern0pt}list\ \isacommand{unfolding}\isamarkupfalse%
\ prufer{\isacharunderscore}{\kern0pt}sequences{\isacharunderscore}{\kern0pt}def\ n{\isacharunderscore}{\kern0pt}sequences{\isacharunderscore}{\kern0pt}def\ \isacommand{by}\isamarkupfalse%
\ auto\isanewline
\isacommand{next}\isamarkupfalse%
\isanewline
\ \ \isacommand{case}\isamarkupfalse%
\ {\isacharparenleft}{\kern0pt}{\isadigit{3}}\ verts\ e{\isadigit{1}}\ e{\isadigit{2}}\ edges{\isacharparenright}{\kern0pt}\isanewline
\ \ \isacommand{let}\isamarkupfalse%
\ {\isacharquery}{\kern0pt}edge{\isacharunderscore}{\kern0pt}list\ {\isacharequal}{\kern0pt}\ {\isachardoublequoteopen}e{\isadigit{1}}{\isacharhash}{\kern0pt}e{\isadigit{2}}{\isacharhash}{\kern0pt}edges{\isachardoublequoteclose}\isanewline
\ \ \isacommand{interpret}\isamarkupfalse%
\ contxt{\isacharcolon}{\kern0pt}\ tree{\isacharunderscore}{\kern0pt}to{\isacharunderscore}{\kern0pt}prufer{\isacharunderscore}{\kern0pt}seq{\isacharunderscore}{\kern0pt}context\ verts\ {\isacharquery}{\kern0pt}edge{\isacharunderscore}{\kern0pt}list\isanewline
\ \ \ \ \isacommand{using}\isamarkupfalse%
\ tree{\isacharunderscore}{\kern0pt}to{\isacharunderscore}{\kern0pt}prufer{\isacharunderscore}{\kern0pt}seq{\isacharunderscore}{\kern0pt}context{\isachardot}{\kern0pt}intro\ {\isadigit{3}}\ \isacommand{by}\isamarkupfalse%
\ blast\isanewline
\ \ \isacommand{have}\isamarkupfalse%
\ length{\isacharunderscore}{\kern0pt}edge{\isacharunderscore}{\kern0pt}list{\isacharcolon}{\kern0pt}\ {\isachardoublequoteopen}length\ {\isacharquery}{\kern0pt}edge{\isacharunderscore}{\kern0pt}list\ {\isasymge}\ {\isadigit{2}}{\isachardoublequoteclose}\ \isacommand{by}\isamarkupfalse%
\ simp\isanewline
\ \ \isacommand{then}\isamarkupfalse%
\ \isacommand{obtain}\isamarkupfalse%
\ leaf\isanewline
\ \ \ \ \isakeyword{where}\ find{\isacharunderscore}{\kern0pt}leaf{\isacharcolon}{\kern0pt}\ {\isachardoublequoteopen}find\ {\isacharparenleft}{\kern0pt}{\isasymlambda}v{\isachardot}{\kern0pt}\ degree\ v\ {\isacharquery}{\kern0pt}edge{\isacharunderscore}{\kern0pt}list\ {\isacharequal}{\kern0pt}\ {\isadigit{1}}{\isacharparenright}{\kern0pt}\ verts\ {\isacharequal}{\kern0pt}\ Some\ leaf{\isachardoublequoteclose}\isanewline
\ \ \ \ \ \ \isakeyword{and}\ contxt{\isacharprime}{\kern0pt}{\isacharcolon}{\kern0pt}\ {\isachardoublequoteopen}tree{\isacharunderscore}{\kern0pt}to{\isacharunderscore}{\kern0pt}prufer{\isacharunderscore}{\kern0pt}seq{\isacharunderscore}{\kern0pt}context\ {\isacharparenleft}{\kern0pt}remove{\isadigit{1}}\ leaf\ verts{\isacharparenright}{\kern0pt}\ {\isacharparenleft}{\kern0pt}remove{\isacharunderscore}{\kern0pt}vertex\ leaf\ {\isacharquery}{\kern0pt}edge{\isacharunderscore}{\kern0pt}list{\isacharparenright}{\kern0pt}{\isachardoublequoteclose}\isanewline
\ \ \ \ \isacommand{using}\isamarkupfalse%
\ contxt{\isachardot}{\kern0pt}obtain{\isacharunderscore}{\kern0pt}leaf{\isacharunderscore}{\kern0pt}tree{\isacharunderscore}{\kern0pt}to{\isacharunderscore}{\kern0pt}prufer{\isacharunderscore}{\kern0pt}seq\ {\isadigit{3}}\ \isacommand{by}\isamarkupfalse%
\ blast\isanewline
\ \ \isacommand{then}\isamarkupfalse%
\ \isacommand{interpret}\isamarkupfalse%
\ contxt{\isacharprime}{\kern0pt}{\isacharcolon}{\kern0pt}\ tree{\isacharunderscore}{\kern0pt}to{\isacharunderscore}{\kern0pt}prufer{\isacharunderscore}{\kern0pt}seq{\isacharunderscore}{\kern0pt}context\ {\isachardoublequoteopen}remove{\isadigit{1}}\ leaf\ verts{\isachardoublequoteclose}\ {\isachardoublequoteopen}remove{\isacharunderscore}{\kern0pt}vertex\ leaf\ {\isacharquery}{\kern0pt}edge{\isacharunderscore}{\kern0pt}list{\isachardoublequoteclose}\ \isacommand{by}\isamarkupfalse%
\ simp\isanewline
\isanewline
\ \ \isacommand{let}\isamarkupfalse%
\ {\isacharquery}{\kern0pt}neigh\ {\isacharequal}{\kern0pt}\ {\isachardoublequoteopen}neighbor\ leaf\ {\isacharquery}{\kern0pt}edge{\isacharunderscore}{\kern0pt}list{\isachardoublequoteclose}\isanewline
\ \ \isacommand{have}\isamarkupfalse%
\ degree{\isacharcolon}{\kern0pt}\ {\isachardoublequoteopen}degree\ leaf\ {\isacharquery}{\kern0pt}edge{\isacharunderscore}{\kern0pt}list\ {\isasymge}\ {\isadigit{1}}{\isachardoublequoteclose}\ \isacommand{using}\isamarkupfalse%
\ find{\isacharunderscore}{\kern0pt}Some\ find{\isacharunderscore}{\kern0pt}leaf\ \isacommand{by}\isamarkupfalse%
\ fastforce\isanewline
\ \ \isacommand{have}\isamarkupfalse%
\ {\isachardoublequoteopen}{\isacharquery}{\kern0pt}neigh\ {\isasymin}\ set\ verts{\isachardoublequoteclose}\ \isacommand{using}\isamarkupfalse%
\ neighbor{\isacharunderscore}{\kern0pt}edge{\isacharunderscore}{\kern0pt}in{\isacharunderscore}{\kern0pt}edges{\isacharbrackleft}{\kern0pt}OF\ degree{\isacharbrackright}{\kern0pt}\ contxt{\isachardot}{\kern0pt}t{\isachardot}{\kern0pt}wellformed{\isacharunderscore}{\kern0pt}alt{\isacharunderscore}{\kern0pt}fst\ \isacommand{by}\isamarkupfalse%
\ blast\isanewline
\ \ \isacommand{then}\isamarkupfalse%
\ \isacommand{show}\isamarkupfalse%
\ {\isacharquery}{\kern0pt}case\ \isacommand{using}\isamarkupfalse%
\ find{\isacharunderscore}{\kern0pt}leaf\ {\isachardoublequoteopen}{\isadigit{3}}{\isachardot}{\kern0pt}IH{\isachardoublequoteclose}\ contxt{\isacharprime}{\kern0pt}\ \isacommand{unfolding}\isamarkupfalse%
\ prufer{\isacharunderscore}{\kern0pt}sequences{\isacharunderscore}{\kern0pt}def\ n{\isacharunderscore}{\kern0pt}sequences{\isacharunderscore}{\kern0pt}def\isanewline
\ \ \ \ \isacommand{apply}\isamarkupfalse%
\ auto\isanewline
\ \ \ \ \isacommand{apply}\isamarkupfalse%
\ {\isacharparenleft}{\kern0pt}meson\ notin{\isacharunderscore}{\kern0pt}set{\isacharunderscore}{\kern0pt}remove{\isadigit{1}}\ subset{\isacharunderscore}{\kern0pt}code{\isacharparenleft}{\kern0pt}{\isadigit{1}}{\isacharparenright}{\kern0pt}{\isacharparenright}{\kern0pt}\isanewline
\ \ \ \ \isacommand{by}\isamarkupfalse%
\ {\isacharparenleft}{\kern0pt}metis\ Suc{\isacharunderscore}{\kern0pt}diff{\isacharunderscore}{\kern0pt}le\ Suc{\isacharunderscore}{\kern0pt}length{\isacharunderscore}{\kern0pt}remove{\isadigit{1}}\ contxt{\isacharprime}{\kern0pt}{\isachardot}{\kern0pt}verts{\isacharunderscore}{\kern0pt}length\ contxt{\isachardot}{\kern0pt}obtain{\isacharunderscore}{\kern0pt}leaf{\isacharunderscore}{\kern0pt}tree{\isacharunderscore}{\kern0pt}to{\isacharunderscore}{\kern0pt}prufer{\isacharunderscore}{\kern0pt}seq\ find{\isacharunderscore}{\kern0pt}leaf\ length{\isacharunderscore}{\kern0pt}edge{\isacharunderscore}{\kern0pt}list\ option{\isachardot}{\kern0pt}simps{\isacharparenleft}{\kern0pt}{\isadigit{1}}{\isacharparenright}{\kern0pt}{\isacharparenright}{\kern0pt}\isanewline
\isacommand{qed}\isamarkupfalse%
%
\endisatagproof
{\isafoldproof}%
%
\isadelimproof
\isanewline
%
\endisadelimproof
\isanewline
\isacommand{lemma}\isamarkupfalse%
\ prufer{\isacharunderscore}{\kern0pt}seq{\isacharunderscore}{\kern0pt}in{\isacharunderscore}{\kern0pt}verts{\isacharcolon}{\kern0pt}\ {\isachardoublequoteopen}v\ {\isasymin}\ set\ {\isacharparenleft}{\kern0pt}tree{\isacharunderscore}{\kern0pt}to{\isacharunderscore}{\kern0pt}prufer{\isacharunderscore}{\kern0pt}seq\ verts\ edge{\isacharunderscore}{\kern0pt}list{\isacharparenright}{\kern0pt}\ {\isasymLongrightarrow}\ v\ {\isasymin}\ set\ verts{\isachardoublequoteclose}\isanewline
%
\isadelimproof
\ \ %
\endisadelimproof
%
\isatagproof
\isacommand{using}\isamarkupfalse%
\ pruf{\isacharunderscore}{\kern0pt}seq{\isacharunderscore}{\kern0pt}tree{\isacharunderscore}{\kern0pt}to{\isacharunderscore}{\kern0pt}prufer{\isacharunderscore}{\kern0pt}seq\ \isacommand{unfolding}\isamarkupfalse%
\ prufer{\isacharunderscore}{\kern0pt}sequences{\isacharunderscore}{\kern0pt}def\ n{\isacharunderscore}{\kern0pt}sequences{\isacharunderscore}{\kern0pt}def\ \isacommand{by}\isamarkupfalse%
\ auto%
\endisatagproof
{\isafoldproof}%
%
\isadelimproof
\isanewline
%
\endisadelimproof
\isanewline
\isacommand{lemma}\isamarkupfalse%
\ degree{\isacharunderscore}{\kern0pt}remove{\isacharunderscore}{\kern0pt}vertex{\isacharunderscore}{\kern0pt}non{\isacharunderscore}{\kern0pt}adjacent{\isacharcolon}{\kern0pt}\isanewline
\ \ \isakeyword{assumes}\ {\isachardoublequoteopen}v\ {\isasymnoteq}\ u{\isachardoublequoteclose}\isanewline
\ \ \ \ \isakeyword{and}\ non{\isacharunderscore}{\kern0pt}adjacent{\isacharcolon}{\kern0pt}\ {\isachardoublequoteopen}{\isacharbraceleft}{\kern0pt}v{\isacharcomma}{\kern0pt}u{\isacharbraceright}{\kern0pt}\ {\isasymnotin}\ edges{\isacharunderscore}{\kern0pt}of{\isacharunderscore}{\kern0pt}edge{\isacharunderscore}{\kern0pt}list\ edge{\isacharunderscore}{\kern0pt}list{\isachardoublequoteclose}\isanewline
\ \ \isakeyword{shows}\ {\isachardoublequoteopen}degree\ u\ {\isacharparenleft}{\kern0pt}remove{\isacharunderscore}{\kern0pt}vertex\ v\ edge{\isacharunderscore}{\kern0pt}list{\isacharparenright}{\kern0pt}\ {\isacharequal}{\kern0pt}\ degree\ u\ edge{\isacharunderscore}{\kern0pt}list{\isachardoublequoteclose}\isanewline
%
\isadelimproof
%
\endisadelimproof
%
\isatagproof
\isacommand{proof}\isamarkupfalse%
\ {\isacharminus}{\kern0pt}\isanewline
\ \ \isacommand{have}\isamarkupfalse%
\ {\isachardoublequoteopen}{\isacharparenleft}{\kern0pt}v{\isacharcomma}{\kern0pt}u{\isacharparenright}{\kern0pt}\ {\isasymnotin}\ set\ edge{\isacharunderscore}{\kern0pt}list\ {\isasymand}\ {\isacharparenleft}{\kern0pt}u{\isacharcomma}{\kern0pt}v{\isacharparenright}{\kern0pt}\ {\isasymnotin}\ set\ edge{\isacharunderscore}{\kern0pt}list{\isachardoublequoteclose}\ \isacommand{using}\isamarkupfalse%
\ non{\isacharunderscore}{\kern0pt}adjacent\ \isacommand{unfolding}\isamarkupfalse%
\ edges{\isacharunderscore}{\kern0pt}of{\isacharunderscore}{\kern0pt}edge{\isacharunderscore}{\kern0pt}list{\isacharunderscore}{\kern0pt}def\ \isacommand{by}\isamarkupfalse%
\ force\isanewline
\ \ \isacommand{then}\isamarkupfalse%
\ \isacommand{have}\isamarkupfalse%
\ {\isachardoublequoteopen}set\ {\isacharparenleft}{\kern0pt}incident{\isacharunderscore}{\kern0pt}edges\ u\ {\isacharparenleft}{\kern0pt}remove{\isacharunderscore}{\kern0pt}vertex\ v\ edge{\isacharunderscore}{\kern0pt}list{\isacharparenright}{\kern0pt}{\isacharparenright}{\kern0pt}\ {\isacharequal}{\kern0pt}\ set\ {\isacharparenleft}{\kern0pt}incident{\isacharunderscore}{\kern0pt}edges\ u\ edge{\isacharunderscore}{\kern0pt}list{\isacharparenright}{\kern0pt}{\isachardoublequoteclose}\ \isacommand{unfolding}\isamarkupfalse%
\ incident{\isacharunderscore}{\kern0pt}edges{\isacharunderscore}{\kern0pt}def\ edges{\isacharunderscore}{\kern0pt}of{\isacharunderscore}{\kern0pt}edge{\isacharunderscore}{\kern0pt}list{\isacharunderscore}{\kern0pt}def\ remove{\isacharunderscore}{\kern0pt}vertex{\isacharunderscore}{\kern0pt}def\ \isacommand{using}\isamarkupfalse%
\ filter{\isacharunderscore}{\kern0pt}filter\ {\isacartoucheopen}v\ {\isasymnoteq}\ u{\isacartoucheclose}\ \isacommand{by}\isamarkupfalse%
\ auto\isanewline
\ \ \isacommand{then}\isamarkupfalse%
\ \isacommand{show}\isamarkupfalse%
\ {\isacharquery}{\kern0pt}thesis\ \isacommand{using}\isamarkupfalse%
\ distinct{\isacharunderscore}{\kern0pt}edges\ distinct{\isacharunderscore}{\kern0pt}remove{\isacharunderscore}{\kern0pt}vertex\ distinct{\isacharunderscore}{\kern0pt}card\ distinct{\isacharunderscore}{\kern0pt}filter\ distinct{\isacharunderscore}{\kern0pt}map\ incident{\isacharunderscore}{\kern0pt}edges{\isacharunderscore}{\kern0pt}def\ \isacommand{by}\isamarkupfalse%
\ metis\isanewline
\isacommand{qed}\isamarkupfalse%
%
\endisatagproof
{\isafoldproof}%
%
\isadelimproof
\isanewline
%
\endisadelimproof
\isanewline
\isacommand{lemma}\isamarkupfalse%
\ count{\isacharunderscore}{\kern0pt}list{\isacharunderscore}{\kern0pt}pruf{\isacharunderscore}{\kern0pt}seq{\isacharunderscore}{\kern0pt}degree{\isacharcolon}{\kern0pt}\isanewline
\ \ \isakeyword{assumes}\ v{\isacharunderscore}{\kern0pt}in{\isacharunderscore}{\kern0pt}verts{\isacharcolon}{\kern0pt}\ {\isachardoublequoteopen}v\ {\isasymin}\ set\ verts{\isachardoublequoteclose}\isanewline
\ \ \isakeyword{shows}\ {\isachardoublequoteopen}Suc\ {\isacharparenleft}{\kern0pt}count{\isacharunderscore}{\kern0pt}list\ {\isacharparenleft}{\kern0pt}tree{\isacharunderscore}{\kern0pt}to{\isacharunderscore}{\kern0pt}prufer{\isacharunderscore}{\kern0pt}seq\ verts\ edge{\isacharunderscore}{\kern0pt}list{\isacharparenright}{\kern0pt}\ v{\isacharparenright}{\kern0pt}\ {\isacharequal}{\kern0pt}\ degree\ v\ edge{\isacharunderscore}{\kern0pt}list{\isachardoublequoteclose}\isanewline
%
\isadelimproof
\ \ %
\endisadelimproof
%
\isatagproof
\isacommand{using}\isamarkupfalse%
\ v{\isacharunderscore}{\kern0pt}in{\isacharunderscore}{\kern0pt}verts\ tree{\isacharunderscore}{\kern0pt}to{\isacharunderscore}{\kern0pt}prufer{\isacharunderscore}{\kern0pt}seq{\isacharunderscore}{\kern0pt}context{\isacharunderscore}{\kern0pt}axioms\isanewline
\isacommand{proof}\isamarkupfalse%
\ {\isacharparenleft}{\kern0pt}induction\ verts\ edge{\isacharunderscore}{\kern0pt}list\ rule{\isacharcolon}{\kern0pt}\ tree{\isacharunderscore}{\kern0pt}to{\isacharunderscore}{\kern0pt}prufer{\isacharunderscore}{\kern0pt}seq{\isachardot}{\kern0pt}induct{\isacharparenright}{\kern0pt}\isanewline
\ \ \isacommand{case}\isamarkupfalse%
\ {\isacharparenleft}{\kern0pt}{\isadigit{1}}\ verts{\isacharparenright}{\kern0pt}\isanewline
\ \ \isacommand{then}\isamarkupfalse%
\ \isacommand{interpret}\isamarkupfalse%
\ contxt{\isacharcolon}{\kern0pt}\ tree{\isacharunderscore}{\kern0pt}to{\isacharunderscore}{\kern0pt}prufer{\isacharunderscore}{\kern0pt}seq{\isacharunderscore}{\kern0pt}context\ verts\ {\isachardoublequoteopen}{\isacharbrackleft}{\kern0pt}{\isacharbrackright}{\kern0pt}{\isachardoublequoteclose}\ \isacommand{using}\isamarkupfalse%
\ tree{\isacharunderscore}{\kern0pt}to{\isacharunderscore}{\kern0pt}prufer{\isacharunderscore}{\kern0pt}seq{\isacharunderscore}{\kern0pt}context{\isachardot}{\kern0pt}intro\ \isacommand{by}\isamarkupfalse%
\ blast\isanewline
\ \ \isacommand{show}\isamarkupfalse%
\ {\isacharquery}{\kern0pt}case\ \isacommand{using}\isamarkupfalse%
\ contxt{\isachardot}{\kern0pt}length{\isacharunderscore}{\kern0pt}edge{\isacharunderscore}{\kern0pt}list\ \isacommand{by}\isamarkupfalse%
\ auto\isanewline
\isacommand{next}\isamarkupfalse%
\isanewline
\ \ \isacommand{case}\isamarkupfalse%
\ {\isacharparenleft}{\kern0pt}{\isadigit{2}}\ verts\ u\ w{\isacharparenright}{\kern0pt}\isanewline
\ \ \isacommand{then}\isamarkupfalse%
\ \isacommand{interpret}\isamarkupfalse%
\ contxt{\isacharcolon}{\kern0pt}\ tree{\isacharunderscore}{\kern0pt}to{\isacharunderscore}{\kern0pt}prufer{\isacharunderscore}{\kern0pt}seq{\isacharunderscore}{\kern0pt}context\ verts\ {\isachardoublequoteopen}{\isacharbrackleft}{\kern0pt}{\isacharparenleft}{\kern0pt}u{\isacharcomma}{\kern0pt}w{\isacharparenright}{\kern0pt}{\isacharbrackright}{\kern0pt}{\isachardoublequoteclose}\ \isacommand{by}\isamarkupfalse%
\ simp\isanewline
\ \ \isacommand{interpret}\isamarkupfalse%
\ tr{\isacharcolon}{\kern0pt}\ tree\ {\isachardoublequoteopen}set\ verts{\isachardoublequoteclose}\ {\isachardoublequoteopen}{\isacharbraceleft}{\kern0pt}{\isacharbraceleft}{\kern0pt}u{\isacharcomma}{\kern0pt}w{\isacharbraceright}{\kern0pt}{\isacharbraceright}{\kern0pt}{\isachardoublequoteclose}\ \isacommand{using}\isamarkupfalse%
\ contxt{\isachardot}{\kern0pt}tree\ \isacommand{unfolding}\isamarkupfalse%
\ edges{\isacharunderscore}{\kern0pt}of{\isacharunderscore}{\kern0pt}edge{\isacharunderscore}{\kern0pt}list{\isacharunderscore}{\kern0pt}def\ \isacommand{by}\isamarkupfalse%
\ simp\isanewline
\ \ \isacommand{have}\isamarkupfalse%
\ {\isachardoublequoteopen}set\ verts\ {\isacharequal}{\kern0pt}\ {\isacharbraceleft}{\kern0pt}u{\isacharcomma}{\kern0pt}w{\isacharbraceright}{\kern0pt}{\isachardoublequoteclose}\ \isacommand{using}\isamarkupfalse%
\ tr{\isachardot}{\kern0pt}V{\isacharunderscore}{\kern0pt}Union{\isacharunderscore}{\kern0pt}E\ contxt{\isachardot}{\kern0pt}non{\isacharunderscore}{\kern0pt}trivial\ \isacommand{by}\isamarkupfalse%
\ blast\isanewline
\ \ \isacommand{then}\isamarkupfalse%
\ \isacommand{show}\isamarkupfalse%
\ {\isacharquery}{\kern0pt}case\ \isacommand{unfolding}\isamarkupfalse%
\ incident{\isacharunderscore}{\kern0pt}edges{\isacharunderscore}{\kern0pt}def\ \isacommand{using}\isamarkupfalse%
\ {\isadigit{2}}\ \isacommand{by}\isamarkupfalse%
\ auto\isanewline
\isacommand{next}\isamarkupfalse%
\isanewline
\ \ \isacommand{case}\isamarkupfalse%
\ {\isacharparenleft}{\kern0pt}{\isadigit{3}}\ verts\ e{\isadigit{1}}\ e{\isadigit{2}}\ edges{\isacharparenright}{\kern0pt}\isanewline
\ \ \isacommand{let}\isamarkupfalse%
\ {\isacharquery}{\kern0pt}edge{\isacharunderscore}{\kern0pt}list\ {\isacharequal}{\kern0pt}\ {\isachardoublequoteopen}e{\isadigit{1}}{\isacharhash}{\kern0pt}e{\isadigit{2}}{\isacharhash}{\kern0pt}edges{\isachardoublequoteclose}\isanewline
\ \ \isacommand{interpret}\isamarkupfalse%
\ contxt{\isacharcolon}{\kern0pt}\ tree{\isacharunderscore}{\kern0pt}to{\isacharunderscore}{\kern0pt}prufer{\isacharunderscore}{\kern0pt}seq{\isacharunderscore}{\kern0pt}context\ verts\ {\isacharquery}{\kern0pt}edge{\isacharunderscore}{\kern0pt}list\ \isacommand{using}\isamarkupfalse%
\ tree{\isacharunderscore}{\kern0pt}to{\isacharunderscore}{\kern0pt}prufer{\isacharunderscore}{\kern0pt}seq{\isacharunderscore}{\kern0pt}context{\isachardot}{\kern0pt}intro\ {\isadigit{3}}\ \isacommand{by}\isamarkupfalse%
\ blast\isanewline
\ \ \isacommand{have}\isamarkupfalse%
\ {\isachardoublequoteopen}length\ {\isacharquery}{\kern0pt}edge{\isacharunderscore}{\kern0pt}list\ {\isasymge}\ {\isadigit{2}}{\isachardoublequoteclose}\ \isacommand{by}\isamarkupfalse%
\ simp\isanewline
\ \ \isacommand{then}\isamarkupfalse%
\ \isacommand{obtain}\isamarkupfalse%
\ leaf\isanewline
\ \ \ \ \isakeyword{where}\ find{\isacharunderscore}{\kern0pt}leaf{\isacharcolon}{\kern0pt}\ {\isachardoublequoteopen}find\ {\isacharparenleft}{\kern0pt}{\isasymlambda}v{\isachardot}{\kern0pt}\ degree\ v\ {\isacharquery}{\kern0pt}edge{\isacharunderscore}{\kern0pt}list\ {\isacharequal}{\kern0pt}\ {\isadigit{1}}{\isacharparenright}{\kern0pt}\ verts\ {\isacharequal}{\kern0pt}\ Some\ leaf{\isachardoublequoteclose}\isanewline
\ \ \ \ \ \ \isakeyword{and}\ leaf{\isacharcolon}{\kern0pt}\ {\isachardoublequoteopen}contxt{\isachardot}{\kern0pt}t{\isachardot}{\kern0pt}leaf\ leaf{\isachardoublequoteclose}\isanewline
\ \ \ \ \ \ \isakeyword{and}\ leaf{\isacharunderscore}{\kern0pt}in{\isacharunderscore}{\kern0pt}verts{\isacharcolon}{\kern0pt}\ {\isachardoublequoteopen}leaf\ {\isasymin}\ set\ verts{\isachardoublequoteclose}\isanewline
\ \ \ \ \ \ \isakeyword{and}\ contxt{\isacharprime}{\kern0pt}{\isacharcolon}{\kern0pt}\ {\isachardoublequoteopen}tree{\isacharunderscore}{\kern0pt}to{\isacharunderscore}{\kern0pt}prufer{\isacharunderscore}{\kern0pt}seq{\isacharunderscore}{\kern0pt}context\ {\isacharparenleft}{\kern0pt}remove{\isadigit{1}}\ leaf\ verts{\isacharparenright}{\kern0pt}\ {\isacharparenleft}{\kern0pt}remove{\isacharunderscore}{\kern0pt}vertex\ leaf\ {\isacharquery}{\kern0pt}edge{\isacharunderscore}{\kern0pt}list{\isacharparenright}{\kern0pt}{\isachardoublequoteclose}\isanewline
\ \ \ \ \isacommand{using}\isamarkupfalse%
\ contxt{\isachardot}{\kern0pt}obtain{\isacharunderscore}{\kern0pt}leaf{\isacharunderscore}{\kern0pt}tree{\isacharunderscore}{\kern0pt}to{\isacharunderscore}{\kern0pt}prufer{\isacharunderscore}{\kern0pt}seq\ {\isadigit{3}}\ \isacommand{by}\isamarkupfalse%
\ blast\isanewline
\ \ \isacommand{then}\isamarkupfalse%
\ \isacommand{interpret}\isamarkupfalse%
\ contxt{\isacharprime}{\kern0pt}{\isacharcolon}{\kern0pt}\ tree{\isacharunderscore}{\kern0pt}to{\isacharunderscore}{\kern0pt}prufer{\isacharunderscore}{\kern0pt}seq{\isacharunderscore}{\kern0pt}context\ {\isachardoublequoteopen}remove{\isadigit{1}}\ leaf\ verts{\isachardoublequoteclose}\ {\isachardoublequoteopen}remove{\isacharunderscore}{\kern0pt}vertex\ leaf\ {\isacharquery}{\kern0pt}edge{\isacharunderscore}{\kern0pt}list{\isachardoublequoteclose}\ \isacommand{using}\isamarkupfalse%
\ tree{\isacharunderscore}{\kern0pt}to{\isacharunderscore}{\kern0pt}prufer{\isacharunderscore}{\kern0pt}seq{\isacharunderscore}{\kern0pt}context{\isachardot}{\kern0pt}intro\ \isacommand{by}\isamarkupfalse%
\ blast\isanewline
\ \ \isacommand{let}\isamarkupfalse%
\ {\isacharquery}{\kern0pt}neigh\ {\isacharequal}{\kern0pt}\ {\isachardoublequoteopen}neighbor\ leaf\ {\isacharquery}{\kern0pt}edge{\isacharunderscore}{\kern0pt}list{\isachardoublequoteclose}\isanewline
\ \ \isacommand{have}\isamarkupfalse%
\ degree{\isacharunderscore}{\kern0pt}leaf{\isacharcolon}{\kern0pt}\ {\isachardoublequoteopen}degree\ leaf\ {\isacharquery}{\kern0pt}edge{\isacharunderscore}{\kern0pt}list\ {\isacharequal}{\kern0pt}\ {\isadigit{1}}{\isachardoublequoteclose}\ \isacommand{using}\isamarkupfalse%
\ find{\isacharunderscore}{\kern0pt}leaf\ find{\isacharunderscore}{\kern0pt}Some\ \isacommand{by}\isamarkupfalse%
\ fast\isanewline
\ \ \isacommand{show}\isamarkupfalse%
\ {\isacharquery}{\kern0pt}case\isanewline
\ \ \isacommand{proof}\isamarkupfalse%
\ {\isacharparenleft}{\kern0pt}cases\ {\isachardoublequoteopen}v\ {\isacharequal}{\kern0pt}\ leaf{\isachardoublequoteclose}{\isacharparenright}{\kern0pt}\isanewline
\ \ \ \ \isacommand{case}\isamarkupfalse%
\ True\isanewline
\ \ \ \ \isacommand{have}\isamarkupfalse%
\ {\isachardoublequoteopen}leaf\ {\isasymnotin}\ set\ {\isacharparenleft}{\kern0pt}remove{\isadigit{1}}\ leaf\ verts{\isacharparenright}{\kern0pt}{\isachardoublequoteclose}\ \isacommand{using}\isamarkupfalse%
\ contxt{\isachardot}{\kern0pt}distinct{\isacharunderscore}{\kern0pt}verts\ set{\isacharunderscore}{\kern0pt}remove{\isadigit{1}}{\isacharunderscore}{\kern0pt}eq\ \isacommand{by}\isamarkupfalse%
\ auto\isanewline
\ \ \ \ \isacommand{then}\isamarkupfalse%
\ \isacommand{have}\isamarkupfalse%
\ leaf{\isacharunderscore}{\kern0pt}notin{\isacharunderscore}{\kern0pt}pruf{\isacharunderscore}{\kern0pt}seq{\isacharprime}{\kern0pt}{\isacharcolon}{\kern0pt}\ {\isachardoublequoteopen}leaf\ {\isasymnotin}\ set\ {\isacharparenleft}{\kern0pt}tree{\isacharunderscore}{\kern0pt}to{\isacharunderscore}{\kern0pt}prufer{\isacharunderscore}{\kern0pt}seq\ {\isacharparenleft}{\kern0pt}remove{\isadigit{1}}\ leaf\ verts{\isacharparenright}{\kern0pt}\ {\isacharparenleft}{\kern0pt}remove{\isacharunderscore}{\kern0pt}vertex\ leaf\ {\isacharparenleft}{\kern0pt}e{\isadigit{1}}\ {\isacharhash}{\kern0pt}\ e{\isadigit{2}}\ {\isacharhash}{\kern0pt}\ edges{\isacharparenright}{\kern0pt}{\isacharparenright}{\kern0pt}{\isacharparenright}{\kern0pt}{\isachardoublequoteclose}\isanewline
\ \ \ \ \ \ \isacommand{using}\isamarkupfalse%
\ contxt{\isacharprime}{\kern0pt}{\isachardot}{\kern0pt}prufer{\isacharunderscore}{\kern0pt}seq{\isacharunderscore}{\kern0pt}in{\isacharunderscore}{\kern0pt}verts\ True\ \isacommand{by}\isamarkupfalse%
\ blast\isanewline
\isanewline
\ \ \ \ \isacommand{have}\isamarkupfalse%
\ {\isachardoublequoteopen}neighbor\ leaf\ {\isacharquery}{\kern0pt}edge{\isacharunderscore}{\kern0pt}list\ {\isasymnoteq}\ leaf{\isachardoublequoteclose}\isanewline
\ \ \ \ \ \ \isacommand{using}\isamarkupfalse%
\ degree{\isacharunderscore}{\kern0pt}leaf\ \isacommand{by}\isamarkupfalse%
\ {\isacharparenleft}{\kern0pt}simp\ add{\isacharcolon}{\kern0pt}\ contxt{\isachardot}{\kern0pt}t{\isachardot}{\kern0pt}edge{\isacharunderscore}{\kern0pt}vertices{\isacharunderscore}{\kern0pt}not{\isacharunderscore}{\kern0pt}equal\ neighbor{\isacharunderscore}{\kern0pt}edge{\isacharunderscore}{\kern0pt}in{\isacharunderscore}{\kern0pt}edges{\isacharparenright}{\kern0pt}\isanewline
\ \ \ \ \isacommand{then}\isamarkupfalse%
\ \isacommand{show}\isamarkupfalse%
\ {\isacharquery}{\kern0pt}thesis\ \isacommand{using}\isamarkupfalse%
\ find{\isacharunderscore}{\kern0pt}leaf\ True\ leaf{\isacharunderscore}{\kern0pt}notin{\isacharunderscore}{\kern0pt}pruf{\isacharunderscore}{\kern0pt}seq{\isacharprime}{\kern0pt}\ degree{\isacharunderscore}{\kern0pt}leaf\ \isacommand{by}\isamarkupfalse%
\ auto\isanewline
\ \ \isacommand{next}\isamarkupfalse%
\isanewline
\ \ \ \ \isacommand{case}\isamarkupfalse%
\ False\isanewline
\ \ \ \ \isacommand{then}\isamarkupfalse%
\ \isacommand{have}\isamarkupfalse%
\ {\isachardoublequoteopen}v\ {\isasymin}\ set\ {\isacharparenleft}{\kern0pt}remove{\isadigit{1}}\ leaf\ verts{\isacharparenright}{\kern0pt}{\isachardoublequoteclose}\ \isacommand{using}\isamarkupfalse%
\ {\isadigit{3}}\ set{\isacharunderscore}{\kern0pt}remove{\isadigit{1}}{\isacharunderscore}{\kern0pt}eq\ \isacommand{by}\isamarkupfalse%
\ auto\isanewline
\ \ \ \ \isacommand{then}\isamarkupfalse%
\ \isacommand{have}\isamarkupfalse%
\ IH{\isacharcolon}{\kern0pt}\ {\isachardoublequoteopen}Suc\ {\isacharparenleft}{\kern0pt}count{\isacharunderscore}{\kern0pt}list\ {\isacharparenleft}{\kern0pt}tree{\isacharunderscore}{\kern0pt}to{\isacharunderscore}{\kern0pt}prufer{\isacharunderscore}{\kern0pt}seq\ {\isacharparenleft}{\kern0pt}remove{\isadigit{1}}\ leaf\ verts{\isacharparenright}{\kern0pt}\ {\isacharparenleft}{\kern0pt}remove{\isacharunderscore}{\kern0pt}vertex\ leaf\ {\isacharquery}{\kern0pt}edge{\isacharunderscore}{\kern0pt}list{\isacharparenright}{\kern0pt}{\isacharparenright}{\kern0pt}\ v{\isacharparenright}{\kern0pt}\isanewline
\ \ \ \ \ \ {\isacharequal}{\kern0pt}\ degree\ v\ {\isacharparenleft}{\kern0pt}remove{\isacharunderscore}{\kern0pt}vertex\ leaf\ {\isacharquery}{\kern0pt}edge{\isacharunderscore}{\kern0pt}list{\isacharparenright}{\kern0pt}{\isachardoublequoteclose}\ \isacommand{using}\isamarkupfalse%
\ {\isachardoublequoteopen}{\isadigit{3}}{\isachardot}{\kern0pt}IH{\isachardoublequoteclose}\ find{\isacharunderscore}{\kern0pt}leaf\ contxt{\isacharprime}{\kern0pt}\ \isacommand{by}\isamarkupfalse%
\ blast\isanewline
\ \ \ \ \isacommand{then}\isamarkupfalse%
\ \isacommand{show}\isamarkupfalse%
\ {\isacharquery}{\kern0pt}thesis\isanewline
\ \ \ \ \isacommand{proof}\isamarkupfalse%
\ {\isacharparenleft}{\kern0pt}cases\ {\isachardoublequoteopen}v\ {\isacharequal}{\kern0pt}\ {\isacharquery}{\kern0pt}neigh{\isachardoublequoteclose}{\isacharparenright}{\kern0pt}\isanewline
\ \ \ \ \ \ \isacommand{case}\isamarkupfalse%
\ True\isanewline
\ \ \ \ \ \ \isacommand{then}\isamarkupfalse%
\ \isacommand{show}\isamarkupfalse%
\ {\isacharquery}{\kern0pt}thesis\ \isacommand{using}\isamarkupfalse%
\ degree{\isacharunderscore}{\kern0pt}neighbor{\isacharunderscore}{\kern0pt}remove{\isacharunderscore}{\kern0pt}vertex{\isacharbrackleft}{\kern0pt}OF\ degree{\isacharunderscore}{\kern0pt}leaf{\isacharbrackright}{\kern0pt}\ find{\isacharunderscore}{\kern0pt}leaf\ IH\ \isacommand{by}\isamarkupfalse%
\ auto\isanewline
\ \ \ \ \isacommand{next}\isamarkupfalse%
\isanewline
\ \ \ \ \ \ \isacommand{case}\isamarkupfalse%
\ False\isanewline
\ \ \ \ \ \ \isacommand{have}\isamarkupfalse%
\ {\isachardoublequoteopen}{\isacharbraceleft}{\kern0pt}leaf{\isacharcomma}{\kern0pt}\ v{\isacharbraceright}{\kern0pt}\ {\isasymnotin}\ edges{\isacharunderscore}{\kern0pt}of{\isacharunderscore}{\kern0pt}edge{\isacharunderscore}{\kern0pt}list\ {\isacharquery}{\kern0pt}edge{\isacharunderscore}{\kern0pt}list{\isachardoublequoteclose}\isanewline
\ \ \ \ \ \ \isacommand{proof}\isamarkupfalse%
\isanewline
\ \ \ \ \ \ \ \ \isacommand{assume}\isamarkupfalse%
\ {\isachardoublequoteopen}{\isacharbraceleft}{\kern0pt}leaf{\isacharcomma}{\kern0pt}\ v{\isacharbraceright}{\kern0pt}\ {\isasymin}\ edges{\isacharunderscore}{\kern0pt}of{\isacharunderscore}{\kern0pt}edge{\isacharunderscore}{\kern0pt}list\ {\isacharquery}{\kern0pt}edge{\isacharunderscore}{\kern0pt}list{\isachardoublequoteclose}\isanewline
\ \ \ \ \ \ \ \ \isacommand{then}\isamarkupfalse%
\ \isacommand{have}\isamarkupfalse%
\ leaf{\isacharunderscore}{\kern0pt}v{\isacharunderscore}{\kern0pt}edge{\isacharcolon}{\kern0pt}\ {\isachardoublequoteopen}{\isacharbraceleft}{\kern0pt}leaf{\isacharcomma}{\kern0pt}\ v{\isacharbraceright}{\kern0pt}\ {\isasymin}\ edges{\isacharunderscore}{\kern0pt}of{\isacharunderscore}{\kern0pt}edge{\isacharunderscore}{\kern0pt}list\ {\isacharparenleft}{\kern0pt}incident{\isacharunderscore}{\kern0pt}edges\ leaf\ {\isacharquery}{\kern0pt}edge{\isacharunderscore}{\kern0pt}list{\isacharparenright}{\kern0pt}{\isachardoublequoteclose}\isanewline
\ \ \ \ \ \ \ \ \ \ \isacommand{unfolding}\isamarkupfalse%
\ contxt{\isachardot}{\kern0pt}incident{\isacharunderscore}{\kern0pt}edges{\isacharunderscore}{\kern0pt}correct\ contxt{\isachardot}{\kern0pt}t{\isachardot}{\kern0pt}incident{\isacharunderscore}{\kern0pt}edges{\isacharunderscore}{\kern0pt}def\ contxt{\isachardot}{\kern0pt}t{\isachardot}{\kern0pt}incident{\isacharunderscore}{\kern0pt}def\ \isacommand{by}\isamarkupfalse%
\ simp\isanewline
\ \ \ \ \ \ \ \ \isacommand{have}\isamarkupfalse%
\ {\isachardoublequoteopen}{\isacharbraceleft}{\kern0pt}{\isacharquery}{\kern0pt}neigh{\isacharcomma}{\kern0pt}\ leaf{\isacharbraceright}{\kern0pt}\ {\isasymin}\ edges{\isacharunderscore}{\kern0pt}of{\isacharunderscore}{\kern0pt}edge{\isacharunderscore}{\kern0pt}list\ {\isacharquery}{\kern0pt}edge{\isacharunderscore}{\kern0pt}list{\isachardoublequoteclose}\ \isacommand{using}\isamarkupfalse%
\ neighbor{\isacharunderscore}{\kern0pt}edge{\isacharunderscore}{\kern0pt}in{\isacharunderscore}{\kern0pt}edges\ degree{\isacharunderscore}{\kern0pt}leaf\ degree{\isacharunderscore}{\kern0pt}length{\isacharunderscore}{\kern0pt}filter\ \isacommand{by}\isamarkupfalse%
\ force\isanewline
\ \ \ \ \ \ \ \ \isacommand{then}\isamarkupfalse%
\ \isacommand{have}\isamarkupfalse%
\ {\isachardoublequoteopen}{\isacharbraceleft}{\kern0pt}{\isacharquery}{\kern0pt}neigh{\isacharcomma}{\kern0pt}\ leaf{\isacharbraceright}{\kern0pt}\ {\isasymin}\ edges{\isacharunderscore}{\kern0pt}of{\isacharunderscore}{\kern0pt}edge{\isacharunderscore}{\kern0pt}list\ {\isacharparenleft}{\kern0pt}incident{\isacharunderscore}{\kern0pt}edges\ leaf\ {\isacharquery}{\kern0pt}edge{\isacharunderscore}{\kern0pt}list{\isacharparenright}{\kern0pt}{\isachardoublequoteclose}\isanewline
\ \ \ \ \ \ \ \ \ \ \isacommand{unfolding}\isamarkupfalse%
\ contxt{\isachardot}{\kern0pt}incident{\isacharunderscore}{\kern0pt}edges{\isacharunderscore}{\kern0pt}correct\ contxt{\isachardot}{\kern0pt}t{\isachardot}{\kern0pt}incident{\isacharunderscore}{\kern0pt}edges{\isacharunderscore}{\kern0pt}def\ contxt{\isachardot}{\kern0pt}t{\isachardot}{\kern0pt}incident{\isacharunderscore}{\kern0pt}def\ \isacommand{by}\isamarkupfalse%
\ simp\isanewline
\ \ \ \ \ \ \ \ \isacommand{then}\isamarkupfalse%
\ \isacommand{show}\isamarkupfalse%
\ False\ \isacommand{using}\isamarkupfalse%
\ leaf{\isacharunderscore}{\kern0pt}v{\isacharunderscore}{\kern0pt}edge\ degree{\isacharunderscore}{\kern0pt}leaf\isanewline
\ \ \ \ \ \ \ \ \ \ \isacommand{by}\isamarkupfalse%
\ {\isacharparenleft}{\kern0pt}metis\ False\ One{\isacharunderscore}{\kern0pt}nat{\isacharunderscore}{\kern0pt}def\ card{\isacharunderscore}{\kern0pt}le{\isacharunderscore}{\kern0pt}Suc{\isadigit{0}}{\isacharunderscore}{\kern0pt}iff{\isacharunderscore}{\kern0pt}eq\ contxt{\isachardot}{\kern0pt}degree{\isacharunderscore}{\kern0pt}correct\ contxt{\isachardot}{\kern0pt}incident{\isacharunderscore}{\kern0pt}edges{\isacharunderscore}{\kern0pt}correct\isanewline
\ \ \ \ \ \ \ \ \ \ \ \ \ \ contxt{\isachardot}{\kern0pt}t{\isachardot}{\kern0pt}alt{\isacharunderscore}{\kern0pt}degree{\isacharunderscore}{\kern0pt}def\ contxt{\isachardot}{\kern0pt}t{\isachardot}{\kern0pt}fin{\isacharunderscore}{\kern0pt}edges\ contxt{\isachardot}{\kern0pt}t{\isachardot}{\kern0pt}finite{\isacharunderscore}{\kern0pt}incident{\isacharunderscore}{\kern0pt}edges\ insert{\isacharunderscore}{\kern0pt}iff\ le{\isacharunderscore}{\kern0pt}numeral{\isacharunderscore}{\kern0pt}extra{\isacharparenleft}{\kern0pt}{\isadigit{4}}{\isacharparenright}{\kern0pt}\ singletonD{\isacharparenright}{\kern0pt}\isanewline
\ \ \ \ \ \ \isacommand{qed}\isamarkupfalse%
\isanewline
\ \ \ \ \ \ \isacommand{then}\isamarkupfalse%
\ \isacommand{show}\isamarkupfalse%
\ {\isacharquery}{\kern0pt}thesis\ \isacommand{using}\isamarkupfalse%
\ False\ find{\isacharunderscore}{\kern0pt}leaf\ IH\ find{\isacharunderscore}{\kern0pt}leaf\ \ {\isacartoucheopen}v\ {\isasymnoteq}\ leaf{\isacartoucheclose}\ contxt{\isachardot}{\kern0pt}degree{\isacharunderscore}{\kern0pt}remove{\isacharunderscore}{\kern0pt}vertex{\isacharunderscore}{\kern0pt}non{\isacharunderscore}{\kern0pt}adjacent\ \isacommand{by}\isamarkupfalse%
\ auto\isanewline
\ \ \ \ \isacommand{qed}\isamarkupfalse%
\isanewline
\ \ \isacommand{qed}\isamarkupfalse%
\isanewline
\isacommand{qed}\isamarkupfalse%
%
\endisatagproof
{\isafoldproof}%
%
\isadelimproof
\isanewline
%
\endisadelimproof
\isanewline
\isacommand{lemma}\isamarkupfalse%
\ notin{\isacharunderscore}{\kern0pt}set{\isacharunderscore}{\kern0pt}tree{\isacharunderscore}{\kern0pt}to{\isacharunderscore}{\kern0pt}prufer{\isacharunderscore}{\kern0pt}seq{\isacharcolon}{\kern0pt}\isanewline
\ \ \isakeyword{assumes}\ v{\isacharunderscore}{\kern0pt}in{\isacharunderscore}{\kern0pt}verts{\isacharcolon}{\kern0pt}\ {\isachardoublequoteopen}v\ {\isasymin}\ set\ verts{\isachardoublequoteclose}\isanewline
\ \ \isakeyword{shows}\ {\isachardoublequoteopen}v\ {\isasymnotin}\ set\ {\isacharparenleft}{\kern0pt}tree{\isacharunderscore}{\kern0pt}to{\isacharunderscore}{\kern0pt}prufer{\isacharunderscore}{\kern0pt}seq\ verts\ edge{\isacharunderscore}{\kern0pt}list{\isacharparenright}{\kern0pt}\ {\isasymlongleftrightarrow}\ degree\ v\ edge{\isacharunderscore}{\kern0pt}list\ {\isacharequal}{\kern0pt}\ {\isadigit{1}}{\isachardoublequoteclose}\isanewline
%
\isadelimproof
\ \ %
\endisadelimproof
%
\isatagproof
\isacommand{using}\isamarkupfalse%
\ count{\isacharunderscore}{\kern0pt}list{\isacharunderscore}{\kern0pt}pruf{\isacharunderscore}{\kern0pt}seq{\isacharunderscore}{\kern0pt}degree\ assms\ count{\isacharunderscore}{\kern0pt}list{\isacharunderscore}{\kern0pt}zero{\isacharunderscore}{\kern0pt}not{\isacharunderscore}{\kern0pt}elem\ \isacommand{by}\isamarkupfalse%
\ force%
\endisatagproof
{\isafoldproof}%
%
\isadelimproof
\isanewline
%
\endisadelimproof
\isanewline
\isacommand{lemma}\isamarkupfalse%
\ find{\isacharunderscore}{\kern0pt}Some{\isacharunderscore}{\kern0pt}impl{\isacharunderscore}{\kern0pt}eq{\isacharcolon}{\kern0pt}\ {\isachardoublequoteopen}find\ P\ xs\ {\isacharequal}{\kern0pt}\ Some\ x\ {\isasymLongrightarrow}\ {\isasymforall}x{\isachardot}{\kern0pt}\ Q\ x\ {\isasymlongrightarrow}\ P\ x\ {\isasymLongrightarrow}\ Q\ x\ {\isasymLongrightarrow}\ find\ Q\ xs\ {\isacharequal}{\kern0pt}\ Some\ x{\isachardoublequoteclose}\isanewline
%
\isadelimproof
\ \ %
\endisadelimproof
%
\isatagproof
\isacommand{by}\isamarkupfalse%
\ {\isacharparenleft}{\kern0pt}induction\ xs{\isacharparenright}{\kern0pt}\ {\isacharparenleft}{\kern0pt}auto\ split{\isacharcolon}{\kern0pt}\ if{\isacharunderscore}{\kern0pt}splits{\isacharparenright}{\kern0pt}%
\endisatagproof
{\isafoldproof}%
%
\isadelimproof
\ \isanewline
%
\endisadelimproof
\isanewline
\isanewline
\isacommand{lemma}\isamarkupfalse%
\ pruf{\isacharunderscore}{\kern0pt}seq{\isacharunderscore}{\kern0pt}to{\isacharunderscore}{\kern0pt}tree{\isacharunderscore}{\kern0pt}to{\isacharunderscore}{\kern0pt}pruf{\isacharunderscore}{\kern0pt}seq{\isacharcolon}{\kern0pt}\ {\isachardoublequoteopen}edges{\isacharunderscore}{\kern0pt}of{\isacharunderscore}{\kern0pt}edge{\isacharunderscore}{\kern0pt}list\ {\isacharparenleft}{\kern0pt}prufer{\isacharunderscore}{\kern0pt}seq{\isacharunderscore}{\kern0pt}to{\isacharunderscore}{\kern0pt}tree{\isacharunderscore}{\kern0pt}edges\ verts\ {\isacharparenleft}{\kern0pt}tree{\isacharunderscore}{\kern0pt}to{\isacharunderscore}{\kern0pt}prufer{\isacharunderscore}{\kern0pt}seq\ verts\ edge{\isacharunderscore}{\kern0pt}list{\isacharparenright}{\kern0pt}{\isacharparenright}{\kern0pt}\ {\isacharequal}{\kern0pt}\ edges{\isacharunderscore}{\kern0pt}of{\isacharunderscore}{\kern0pt}edge{\isacharunderscore}{\kern0pt}list\ edge{\isacharunderscore}{\kern0pt}list{\isachardoublequoteclose}\isanewline
%
\isadelimproof
\ \ %
\endisadelimproof
%
\isatagproof
\isacommand{using}\isamarkupfalse%
\ tree{\isacharunderscore}{\kern0pt}to{\isacharunderscore}{\kern0pt}prufer{\isacharunderscore}{\kern0pt}seq{\isacharunderscore}{\kern0pt}context{\isacharunderscore}{\kern0pt}axioms\isanewline
\isacommand{proof}\isamarkupfalse%
\ {\isacharparenleft}{\kern0pt}induction\ verts\ edge{\isacharunderscore}{\kern0pt}list\ rule{\isacharcolon}{\kern0pt}\ tree{\isacharunderscore}{\kern0pt}to{\isacharunderscore}{\kern0pt}prufer{\isacharunderscore}{\kern0pt}seq{\isachardot}{\kern0pt}induct{\isacharparenright}{\kern0pt}\isanewline
\ \ \isacommand{case}\isamarkupfalse%
\ {\isacharparenleft}{\kern0pt}{\isadigit{1}}\ verts{\isacharparenright}{\kern0pt}\isanewline
\ \ \isacommand{then}\isamarkupfalse%
\ \isacommand{interpret}\isamarkupfalse%
\ contxt{\isacharcolon}{\kern0pt}\ tree{\isacharunderscore}{\kern0pt}to{\isacharunderscore}{\kern0pt}prufer{\isacharunderscore}{\kern0pt}seq{\isacharunderscore}{\kern0pt}context\ verts\ {\isachardoublequoteopen}{\isacharbrackleft}{\kern0pt}{\isacharbrackright}{\kern0pt}{\isachardoublequoteclose}\ \isacommand{using}\isamarkupfalse%
\ tree{\isacharunderscore}{\kern0pt}to{\isacharunderscore}{\kern0pt}prufer{\isacharunderscore}{\kern0pt}seq{\isacharunderscore}{\kern0pt}context{\isachardot}{\kern0pt}intro\ \isacommand{by}\isamarkupfalse%
\ blast\isanewline
\ \ \isacommand{show}\isamarkupfalse%
\ {\isacharquery}{\kern0pt}case\ \isacommand{using}\isamarkupfalse%
\ contxt{\isachardot}{\kern0pt}length{\isacharunderscore}{\kern0pt}edge{\isacharunderscore}{\kern0pt}list\ \isacommand{by}\isamarkupfalse%
\ auto\isanewline
\isacommand{next}\isamarkupfalse%
\isanewline
\ \ \isacommand{case}\isamarkupfalse%
\ {\isacharparenleft}{\kern0pt}{\isadigit{2}}\ verts\ u\ w{\isacharparenright}{\kern0pt}\isanewline
\ \ \isacommand{then}\isamarkupfalse%
\ \isacommand{interpret}\isamarkupfalse%
\ contxt{\isacharcolon}{\kern0pt}\ tree{\isacharunderscore}{\kern0pt}to{\isacharunderscore}{\kern0pt}prufer{\isacharunderscore}{\kern0pt}seq{\isacharunderscore}{\kern0pt}context\ verts\ {\isachardoublequoteopen}{\isacharbrackleft}{\kern0pt}{\isacharparenleft}{\kern0pt}u{\isacharcomma}{\kern0pt}\ w{\isacharparenright}{\kern0pt}{\isacharbrackright}{\kern0pt}{\isachardoublequoteclose}\ \isacommand{by}\isamarkupfalse%
\ simp\isanewline
\ \ \isacommand{interpret}\isamarkupfalse%
\ tr{\isacharcolon}{\kern0pt}\ tree\ {\isachardoublequoteopen}set\ verts{\isachardoublequoteclose}\ {\isachardoublequoteopen}{\isacharbraceleft}{\kern0pt}{\isacharbraceleft}{\kern0pt}u{\isacharcomma}{\kern0pt}w{\isacharbraceright}{\kern0pt}{\isacharbraceright}{\kern0pt}{\isachardoublequoteclose}\ \isacommand{using}\isamarkupfalse%
\ contxt{\isachardot}{\kern0pt}tree\ \isacommand{unfolding}\isamarkupfalse%
\ edges{\isacharunderscore}{\kern0pt}of{\isacharunderscore}{\kern0pt}edge{\isacharunderscore}{\kern0pt}list{\isacharunderscore}{\kern0pt}def\ \isacommand{by}\isamarkupfalse%
\ simp\isanewline
\ \ \isacommand{have}\isamarkupfalse%
\ card{\isacharunderscore}{\kern0pt}verts{\isacharcolon}{\kern0pt}\ {\isachardoublequoteopen}card\ {\isacharparenleft}{\kern0pt}set\ verts{\isacharparenright}{\kern0pt}\ {\isacharequal}{\kern0pt}\ {\isadigit{2}}{\isachardoublequoteclose}\ \isacommand{using}\isamarkupfalse%
\ tr{\isachardot}{\kern0pt}card{\isacharunderscore}{\kern0pt}V{\isacharunderscore}{\kern0pt}card{\isacharunderscore}{\kern0pt}E\ \isacommand{by}\isamarkupfalse%
\ force\isanewline
\ \ \isacommand{then}\isamarkupfalse%
\ \isacommand{have}\isamarkupfalse%
\ set{\isacharunderscore}{\kern0pt}verts{\isacharcolon}{\kern0pt}\ {\isachardoublequoteopen}set\ verts\ {\isacharequal}{\kern0pt}\ {\isacharbraceleft}{\kern0pt}u{\isacharcomma}{\kern0pt}w{\isacharbraceright}{\kern0pt}{\isachardoublequoteclose}\ \isacommand{using}\isamarkupfalse%
\ tr{\isachardot}{\kern0pt}V{\isacharunderscore}{\kern0pt}Union{\isacharunderscore}{\kern0pt}E\ contxt{\isachardot}{\kern0pt}non{\isacharunderscore}{\kern0pt}trivial\ \isacommand{by}\isamarkupfalse%
\ simp\isanewline
\ \ \isacommand{have}\isamarkupfalse%
\ {\isachardoublequoteopen}length\ verts\ {\isacharequal}{\kern0pt}\ Suc\ {\isacharparenleft}{\kern0pt}Suc\ {\isadigit{0}}{\isacharparenright}{\kern0pt}{\isachardoublequoteclose}\ \isacommand{using}\isamarkupfalse%
\ contxt{\isachardot}{\kern0pt}distinct{\isacharunderscore}{\kern0pt}verts\ card{\isacharunderscore}{\kern0pt}verts\ distinct{\isacharunderscore}{\kern0pt}card\ \isacommand{by}\isamarkupfalse%
\ fastforce\isanewline
\ \ \isacommand{then}\isamarkupfalse%
\ \isacommand{have}\isamarkupfalse%
\ {\isachardoublequoteopen}{\isasymexists}a\ b{\isachardot}{\kern0pt}\ verts\ {\isacharequal}{\kern0pt}\ {\isacharbrackleft}{\kern0pt}a{\isacharcomma}{\kern0pt}b{\isacharbrackright}{\kern0pt}{\isachardoublequoteclose}\ \isacommand{by}\isamarkupfalse%
\ {\isacharparenleft}{\kern0pt}metis\ length{\isacharunderscore}{\kern0pt}{\isadigit{0}}{\isacharunderscore}{\kern0pt}conv\ length{\isacharunderscore}{\kern0pt}Suc{\isacharunderscore}{\kern0pt}conv{\isacharparenright}{\kern0pt}\isanewline
\ \ \isacommand{then}\isamarkupfalse%
\ \isacommand{show}\isamarkupfalse%
\ {\isacharquery}{\kern0pt}case\ \isacommand{unfolding}\isamarkupfalse%
\ edges{\isacharunderscore}{\kern0pt}of{\isacharunderscore}{\kern0pt}edge{\isacharunderscore}{\kern0pt}list{\isacharunderscore}{\kern0pt}def\ \isacommand{using}\isamarkupfalse%
\ set{\isacharunderscore}{\kern0pt}verts\ \isacommand{by}\isamarkupfalse%
\ force\isanewline
\isacommand{next}\isamarkupfalse%
\isanewline
\ \ \isacommand{case}\isamarkupfalse%
\ {\isacharparenleft}{\kern0pt}{\isadigit{3}}\ verts\ e{\isadigit{1}}\ e{\isadigit{2}}\ es{\isacharparenright}{\kern0pt}\isanewline
\ \ \isacommand{let}\isamarkupfalse%
\ {\isacharquery}{\kern0pt}edge{\isacharunderscore}{\kern0pt}list\ {\isacharequal}{\kern0pt}\ {\isachardoublequoteopen}e{\isadigit{1}}{\isacharhash}{\kern0pt}e{\isadigit{2}}{\isacharhash}{\kern0pt}es{\isachardoublequoteclose}\isanewline
\ \ \isacommand{interpret}\isamarkupfalse%
\ contxt{\isacharcolon}{\kern0pt}\ tree{\isacharunderscore}{\kern0pt}to{\isacharunderscore}{\kern0pt}prufer{\isacharunderscore}{\kern0pt}seq{\isacharunderscore}{\kern0pt}context\ verts\ {\isacharquery}{\kern0pt}edge{\isacharunderscore}{\kern0pt}list\ \isacommand{using}\isamarkupfalse%
\ {\isadigit{3}}\ tree{\isacharunderscore}{\kern0pt}to{\isacharunderscore}{\kern0pt}prufer{\isacharunderscore}{\kern0pt}seq{\isacharunderscore}{\kern0pt}context{\isachardot}{\kern0pt}intro\ \isacommand{by}\isamarkupfalse%
\ blast\isanewline
\ \ \isacommand{have}\isamarkupfalse%
\ {\isachardoublequoteopen}length\ {\isacharquery}{\kern0pt}edge{\isacharunderscore}{\kern0pt}list\ {\isasymge}\ {\isadigit{2}}{\isachardoublequoteclose}\ \isacommand{by}\isamarkupfalse%
\ simp\isanewline
\ \ \isacommand{then}\isamarkupfalse%
\ \isacommand{obtain}\isamarkupfalse%
\ leaf\isanewline
\ \ \ \ \isakeyword{where}\ find{\isacharunderscore}{\kern0pt}leaf{\isacharcolon}{\kern0pt}\ {\isachardoublequoteopen}find\ {\isacharparenleft}{\kern0pt}{\isasymlambda}v{\isachardot}{\kern0pt}\ degree\ v\ {\isacharquery}{\kern0pt}edge{\isacharunderscore}{\kern0pt}list\ {\isacharequal}{\kern0pt}\ {\isadigit{1}}{\isacharparenright}{\kern0pt}\ verts\ {\isacharequal}{\kern0pt}\ Some\ leaf{\isachardoublequoteclose}\isanewline
\ \ \ \ \isakeyword{and}\ leaf{\isacharcolon}{\kern0pt}\ {\isachardoublequoteopen}contxt{\isachardot}{\kern0pt}t{\isachardot}{\kern0pt}leaf\ leaf{\isachardoublequoteclose}\isanewline
\ \ \ \ \isakeyword{and}\ leaf{\isacharunderscore}{\kern0pt}in{\isacharunderscore}{\kern0pt}verts{\isacharcolon}{\kern0pt}\ {\isachardoublequoteopen}leaf\ {\isasymin}\ set\ verts{\isachardoublequoteclose}\isanewline
\ \ \ \ \isakeyword{and}\ contxt{\isacharprime}{\kern0pt}{\isacharcolon}{\kern0pt}\ {\isachardoublequoteopen}tree{\isacharunderscore}{\kern0pt}to{\isacharunderscore}{\kern0pt}prufer{\isacharunderscore}{\kern0pt}seq{\isacharunderscore}{\kern0pt}context\ {\isacharparenleft}{\kern0pt}remove{\isadigit{1}}\ leaf\ verts{\isacharparenright}{\kern0pt}\ {\isacharparenleft}{\kern0pt}remove{\isacharunderscore}{\kern0pt}vertex\ leaf\ {\isacharquery}{\kern0pt}edge{\isacharunderscore}{\kern0pt}list{\isacharparenright}{\kern0pt}{\isachardoublequoteclose}\isanewline
\ \ \ \ \isacommand{using}\isamarkupfalse%
\ contxt{\isachardot}{\kern0pt}obtain{\isacharunderscore}{\kern0pt}leaf{\isacharunderscore}{\kern0pt}tree{\isacharunderscore}{\kern0pt}to{\isacharunderscore}{\kern0pt}prufer{\isacharunderscore}{\kern0pt}seq\ {\isadigit{3}}\ \isacommand{by}\isamarkupfalse%
\ blast\isanewline
\ \ \isacommand{then}\isamarkupfalse%
\ \isacommand{interpret}\isamarkupfalse%
\ contxt{\isacharprime}{\kern0pt}{\isacharcolon}{\kern0pt}\ tree{\isacharunderscore}{\kern0pt}to{\isacharunderscore}{\kern0pt}prufer{\isacharunderscore}{\kern0pt}seq{\isacharunderscore}{\kern0pt}context\ {\isachardoublequoteopen}remove{\isadigit{1}}\ leaf\ verts{\isachardoublequoteclose}\ {\isachardoublequoteopen}remove{\isacharunderscore}{\kern0pt}vertex\ leaf\ {\isacharquery}{\kern0pt}edge{\isacharunderscore}{\kern0pt}list{\isachardoublequoteclose}\ \isacommand{by}\isamarkupfalse%
\ simp\isanewline
\isanewline
\ \ \isacommand{have}\isamarkupfalse%
\ degree{\isacharunderscore}{\kern0pt}leaf{\isacharcolon}{\kern0pt}\ {\isachardoublequoteopen}degree\ leaf\ {\isacharquery}{\kern0pt}edge{\isacharunderscore}{\kern0pt}list\ {\isacharequal}{\kern0pt}\ {\isadigit{1}}{\isachardoublequoteclose}\ \isacommand{using}\isamarkupfalse%
\ find{\isacharunderscore}{\kern0pt}leaf\ find{\isacharunderscore}{\kern0pt}Some\ \isacommand{by}\isamarkupfalse%
\ fast\isanewline
\ \ \isacommand{have}\isamarkupfalse%
\ find{\isacharunderscore}{\kern0pt}not{\isacharunderscore}{\kern0pt}in{\isacharunderscore}{\kern0pt}seq{\isacharcolon}{\kern0pt}\ {\isachardoublequoteopen}find\ {\isacharparenleft}{\kern0pt}{\isasymlambda}v{\isachardot}{\kern0pt}\ v\ {\isasymnotin}\ set\ {\isacharparenleft}{\kern0pt}tree{\isacharunderscore}{\kern0pt}to{\isacharunderscore}{\kern0pt}prufer{\isacharunderscore}{\kern0pt}seq\ verts\ {\isacharquery}{\kern0pt}edge{\isacharunderscore}{\kern0pt}list{\isacharparenright}{\kern0pt}{\isacharparenright}{\kern0pt}\ verts\ {\isacharequal}{\kern0pt}\ Some\ leaf{\isachardoublequoteclose}\isanewline
\ \ \ \ \isacommand{using}\isamarkupfalse%
\ find{\isacharunderscore}{\kern0pt}leaf\ contxt{\isachardot}{\kern0pt}notin{\isacharunderscore}{\kern0pt}set{\isacharunderscore}{\kern0pt}tree{\isacharunderscore}{\kern0pt}to{\isacharunderscore}{\kern0pt}prufer{\isacharunderscore}{\kern0pt}seq\ find{\isacharunderscore}{\kern0pt}cong\ \isacommand{by}\isamarkupfalse%
\ force\isanewline
\ \ \isacommand{show}\isamarkupfalse%
\ {\isacharquery}{\kern0pt}case\ \isacommand{using}\isamarkupfalse%
\ find{\isacharunderscore}{\kern0pt}not{\isacharunderscore}{\kern0pt}in{\isacharunderscore}{\kern0pt}seq\ find{\isacharunderscore}{\kern0pt}leaf\ {\isachardoublequoteopen}{\isadigit{3}}{\isachardot}{\kern0pt}IH{\isachardoublequoteclose}\ find{\isacharunderscore}{\kern0pt}leaf\ contxt{\isacharprime}{\kern0pt}\ insert{\isacharunderscore}{\kern0pt}remove{\isacharunderscore}{\kern0pt}leaf{\isacharbrackleft}{\kern0pt}OF\ degree{\isacharunderscore}{\kern0pt}leaf{\isacharbrackright}{\kern0pt}\isanewline
\ \ \ \ \isacommand{unfolding}\isamarkupfalse%
\ edges{\isacharunderscore}{\kern0pt}of{\isacharunderscore}{\kern0pt}edge{\isacharunderscore}{\kern0pt}list{\isacharunderscore}{\kern0pt}def\ \isacommand{by}\isamarkupfalse%
\ simp\isanewline
\isacommand{qed}\isamarkupfalse%
%
\endisatagproof
{\isafoldproof}%
%
\isadelimproof
\isanewline
%
\endisadelimproof
\isanewline
\isacommand{end}\isamarkupfalse%
\isanewline
\isanewline
\isacommand{context}\isamarkupfalse%
\ prufer{\isacharunderscore}{\kern0pt}seq{\isacharunderscore}{\kern0pt}to{\isacharunderscore}{\kern0pt}tree{\isacharunderscore}{\kern0pt}context\isanewline
\isakeyword{begin}\isanewline
\isanewline
\isacommand{lemma}\isamarkupfalse%
\ tree{\isacharunderscore}{\kern0pt}labeled{\isacharunderscore}{\kern0pt}tree{\isacharunderscore}{\kern0pt}enum{\isacharcolon}{\kern0pt}\isanewline
\ \ \isakeyword{assumes}\ t{\isacharcolon}{\kern0pt}\ {\isachardoublequoteopen}tree\ {\isacharparenleft}{\kern0pt}set\ verts{\isacharparenright}{\kern0pt}\ E{\isachardoublequoteclose}\isanewline
\ \ \isakeyword{shows}\ {\isachardoublequoteopen}{\isacharparenleft}{\kern0pt}set\ verts{\isacharcomma}{\kern0pt}\ E{\isacharparenright}{\kern0pt}\ {\isasymin}\ set\ {\isacharparenleft}{\kern0pt}labeled{\isacharunderscore}{\kern0pt}tree{\isacharunderscore}{\kern0pt}enum\ verts{\isacharparenright}{\kern0pt}{\isachardoublequoteclose}\isanewline
%
\isadelimproof
%
\endisadelimproof
%
\isatagproof
\isacommand{proof}\isamarkupfalse%
{\isacharminus}{\kern0pt}\isanewline
\ \ \isacommand{interpret}\isamarkupfalse%
\ t{\isacharcolon}{\kern0pt}\ tree\ {\isachardoublequoteopen}set\ verts{\isachardoublequoteclose}\ E\ \isacommand{using}\isamarkupfalse%
\ t\ \isacommand{{\isachardot}{\kern0pt}}\isamarkupfalse%
\isanewline
\ \ \isacommand{obtain}\isamarkupfalse%
\ edges\ \isakeyword{where}\ set{\isacharunderscore}{\kern0pt}edges{\isacharcolon}{\kern0pt}\ {\isachardoublequoteopen}set\ edges\ {\isacharequal}{\kern0pt}\ E{\isachardoublequoteclose}\ \isakeyword{and}\ \ distinct{\isacharunderscore}{\kern0pt}edges{\isacharcolon}{\kern0pt}\ {\isachardoublequoteopen}distinct\ edges{\isachardoublequoteclose}\ \isacommand{using}\isamarkupfalse%
\ finite{\isacharunderscore}{\kern0pt}distinct{\isacharunderscore}{\kern0pt}list\ t{\isachardot}{\kern0pt}fin{\isacharunderscore}{\kern0pt}edges\ \isacommand{by}\isamarkupfalse%
\ blast\isanewline
\ \ \isacommand{let}\isamarkupfalse%
\ {\isacharquery}{\kern0pt}edge{\isacharunderscore}{\kern0pt}list\ {\isacharequal}{\kern0pt}\ {\isachardoublequoteopen}map\ {\isacharparenleft}{\kern0pt}{\isasymlambda}e{\isachardot}{\kern0pt}\ SOME\ uv{\isachardot}{\kern0pt}\ mk{\isacharunderscore}{\kern0pt}edge\ uv\ {\isacharequal}{\kern0pt}\ e{\isacharparenright}{\kern0pt}\ edges{\isachardoublequoteclose}\isanewline
\ \ \isacommand{have}\isamarkupfalse%
\ {\isachardoublequoteopen}{\isasymforall}e{\isasymin}E{\isachardot}{\kern0pt}\ {\isasymexists}uv{\isachardot}{\kern0pt}\ mk{\isacharunderscore}{\kern0pt}edge\ uv\ {\isacharequal}{\kern0pt}\ e{\isachardoublequoteclose}\ \isacommand{using}\isamarkupfalse%
\ t{\isachardot}{\kern0pt}two{\isacharunderscore}{\kern0pt}edges\ card{\isacharunderscore}{\kern0pt}{\isadigit{2}}{\isacharunderscore}{\kern0pt}iff\ \isacommand{by}\isamarkupfalse%
\ {\isacharparenleft}{\kern0pt}metis\ mk{\isacharunderscore}{\kern0pt}edge{\isachardot}{\kern0pt}simps{\isacharparenright}{\kern0pt}\isanewline
\ \ \isacommand{then}\isamarkupfalse%
\ \isacommand{have}\isamarkupfalse%
\ {\isachardoublequoteopen}{\isasymAnd}e{\isachardot}{\kern0pt}\ e\ {\isasymin}\ E\ {\isasymLongrightarrow}\ {\isacharparenleft}{\kern0pt}mk{\isacharunderscore}{\kern0pt}edge\ o\ {\isacharparenleft}{\kern0pt}{\isasymlambda}e{\isachardot}{\kern0pt}\ SOME\ uv{\isachardot}{\kern0pt}\ mk{\isacharunderscore}{\kern0pt}edge\ uv\ {\isacharequal}{\kern0pt}\ e{\isacharparenright}{\kern0pt}{\isacharparenright}{\kern0pt}\ e\ {\isacharequal}{\kern0pt}\ e{\isachardoublequoteclose}\ \isacommand{using}\isamarkupfalse%
\ someI{\isacharunderscore}{\kern0pt}ex\isanewline
\ \ \ \ \isacommand{by}\isamarkupfalse%
\ {\isacharparenleft}{\kern0pt}smt\ {\isacharparenleft}{\kern0pt}verit{\isacharcomma}{\kern0pt}\ del{\isacharunderscore}{\kern0pt}insts{\isacharparenright}{\kern0pt}\ comp{\isacharunderscore}{\kern0pt}apply{\isacharparenright}{\kern0pt}\isanewline
\ \ \isacommand{then}\isamarkupfalse%
\ \isacommand{have}\isamarkupfalse%
\ map{\isacharunderscore}{\kern0pt}edges{\isacharcolon}{\kern0pt}\ {\isachardoublequoteopen}map\ mk{\isacharunderscore}{\kern0pt}edge\ {\isacharquery}{\kern0pt}edge{\isacharunderscore}{\kern0pt}list\ {\isacharequal}{\kern0pt}\ edges{\isachardoublequoteclose}\ \isacommand{unfolding}\isamarkupfalse%
\ map{\isacharunderscore}{\kern0pt}map\ \isacommand{using}\isamarkupfalse%
\ map{\isacharunderscore}{\kern0pt}idI\ set{\isacharunderscore}{\kern0pt}edges\ \isacommand{by}\isamarkupfalse%
\ blast\isanewline
\ \ \isacommand{then}\isamarkupfalse%
\ \isacommand{have}\isamarkupfalse%
\ edge{\isacharunderscore}{\kern0pt}list{\isacharcolon}{\kern0pt}\ {\isachardoublequoteopen}edges{\isacharunderscore}{\kern0pt}of{\isacharunderscore}{\kern0pt}edge{\isacharunderscore}{\kern0pt}list\ {\isacharquery}{\kern0pt}edge{\isacharunderscore}{\kern0pt}list\ {\isacharequal}{\kern0pt}\ E{\isachardoublequoteclose}\ \isacommand{unfolding}\isamarkupfalse%
\ edges{\isacharunderscore}{\kern0pt}of{\isacharunderscore}{\kern0pt}edge{\isacharunderscore}{\kern0pt}list{\isacharunderscore}{\kern0pt}def\ \isacommand{using}\isamarkupfalse%
\ set{\isacharunderscore}{\kern0pt}edges\ set{\isacharunderscore}{\kern0pt}map\ \isacommand{by}\isamarkupfalse%
\ metis\isanewline
\ \ \isacommand{have}\isamarkupfalse%
\ distinct{\isacharunderscore}{\kern0pt}edge{\isacharunderscore}{\kern0pt}list{\isacharcolon}{\kern0pt}\ {\isachardoublequoteopen}distinct\ {\isacharparenleft}{\kern0pt}map\ mk{\isacharunderscore}{\kern0pt}edge\ {\isacharquery}{\kern0pt}edge{\isacharunderscore}{\kern0pt}list{\isacharparenright}{\kern0pt}{\isachardoublequoteclose}\ \isacommand{using}\isamarkupfalse%
\ distinct{\isacharunderscore}{\kern0pt}edges\ map{\isacharunderscore}{\kern0pt}edges\ \isacommand{by}\isamarkupfalse%
\ metis\isanewline
\ \ \isanewline
\ \ \isacommand{then}\isamarkupfalse%
\ \isacommand{interpret}\isamarkupfalse%
\ contxt{\isacharcolon}{\kern0pt}\ tree{\isacharunderscore}{\kern0pt}to{\isacharunderscore}{\kern0pt}prufer{\isacharunderscore}{\kern0pt}seq{\isacharunderscore}{\kern0pt}context\ verts\ {\isacharquery}{\kern0pt}edge{\isacharunderscore}{\kern0pt}list\ \isacommand{using}\isamarkupfalse%
\ t\ tree{\isacharunderscore}{\kern0pt}to{\isacharunderscore}{\kern0pt}prufer{\isacharunderscore}{\kern0pt}seq{\isacharunderscore}{\kern0pt}context{\isachardot}{\kern0pt}intro\ distinct{\isacharunderscore}{\kern0pt}verts\ edge{\isacharunderscore}{\kern0pt}list\ card{\isacharunderscore}{\kern0pt}verts\ \isacommand{by}\isamarkupfalse%
\ blast\isanewline
\ \ \isacommand{show}\isamarkupfalse%
\ {\isacharquery}{\kern0pt}thesis\isanewline
\ \ \ \ \isacommand{using}\isamarkupfalse%
\ contxt{\isachardot}{\kern0pt}pruf{\isacharunderscore}{\kern0pt}seq{\isacharunderscore}{\kern0pt}tree{\isacharunderscore}{\kern0pt}to{\isacharunderscore}{\kern0pt}prufer{\isacharunderscore}{\kern0pt}seq\ contxt{\isachardot}{\kern0pt}pruf{\isacharunderscore}{\kern0pt}seq{\isacharunderscore}{\kern0pt}to{\isacharunderscore}{\kern0pt}tree{\isacharunderscore}{\kern0pt}to{\isacharunderscore}{\kern0pt}pruf{\isacharunderscore}{\kern0pt}seq\ n{\isacharunderscore}{\kern0pt}sequence{\isacharunderscore}{\kern0pt}enum{\isacharunderscore}{\kern0pt}correct\ distinct{\isacharunderscore}{\kern0pt}edge{\isacharunderscore}{\kern0pt}list\ edge{\isacharunderscore}{\kern0pt}list\isanewline
\ \ \ \ \isacommand{unfolding}\isamarkupfalse%
\ prufer{\isacharunderscore}{\kern0pt}sequences{\isacharunderscore}{\kern0pt}def\ prufer{\isacharunderscore}{\kern0pt}seq{\isacharunderscore}{\kern0pt}to{\isacharunderscore}{\kern0pt}tree{\isacharunderscore}{\kern0pt}def\ labeled{\isacharunderscore}{\kern0pt}tree{\isacharunderscore}{\kern0pt}enum{\isacharunderscore}{\kern0pt}def\ \isacommand{by}\isamarkupfalse%
\ auto\isanewline
\isacommand{qed}\isamarkupfalse%
%
\endisatagproof
{\isafoldproof}%
%
\isadelimproof
\isanewline
%
\endisadelimproof
\isanewline
\isacommand{lemma}\isamarkupfalse%
\ V{\isacharunderscore}{\kern0pt}labeled{\isacharunderscore}{\kern0pt}tree{\isacharunderscore}{\kern0pt}enum{\isacharunderscore}{\kern0pt}verts{\isacharcolon}{\kern0pt}\ {\isachardoublequoteopen}{\isacharparenleft}{\kern0pt}V{\isacharcomma}{\kern0pt}E{\isacharparenright}{\kern0pt}\ {\isasymin}\ set\ {\isacharparenleft}{\kern0pt}labeled{\isacharunderscore}{\kern0pt}tree{\isacharunderscore}{\kern0pt}enum\ verts{\isacharparenright}{\kern0pt}\ {\isasymLongrightarrow}\ V\ {\isacharequal}{\kern0pt}\ set\ verts{\isachardoublequoteclose}\isanewline
%
\isadelimproof
\ \ %
\endisadelimproof
%
\isatagproof
\isacommand{unfolding}\isamarkupfalse%
\ labeled{\isacharunderscore}{\kern0pt}tree{\isacharunderscore}{\kern0pt}enum{\isacharunderscore}{\kern0pt}def\ \isacommand{by}\isamarkupfalse%
\ {\isacharparenleft}{\kern0pt}metis\ Pair{\isacharunderscore}{\kern0pt}inject\ ex{\isacharunderscore}{\kern0pt}map{\isacharunderscore}{\kern0pt}conv\ prufer{\isacharunderscore}{\kern0pt}seq{\isacharunderscore}{\kern0pt}to{\isacharunderscore}{\kern0pt}tree{\isacharunderscore}{\kern0pt}def{\isacharparenright}{\kern0pt}%
\endisatagproof
{\isafoldproof}%
%
\isadelimproof
\isanewline
%
\endisadelimproof
\isanewline
\isacommand{theorem}\isamarkupfalse%
\ labeled{\isacharunderscore}{\kern0pt}tree{\isacharunderscore}{\kern0pt}enum{\isacharunderscore}{\kern0pt}correct{\isacharcolon}{\kern0pt}\ {\isachardoublequoteopen}set\ {\isacharparenleft}{\kern0pt}labeled{\isacharunderscore}{\kern0pt}tree{\isacharunderscore}{\kern0pt}enum\ verts{\isacharparenright}{\kern0pt}\ {\isacharequal}{\kern0pt}\ labeled{\isacharunderscore}{\kern0pt}trees\ {\isacharparenleft}{\kern0pt}set\ verts{\isacharparenright}{\kern0pt}{\isachardoublequoteclose}\isanewline
%
\isadelimproof
\ \ %
\endisadelimproof
%
\isatagproof
\isacommand{using}\isamarkupfalse%
\ labeled{\isacharunderscore}{\kern0pt}tree{\isacharunderscore}{\kern0pt}enum{\isacharunderscore}{\kern0pt}tree\ V{\isacharunderscore}{\kern0pt}labeled{\isacharunderscore}{\kern0pt}tree{\isacharunderscore}{\kern0pt}enum{\isacharunderscore}{\kern0pt}verts\ tree{\isacharunderscore}{\kern0pt}labeled{\isacharunderscore}{\kern0pt}tree{\isacharunderscore}{\kern0pt}enum\ \isacommand{unfolding}\isamarkupfalse%
\ labeled{\isacharunderscore}{\kern0pt}trees{\isacharunderscore}{\kern0pt}def\ \isacommand{by}\isamarkupfalse%
\ auto%
\endisatagproof
{\isafoldproof}%
%
\isadelimproof
%
\endisadelimproof
%
\isadelimdocument
%
\endisadelimdocument
%
\isatagdocument
%
\isamarkupsubsection{Distinctness%
}
\isamarkuptrue%
%
\endisatagdocument
{\isafolddocument}%
%
\isadelimdocument
%
\endisadelimdocument
\isacommand{lemma}\isamarkupfalse%
\ count{\isacharunderscore}{\kern0pt}list{\isacharunderscore}{\kern0pt}degree{\isacharcolon}{\kern0pt}\ {\isachardoublequoteopen}seq\ {\isasymin}\ prufer{\isacharunderscore}{\kern0pt}sequences\ verts\ {\isasymLongrightarrow}\ v\ {\isasymin}\ set\ verts\ {\isasymLongrightarrow}\ Suc\ {\isacharparenleft}{\kern0pt}count{\isacharunderscore}{\kern0pt}list\ seq\ v{\isacharparenright}{\kern0pt}\ {\isacharequal}{\kern0pt}\ degree\ v\ {\isacharparenleft}{\kern0pt}prufer{\isacharunderscore}{\kern0pt}seq{\isacharunderscore}{\kern0pt}to{\isacharunderscore}{\kern0pt}tree{\isacharunderscore}{\kern0pt}edges\ verts\ seq{\isacharparenright}{\kern0pt}{\isachardoublequoteclose}\isanewline
%
\isadelimproof
\ \ %
\endisadelimproof
%
\isatagproof
\isacommand{using}\isamarkupfalse%
\ verts{\isacharunderscore}{\kern0pt}length\ distinct{\isacharunderscore}{\kern0pt}verts\isanewline
\isacommand{proof}\isamarkupfalse%
\ {\isacharparenleft}{\kern0pt}induction\ verts\ seq\ rule{\isacharcolon}{\kern0pt}\ prufer{\isacharunderscore}{\kern0pt}seq{\isacharunderscore}{\kern0pt}to{\isacharunderscore}{\kern0pt}tree{\isacharunderscore}{\kern0pt}edges{\isachardot}{\kern0pt}induct{\isacharparenright}{\kern0pt}\isanewline
\ \ \isacommand{case}\isamarkupfalse%
\ {\isacharparenleft}{\kern0pt}{\isadigit{1}}\ u\ w{\isacharparenright}{\kern0pt}\isanewline
\ \ \isacommand{then}\isamarkupfalse%
\ \isacommand{show}\isamarkupfalse%
\ {\isacharquery}{\kern0pt}case\ \isacommand{unfolding}\isamarkupfalse%
\ incident{\isacharunderscore}{\kern0pt}edges{\isacharunderscore}{\kern0pt}def\ \isacommand{by}\isamarkupfalse%
\ auto\isanewline
\isacommand{next}\isamarkupfalse%
\isanewline
\ \ \isacommand{case}\isamarkupfalse%
\ {\isacharparenleft}{\kern0pt}{\isadigit{2}}\ verts\ a\ seq{\isacharparenright}{\kern0pt}\isanewline
\ \ \isacommand{then}\isamarkupfalse%
\ \isacommand{interpret}\isamarkupfalse%
\ contxt{\isacharcolon}{\kern0pt}\ prufer{\isacharunderscore}{\kern0pt}seq{\isacharunderscore}{\kern0pt}to{\isacharunderscore}{\kern0pt}tree{\isacharunderscore}{\kern0pt}context\ verts\ \isacommand{by}\isamarkupfalse%
\ unfold{\isacharunderscore}{\kern0pt}locales\isanewline
\ \ \isacommand{obtain}\isamarkupfalse%
\ leaf\isanewline
\ \ \ \ \isakeyword{where}\ leaf{\isacharunderscore}{\kern0pt}find{\isacharcolon}{\kern0pt}\ {\isachardoublequoteopen}find\ {\isacharparenleft}{\kern0pt}{\isasymlambda}x{\isachardot}{\kern0pt}\ x\ {\isasymnotin}\ set\ {\isacharparenleft}{\kern0pt}a\ {\isacharhash}{\kern0pt}\ seq{\isacharparenright}{\kern0pt}{\isacharparenright}{\kern0pt}\ verts\ {\isacharequal}{\kern0pt}\ Some\ leaf{\isachardoublequoteclose}\isanewline
\ \ \ \ \ \ \isakeyword{and}\ leaf{\isacharunderscore}{\kern0pt}not{\isacharunderscore}{\kern0pt}in{\isacharunderscore}{\kern0pt}seq{\isacharcolon}{\kern0pt}\ {\isachardoublequoteopen}leaf\ {\isasymnotin}\ set\ {\isacharparenleft}{\kern0pt}a{\isacharhash}{\kern0pt}seq{\isacharparenright}{\kern0pt}{\isachardoublequoteclose}\isanewline
\ \ \ \ \ \ \isakeyword{and}\ seq{\isacharunderscore}{\kern0pt}in{\isacharunderscore}{\kern0pt}verts{\isacharprime}{\kern0pt}{\isacharcolon}{\kern0pt}\ {\isachardoublequoteopen}seq\ {\isasymin}\ prufer{\isacharunderscore}{\kern0pt}sequences\ {\isacharparenleft}{\kern0pt}remove{\isadigit{1}}\ leaf\ verts{\isacharparenright}{\kern0pt}{\isachardoublequoteclose}\isanewline
\ \ \ \ \ \ \isakeyword{and}\ len{\isacharunderscore}{\kern0pt}verts{\isacharprime}{\kern0pt}{\isacharcolon}{\kern0pt}\ {\isachardoublequoteopen}{\isadigit{2}}\ {\isasymle}\ length\ {\isacharparenleft}{\kern0pt}remove{\isadigit{1}}\ leaf\ verts{\isacharparenright}{\kern0pt}{\isachardoublequoteclose}\isanewline
\ \ \ \ \ \ \isakeyword{and}\ distinct{\isacharunderscore}{\kern0pt}verts{\isacharprime}{\kern0pt}{\isacharcolon}{\kern0pt}\ {\isachardoublequoteopen}distinct\ {\isacharparenleft}{\kern0pt}remove{\isadigit{1}}\ leaf\ verts{\isacharparenright}{\kern0pt}{\isachardoublequoteclose}\isanewline
\ \ \ \ \ \ \isakeyword{and}\ leaf{\isacharunderscore}{\kern0pt}in{\isacharunderscore}{\kern0pt}verts{\isacharcolon}{\kern0pt}\ {\isachardoublequoteopen}leaf\ {\isasymin}\ set\ verts{\isachardoublequoteclose}\ \isacommand{using}\isamarkupfalse%
\ contxt{\isachardot}{\kern0pt}obtain{\isacharunderscore}{\kern0pt}b{\isacharunderscore}{\kern0pt}prufer{\isacharunderscore}{\kern0pt}seq{\isacharunderscore}{\kern0pt}to{\isacharunderscore}{\kern0pt}tree{\isacharunderscore}{\kern0pt}edges\ {\isadigit{2}}\ \isacommand{by}\isamarkupfalse%
\ blast\isanewline
\ \ \isacommand{interpret}\isamarkupfalse%
\ contxt{\isacharprime}{\kern0pt}{\isacharcolon}{\kern0pt}\ prufer{\isacharunderscore}{\kern0pt}seq{\isacharunderscore}{\kern0pt}to{\isacharunderscore}{\kern0pt}tree{\isacharunderscore}{\kern0pt}context\ {\isachardoublequoteopen}remove{\isadigit{1}}\ leaf\ verts{\isachardoublequoteclose}\ \isacommand{using}\isamarkupfalse%
\ len{\isacharunderscore}{\kern0pt}verts{\isacharprime}{\kern0pt}\ distinct{\isacharunderscore}{\kern0pt}verts{\isacharprime}{\kern0pt}\ \isacommand{by}\isamarkupfalse%
\ unfold{\isacharunderscore}{\kern0pt}locales\isanewline
\ \ \isacommand{show}\isamarkupfalse%
\ {\isacharquery}{\kern0pt}case\isanewline
\ \ \isacommand{proof}\isamarkupfalse%
\ {\isacharparenleft}{\kern0pt}cases\ {\isachardoublequoteopen}v\ {\isacharequal}{\kern0pt}\ leaf{\isachardoublequoteclose}{\isacharparenright}{\kern0pt}\isanewline
\ \ \ \ \isacommand{case}\isamarkupfalse%
\ True\isanewline
\ \ \ \ \isacommand{then}\isamarkupfalse%
\ \isacommand{have}\isamarkupfalse%
\ {\isachardoublequoteopen}a\ {\isasymnoteq}\ leaf{\isachardoublequoteclose}\ \isacommand{using}\isamarkupfalse%
\ {\isadigit{2}}\ leaf{\isacharunderscore}{\kern0pt}not{\isacharunderscore}{\kern0pt}in{\isacharunderscore}{\kern0pt}seq\ \isacommand{by}\isamarkupfalse%
\ auto\isanewline
\ \ \ \ \isacommand{interpret}\isamarkupfalse%
\ t{\isacharcolon}{\kern0pt}\ tree\ {\isachardoublequoteopen}set\ {\isacharparenleft}{\kern0pt}remove{\isadigit{1}}\ leaf\ verts{\isacharparenright}{\kern0pt}{\isachardoublequoteclose}\ {\isachardoublequoteopen}edges{\isacharunderscore}{\kern0pt}of{\isacharunderscore}{\kern0pt}edge{\isacharunderscore}{\kern0pt}list\ {\isacharparenleft}{\kern0pt}prufer{\isacharunderscore}{\kern0pt}seq{\isacharunderscore}{\kern0pt}to{\isacharunderscore}{\kern0pt}tree{\isacharunderscore}{\kern0pt}edges\ {\isacharparenleft}{\kern0pt}remove{\isadigit{1}}\ leaf\ verts{\isacharparenright}{\kern0pt}\ seq{\isacharparenright}{\kern0pt}{\isachardoublequoteclose}\isanewline
\ \ \ \ \ \ \isacommand{using}\isamarkupfalse%
\ contxt{\isacharprime}{\kern0pt}{\isachardot}{\kern0pt}prufer{\isacharunderscore}{\kern0pt}seq{\isacharunderscore}{\kern0pt}to{\isacharunderscore}{\kern0pt}tree{\isacharunderscore}{\kern0pt}edges{\isacharunderscore}{\kern0pt}tree\ seq{\isacharunderscore}{\kern0pt}in{\isacharunderscore}{\kern0pt}verts{\isacharprime}{\kern0pt}\ \isacommand{by}\isamarkupfalse%
\ auto\isanewline
\ \ \ \ \isacommand{have}\isamarkupfalse%
\ {\isacharbrackleft}{\kern0pt}simp{\isacharbrackright}{\kern0pt}{\isacharcolon}{\kern0pt}\ {\isachardoublequoteopen}set\ {\isacharparenleft}{\kern0pt}remove{\isadigit{1}}\ leaf\ verts{\isacharparenright}{\kern0pt}\ {\isacharequal}{\kern0pt}\ set\ verts\ {\isacharminus}{\kern0pt}\ {\isacharbraceleft}{\kern0pt}leaf{\isacharbraceright}{\kern0pt}{\isachardoublequoteclose}\ \isacommand{using}\isamarkupfalse%
\ set{\isacharunderscore}{\kern0pt}remove{\isadigit{1}}{\isacharunderscore}{\kern0pt}eq\ {\isadigit{2}}\ \isacommand{by}\isamarkupfalse%
\ auto\isanewline
\ \ \ \ \isacommand{then}\isamarkupfalse%
\ \isacommand{have}\isamarkupfalse%
\ {\isachardoublequoteopen}{\isasymforall}{\isacharparenleft}{\kern0pt}u{\isacharcomma}{\kern0pt}w{\isacharparenright}{\kern0pt}{\isasymin}set\ {\isacharparenleft}{\kern0pt}prufer{\isacharunderscore}{\kern0pt}seq{\isacharunderscore}{\kern0pt}to{\isacharunderscore}{\kern0pt}tree{\isacharunderscore}{\kern0pt}edges\ {\isacharparenleft}{\kern0pt}remove{\isadigit{1}}\ leaf\ verts{\isacharparenright}{\kern0pt}\ seq{\isacharparenright}{\kern0pt}{\isachardot}{\kern0pt}\ u\ {\isasymnoteq}\ leaf\ {\isasymand}\ w\ {\isasymnoteq}\ leaf{\isachardoublequoteclose}\isanewline
\ \ \ \ \ \ \isacommand{using}\isamarkupfalse%
\ t{\isachardot}{\kern0pt}wellformed\ in{\isacharunderscore}{\kern0pt}mk{\isacharunderscore}{\kern0pt}edge{\isacharunderscore}{\kern0pt}img\ \isacommand{unfolding}\isamarkupfalse%
\ edges{\isacharunderscore}{\kern0pt}of{\isacharunderscore}{\kern0pt}edge{\isacharunderscore}{\kern0pt}list{\isacharunderscore}{\kern0pt}def\ \isacommand{apply}\isamarkupfalse%
\ auto\ \isacommand{by}\isamarkupfalse%
\ fast{\isacharplus}{\kern0pt}\isanewline
\ \ \ \ \isacommand{then}\isamarkupfalse%
\ \isacommand{have}\isamarkupfalse%
\ {\isachardoublequoteopen}degree\ v\ {\isacharparenleft}{\kern0pt}prufer{\isacharunderscore}{\kern0pt}seq{\isacharunderscore}{\kern0pt}to{\isacharunderscore}{\kern0pt}tree{\isacharunderscore}{\kern0pt}edges\ {\isacharparenleft}{\kern0pt}remove{\isadigit{1}}\ leaf\ verts{\isacharparenright}{\kern0pt}\ seq{\isacharparenright}{\kern0pt}\ {\isacharequal}{\kern0pt}\ {\isadigit{0}}{\isachardoublequoteclose}\isanewline
\ \ \ \ \ \ \isacommand{unfolding}\isamarkupfalse%
\ incident{\isacharunderscore}{\kern0pt}edges{\isacharunderscore}{\kern0pt}def\ filter{\isacharunderscore}{\kern0pt}False\ True\ \isacommand{by}\isamarkupfalse%
\ {\isacharparenleft}{\kern0pt}auto\ split{\isacharcolon}{\kern0pt}\ prod{\isachardot}{\kern0pt}splits{\isacharparenright}{\kern0pt}\isanewline
\ \ \ \ \isacommand{then}\isamarkupfalse%
\ \isacommand{show}\isamarkupfalse%
\ {\isacharquery}{\kern0pt}thesis\ \isacommand{using}\isamarkupfalse%
\ {\isacartoucheopen}a{\isasymnoteq}leaf{\isacartoucheclose}\ True\ leaf{\isacharunderscore}{\kern0pt}find\ leaf{\isacharunderscore}{\kern0pt}not{\isacharunderscore}{\kern0pt}in{\isacharunderscore}{\kern0pt}seq\ \isacommand{unfolding}\isamarkupfalse%
\ incident{\isacharunderscore}{\kern0pt}edges{\isacharunderscore}{\kern0pt}def\ \isacommand{by}\isamarkupfalse%
\ simp\isanewline
\ \ \isacommand{next}\isamarkupfalse%
\isanewline
\ \ \ \ \isacommand{case}\isamarkupfalse%
\ False\isanewline
\ \ \ \ \isacommand{then}\isamarkupfalse%
\ \isacommand{show}\isamarkupfalse%
\ {\isacharquery}{\kern0pt}thesis\ \isacommand{using}\isamarkupfalse%
\ {\isadigit{2}}\ leaf{\isacharunderscore}{\kern0pt}find\ seq{\isacharunderscore}{\kern0pt}in{\isacharunderscore}{\kern0pt}verts{\isacharprime}{\kern0pt}\ len{\isacharunderscore}{\kern0pt}verts{\isacharprime}{\kern0pt}\ \isacommand{unfolding}\isamarkupfalse%
\ incident{\isacharunderscore}{\kern0pt}edges{\isacharunderscore}{\kern0pt}def\ \isacommand{by}\isamarkupfalse%
\ auto\isanewline
\ \ \isacommand{qed}\isamarkupfalse%
\isanewline
\isacommand{qed}\isamarkupfalse%
\ {\isacharparenleft}{\kern0pt}auto\ simp{\isacharcolon}{\kern0pt}\ prufer{\isacharunderscore}{\kern0pt}sequences{\isacharunderscore}{\kern0pt}def\ n{\isacharunderscore}{\kern0pt}sequences{\isacharunderscore}{\kern0pt}def{\isacharparenright}{\kern0pt}%
\endisatagproof
{\isafoldproof}%
%
\isadelimproof
\isanewline
%
\endisadelimproof
\isanewline
\isacommand{lemma}\isamarkupfalse%
\ vert{\isacharunderscore}{\kern0pt}notin{\isacharunderscore}{\kern0pt}pruf{\isacharunderscore}{\kern0pt}seq{\isacharunderscore}{\kern0pt}leaf{\isacharcolon}{\kern0pt}\ {\isachardoublequoteopen}seq\ {\isasymin}\ prufer{\isacharunderscore}{\kern0pt}sequences\ verts\ {\isasymLongrightarrow}\ v\ {\isasymin}\ set\ verts\ {\isasymLongrightarrow}\ v\ {\isasymnotin}\ set\ seq\ {\isasymlongleftrightarrow}\ degree\ v\ {\isacharparenleft}{\kern0pt}prufer{\isacharunderscore}{\kern0pt}seq{\isacharunderscore}{\kern0pt}to{\isacharunderscore}{\kern0pt}tree{\isacharunderscore}{\kern0pt}edges\ verts\ seq{\isacharparenright}{\kern0pt}\ {\isacharequal}{\kern0pt}\ {\isadigit{1}}{\isachardoublequoteclose}\isanewline
%
\isadelimproof
\ \ %
\endisadelimproof
%
\isatagproof
\isacommand{using}\isamarkupfalse%
\ count{\isacharunderscore}{\kern0pt}list{\isacharunderscore}{\kern0pt}degree\ count{\isacharunderscore}{\kern0pt}list{\isacharunderscore}{\kern0pt}zero{\isacharunderscore}{\kern0pt}not{\isacharunderscore}{\kern0pt}elem\ \isacommand{by}\isamarkupfalse%
\ fastforce%
\endisatagproof
{\isafoldproof}%
%
\isadelimproof
\isanewline
%
\endisadelimproof
\isanewline
\isacommand{lemma}\isamarkupfalse%
\ inj{\isacharunderscore}{\kern0pt}prufer{\isacharunderscore}{\kern0pt}seq{\isacharunderscore}{\kern0pt}to{\isacharunderscore}{\kern0pt}tree{\isacharunderscore}{\kern0pt}edges{\isacharcolon}{\kern0pt}\isanewline
\ \ \isakeyword{assumes}\ pruf{\isacharunderscore}{\kern0pt}seq{\isadigit{1}}{\isacharcolon}{\kern0pt}\ {\isachardoublequoteopen}seq{\isadigit{1}}\ {\isasymin}\ prufer{\isacharunderscore}{\kern0pt}sequences\ verts{\isachardoublequoteclose}\isanewline
\ \ \ \ \isakeyword{and}\ pruf{\isacharunderscore}{\kern0pt}seq{\isadigit{2}}{\isacharcolon}{\kern0pt}\ {\isachardoublequoteopen}seq{\isadigit{2}}\ {\isasymin}\ prufer{\isacharunderscore}{\kern0pt}sequences\ verts{\isachardoublequoteclose}\isanewline
\ \ \ \ \isakeyword{and}\ seq{\isacharunderscore}{\kern0pt}ne{\isacharcolon}{\kern0pt}\ {\isachardoublequoteopen}seq{\isadigit{1}}\ {\isasymnoteq}\ seq{\isadigit{2}}{\isachardoublequoteclose}\isanewline
\ \ \isakeyword{shows}\ {\isachardoublequoteopen}edges{\isacharunderscore}{\kern0pt}of{\isacharunderscore}{\kern0pt}edge{\isacharunderscore}{\kern0pt}list\ {\isacharparenleft}{\kern0pt}prufer{\isacharunderscore}{\kern0pt}seq{\isacharunderscore}{\kern0pt}to{\isacharunderscore}{\kern0pt}tree{\isacharunderscore}{\kern0pt}edges\ verts\ seq{\isadigit{1}}{\isacharparenright}{\kern0pt}\ {\isasymnoteq}\ edges{\isacharunderscore}{\kern0pt}of{\isacharunderscore}{\kern0pt}edge{\isacharunderscore}{\kern0pt}list\ {\isacharparenleft}{\kern0pt}prufer{\isacharunderscore}{\kern0pt}seq{\isacharunderscore}{\kern0pt}to{\isacharunderscore}{\kern0pt}tree{\isacharunderscore}{\kern0pt}edges\ verts\ seq{\isadigit{2}}{\isacharparenright}{\kern0pt}{\isachardoublequoteclose}\ {\isacharparenleft}{\kern0pt}\isakeyword{is}\ {\isachardoublequoteopen}{\isacharquery}{\kern0pt}l\ {\isasymnoteq}\ {\isacharquery}{\kern0pt}r{\isachardoublequoteclose}{\isacharparenright}{\kern0pt}\isanewline
%
\isadelimproof
%
\endisadelimproof
%
\isatagproof
\isacommand{proof}\isamarkupfalse%
\isanewline
\ \ \isacommand{assume}\isamarkupfalse%
\ trees{\isacharunderscore}{\kern0pt}eq{\isacharcolon}{\kern0pt}\ {\isachardoublequoteopen}{\isacharquery}{\kern0pt}l\ {\isacharequal}{\kern0pt}\ {\isacharquery}{\kern0pt}r{\isachardoublequoteclose}\isanewline
\ \ \isacommand{have}\isamarkupfalse%
\ {\isachardoublequoteopen}length\ seq{\isadigit{1}}\ {\isacharequal}{\kern0pt}\ length\ seq{\isadigit{2}}{\isachardoublequoteclose}\ \isacommand{using}\isamarkupfalse%
\ pruf{\isacharunderscore}{\kern0pt}seq{\isadigit{1}}\ pruf{\isacharunderscore}{\kern0pt}seq{\isadigit{2}}\ \isacommand{unfolding}\isamarkupfalse%
\ prufer{\isacharunderscore}{\kern0pt}sequences{\isacharunderscore}{\kern0pt}def\ n{\isacharunderscore}{\kern0pt}sequences{\isacharunderscore}{\kern0pt}def\ \isacommand{by}\isamarkupfalse%
\ simp\isanewline
\ \ \isacommand{then}\isamarkupfalse%
\ \isacommand{show}\isamarkupfalse%
\ False\isanewline
\ \ \ \ \isacommand{using}\isamarkupfalse%
\ assms\ {\isacartoucheopen}{\isacharquery}{\kern0pt}l\ {\isacharequal}{\kern0pt}\ {\isacharquery}{\kern0pt}r{\isacartoucheclose}\ prufer{\isacharunderscore}{\kern0pt}seq{\isacharunderscore}{\kern0pt}to{\isacharunderscore}{\kern0pt}tree{\isacharunderscore}{\kern0pt}context{\isacharunderscore}{\kern0pt}axioms\isanewline
\ \ \isacommand{proof}\isamarkupfalse%
\ {\isacharparenleft}{\kern0pt}induction\ seq{\isadigit{1}}\ seq{\isadigit{2}}\ arbitrary{\isacharcolon}{\kern0pt}\ verts\ rule{\isacharcolon}{\kern0pt}\ list{\isacharunderscore}{\kern0pt}induct{\isadigit{2}}{\isacharparenright}{\kern0pt}\isanewline
\ \ \ \ \isacommand{case}\isamarkupfalse%
\ Nil\isanewline
\ \ \ \ \isacommand{then}\isamarkupfalse%
\ \isacommand{show}\isamarkupfalse%
\ {\isacharquery}{\kern0pt}case\ \isacommand{by}\isamarkupfalse%
\ simp\isanewline
\ \ \isacommand{next}\isamarkupfalse%
\isanewline
\ \ \ \ \isacommand{case}\isamarkupfalse%
\ {\isacharparenleft}{\kern0pt}Cons\ x\ xs\ y\ ys{\isacharparenright}{\kern0pt}\isanewline
\ \ \ \ \isacommand{then}\isamarkupfalse%
\ \isacommand{interpret}\isamarkupfalse%
\ prufer{\isacharunderscore}{\kern0pt}seq{\isacharunderscore}{\kern0pt}to{\isacharunderscore}{\kern0pt}tree{\isacharunderscore}{\kern0pt}context\ verts\ \isacommand{by}\isamarkupfalse%
\ simp\isanewline
\ \ \ \ \isacommand{interpret}\isamarkupfalse%
\ t{\isadigit{1}}{\isacharcolon}{\kern0pt}\ tree\ {\isachardoublequoteopen}set\ verts{\isachardoublequoteclose}\ {\isachardoublequoteopen}edges{\isacharunderscore}{\kern0pt}of{\isacharunderscore}{\kern0pt}edge{\isacharunderscore}{\kern0pt}list\ {\isacharparenleft}{\kern0pt}prufer{\isacharunderscore}{\kern0pt}seq{\isacharunderscore}{\kern0pt}to{\isacharunderscore}{\kern0pt}tree{\isacharunderscore}{\kern0pt}edges\ verts\ {\isacharparenleft}{\kern0pt}x{\isacharhash}{\kern0pt}xs{\isacharparenright}{\kern0pt}{\isacharparenright}{\kern0pt}{\isachardoublequoteclose}\ \isacommand{using}\isamarkupfalse%
\ Cons{\isacharparenleft}{\kern0pt}{\isadigit{3}}{\isacharparenright}{\kern0pt}\ prufer{\isacharunderscore}{\kern0pt}seq{\isacharunderscore}{\kern0pt}to{\isacharunderscore}{\kern0pt}tree{\isacharunderscore}{\kern0pt}edges{\isacharunderscore}{\kern0pt}tree\ \isacommand{by}\isamarkupfalse%
\ fastforce\isanewline
\ \ \ \ \isacommand{interpret}\isamarkupfalse%
\ t{\isadigit{2}}{\isacharcolon}{\kern0pt}\ tree\ {\isachardoublequoteopen}set\ verts{\isachardoublequoteclose}\ {\isachardoublequoteopen}edges{\isacharunderscore}{\kern0pt}of{\isacharunderscore}{\kern0pt}edge{\isacharunderscore}{\kern0pt}list\ {\isacharparenleft}{\kern0pt}prufer{\isacharunderscore}{\kern0pt}seq{\isacharunderscore}{\kern0pt}to{\isacharunderscore}{\kern0pt}tree{\isacharunderscore}{\kern0pt}edges\ verts\ {\isacharparenleft}{\kern0pt}y{\isacharhash}{\kern0pt}ys{\isacharparenright}{\kern0pt}{\isacharparenright}{\kern0pt}{\isachardoublequoteclose}\ \isacommand{using}\isamarkupfalse%
\ Cons{\isacharparenleft}{\kern0pt}{\isadigit{4}}{\isacharparenright}{\kern0pt}\ prufer{\isacharunderscore}{\kern0pt}seq{\isacharunderscore}{\kern0pt}to{\isacharunderscore}{\kern0pt}tree{\isacharunderscore}{\kern0pt}edges{\isacharunderscore}{\kern0pt}tree\ \isacommand{by}\isamarkupfalse%
\ fastforce\isanewline
\ \ \ \ \isacommand{obtain}\isamarkupfalse%
\ leaf\ \isakeyword{where}\ find{\isacharunderscore}{\kern0pt}leaf{\isacharcolon}{\kern0pt}\ {\isachardoublequoteopen}find\ {\isacharparenleft}{\kern0pt}{\isasymlambda}v{\isachardot}{\kern0pt}\ v\ {\isasymnotin}\ set\ {\isacharparenleft}{\kern0pt}x{\isacharhash}{\kern0pt}xs{\isacharparenright}{\kern0pt}{\isacharparenright}{\kern0pt}\ verts\ {\isacharequal}{\kern0pt}\ Some\ leaf{\isachardoublequoteclose}\isanewline
\ \ \ \ \ \ \isakeyword{and}\ pruf{\isacharunderscore}{\kern0pt}seq{\isadigit{1}}{\isacharprime}{\kern0pt}{\isacharcolon}{\kern0pt}\ {\isachardoublequoteopen}xs\ {\isasymin}\ prufer{\isacharunderscore}{\kern0pt}sequences\ {\isacharparenleft}{\kern0pt}remove{\isadigit{1}}\ leaf\ verts{\isacharparenright}{\kern0pt}{\isachardoublequoteclose}\isanewline
\ \ \ \ \ \ \isakeyword{and}\ {\isachardoublequoteopen}length\ {\isacharparenleft}{\kern0pt}remove{\isadigit{1}}\ leaf\ verts{\isacharparenright}{\kern0pt}\ {\isasymge}\ {\isadigit{2}}{\isachardoublequoteclose}\isanewline
\ \ \ \ \ \ \isakeyword{and}\ {\isachardoublequoteopen}distinct\ {\isacharparenleft}{\kern0pt}remove{\isadigit{1}}\ leaf\ verts{\isacharparenright}{\kern0pt}{\isachardoublequoteclose}\ \isacommand{using}\isamarkupfalse%
\ obtain{\isacharunderscore}{\kern0pt}b{\isacharunderscore}{\kern0pt}prufer{\isacharunderscore}{\kern0pt}seq{\isacharunderscore}{\kern0pt}to{\isacharunderscore}{\kern0pt}tree{\isacharunderscore}{\kern0pt}edges\ Cons{\isacharparenleft}{\kern0pt}{\isadigit{3}}{\isacharparenright}{\kern0pt}\ \isacommand{by}\isamarkupfalse%
\ blast\isanewline
\ \ \ \ \isacommand{then}\isamarkupfalse%
\ \isacommand{interpret}\isamarkupfalse%
\ contxt{\isacharprime}{\kern0pt}{\isacharcolon}{\kern0pt}\ prufer{\isacharunderscore}{\kern0pt}seq{\isacharunderscore}{\kern0pt}to{\isacharunderscore}{\kern0pt}tree{\isacharunderscore}{\kern0pt}context\ {\isachardoublequoteopen}remove{\isadigit{1}}\ leaf\ verts{\isachardoublequoteclose}\ \isacommand{by}\isamarkupfalse%
\ {\isacharparenleft}{\kern0pt}unfold{\isacharunderscore}{\kern0pt}locales{\isacharcomma}{\kern0pt}\ simp{\isacharparenright}{\kern0pt}\isanewline
\ \ \ \ \isacommand{obtain}\isamarkupfalse%
\ leaf{\isadigit{2}}\ \isakeyword{where}\ find{\isacharunderscore}{\kern0pt}leaf{\isadigit{2}}{\isacharcolon}{\kern0pt}\ {\isachardoublequoteopen}find\ {\isacharparenleft}{\kern0pt}{\isasymlambda}v{\isachardot}{\kern0pt}\ v\ {\isasymnotin}\ set\ {\isacharparenleft}{\kern0pt}y{\isacharhash}{\kern0pt}ys{\isacharparenright}{\kern0pt}{\isacharparenright}{\kern0pt}\ verts\ {\isacharequal}{\kern0pt}\ Some\ leaf{\isadigit{2}}{\isachardoublequoteclose}\isanewline
\ \ \ \ \ \ \isakeyword{and}\ pruf{\isacharunderscore}{\kern0pt}seq{\isadigit{2}}{\isacharprime}{\kern0pt}{\isacharcolon}{\kern0pt}\ {\isachardoublequoteopen}ys\ {\isasymin}\ prufer{\isacharunderscore}{\kern0pt}sequences\ {\isacharparenleft}{\kern0pt}remove{\isadigit{1}}\ leaf{\isadigit{2}}\ verts{\isacharparenright}{\kern0pt}{\isachardoublequoteclose}\ \isacommand{using}\isamarkupfalse%
\ obtain{\isacharunderscore}{\kern0pt}b{\isacharunderscore}{\kern0pt}prufer{\isacharunderscore}{\kern0pt}seq{\isacharunderscore}{\kern0pt}to{\isacharunderscore}{\kern0pt}tree{\isacharunderscore}{\kern0pt}edges\ Cons{\isacharparenleft}{\kern0pt}{\isadigit{4}}{\isacharparenright}{\kern0pt}\ \isacommand{by}\isamarkupfalse%
\ blast\isanewline
\ \ \ \ \isacommand{interpret}\isamarkupfalse%
\ ttps{\isacharunderscore}{\kern0pt}contxt{\isadigit{1}}{\isacharcolon}{\kern0pt}\ tree{\isacharunderscore}{\kern0pt}to{\isacharunderscore}{\kern0pt}prufer{\isacharunderscore}{\kern0pt}seq{\isacharunderscore}{\kern0pt}context\ verts\ {\isachardoublequoteopen}prufer{\isacharunderscore}{\kern0pt}seq{\isacharunderscore}{\kern0pt}to{\isacharunderscore}{\kern0pt}tree{\isacharunderscore}{\kern0pt}edges\ verts\ {\isacharparenleft}{\kern0pt}x{\isacharhash}{\kern0pt}xs{\isacharparenright}{\kern0pt}{\isachardoublequoteclose}\isanewline
\ \ \ \ \ \ \isacommand{using}\isamarkupfalse%
\ distinct{\isacharunderscore}{\kern0pt}verts\ verts{\isacharunderscore}{\kern0pt}length\ distinct{\isacharunderscore}{\kern0pt}prufer{\isacharunderscore}{\kern0pt}seq{\isacharunderscore}{\kern0pt}to{\isacharunderscore}{\kern0pt}tree{\isacharbrackleft}{\kern0pt}OF\ Cons{\isacharparenleft}{\kern0pt}{\isadigit{3}}{\isacharparenright}{\kern0pt}{\isacharbrackright}{\kern0pt}\ \isacommand{by}\isamarkupfalse%
\ {\isacharparenleft}{\kern0pt}unfold{\isacharunderscore}{\kern0pt}locales{\isacharcomma}{\kern0pt}\ auto\ simp{\isacharcolon}{\kern0pt}\ distinct{\isacharunderscore}{\kern0pt}card{\isacharparenright}{\kern0pt}\isanewline
\ \ \ \ \isacommand{interpret}\isamarkupfalse%
\ ttps{\isacharunderscore}{\kern0pt}contxt{\isadigit{2}}{\isacharcolon}{\kern0pt}\ tree{\isacharunderscore}{\kern0pt}to{\isacharunderscore}{\kern0pt}prufer{\isacharunderscore}{\kern0pt}seq{\isacharunderscore}{\kern0pt}context\ verts\ {\isachardoublequoteopen}prufer{\isacharunderscore}{\kern0pt}seq{\isacharunderscore}{\kern0pt}to{\isacharunderscore}{\kern0pt}tree{\isacharunderscore}{\kern0pt}edges\ verts\ {\isacharparenleft}{\kern0pt}y{\isacharhash}{\kern0pt}ys{\isacharparenright}{\kern0pt}{\isachardoublequoteclose}\isanewline
\ \ \ \ \ \ \isacommand{using}\isamarkupfalse%
\ distinct{\isacharunderscore}{\kern0pt}verts\ verts{\isacharunderscore}{\kern0pt}length\ distinct{\isacharunderscore}{\kern0pt}prufer{\isacharunderscore}{\kern0pt}seq{\isacharunderscore}{\kern0pt}to{\isacharunderscore}{\kern0pt}tree{\isacharbrackleft}{\kern0pt}OF\ Cons{\isacharparenleft}{\kern0pt}{\isadigit{4}}{\isacharparenright}{\kern0pt}{\isacharbrackright}{\kern0pt}\ \isacommand{by}\isamarkupfalse%
\ {\isacharparenleft}{\kern0pt}unfold{\isacharunderscore}{\kern0pt}locales{\isacharcomma}{\kern0pt}\ auto\ simp{\isacharcolon}{\kern0pt}\ distinct{\isacharunderscore}{\kern0pt}card{\isacharparenright}{\kern0pt}\isanewline
\ \ \ \ \isacommand{have}\isamarkupfalse%
\ {\isadigit{1}}{\isacharcolon}{\kern0pt}\ {\isachardoublequoteopen}find\ {\isacharparenleft}{\kern0pt}{\isasymlambda}v{\isachardot}{\kern0pt}\ v\ {\isasymnotin}\ set\ {\isacharparenleft}{\kern0pt}x{\isacharhash}{\kern0pt}xs{\isacharparenright}{\kern0pt}{\isacharparenright}{\kern0pt}\ verts\ {\isacharequal}{\kern0pt}\ find\ {\isacharparenleft}{\kern0pt}{\isasymlambda}v{\isachardot}{\kern0pt}\ t{\isadigit{1}}{\isachardot}{\kern0pt}leaf\ v{\isacharparenright}{\kern0pt}\ verts{\isachardoublequoteclose}\ \isacommand{using}\isamarkupfalse%
\ vert{\isacharunderscore}{\kern0pt}notin{\isacharunderscore}{\kern0pt}pruf{\isacharunderscore}{\kern0pt}seq{\isacharunderscore}{\kern0pt}leaf{\isacharbrackleft}{\kern0pt}OF\ Cons{\isacharparenleft}{\kern0pt}{\isadigit{3}}{\isacharparenright}{\kern0pt}{\isacharbrackright}{\kern0pt}\ ttps{\isacharunderscore}{\kern0pt}contxt{\isadigit{1}}{\isachardot}{\kern0pt}degree{\isacharunderscore}{\kern0pt}correct\ find{\isacharunderscore}{\kern0pt}cong\ \isacommand{unfolding}\isamarkupfalse%
\ t{\isadigit{1}}{\isachardot}{\kern0pt}leaf{\isacharunderscore}{\kern0pt}def\ \isacommand{by}\isamarkupfalse%
\ force\isanewline
\ \ \ \ \isacommand{have}\isamarkupfalse%
\ {\isadigit{2}}{\isacharcolon}{\kern0pt}\ {\isachardoublequoteopen}find\ {\isacharparenleft}{\kern0pt}{\isasymlambda}v{\isachardot}{\kern0pt}\ v\ {\isasymnotin}\ set\ {\isacharparenleft}{\kern0pt}y{\isacharhash}{\kern0pt}ys{\isacharparenright}{\kern0pt}{\isacharparenright}{\kern0pt}\ verts\ {\isacharequal}{\kern0pt}\ find\ {\isacharparenleft}{\kern0pt}{\isasymlambda}v{\isachardot}{\kern0pt}\ t{\isadigit{2}}{\isachardot}{\kern0pt}leaf\ v{\isacharparenright}{\kern0pt}\ verts{\isachardoublequoteclose}\ \isacommand{using}\isamarkupfalse%
\ vert{\isacharunderscore}{\kern0pt}notin{\isacharunderscore}{\kern0pt}pruf{\isacharunderscore}{\kern0pt}seq{\isacharunderscore}{\kern0pt}leaf{\isacharbrackleft}{\kern0pt}OF\ Cons{\isacharparenleft}{\kern0pt}{\isadigit{4}}{\isacharparenright}{\kern0pt}{\isacharbrackright}{\kern0pt}\ ttps{\isacharunderscore}{\kern0pt}contxt{\isadigit{2}}{\isachardot}{\kern0pt}degree{\isacharunderscore}{\kern0pt}correct\ find{\isacharunderscore}{\kern0pt}cong\ \isacommand{unfolding}\isamarkupfalse%
\ t{\isadigit{2}}{\isachardot}{\kern0pt}leaf{\isacharunderscore}{\kern0pt}def\ \isacommand{by}\isamarkupfalse%
\ force\isanewline
\ \ \ \ \isacommand{have}\isamarkupfalse%
\ {\isachardoublequoteopen}find\ {\isacharparenleft}{\kern0pt}{\isasymlambda}v{\isachardot}{\kern0pt}\ v\ {\isasymnotin}\ set\ {\isacharparenleft}{\kern0pt}x{\isacharhash}{\kern0pt}xs{\isacharparenright}{\kern0pt}{\isacharparenright}{\kern0pt}\ verts\ {\isacharequal}{\kern0pt}\ find\ {\isacharparenleft}{\kern0pt}{\isasymlambda}v{\isachardot}{\kern0pt}\ v\ {\isasymnotin}\ set\ {\isacharparenleft}{\kern0pt}y{\isacharhash}{\kern0pt}ys{\isacharparenright}{\kern0pt}{\isacharparenright}{\kern0pt}\ verts{\isachardoublequoteclose}\ \isacommand{using}\isamarkupfalse%
\ Cons{\isacharparenleft}{\kern0pt}{\isadigit{6}}{\isacharparenright}{\kern0pt}\ {\isadigit{1}}\ {\isadigit{2}}\ \isacommand{unfolding}\isamarkupfalse%
\ t{\isadigit{1}}{\isachardot}{\kern0pt}leaf{\isacharunderscore}{\kern0pt}def\ t{\isadigit{2}}{\isachardot}{\kern0pt}leaf{\isacharunderscore}{\kern0pt}def\ \isacommand{by}\isamarkupfalse%
\ simp\isanewline
\ \ \ \ \isacommand{have}\isamarkupfalse%
\ leafs{\isacharunderscore}{\kern0pt}eq{\isacharcolon}{\kern0pt}\ {\isachardoublequoteopen}leaf{\isadigit{2}}\ {\isacharequal}{\kern0pt}\ leaf{\isachardoublequoteclose}\ \isacommand{using}\isamarkupfalse%
\ Cons{\isacharparenleft}{\kern0pt}{\isadigit{6}}{\isacharparenright}{\kern0pt}\ {\isadigit{1}}\ {\isadigit{2}}\ find{\isacharunderscore}{\kern0pt}leaf\ find{\isacharunderscore}{\kern0pt}leaf{\isadigit{2}}\ \isacommand{unfolding}\isamarkupfalse%
\ t{\isadigit{1}}{\isachardot}{\kern0pt}leaf{\isacharunderscore}{\kern0pt}def\ t{\isadigit{2}}{\isachardot}{\kern0pt}leaf{\isacharunderscore}{\kern0pt}def\ \isacommand{by}\isamarkupfalse%
\ simp\isanewline
\ \ \ \ \isacommand{have}\isamarkupfalse%
\ leaf{\isacharunderscore}{\kern0pt}not{\isacharunderscore}{\kern0pt}in{\isacharunderscore}{\kern0pt}verts{\isacharprime}{\kern0pt}{\isacharcolon}{\kern0pt}\ {\isachardoublequoteopen}leaf\ {\isasymnotin}\ set\ {\isacharparenleft}{\kern0pt}remove{\isadigit{1}}\ leaf\ verts{\isacharparenright}{\kern0pt}{\isachardoublequoteclose}\ \isacommand{using}\isamarkupfalse%
\ distinct{\isacharunderscore}{\kern0pt}verts\ set{\isacharunderscore}{\kern0pt}remove{\isadigit{1}}{\isacharunderscore}{\kern0pt}eq\ \isacommand{by}\isamarkupfalse%
\ simp\isanewline
\ \ \ \ \isacommand{show}\isamarkupfalse%
\ False\isanewline
\ \ \ \ \isacommand{proof}\isamarkupfalse%
\ {\isacharparenleft}{\kern0pt}cases\ {\isachardoublequoteopen}y\ {\isacharequal}{\kern0pt}\ x{\isachardoublequoteclose}{\isacharparenright}{\kern0pt}\isanewline
\ \ \ \ \ \ \isacommand{case}\isamarkupfalse%
\ True\isanewline
\ \ \ \ \ \ \isacommand{then}\isamarkupfalse%
\ \isacommand{have}\isamarkupfalse%
\ {\isachardoublequoteopen}xs\ {\isasymnoteq}\ ys{\isachardoublequoteclose}\ \isacommand{using}\isamarkupfalse%
\ Cons\ \isacommand{by}\isamarkupfalse%
\ simp\isanewline
\ \ \ \ \ \ \isacommand{have}\isamarkupfalse%
\ {\isadigit{1}}{\isacharcolon}{\kern0pt}\ {\isachardoublequoteopen}{\isacharbraceleft}{\kern0pt}x{\isacharcomma}{\kern0pt}\ leaf{\isacharbraceright}{\kern0pt}\ {\isasymnotin}\ edges{\isacharunderscore}{\kern0pt}of{\isacharunderscore}{\kern0pt}edge{\isacharunderscore}{\kern0pt}list\ {\isacharparenleft}{\kern0pt}prufer{\isacharunderscore}{\kern0pt}seq{\isacharunderscore}{\kern0pt}to{\isacharunderscore}{\kern0pt}tree{\isacharunderscore}{\kern0pt}edges\ {\isacharparenleft}{\kern0pt}remove{\isadigit{1}}\ leaf\ verts{\isacharparenright}{\kern0pt}\ xs{\isacharparenright}{\kern0pt}{\isachardoublequoteclose}\ \isacommand{using}\isamarkupfalse%
\ contxt{\isacharprime}{\kern0pt}{\isachardot}{\kern0pt}prufer{\isacharunderscore}{\kern0pt}seq{\isacharunderscore}{\kern0pt}to{\isacharunderscore}{\kern0pt}tree{\isacharunderscore}{\kern0pt}edges{\isacharunderscore}{\kern0pt}wf\ pruf{\isacharunderscore}{\kern0pt}seq{\isadigit{1}}{\isacharprime}{\kern0pt}\ leaf{\isacharunderscore}{\kern0pt}not{\isacharunderscore}{\kern0pt}in{\isacharunderscore}{\kern0pt}verts{\isacharprime}{\kern0pt}\ \isacommand{by}\isamarkupfalse%
\ blast\isanewline
\ \ \ \ \ \ \isacommand{have}\isamarkupfalse%
\ {\isadigit{2}}{\isacharcolon}{\kern0pt}\ {\isachardoublequoteopen}{\isacharbraceleft}{\kern0pt}x{\isacharcomma}{\kern0pt}\ leaf{\isacharbraceright}{\kern0pt}\ {\isasymnotin}\ edges{\isacharunderscore}{\kern0pt}of{\isacharunderscore}{\kern0pt}edge{\isacharunderscore}{\kern0pt}list\ {\isacharparenleft}{\kern0pt}prufer{\isacharunderscore}{\kern0pt}seq{\isacharunderscore}{\kern0pt}to{\isacharunderscore}{\kern0pt}tree{\isacharunderscore}{\kern0pt}edges\ {\isacharparenleft}{\kern0pt}remove{\isadigit{1}}\ leaf\ verts{\isacharparenright}{\kern0pt}\ ys{\isacharparenright}{\kern0pt}{\isachardoublequoteclose}\ \isacommand{using}\isamarkupfalse%
\ contxt{\isacharprime}{\kern0pt}{\isachardot}{\kern0pt}prufer{\isacharunderscore}{\kern0pt}seq{\isacharunderscore}{\kern0pt}to{\isacharunderscore}{\kern0pt}tree{\isacharunderscore}{\kern0pt}edges{\isacharunderscore}{\kern0pt}wf\ pruf{\isacharunderscore}{\kern0pt}seq{\isadigit{2}}{\isacharprime}{\kern0pt}\ leaf{\isacharunderscore}{\kern0pt}not{\isacharunderscore}{\kern0pt}in{\isacharunderscore}{\kern0pt}verts{\isacharprime}{\kern0pt}\ True\ leafs{\isacharunderscore}{\kern0pt}eq\ \isacommand{by}\isamarkupfalse%
\ blast\isanewline
\ \ \ \ \ \ \isacommand{then}\isamarkupfalse%
\ \isacommand{have}\isamarkupfalse%
\ {\isachardoublequoteopen}edges{\isacharunderscore}{\kern0pt}of{\isacharunderscore}{\kern0pt}edge{\isacharunderscore}{\kern0pt}list\ {\isacharparenleft}{\kern0pt}prufer{\isacharunderscore}{\kern0pt}seq{\isacharunderscore}{\kern0pt}to{\isacharunderscore}{\kern0pt}tree{\isacharunderscore}{\kern0pt}edges\ {\isacharparenleft}{\kern0pt}remove{\isadigit{1}}\ leaf\ verts{\isacharparenright}{\kern0pt}\ xs{\isacharparenright}{\kern0pt}\ {\isacharequal}{\kern0pt}\ edges{\isacharunderscore}{\kern0pt}of{\isacharunderscore}{\kern0pt}edge{\isacharunderscore}{\kern0pt}list\ {\isacharparenleft}{\kern0pt}prufer{\isacharunderscore}{\kern0pt}seq{\isacharunderscore}{\kern0pt}to{\isacharunderscore}{\kern0pt}tree{\isacharunderscore}{\kern0pt}edges\ {\isacharparenleft}{\kern0pt}remove{\isadigit{1}}\ leaf\ verts{\isacharparenright}{\kern0pt}\ ys{\isacharparenright}{\kern0pt}{\isachardoublequoteclose}\isanewline
\ \ \ \ \ \ \ \ \isacommand{using}\isamarkupfalse%
\ Cons{\isacharparenleft}{\kern0pt}{\isadigit{6}}{\isacharparenright}{\kern0pt}\ find{\isacharunderscore}{\kern0pt}leaf\ find{\isacharunderscore}{\kern0pt}leaf{\isadigit{2}}\ leafs{\isacharunderscore}{\kern0pt}eq\ True\ insert{\isacharunderscore}{\kern0pt}ident{\isacharbrackleft}{\kern0pt}OF\ {\isadigit{1}}\ {\isadigit{2}}{\isacharbrackright}{\kern0pt}\ \isacommand{unfolding}\isamarkupfalse%
\ edges{\isacharunderscore}{\kern0pt}of{\isacharunderscore}{\kern0pt}edge{\isacharunderscore}{\kern0pt}list{\isacharunderscore}{\kern0pt}def\ \isacommand{by}\isamarkupfalse%
\ simp\isanewline
\ \ \ \ \ \ \isacommand{then}\isamarkupfalse%
\ \isacommand{show}\isamarkupfalse%
\ {\isacharquery}{\kern0pt}thesis\ \isacommand{using}\isamarkupfalse%
\ True\ leafs{\isacharunderscore}{\kern0pt}eq\ Cons{\isachardot}{\kern0pt}IH\ pruf{\isacharunderscore}{\kern0pt}seq{\isadigit{1}}{\isacharprime}{\kern0pt}\ pruf{\isacharunderscore}{\kern0pt}seq{\isadigit{2}}{\isacharprime}{\kern0pt}\ leafs{\isacharunderscore}{\kern0pt}eq\ Cons{\isacharparenleft}{\kern0pt}{\isadigit{6}}{\isacharparenright}{\kern0pt}\ find{\isacharunderscore}{\kern0pt}leaf\isanewline
\ \ \ \ \ \ \ \ \ \ find{\isacharunderscore}{\kern0pt}leaf{\isadigit{2}}\ {\isacartoucheopen}xs\ {\isasymnoteq}\ ys{\isacartoucheclose}\ contxt{\isacharprime}{\kern0pt}{\isachardot}{\kern0pt}prufer{\isacharunderscore}{\kern0pt}seq{\isacharunderscore}{\kern0pt}to{\isacharunderscore}{\kern0pt}tree{\isacharunderscore}{\kern0pt}context{\isacharunderscore}{\kern0pt}axioms\ \isacommand{unfolding}\isamarkupfalse%
\ edges{\isacharunderscore}{\kern0pt}of{\isacharunderscore}{\kern0pt}edge{\isacharunderscore}{\kern0pt}list{\isacharunderscore}{\kern0pt}def\ \isacommand{by}\isamarkupfalse%
\ auto\isanewline
\ \ \ \ \isacommand{next}\isamarkupfalse%
\isanewline
\ \ \ \ \ \ \isacommand{case}\isamarkupfalse%
\ False\isanewline
\ \ \ \ \ \ \isacommand{then}\isamarkupfalse%
\ \isacommand{have}\isamarkupfalse%
\ {\isachardoublequoteopen}{\isacharbraceleft}{\kern0pt}x{\isacharcomma}{\kern0pt}\ leaf{\isacharbraceright}{\kern0pt}\ {\isasymnotin}\ edges{\isacharunderscore}{\kern0pt}of{\isacharunderscore}{\kern0pt}edge{\isacharunderscore}{\kern0pt}list\ {\isacharparenleft}{\kern0pt}prufer{\isacharunderscore}{\kern0pt}seq{\isacharunderscore}{\kern0pt}to{\isacharunderscore}{\kern0pt}tree{\isacharunderscore}{\kern0pt}edges\ {\isacharparenleft}{\kern0pt}remove{\isadigit{1}}\ leaf\ verts{\isacharparenright}{\kern0pt}\ ys{\isacharparenright}{\kern0pt}{\isachardoublequoteclose}\ \isacommand{using}\isamarkupfalse%
\ find{\isacharunderscore}{\kern0pt}leaf{\isadigit{2}}\ leafs{\isacharunderscore}{\kern0pt}eq\ contxt{\isacharprime}{\kern0pt}{\isachardot}{\kern0pt}prufer{\isacharunderscore}{\kern0pt}seq{\isacharunderscore}{\kern0pt}to{\isacharunderscore}{\kern0pt}tree{\isacharunderscore}{\kern0pt}edges{\isacharunderscore}{\kern0pt}wf\ pruf{\isacharunderscore}{\kern0pt}seq{\isadigit{2}}{\isacharprime}{\kern0pt}\ leaf{\isacharunderscore}{\kern0pt}not{\isacharunderscore}{\kern0pt}in{\isacharunderscore}{\kern0pt}verts{\isacharprime}{\kern0pt}\ \isacommand{by}\isamarkupfalse%
\ auto\isanewline
\ \ \ \ \ \ \isacommand{then}\isamarkupfalse%
\ \isacommand{show}\isamarkupfalse%
\ {\isacharquery}{\kern0pt}thesis\ \isacommand{using}\isamarkupfalse%
\ Cons{\isacharparenleft}{\kern0pt}{\isadigit{6}}{\isacharparenright}{\kern0pt}\ find{\isacharunderscore}{\kern0pt}leaf\ find{\isacharunderscore}{\kern0pt}leaf{\isadigit{2}}\ leafs{\isacharunderscore}{\kern0pt}eq\ False\ \isacommand{unfolding}\isamarkupfalse%
\ edges{\isacharunderscore}{\kern0pt}of{\isacharunderscore}{\kern0pt}edge{\isacharunderscore}{\kern0pt}list{\isacharunderscore}{\kern0pt}def\isanewline
\ \ \ \ \ \ \ \ \isacommand{by}\isamarkupfalse%
\ {\isacharparenleft}{\kern0pt}auto{\isacharcomma}{\kern0pt}\ metis\ {\isacharparenleft}{\kern0pt}no{\isacharunderscore}{\kern0pt}types{\isacharcomma}{\kern0pt}\ lifting{\isacharparenright}{\kern0pt}\ doubleton{\isacharunderscore}{\kern0pt}eq{\isacharunderscore}{\kern0pt}iff\ insert{\isacharunderscore}{\kern0pt}iff{\isacharparenright}{\kern0pt}\isanewline
\ \ \ \ \isacommand{qed}\isamarkupfalse%
\isanewline
\ \ \isacommand{qed}\isamarkupfalse%
\isanewline
\isacommand{qed}\isamarkupfalse%
%
\endisatagproof
{\isafoldproof}%
%
\isadelimproof
\isanewline
%
\endisadelimproof
\isanewline
\isacommand{lemma}\isamarkupfalse%
\ inj{\isacharunderscore}{\kern0pt}on{\isacharunderscore}{\kern0pt}prufer{\isacharunderscore}{\kern0pt}seq{\isacharunderscore}{\kern0pt}to{\isacharunderscore}{\kern0pt}tree{\isacharcolon}{\kern0pt}\ {\isachardoublequoteopen}inj{\isacharunderscore}{\kern0pt}on\ {\isacharparenleft}{\kern0pt}prufer{\isacharunderscore}{\kern0pt}seq{\isacharunderscore}{\kern0pt}to{\isacharunderscore}{\kern0pt}tree\ verts{\isacharparenright}{\kern0pt}\ {\isacharparenleft}{\kern0pt}prufer{\isacharunderscore}{\kern0pt}sequences\ verts{\isacharparenright}{\kern0pt}{\isachardoublequoteclose}\isanewline
%
\isadelimproof
\ \ %
\endisadelimproof
%
\isatagproof
\isacommand{unfolding}\isamarkupfalse%
\ inj{\isacharunderscore}{\kern0pt}on{\isacharunderscore}{\kern0pt}def\ prufer{\isacharunderscore}{\kern0pt}seq{\isacharunderscore}{\kern0pt}to{\isacharunderscore}{\kern0pt}tree{\isacharunderscore}{\kern0pt}def\ \isacommand{using}\isamarkupfalse%
\ inj{\isacharunderscore}{\kern0pt}prufer{\isacharunderscore}{\kern0pt}seq{\isacharunderscore}{\kern0pt}to{\isacharunderscore}{\kern0pt}tree{\isacharunderscore}{\kern0pt}edges\ \isacommand{by}\isamarkupfalse%
\ auto%
\endisatagproof
{\isafoldproof}%
%
\isadelimproof
\isanewline
%
\endisadelimproof
\isanewline
\isacommand{theorem}\isamarkupfalse%
\ labeled{\isacharunderscore}{\kern0pt}tree{\isacharunderscore}{\kern0pt}enum{\isacharunderscore}{\kern0pt}distinct{\isacharcolon}{\kern0pt}\ {\isachardoublequoteopen}distinct\ {\isacharparenleft}{\kern0pt}labeled{\isacharunderscore}{\kern0pt}tree{\isacharunderscore}{\kern0pt}enum\ verts{\isacharparenright}{\kern0pt}{\isachardoublequoteclose}\isanewline
%
\isadelimproof
\ \ %
\endisadelimproof
%
\isatagproof
\isacommand{unfolding}\isamarkupfalse%
\ labeled{\isacharunderscore}{\kern0pt}tree{\isacharunderscore}{\kern0pt}enum{\isacharunderscore}{\kern0pt}def\ \isacommand{using}\isamarkupfalse%
\ inj{\isacharunderscore}{\kern0pt}on{\isacharunderscore}{\kern0pt}prufer{\isacharunderscore}{\kern0pt}seq{\isacharunderscore}{\kern0pt}to{\isacharunderscore}{\kern0pt}tree\isanewline
\ \ \isacommand{by}\isamarkupfalse%
\ {\isacharparenleft}{\kern0pt}simp\ add{\isacharcolon}{\kern0pt}\ distinct{\isacharunderscore}{\kern0pt}map\ n{\isacharunderscore}{\kern0pt}sequence{\isacharunderscore}{\kern0pt}enum{\isacharunderscore}{\kern0pt}correct\ n{\isacharunderscore}{\kern0pt}sequence{\isacharunderscore}{\kern0pt}enum{\isacharunderscore}{\kern0pt}distinct\ prufer{\isacharunderscore}{\kern0pt}sequences{\isacharunderscore}{\kern0pt}def\ distinct{\isacharunderscore}{\kern0pt}verts{\isacharparenright}{\kern0pt}%
\endisatagproof
{\isafoldproof}%
%
\isadelimproof
\isanewline
%
\endisadelimproof
\isanewline
\isanewline
\isacommand{end}\isamarkupfalse%
\isanewline
%
\isadelimtheory
\isanewline
%
\endisadelimtheory
%
\isatagtheory
\isacommand{end}\isamarkupfalse%
%
\endisatagtheory
{\isafoldtheory}%
%
\isadelimtheory
%
\endisadelimtheory
%
\end{isabellebody}%
\endinput
%:%file=Labeled_Tree_Enumeration.tex%:%
%:%11=1%:%
%:%27=3%:%
%:%28=3%:%
%:%29=4%:%
%:%30=5%:%
%:%44=7%:%
%:%54=9%:%
%:%55=9%:%
%:%56=10%:%
%:%63=12%:%
%:%75=14%:%
%:%77=16%:%
%:%78=16%:%
%:%79=17%:%
%:%80=18%:%
%:%81=19%:%
%:%82=19%:%
%:%83=20%:%
%:%84=21%:%
%:%86=23%:%
%:%87=24%:%
%:%88=25%:%
%:%89=25%:%
%:%90=26%:%
%:%91=27%:%
%:%92=28%:%
%:%93=28%:%
%:%94=29%:%
%:%95=30%:%
%:%96=31%:%
%:%97=31%:%
%:%98=32%:%
%:%105=38%:%
%:%117=40%:%
%:%119=42%:%
%:%120=42%:%
%:%121=43%:%
%:%122=44%:%
%:%123=45%:%
%:%124=45%:%
%:%125=46%:%
%:%126=47%:%
%:%127=47%:%
%:%128=48%:%
%:%129=49%:%
%:%130=50%:%
%:%131=51%:%
%:%132=51%:%
%:%133=52%:%
%:%134=53%:%
%:%135=54%:%
%:%136=54%:%
%:%139=55%:%
%:%143=55%:%
%:%144=55%:%
%:%149=55%:%
%:%152=56%:%
%:%153=57%:%
%:%154=57%:%
%:%157=58%:%
%:%161=58%:%
%:%162=58%:%
%:%167=58%:%
%:%170=59%:%
%:%171=60%:%
%:%172=60%:%
%:%173=61%:%
%:%174=62%:%
%:%175=63%:%
%:%177=65%:%
%:%178=66%:%
%:%179=67%:%
%:%180=67%:%
%:%183=68%:%
%:%187=68%:%
%:%188=68%:%
%:%189=68%:%
%:%194=68%:%
%:%197=69%:%
%:%198=70%:%
%:%199=70%:%
%:%202=71%:%
%:%206=71%:%
%:%207=71%:%
%:%208=71%:%
%:%213=71%:%
%:%216=72%:%
%:%217=73%:%
%:%218=73%:%
%:%221=74%:%
%:%225=74%:%
%:%226=74%:%
%:%227=75%:%
%:%228=75%:%
%:%233=75%:%
%:%236=76%:%
%:%237=77%:%
%:%238=77%:%
%:%239=78%:%
%:%240=79%:%
%:%247=80%:%
%:%248=80%:%
%:%249=81%:%
%:%250=81%:%
%:%251=81%:%
%:%252=81%:%
%:%253=82%:%
%:%254=82%:%
%:%255=83%:%
%:%256=83%:%
%:%257=83%:%
%:%258=83%:%
%:%259=83%:%
%:%260=84%:%
%:%266=84%:%
%:%269=85%:%
%:%270=86%:%
%:%271=86%:%
%:%278=87%:%
%:%279=87%:%
%:%280=88%:%
%:%281=88%:%
%:%282=88%:%
%:%283=89%:%
%:%284=89%:%
%:%285=89%:%
%:%286=89%:%
%:%287=89%:%
%:%288=90%:%
%:%289=90%:%
%:%290=90%:%
%:%291=91%:%
%:%292=91%:%
%:%293=91%:%
%:%294=91%:%
%:%295=91%:%
%:%296=91%:%
%:%297=92%:%
%:%303=92%:%
%:%306=93%:%
%:%307=94%:%
%:%308=94%:%
%:%315=95%:%
%:%316=95%:%
%:%317=96%:%
%:%318=96%:%
%:%319=97%:%
%:%320=97%:%
%:%321=97%:%
%:%322=97%:%
%:%323=97%:%
%:%324=98%:%
%:%325=98%:%
%:%326=99%:%
%:%327=99%:%
%:%328=100%:%
%:%329=100%:%
%:%330=101%:%
%:%331=101%:%
%:%332=101%:%
%:%333=102%:%
%:%334=102%:%
%:%335=102%:%
%:%336=103%:%
%:%337=103%:%
%:%338=104%:%
%:%339=104%:%
%:%340=105%:%
%:%341=105%:%
%:%342=105%:%
%:%343=105%:%
%:%344=105%:%
%:%345=106%:%
%:%346=106%:%
%:%347=107%:%
%:%348=107%:%
%:%349=108%:%
%:%350=108%:%
%:%351=108%:%
%:%352=108%:%
%:%353=108%:%
%:%354=108%:%
%:%355=109%:%
%:%356=109%:%
%:%357=109%:%
%:%358=109%:%
%:%359=109%:%
%:%360=109%:%
%:%361=110%:%
%:%362=110%:%
%:%363=111%:%
%:%364=111%:%
%:%365=112%:%
%:%366=112%:%
%:%367=112%:%
%:%368=112%:%
%:%369=112%:%
%:%370=112%:%
%:%371=113%:%
%:%372=113%:%
%:%373=113%:%
%:%374=113%:%
%:%375=113%:%
%:%376=113%:%
%:%377=114%:%
%:%378=114%:%
%:%379=115%:%
%:%380=115%:%
%:%381=116%:%
%:%382=116%:%
%:%383=116%:%
%:%384=116%:%
%:%385=116%:%
%:%386=116%:%
%:%387=117%:%
%:%388=117%:%
%:%389=117%:%
%:%390=117%:%
%:%391=117%:%
%:%392=117%:%
%:%393=118%:%
%:%394=118%:%
%:%395=119%:%
%:%401=119%:%
%:%404=120%:%
%:%405=121%:%
%:%406=121%:%
%:%409=122%:%
%:%413=122%:%
%:%414=122%:%
%:%415=122%:%
%:%416=122%:%
%:%421=122%:%
%:%424=123%:%
%:%425=124%:%
%:%426=124%:%
%:%429=125%:%
%:%433=125%:%
%:%434=125%:%
%:%435=125%:%
%:%440=125%:%
%:%443=126%:%
%:%444=127%:%
%:%445=127%:%
%:%446=128%:%
%:%447=129%:%
%:%454=130%:%
%:%455=130%:%
%:%456=131%:%
%:%457=131%:%
%:%458=132%:%
%:%459=132%:%
%:%460=132%:%
%:%461=132%:%
%:%462=132%:%
%:%463=133%:%
%:%464=133%:%
%:%465=133%:%
%:%466=133%:%
%:%467=134%:%
%:%468=134%:%
%:%469=134%:%
%:%470=134%:%
%:%471=135%:%
%:%472=135%:%
%:%473=135%:%
%:%474=135%:%
%:%475=136%:%
%:%476=136%:%
%:%477=136%:%
%:%478=136%:%
%:%479=137%:%
%:%480=137%:%
%:%481=138%:%
%:%482=138%:%
%:%483=138%:%
%:%484=138%:%
%:%485=138%:%
%:%486=139%:%
%:%487=140%:%
%:%488=140%:%
%:%489=140%:%
%:%490=140%:%
%:%491=141%:%
%:%492=141%:%
%:%493=141%:%
%:%494=141%:%
%:%495=141%:%
%:%496=142%:%
%:%502=142%:%
%:%505=143%:%
%:%506=144%:%
%:%507=144%:%
%:%510=145%:%
%:%514=145%:%
%:%515=145%:%
%:%520=145%:%
%:%523=146%:%
%:%524=147%:%
%:%525=147%:%
%:%526=148%:%
%:%527=149%:%
%:%528=150%:%
%:%529=151%:%
%:%530=151%:%
%:%531=152%:%
%:%532=153%:%
%:%533=154%:%
%:%534=155%:%
%:%535=156%:%
%:%536=157%:%
%:%537=157%:%
%:%540=158%:%
%:%544=158%:%
%:%545=158%:%
%:%546=158%:%
%:%551=158%:%
%:%554=159%:%
%:%555=160%:%
%:%556=160%:%
%:%557=161%:%
%:%558=162%:%
%:%559=163%:%
%:%566=164%:%
%:%567=164%:%
%:%568=165%:%
%:%569=165%:%
%:%570=166%:%
%:%571=166%:%
%:%572=167%:%
%:%573=167%:%
%:%574=167%:%
%:%575=168%:%
%:%576=168%:%
%:%577=169%:%
%:%578=169%:%
%:%579=169%:%
%:%580=169%:%
%:%581=170%:%
%:%587=170%:%
%:%590=171%:%
%:%591=172%:%
%:%592=172%:%
%:%593=173%:%
%:%594=174%:%
%:%595=175%:%
%:%596=176%:%
%:%597=177%:%
%:%598=178%:%
%:%599=179%:%
%:%600=180%:%
%:%607=181%:%
%:%608=181%:%
%:%609=182%:%
%:%610=182%:%
%:%611=183%:%
%:%612=183%:%
%:%613=184%:%
%:%614=184%:%
%:%615=185%:%
%:%616=185%:%
%:%617=186%:%
%:%618=186%:%
%:%619=186%:%
%:%620=187%:%
%:%621=187%:%
%:%622=188%:%
%:%623=188%:%
%:%624=188%:%
%:%625=189%:%
%:%626=189%:%
%:%627=190%:%
%:%628=190%:%
%:%629=191%:%
%:%630=191%:%
%:%631=192%:%
%:%632=192%:%
%:%633=193%:%
%:%634=193%:%
%:%635=194%:%
%:%636=194%:%
%:%637=194%:%
%:%638=194%:%
%:%639=194%:%
%:%640=195%:%
%:%641=195%:%
%:%642=195%:%
%:%643=195%:%
%:%644=196%:%
%:%645=196%:%
%:%646=196%:%
%:%647=196%:%
%:%648=197%:%
%:%649=197%:%
%:%650=197%:%
%:%651=197%:%
%:%652=198%:%
%:%658=198%:%
%:%661=199%:%
%:%662=200%:%
%:%663=200%:%
%:%664=201%:%
%:%665=202%:%
%:%668=203%:%
%:%672=203%:%
%:%673=203%:%
%:%674=204%:%
%:%675=204%:%
%:%676=205%:%
%:%677=205%:%
%:%678=206%:%
%:%679=206%:%
%:%680=206%:%
%:%681=206%:%
%:%682=206%:%
%:%683=207%:%
%:%684=207%:%
%:%685=208%:%
%:%686=208%:%
%:%687=209%:%
%:%688=209%:%
%:%689=209%:%
%:%690=209%:%
%:%691=210%:%
%:%692=210%:%
%:%693=211%:%
%:%694=212%:%
%:%695=213%:%
%:%696=214%:%
%:%697=215%:%
%:%698=216%:%
%:%699=216%:%
%:%700=216%:%
%:%701=217%:%
%:%702=217%:%
%:%703=217%:%
%:%704=218%:%
%:%705=219%:%
%:%706=219%:%
%:%707=220%:%
%:%708=220%:%
%:%709=220%:%
%:%710=221%:%
%:%711=221%:%
%:%712=221%:%
%:%713=222%:%
%:%714=222%:%
%:%715=222%:%
%:%716=222%:%
%:%717=222%:%
%:%718=223%:%
%:%719=223%:%
%:%720=223%:%
%:%721=223%:%
%:%722=223%:%
%:%723=223%:%
%:%724=224%:%
%:%725=224%:%
%:%726=224%:%
%:%727=224%:%
%:%728=225%:%
%:%729=225%:%
%:%734=225%:%
%:%737=226%:%
%:%738=227%:%
%:%739=227%:%
%:%740=228%:%
%:%741=229%:%
%:%742=230%:%
%:%745=231%:%
%:%749=231%:%
%:%750=231%:%
%:%751=232%:%
%:%752=232%:%
%:%753=233%:%
%:%754=233%:%
%:%755=234%:%
%:%756=234%:%
%:%757=235%:%
%:%758=235%:%
%:%759=235%:%
%:%760=235%:%
%:%761=236%:%
%:%762=237%:%
%:%763=237%:%
%:%764=238%:%
%:%765=238%:%
%:%766=239%:%
%:%767=240%:%
%:%768=240%:%
%:%769=241%:%
%:%770=241%:%
%:%771=242%:%
%:%772=242%:%
%:%773=243%:%
%:%774=243%:%
%:%775=243%:%
%:%776=244%:%
%:%777=244%:%
%:%778=244%:%
%:%779=244%:%
%:%780=245%:%
%:%781=245%:%
%:%782=245%:%
%:%783=246%:%
%:%784=246%:%
%:%785=246%:%
%:%786=246%:%
%:%787=247%:%
%:%788=247%:%
%:%789=247%:%
%:%790=248%:%
%:%791=248%:%
%:%792=248%:%
%:%793=248%:%
%:%794=249%:%
%:%795=250%:%
%:%796=250%:%
%:%797=250%:%
%:%798=250%:%
%:%799=250%:%
%:%800=250%:%
%:%801=251%:%
%:%802=251%:%
%:%803=252%:%
%:%804=252%:%
%:%805=253%:%
%:%806=253%:%
%:%807=253%:%
%:%808=253%:%
%:%809=254%:%
%:%810=254%:%
%:%811=255%:%
%:%812=256%:%
%:%813=257%:%
%:%814=258%:%
%:%815=259%:%
%:%816=260%:%
%:%817=261%:%
%:%818=261%:%
%:%819=261%:%
%:%820=262%:%
%:%821=262%:%
%:%822=262%:%
%:%823=263%:%
%:%824=263%:%
%:%825=263%:%
%:%826=264%:%
%:%827=265%:%
%:%828=265%:%
%:%829=265%:%
%:%830=265%:%
%:%831=266%:%
%:%832=267%:%
%:%833=267%:%
%:%834=268%:%
%:%835=268%:%
%:%836=268%:%
%:%837=269%:%
%:%838=269%:%
%:%839=270%:%
%:%840=270%:%
%:%841=270%:%
%:%842=271%:%
%:%843=271%:%
%:%844=272%:%
%:%845=272%:%
%:%846=272%:%
%:%847=272%:%
%:%848=273%:%
%:%849=273%:%
%:%850=273%:%
%:%851=273%:%
%:%852=274%:%
%:%853=274%:%
%:%854=275%:%
%:%855=275%:%
%:%856=275%:%
%:%857=275%:%
%:%858=276%:%
%:%859=277%:%
%:%860=277%:%
%:%861=277%:%
%:%862=277%:%
%:%863=278%:%
%:%864=278%:%
%:%869=278%:%
%:%872=279%:%
%:%873=280%:%
%:%874=280%:%
%:%877=281%:%
%:%881=281%:%
%:%882=281%:%
%:%883=281%:%
%:%884=281%:%
%:%889=281%:%
%:%892=282%:%
%:%893=283%:%
%:%894=283%:%
%:%897=284%:%
%:%901=284%:%
%:%902=284%:%
%:%903=284%:%
%:%904=284%:%
%:%909=284%:%
%:%912=285%:%
%:%913=286%:%
%:%914=287%:%
%:%915=287%:%
%:%916=288%:%
%:%917=289%:%
%:%918=290%:%
%:%921=291%:%
%:%925=291%:%
%:%926=291%:%
%:%927=292%:%
%:%928=292%:%
%:%929=293%:%
%:%930=293%:%
%:%931=294%:%
%:%932=294%:%
%:%933=294%:%
%:%934=294%:%
%:%935=295%:%
%:%936=295%:%
%:%937=295%:%
%:%938=295%:%
%:%939=295%:%
%:%940=296%:%
%:%941=296%:%
%:%942=297%:%
%:%943=297%:%
%:%944=297%:%
%:%945=297%:%
%:%946=297%:%
%:%947=297%:%
%:%948=298%:%
%:%949=298%:%
%:%950=299%:%
%:%951=299%:%
%:%952=300%:%
%:%953=300%:%
%:%954=300%:%
%:%955=300%:%
%:%956=301%:%
%:%957=301%:%
%:%958=302%:%
%:%959=303%:%
%:%960=304%:%
%:%961=305%:%
%:%962=305%:%
%:%963=305%:%
%:%964=306%:%
%:%965=306%:%
%:%966=306%:%
%:%967=306%:%
%:%968=307%:%
%:%969=307%:%
%:%970=307%:%
%:%971=307%:%
%:%972=307%:%
%:%973=308%:%
%:%974=308%:%
%:%975=308%:%
%:%976=309%:%
%:%977=309%:%
%:%978=309%:%
%:%979=310%:%
%:%985=310%:%
%:%988=311%:%
%:%989=312%:%
%:%990=312%:%
%:%993=313%:%
%:%997=313%:%
%:%998=313%:%
%:%999=314%:%
%:%1000=314%:%
%:%1001=315%:%
%:%1002=315%:%
%:%1003=316%:%
%:%1004=316%:%
%:%1005=316%:%
%:%1006=316%:%
%:%1007=317%:%
%:%1008=317%:%
%:%1009=317%:%
%:%1010=317%:%
%:%1011=317%:%
%:%1012=318%:%
%:%1013=318%:%
%:%1014=319%:%
%:%1015=319%:%
%:%1016=319%:%
%:%1017=319%:%
%:%1018=320%:%
%:%1019=320%:%
%:%1020=321%:%
%:%1021=321%:%
%:%1022=322%:%
%:%1023=322%:%
%:%1024=322%:%
%:%1025=322%:%
%:%1026=323%:%
%:%1027=323%:%
%:%1028=324%:%
%:%1029=325%:%
%:%1030=326%:%
%:%1031=326%:%
%:%1032=326%:%
%:%1033=327%:%
%:%1034=327%:%
%:%1035=327%:%
%:%1036=327%:%
%:%1037=328%:%
%:%1038=328%:%
%:%1039=328%:%
%:%1040=328%:%
%:%1041=329%:%
%:%1042=329%:%
%:%1043=329%:%
%:%1044=330%:%
%:%1045=330%:%
%:%1046=330%:%
%:%1047=331%:%
%:%1048=331%:%
%:%1049=331%:%
%:%1050=331%:%
%:%1051=332%:%
%:%1052=332%:%
%:%1053=332%:%
%:%1054=333%:%
%:%1060=333%:%
%:%1063=334%:%
%:%1064=335%:%
%:%1065=335%:%
%:%1066=336%:%
%:%1067=337%:%
%:%1068=337%:%
%:%1069=338%:%
%:%1070=339%:%
%:%1071=340%:%
%:%1072=341%:%
%:%1073=342%:%
%:%1074=343%:%
%:%1075=344%:%
%:%1076=345%:%
%:%1077=346%:%
%:%1078=346%:%
%:%1080=346%:%
%:%1084=346%:%
%:%1085=346%:%
%:%1093=346%:%
%:%1094=347%:%
%:%1095=348%:%
%:%1096=348%:%
%:%1099=349%:%
%:%1103=349%:%
%:%1104=349%:%
%:%1105=349%:%
%:%1111=349%:%
%:%1114=350%:%
%:%1115=351%:%
%:%1116=351%:%
%:%1119=352%:%
%:%1123=352%:%
%:%1124=352%:%
%:%1125=352%:%
%:%1130=352%:%
%:%1133=353%:%
%:%1134=354%:%
%:%1135=354%:%
%:%1137=354%:%
%:%1141=354%:%
%:%1142=354%:%
%:%1143=354%:%
%:%1150=354%:%
%:%1151=355%:%
%:%1152=356%:%
%:%1153=356%:%
%:%1156=357%:%
%:%1160=357%:%
%:%1161=357%:%
%:%1162=357%:%
%:%1163=357%:%
%:%1168=357%:%
%:%1171=358%:%
%:%1172=359%:%
%:%1173=359%:%
%:%1176=360%:%
%:%1180=360%:%
%:%1181=360%:%
%:%1182=360%:%
%:%1187=360%:%
%:%1190=361%:%
%:%1191=362%:%
%:%1192=362%:%
%:%1195=363%:%
%:%1199=363%:%
%:%1200=363%:%
%:%1201=363%:%
%:%1206=363%:%
%:%1209=364%:%
%:%1210=365%:%
%:%1211=365%:%
%:%1214=366%:%
%:%1218=366%:%
%:%1219=366%:%
%:%1220=366%:%
%:%1225=366%:%
%:%1228=367%:%
%:%1229=368%:%
%:%1230=368%:%
%:%1237=369%:%
%:%1238=369%:%
%:%1239=370%:%
%:%1240=370%:%
%:%1241=370%:%
%:%1242=370%:%
%:%1243=370%:%
%:%1244=371%:%
%:%1245=371%:%
%:%1246=371%:%
%:%1247=371%:%
%:%1248=372%:%
%:%1249=372%:%
%:%1250=372%:%
%:%1251=372%:%
%:%1252=372%:%
%:%1253=372%:%
%:%1254=373%:%
%:%1255=373%:%
%:%1256=373%:%
%:%1257=373%:%
%:%1258=373%:%
%:%1259=374%:%
%:%1260=374%:%
%:%1261=374%:%
%:%1262=374%:%
%:%1263=375%:%
%:%1269=375%:%
%:%1272=376%:%
%:%1273=377%:%
%:%1274=377%:%
%:%1275=378%:%
%:%1276=379%:%
%:%1277=380%:%
%:%1278=381%:%
%:%1279=382%:%
%:%1280=383%:%
%:%1287=384%:%
%:%1288=384%:%
%:%1289=385%:%
%:%1290=385%:%
%:%1291=386%:%
%:%1292=386%:%
%:%1293=386%:%
%:%1294=387%:%
%:%1295=387%:%
%:%1296=387%:%
%:%1297=388%:%
%:%1298=388%:%
%:%1299=389%:%
%:%1300=389%:%
%:%1301=389%:%
%:%1302=389%:%
%:%1303=389%:%
%:%1304=390%:%
%:%1305=390%:%
%:%1306=390%:%
%:%1307=391%:%
%:%1308=391%:%
%:%1309=392%:%
%:%1310=392%:%
%:%1311=393%:%
%:%1312=393%:%
%:%1313=393%:%
%:%1314=393%:%
%:%1315=394%:%
%:%1316=394%:%
%:%1317=394%:%
%:%1318=394%:%
%:%1319=394%:%
%:%1320=395%:%
%:%1321=395%:%
%:%1322=395%:%
%:%1323=395%:%
%:%1324=395%:%
%:%1325=396%:%
%:%1326=396%:%
%:%1327=396%:%
%:%1328=396%:%
%:%1329=397%:%
%:%1330=397%:%
%:%1331=397%:%
%:%1332=398%:%
%:%1333=398%:%
%:%1334=398%:%
%:%1335=399%:%
%:%1336=399%:%
%:%1337=399%:%
%:%1338=399%:%
%:%1339=400%:%
%:%1340=400%:%
%:%1341=400%:%
%:%1342=400%:%
%:%1343=400%:%
%:%1344=401%:%
%:%1345=401%:%
%:%1346=401%:%
%:%1347=401%:%
%:%1348=401%:%
%:%1349=402%:%
%:%1355=402%:%
%:%1358=403%:%
%:%1359=404%:%
%:%1360=404%:%
%:%1367=405%:%
%:%1368=405%:%
%:%1369=406%:%
%:%1370=406%:%
%:%1371=406%:%
%:%1372=406%:%
%:%1373=406%:%
%:%1374=407%:%
%:%1375=407%:%
%:%1376=407%:%
%:%1377=407%:%
%:%1378=407%:%
%:%1379=408%:%
%:%1385=408%:%
%:%1388=409%:%
%:%1389=410%:%
%:%1390=410%:%
%:%1393=411%:%
%:%1397=411%:%
%:%1398=411%:%
%:%1399=412%:%
%:%1400=412%:%
%:%1401=413%:%
%:%1402=413%:%
%:%1403=414%:%
%:%1404=414%:%
%:%1405=414%:%
%:%1406=415%:%
%:%1407=415%:%
%:%1408=415%:%
%:%1409=416%:%
%:%1410=416%:%
%:%1411=416%:%
%:%1412=416%:%
%:%1413=417%:%
%:%1414=417%:%
%:%1415=418%:%
%:%1416=418%:%
%:%1417=419%:%
%:%1418=419%:%
%:%1419=419%:%
%:%1420=420%:%
%:%1421=420%:%
%:%1422=420%:%
%:%1423=421%:%
%:%1424=421%:%
%:%1425=421%:%
%:%1426=421%:%
%:%1427=421%:%
%:%1428=422%:%
%:%1429=422%:%
%:%1430=423%:%
%:%1431=423%:%
%:%1432=424%:%
%:%1433=424%:%
%:%1434=425%:%
%:%1435=425%:%
%:%1436=426%:%
%:%1437=426%:%
%:%1438=426%:%
%:%1439=427%:%
%:%1440=427%:%
%:%1441=427%:%
%:%1442=428%:%
%:%1443=428%:%
%:%1444=428%:%
%:%1445=429%:%
%:%1446=430%:%
%:%1447=431%:%
%:%1448=431%:%
%:%1449=431%:%
%:%1450=432%:%
%:%1451=432%:%
%:%1452=432%:%
%:%1453=432%:%
%:%1454=433%:%
%:%1455=434%:%
%:%1456=434%:%
%:%1457=435%:%
%:%1458=435%:%
%:%1459=435%:%
%:%1460=435%:%
%:%1461=436%:%
%:%1462=436%:%
%:%1463=436%:%
%:%1464=436%:%
%:%1465=437%:%
%:%1466=437%:%
%:%1467=437%:%
%:%1468=437%:%
%:%1469=437%:%
%:%1470=438%:%
%:%1471=438%:%
%:%1472=439%:%
%:%1473=439%:%
%:%1474=440%:%
%:%1475=440%:%
%:%1476=441%:%
%:%1482=441%:%
%:%1485=442%:%
%:%1486=443%:%
%:%1487=443%:%
%:%1490=444%:%
%:%1494=444%:%
%:%1495=444%:%
%:%1496=444%:%
%:%1497=444%:%
%:%1502=444%:%
%:%1505=445%:%
%:%1506=446%:%
%:%1507=446%:%
%:%1508=447%:%
%:%1509=448%:%
%:%1510=449%:%
%:%1517=450%:%
%:%1518=450%:%
%:%1519=451%:%
%:%1520=451%:%
%:%1521=451%:%
%:%1522=451%:%
%:%1523=451%:%
%:%1524=452%:%
%:%1525=452%:%
%:%1526=452%:%
%:%1527=452%:%
%:%1528=452%:%
%:%1529=452%:%
%:%1530=453%:%
%:%1531=453%:%
%:%1532=453%:%
%:%1533=453%:%
%:%1534=453%:%
%:%1535=454%:%
%:%1541=454%:%
%:%1544=455%:%
%:%1545=456%:%
%:%1546=456%:%
%:%1547=457%:%
%:%1548=458%:%
%:%1551=459%:%
%:%1555=459%:%
%:%1556=459%:%
%:%1557=460%:%
%:%1558=460%:%
%:%1559=461%:%
%:%1560=461%:%
%:%1561=462%:%
%:%1562=462%:%
%:%1563=462%:%
%:%1564=462%:%
%:%1565=462%:%
%:%1566=463%:%
%:%1567=463%:%
%:%1568=463%:%
%:%1569=463%:%
%:%1570=464%:%
%:%1571=464%:%
%:%1572=465%:%
%:%1573=465%:%
%:%1574=466%:%
%:%1575=466%:%
%:%1576=466%:%
%:%1577=466%:%
%:%1578=467%:%
%:%1579=467%:%
%:%1580=467%:%
%:%1581=467%:%
%:%1582=467%:%
%:%1583=468%:%
%:%1584=468%:%
%:%1585=468%:%
%:%1586=468%:%
%:%1587=469%:%
%:%1588=469%:%
%:%1589=469%:%
%:%1590=469%:%
%:%1591=469%:%
%:%1592=469%:%
%:%1593=470%:%
%:%1594=470%:%
%:%1595=471%:%
%:%1596=471%:%
%:%1597=472%:%
%:%1598=472%:%
%:%1599=473%:%
%:%1600=473%:%
%:%1601=473%:%
%:%1602=473%:%
%:%1603=474%:%
%:%1604=474%:%
%:%1605=474%:%
%:%1606=475%:%
%:%1607=475%:%
%:%1608=475%:%
%:%1609=476%:%
%:%1610=477%:%
%:%1611=478%:%
%:%1612=479%:%
%:%1613=480%:%
%:%1614=480%:%
%:%1615=480%:%
%:%1616=481%:%
%:%1617=481%:%
%:%1618=481%:%
%:%1619=481%:%
%:%1620=481%:%
%:%1621=482%:%
%:%1622=482%:%
%:%1623=483%:%
%:%1624=483%:%
%:%1625=483%:%
%:%1626=483%:%
%:%1627=484%:%
%:%1628=484%:%
%:%1629=485%:%
%:%1630=485%:%
%:%1631=486%:%
%:%1632=486%:%
%:%1633=487%:%
%:%1634=487%:%
%:%1635=487%:%
%:%1636=487%:%
%:%1637=488%:%
%:%1638=488%:%
%:%1639=488%:%
%:%1640=489%:%
%:%1641=489%:%
%:%1642=489%:%
%:%1643=490%:%
%:%1644=491%:%
%:%1645=491%:%
%:%1646=492%:%
%:%1647=492%:%
%:%1648=492%:%
%:%1649=493%:%
%:%1650=493%:%
%:%1651=493%:%
%:%1652=493%:%
%:%1653=493%:%
%:%1654=494%:%
%:%1655=494%:%
%:%1656=495%:%
%:%1657=495%:%
%:%1658=496%:%
%:%1659=496%:%
%:%1660=496%:%
%:%1661=496%:%
%:%1662=496%:%
%:%1663=497%:%
%:%1664=497%:%
%:%1665=497%:%
%:%1666=498%:%
%:%1667=498%:%
%:%1668=498%:%
%:%1669=499%:%
%:%1670=499%:%
%:%1671=499%:%
%:%1672=500%:%
%:%1673=500%:%
%:%1674=501%:%
%:%1675=501%:%
%:%1676=502%:%
%:%1677=502%:%
%:%1678=502%:%
%:%1679=502%:%
%:%1680=502%:%
%:%1681=503%:%
%:%1682=503%:%
%:%1683=504%:%
%:%1684=504%:%
%:%1685=505%:%
%:%1686=505%:%
%:%1687=506%:%
%:%1688=506%:%
%:%1689=507%:%
%:%1690=507%:%
%:%1691=508%:%
%:%1692=508%:%
%:%1693=508%:%
%:%1694=509%:%
%:%1695=509%:%
%:%1696=509%:%
%:%1697=510%:%
%:%1698=510%:%
%:%1699=510%:%
%:%1700=510%:%
%:%1701=511%:%
%:%1702=511%:%
%:%1703=511%:%
%:%1704=512%:%
%:%1705=512%:%
%:%1706=512%:%
%:%1707=513%:%
%:%1708=513%:%
%:%1709=513%:%
%:%1710=513%:%
%:%1711=514%:%
%:%1712=514%:%
%:%1713=515%:%
%:%1714=516%:%
%:%1715=516%:%
%:%1716=517%:%
%:%1717=517%:%
%:%1718=517%:%
%:%1719=517%:%
%:%1720=517%:%
%:%1721=518%:%
%:%1722=518%:%
%:%1723=519%:%
%:%1724=519%:%
%:%1725=520%:%
%:%1731=520%:%
%:%1734=521%:%
%:%1735=522%:%
%:%1736=522%:%
%:%1737=523%:%
%:%1738=524%:%
%:%1741=525%:%
%:%1745=525%:%
%:%1746=525%:%
%:%1747=525%:%
%:%1752=525%:%
%:%1755=526%:%
%:%1756=527%:%
%:%1757=527%:%
%:%1760=528%:%
%:%1764=528%:%
%:%1765=528%:%
%:%1770=528%:%
%:%1773=529%:%
%:%1774=530%:%
%:%1775=531%:%
%:%1776=531%:%
%:%1779=532%:%
%:%1783=532%:%
%:%1784=532%:%
%:%1785=533%:%
%:%1786=533%:%
%:%1787=534%:%
%:%1788=534%:%
%:%1789=535%:%
%:%1790=535%:%
%:%1791=535%:%
%:%1792=535%:%
%:%1793=535%:%
%:%1794=536%:%
%:%1795=536%:%
%:%1796=536%:%
%:%1797=536%:%
%:%1798=537%:%
%:%1799=537%:%
%:%1800=538%:%
%:%1801=538%:%
%:%1802=539%:%
%:%1803=539%:%
%:%1804=539%:%
%:%1805=539%:%
%:%1806=540%:%
%:%1807=540%:%
%:%1808=540%:%
%:%1809=540%:%
%:%1810=540%:%
%:%1811=541%:%
%:%1812=541%:%
%:%1813=541%:%
%:%1814=541%:%
%:%1815=542%:%
%:%1816=542%:%
%:%1817=542%:%
%:%1818=542%:%
%:%1819=542%:%
%:%1820=543%:%
%:%1821=543%:%
%:%1822=543%:%
%:%1823=543%:%
%:%1824=544%:%
%:%1825=544%:%
%:%1826=544%:%
%:%1827=544%:%
%:%1828=545%:%
%:%1829=545%:%
%:%1830=545%:%
%:%1831=545%:%
%:%1832=545%:%
%:%1833=545%:%
%:%1834=546%:%
%:%1835=546%:%
%:%1836=547%:%
%:%1837=547%:%
%:%1838=548%:%
%:%1839=548%:%
%:%1840=549%:%
%:%1841=549%:%
%:%1842=549%:%
%:%1843=549%:%
%:%1844=550%:%
%:%1845=550%:%
%:%1846=550%:%
%:%1847=551%:%
%:%1848=551%:%
%:%1849=551%:%
%:%1850=552%:%
%:%1851=553%:%
%:%1852=554%:%
%:%1853=555%:%
%:%1854=556%:%
%:%1855=556%:%
%:%1856=556%:%
%:%1857=557%:%
%:%1858=557%:%
%:%1859=557%:%
%:%1860=557%:%
%:%1861=558%:%
%:%1862=559%:%
%:%1863=559%:%
%:%1864=559%:%
%:%1865=559%:%
%:%1866=560%:%
%:%1867=560%:%
%:%1868=561%:%
%:%1869=561%:%
%:%1870=561%:%
%:%1871=562%:%
%:%1872=562%:%
%:%1873=562%:%
%:%1874=563%:%
%:%1875=563%:%
%:%1876=563%:%
%:%1877=564%:%
%:%1883=564%:%
%:%1886=565%:%
%:%1887=566%:%
%:%1888=566%:%
%:%1889=567%:%
%:%1890=568%:%
%:%1891=568%:%
%:%1892=569%:%
%:%1893=570%:%
%:%1894=571%:%
%:%1895=571%:%
%:%1896=572%:%
%:%1897=573%:%
%:%1904=574%:%
%:%1905=574%:%
%:%1906=575%:%
%:%1907=575%:%
%:%1908=575%:%
%:%1909=575%:%
%:%1910=576%:%
%:%1911=576%:%
%:%1912=576%:%
%:%1913=576%:%
%:%1914=577%:%
%:%1915=577%:%
%:%1916=578%:%
%:%1917=578%:%
%:%1918=578%:%
%:%1919=578%:%
%:%1920=579%:%
%:%1921=579%:%
%:%1922=579%:%
%:%1923=579%:%
%:%1924=580%:%
%:%1925=580%:%
%:%1926=581%:%
%:%1927=581%:%
%:%1928=581%:%
%:%1929=581%:%
%:%1930=581%:%
%:%1931=581%:%
%:%1932=582%:%
%:%1933=582%:%
%:%1934=582%:%
%:%1935=582%:%
%:%1936=582%:%
%:%1937=582%:%
%:%1938=583%:%
%:%1939=583%:%
%:%1940=583%:%
%:%1941=583%:%
%:%1942=584%:%
%:%1943=585%:%
%:%1944=585%:%
%:%1945=585%:%
%:%1946=585%:%
%:%1947=585%:%
%:%1948=586%:%
%:%1949=586%:%
%:%1950=587%:%
%:%1951=587%:%
%:%1952=588%:%
%:%1953=588%:%
%:%1954=588%:%
%:%1955=589%:%
%:%1961=589%:%
%:%1964=590%:%
%:%1965=591%:%
%:%1966=591%:%
%:%1969=592%:%
%:%1973=592%:%
%:%1974=592%:%
%:%1975=592%:%
%:%1980=592%:%
%:%1983=593%:%
%:%1984=594%:%
%:%1985=594%:%
%:%1988=595%:%
%:%1992=595%:%
%:%1993=595%:%
%:%1994=595%:%
%:%1995=595%:%
%:%2009=598%:%
%:%2019=600%:%
%:%2020=600%:%
%:%2023=601%:%
%:%2027=601%:%
%:%2028=601%:%
%:%2029=602%:%
%:%2030=602%:%
%:%2031=603%:%
%:%2032=603%:%
%:%2033=604%:%
%:%2034=604%:%
%:%2035=604%:%
%:%2036=604%:%
%:%2037=604%:%
%:%2038=605%:%
%:%2039=605%:%
%:%2040=606%:%
%:%2041=606%:%
%:%2042=607%:%
%:%2043=607%:%
%:%2044=607%:%
%:%2045=607%:%
%:%2046=608%:%
%:%2047=608%:%
%:%2048=609%:%
%:%2049=610%:%
%:%2050=611%:%
%:%2051=612%:%
%:%2052=613%:%
%:%2053=614%:%
%:%2054=614%:%
%:%2055=614%:%
%:%2056=615%:%
%:%2057=615%:%
%:%2058=615%:%
%:%2059=615%:%
%:%2060=616%:%
%:%2061=616%:%
%:%2062=617%:%
%:%2063=617%:%
%:%2064=618%:%
%:%2065=618%:%
%:%2066=619%:%
%:%2067=619%:%
%:%2068=619%:%
%:%2069=619%:%
%:%2070=619%:%
%:%2071=620%:%
%:%2072=620%:%
%:%2073=621%:%
%:%2074=621%:%
%:%2075=621%:%
%:%2076=622%:%
%:%2077=622%:%
%:%2078=622%:%
%:%2079=622%:%
%:%2080=623%:%
%:%2081=623%:%
%:%2082=623%:%
%:%2083=624%:%
%:%2084=624%:%
%:%2085=624%:%
%:%2086=624%:%
%:%2087=624%:%
%:%2088=625%:%
%:%2089=625%:%
%:%2090=625%:%
%:%2091=626%:%
%:%2092=626%:%
%:%2093=626%:%
%:%2094=627%:%
%:%2095=627%:%
%:%2096=627%:%
%:%2097=627%:%
%:%2098=627%:%
%:%2099=627%:%
%:%2100=628%:%
%:%2101=628%:%
%:%2102=629%:%
%:%2103=629%:%
%:%2104=630%:%
%:%2105=630%:%
%:%2106=630%:%
%:%2107=630%:%
%:%2108=630%:%
%:%2109=630%:%
%:%2110=631%:%
%:%2111=631%:%
%:%2112=632%:%
%:%2113=632%:%
%:%2118=632%:%
%:%2121=633%:%
%:%2122=634%:%
%:%2123=634%:%
%:%2126=635%:%
%:%2130=635%:%
%:%2131=635%:%
%:%2132=635%:%
%:%2137=635%:%
%:%2140=636%:%
%:%2141=637%:%
%:%2142=637%:%
%:%2143=638%:%
%:%2144=639%:%
%:%2145=640%:%
%:%2146=641%:%
%:%2153=642%:%
%:%2154=642%:%
%:%2155=643%:%
%:%2156=643%:%
%:%2157=644%:%
%:%2158=644%:%
%:%2159=644%:%
%:%2160=644%:%
%:%2161=644%:%
%:%2162=645%:%
%:%2163=645%:%
%:%2164=645%:%
%:%2165=646%:%
%:%2166=646%:%
%:%2167=647%:%
%:%2168=647%:%
%:%2169=648%:%
%:%2170=648%:%
%:%2171=649%:%
%:%2172=649%:%
%:%2173=649%:%
%:%2174=649%:%
%:%2175=650%:%
%:%2176=650%:%
%:%2177=651%:%
%:%2178=651%:%
%:%2179=652%:%
%:%2180=652%:%
%:%2181=652%:%
%:%2182=652%:%
%:%2183=653%:%
%:%2184=653%:%
%:%2185=653%:%
%:%2186=653%:%
%:%2187=654%:%
%:%2188=654%:%
%:%2189=654%:%
%:%2190=654%:%
%:%2191=655%:%
%:%2192=655%:%
%:%2193=656%:%
%:%2194=657%:%
%:%2195=658%:%
%:%2196=658%:%
%:%2197=658%:%
%:%2198=659%:%
%:%2199=659%:%
%:%2200=659%:%
%:%2201=659%:%
%:%2202=660%:%
%:%2203=660%:%
%:%2204=661%:%
%:%2205=661%:%
%:%2206=661%:%
%:%2207=662%:%
%:%2208=662%:%
%:%2209=663%:%
%:%2210=663%:%
%:%2211=663%:%
%:%2212=664%:%
%:%2213=664%:%
%:%2214=665%:%
%:%2215=665%:%
%:%2216=665%:%
%:%2217=666%:%
%:%2218=666%:%
%:%2219=666%:%
%:%2220=666%:%
%:%2221=666%:%
%:%2222=667%:%
%:%2223=667%:%
%:%2224=667%:%
%:%2225=667%:%
%:%2226=667%:%
%:%2227=668%:%
%:%2228=668%:%
%:%2229=668%:%
%:%2230=668%:%
%:%2231=668%:%
%:%2232=669%:%
%:%2233=669%:%
%:%2234=669%:%
%:%2235=669%:%
%:%2236=669%:%
%:%2237=670%:%
%:%2238=670%:%
%:%2239=670%:%
%:%2240=670%:%
%:%2241=671%:%
%:%2242=671%:%
%:%2243=672%:%
%:%2244=672%:%
%:%2245=673%:%
%:%2246=673%:%
%:%2247=674%:%
%:%2248=674%:%
%:%2249=674%:%
%:%2250=674%:%
%:%2251=674%:%
%:%2252=675%:%
%:%2253=675%:%
%:%2254=675%:%
%:%2255=675%:%
%:%2256=676%:%
%:%2257=676%:%
%:%2258=676%:%
%:%2259=676%:%
%:%2260=677%:%
%:%2261=677%:%
%:%2262=677%:%
%:%2263=678%:%
%:%2264=678%:%
%:%2265=678%:%
%:%2266=678%:%
%:%2267=679%:%
%:%2268=679%:%
%:%2269=679%:%
%:%2270=679%:%
%:%2271=680%:%
%:%2272=680%:%
%:%2273=680%:%
%:%2274=681%:%
%:%2275=681%:%
%:%2276=682%:%
%:%2277=682%:%
%:%2278=683%:%
%:%2279=683%:%
%:%2280=683%:%
%:%2281=683%:%
%:%2282=683%:%
%:%2283=684%:%
%:%2284=684%:%
%:%2285=684%:%
%:%2286=684%:%
%:%2287=684%:%
%:%2288=685%:%
%:%2289=685%:%
%:%2290=686%:%
%:%2291=686%:%
%:%2292=687%:%
%:%2293=687%:%
%:%2294=688%:%
%:%2300=688%:%
%:%2303=689%:%
%:%2304=690%:%
%:%2305=690%:%
%:%2308=691%:%
%:%2312=691%:%
%:%2313=691%:%
%:%2314=691%:%
%:%2315=691%:%
%:%2320=691%:%
%:%2323=692%:%
%:%2324=693%:%
%:%2325=693%:%
%:%2328=694%:%
%:%2332=694%:%
%:%2333=694%:%
%:%2334=694%:%
%:%2335=695%:%
%:%2336=695%:%
%:%2341=695%:%
%:%2344=696%:%
%:%2345=697%:%
%:%2346=698%:%
%:%2347=698%:%
%:%2350=699%:%
%:%2355=700%:%



% optional bibliography
\bibliographystyle{abbrv}
\bibliography{root}

\end{document}

%%% Local Variables:
%%% mode: latex
%%% TeX-master: t
%%% End:
