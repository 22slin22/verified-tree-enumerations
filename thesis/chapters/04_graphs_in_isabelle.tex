\chapter{Formalization of Graph Theoretic Results}\label{chapter:isabelle_graphs}

In the previous chapters, the main focus was on the algorithms at hand and the approach to formalize them.
While we mentioned some graph theoretic definitions we took well known definitions and lemmas on trees and graphs for granted.
Some results were already formalized by the undirected graph library but further ones were needed for this project.
This chapter tries to summarize some graph theoretic results that were formalized in the process.
These are well suited to be reused in other projects.

We already discussed the definition of trees as acyclic, connected graphs.
We also provide the standard definition of leaves as vertices of degree 1.
When proving properties of trees it is often useful to prove them by induction on the number of vertices.
To do this we show that removing a leaf from a tree results in another tree.
To simplify proving properties on trees inductively we provide a custom induction rule.
One then only needs to prove the property for the singleton tree and that the property is preserved when adding a new leaf.

\begin{isabellebox}
    \isacommand{lemma}\isamarkupfalse%
    \ tree{\isacharunderscore}{\kern0pt}induct\ {\isacharbrackleft}{\kern0pt}case{\isacharunderscore}{\kern0pt}names\ singolton\ insert{\isacharcomma}{\kern0pt}\ induct\ set{\isacharcolon}{\kern0pt}\ tree{\isacharbrackright}{\kern0pt}{\isacharcolon}{\kern0pt}\isanewline
    \ \ \isakeyword{assumes}\ tree{\isacharcolon}{\kern0pt}\ {\isachardoublequoteopen}tree\ V\ E{\isachardoublequoteclose}\isanewline
    \ \ \ \ \isakeyword{and}\ trivial{\isacharcolon}{\kern0pt}\ {\isachardoublequoteopen}{\isasymAnd}v{\isachardot}{\kern0pt}\ tree\ {\isacharbraceleft}{\kern0pt}v{\isacharbraceright}{\kern0pt}\ {\isacharbraceleft}{\kern0pt}{\isacharbraceright}{\kern0pt}\ {\isasymLongrightarrow}\ P\ {\isacharbraceleft}{\kern0pt}v{\isacharbraceright}{\kern0pt}\ {\isacharbraceleft}{\kern0pt}{\isacharbraceright}{\kern0pt}{\isachardoublequoteclose}\isanewline
    \ \ \ \ \isakeyword{and}\ insert{\isacharcolon}{\kern0pt}\ {\isachardoublequoteopen}{\isasymAnd}l\ v\ V\ E{\isachardot}{\kern0pt}\ tree\ V\ E\ {\isasymLongrightarrow}\ P\ V\ E\ {\isasymLongrightarrow}\ l\ {\isasymnotin}\ V\ {\isasymLongrightarrow}\ v\ {\isasymin}\ V\ {\isasymLongrightarrow}\isanewline
    \ \ \ \ \ \ {\isacharbraceleft}{\kern0pt}l{\isacharcomma}{\kern0pt}v{\isacharbraceright}{\kern0pt}\ {\isasymnotin}\ E\ {\isasymLongrightarrow}\ tree{\isachardot}{\kern0pt}leaf\ {\isacharparenleft}{\kern0pt}insert\ {\isacharbraceleft}{\kern0pt}l{\isacharcomma}{\kern0pt}v{\isacharbraceright}{\kern0pt}\ E{\isacharparenright}{\kern0pt}\ l\ {\isasymLongrightarrow}\isanewline
    \ \ \ \ \ \ P\ {\isacharparenleft}{\kern0pt}insert\ l\ V{\isacharparenright}{\kern0pt}\ {\isacharparenleft}{\kern0pt}insert\ {\isacharbraceleft}{\kern0pt}l{\isacharcomma}{\kern0pt}v{\isacharbraceright}{\kern0pt}\ E{\isacharparenright}{\kern0pt}{\isachardoublequoteclose}\isanewline
    \ \ \isakeyword{shows}\ {\isachardoublequoteopen}P\ V\ E{\isachardoublequoteclose}
\end{isabellebox}

With the additional assumptions in the inductive step, proofs of properties on trees, such as that a tree with $n$ vertices has $n-1$ edges, can be simplified a great deal.

When proving that a graph is a tree, arguing about acyclicity is in many cases difficult as it can require one to argue about all possible paths in the graphs.
Instead, like in Section \ref{section:labeled_tree_enum_correct}, we use the well-known fact that a connected graph with $n$ vertices and $n-1$ edges is a tree.
This lemma turns out to require a fair bit of setup.
First, we define a spanning tree of a graph $G$ to be a subgraph of $G$ that is a tree and that contains all vertices of $G$.

\begin{isabellebox}
    \isacommand{locale}\isamarkupfalse%
    \ spanning{\isacharunderscore}{\kern0pt}tree\ {\isacharequal}{\kern0pt}\ ulgraph\ V\ E\ {\isacharplus}{\kern0pt}\ T{\isacharcolon}{\kern0pt}\ tree\ V\ T\ \isakeyword{for}\ V\ E\ T\ {\isacharplus}{\kern0pt}\isanewline
    \ \ \isakeyword{assumes}\ subgraph{\isacharcolon}{\kern0pt}\ {\isachardoublequoteopen}T\ {\isasymsubseteq}\ E{\isachardoublequoteclose}
\end{isabellebox}

Next, we prove that every finite, connected graph has a spanning tree.
This is done inductively by removing edges of cycles, which keeps the graph connected, until the graph is acyclic.
Now to prove the lemma we were interested in initially, that every connected graph with $n$ vertices and $n-1$ edges is a tree,
let us assume for contradiction that there exists such a graph that is not a tree.
As the graph is connected, it has a spanning tree and since it is not a tree itself, the spanning tree has to be a strict subgraph.
Since a spanning tree is a tree it has $n-1$ edges by the statement proven earlier.
That means the original graph has more than $n-1$ edges, which is a contradiction.
That concludes the proof.

Another concept that had not yet been formalized is connected components.
This was required for rooted trees, where removing the root node leaves the principal subtrees as connected components.
For a formal definition we first define a relation on nodes that indicates connectivity.

\begin{isabellebox}
    \isacommand{abbreviation}\isamarkupfalse%
    \ {\isachardoublequoteopen}vert{\isacharunderscore}{\kern0pt}connected{\isacharunderscore}{\kern0pt}rel\ {\isasymequiv}\ {\isacharbraceleft}{\kern0pt}{\isacharparenleft}{\kern0pt}u{\isacharcomma}{\kern0pt}v{\isacharparenright}{\kern0pt}{\isachardot}{\kern0pt}\ vert{\isacharunderscore}{\kern0pt}connected\ u\ v{\isacharbraceright}{\kern0pt}{\isachardoublequoteclose}
\end{isabellebox}

We prove this to be an equivalence relation and define connected components to be the equivalence classes of this relation.

\begin{isabellebox}
    \isacommand{definition}\isamarkupfalse%
    \ connected{\isacharunderscore}{\kern0pt}components\ {\isacharcolon}{\kern0pt}{\isacharcolon}{\kern0pt}\ {\isachardoublequoteopen}{\isacharprime}{\kern0pt}a\ set\ set{\isachardoublequoteclose}\ \isakeyword{where}\isanewline
    \ \ {\isachardoublequoteopen}connected{\isacharunderscore}{\kern0pt}components\ {\isacharequal}{\kern0pt}\ V\ {\isacharslash}{\kern0pt}{\isacharslash}{\kern0pt}\ vert{\isacharunderscore}{\kern0pt}connected{\isacharunderscore}{\kern0pt}rel{\isachardoublequoteclose}
\end{isabellebox}

A couple of interesting properties of connected components that were needed for the project were proven as well.

Additionally, there are dozens more lemmas for graphs, paths, connectivity, subgraphs, vertex degrees etc. formalized.
One well known fact which might be of interest for other projects is the handshaking lemma which relates the number of edges with the sum of vertex degrees.

\begin{isabellebox}
    \isacommand{lemma}\isamarkupfalse%
    \ {\isacharparenleft}{\kern0pt}\isakeyword{in}\ fin{\isacharunderscore}{\kern0pt}ulgraph{\isacharparenright}{\kern0pt}\ handshaking{\isacharcolon}{\kern0pt}\ {\isachardoublequoteopen}{\isacharparenleft}{\kern0pt}{\isasymSum}v{\isasymin}V{\isachardot}{\kern0pt}\ degree\ v{\isacharparenright}{\kern0pt}\ {\isacharequal}{\kern0pt}\ {\isadigit{2}}\ {\isacharasterisk}{\kern0pt}\ card\ E{\isachardoublequoteclose}
\end{isabellebox}

The interested reader is referred to the Archive of Formal Proofs entry for the full development.
